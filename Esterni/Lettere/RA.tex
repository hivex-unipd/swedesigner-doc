\documentclass[a4paper]{letter} % non possiamo usare hx-ambiente
\usepackage{accanthis} % scegliere da http://www.tug.dk/FontCatalogue/
\usepackage[T1]{fontenc}
\usepackage[utf8]{inputenc}
\usepackage[italian]{babel}
\usepackage{microtype}
\usepackage{eurosym}
\usepackage{siunitx}
\usepackage{graphicx}
\usepackage{hyperref}
\usepackage{../../util/hx-macro}
\usepackage{../../util/hx-vers}
\date{Padova, \today}
\signature{\LB\\Responsabile \hx{}\\
\includegraphics[scale=1]{../Piano_di_progetto/img/firmalb.jpg}}

\begin{document}

\begin{letter}{Egr. Prof. Tullio Vardanega\\
Università degli Studi di Padova\\
Dipartimento di Matematica “Tullio Levi-Civita”\\
via Trieste, 63\\
35121 Padova}

\includegraphics[width=80px]{../../util/hivex_logo3.png}

\opening{Egregio Professor Vardanega,}

Con la presente, il gruppo \hx{} intende comunicarLe la partecipazione alla Revisione di Accettazione del progetto didattico.

Le alleghiamo i seguenti documenti, che trattano in maniera approfondita gli aspetti di pianificazione, normazione, qualifica, analisi e progettazione:
\begin{itemize}
	\item \NdP;
	\item \PdP;
	\item \PdQ;
	\item \AdR;
	\item \ST;
	\item \DP;
	\item \Glossario;
	\item \MU;
	\item Verbali delle riunioni esterne:
	\begin{itemize}
		\item \emph{VE\_16-12-15.pdf};
		\item \emph{VE\_16-12-21.pdf};
		\item \emph{VE\_17-01-10.pdf};
		\item \emph{VE\_17-03-20.pdf};
		\item \emph{VE\_17-03-23.pdf};
		\item \emph{VE\_17-05-04.pdf}; % TODO
	\end{itemize}
	\item Verbali delle riunioni interne:
	\begin{itemize}
		\item \emph{VI\_16-11-29.pdf};
		\item \emph{VI\_16-12-12.pdf};
		\item \emph{VI\_16-12-20.pdf};
		\item \emph{VI\_16-12-22.pdf};
		\item \emph{VI\_16-12-23.pdf};
		\item \emph{VI\_17-01-02.pdf};
		\item \emph{VI\_17-02-01.pdf};
		\item \emph{VI\_17-02-08.pdf};
		\item \emph{VI\_17-02-15.pdf};
		\item \emph{VI\_17-03-17.pdf};
		\item \emph{VI\_17-05-01.pdf}.
	\end{itemize}
\end{itemize}

Consegniamo il prodotto richiesto in data \textbf{12 maggio 2017} con un consuntivo di \num{11349} \euro{} (euro), come rendicontato nel documento \PdP. All'indirizzo web \url{https://github.com/hivex-unipd/swedesigner} può reperire il codice sorgente del prodotto completo, \emph{open source} sotto licenza pubblica Mozilla 2.0.

\closing{Distinti Saluti,}

\end{letter}

\end{document}
