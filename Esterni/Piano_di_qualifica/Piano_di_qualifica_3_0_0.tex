% Piano di Qualifica
% da compilare con il comando pdflatex Piano_di_Qualifica_x.x.x.tex

% Dichiarazioni di ambiente e inclusione di pacchetti
% da usare tramite il comando % Dichiarazioni di ambiente e inclusione di pacchetti
% da usare tramite il comando % Dichiarazioni di ambiente e inclusione di pacchetti
% da usare tramite il comando \input{../../util/hx-ambiente}

\documentclass[a4paper,titlepage]{article}
\usepackage[T1]{fontenc}
\usepackage[utf8]{inputenc}
\usepackage[english,italian]{babel}
\usepackage{microtype}
\usepackage{lmodern}
\usepackage{underscore}
\usepackage{graphicx}
\usepackage{eurosym}
\usepackage{float}
\usepackage{fancyhdr}
\usepackage[table,dvipsnames]{xcolor}
\usepackage{multirow}
\usepackage{longtable}
\usepackage{chngpage}
\usepackage{grffile}
\usepackage[titles]{tocloft}
\usepackage{hyperref}
\hypersetup{hidelinks}

\usepackage{../../util/hx-vers}
\usepackage{../../util/hx-macro}
\usepackage{../../util/hx-front}

% solo se si vuole una nuova pagina ad ogni \section:
\usepackage{titlesec}
\newcommand{\sectionbreak}{\clearpage}

% stile di pagina:
\pagestyle{fancy}

% solo se si vuole eliminare l'indentazione ad ogni paragrafo:
\setlength{\parindent}{0pt}

% intestazione:
\lhead{\Large{\proj}}
\rhead{\includegraphics[keepaspectratio=true,width=50px]{../../util/hivex_logo2.png}}
\renewcommand{\headrulewidth}{0.4pt}

% pie' di pagina:
\lfoot{\email}
\rfoot{\thepage}
\cfoot{}
\renewcommand{\footrulewidth}{0.4pt}

% spazio verticale tra le celle di una tabella:
\renewcommand{\arraystretch}{1.5}

% profondità di indicizzazione:
\setcounter{tocdepth}{4}
\setcounter{secnumdepth}{4}

% numerazione innestata per elenchi numerati:
\renewcommand{\labelenumii}{\theenumii}
\renewcommand{\theenumii}{\theenumi.\arabic{enumii}.}


\documentclass[a4paper,titlepage]{article}
\usepackage[T1]{fontenc}
\usepackage[utf8]{inputenc}
\usepackage[english,italian]{babel}
\usepackage{microtype}
\usepackage{lmodern}
\usepackage{underscore}
\usepackage{graphicx}
\usepackage{eurosym}
\usepackage{float}
\usepackage{fancyhdr}
\usepackage[table,dvipsnames]{xcolor}
\usepackage{multirow}
\usepackage{longtable}
\usepackage{chngpage}
\usepackage{grffile}
\usepackage[titles]{tocloft}
\usepackage{hyperref}
\hypersetup{hidelinks}

\usepackage{../../util/hx-vers}
\usepackage{../../util/hx-macro}
\usepackage{../../util/hx-front}

% solo se si vuole una nuova pagina ad ogni \section:
\usepackage{titlesec}
\newcommand{\sectionbreak}{\clearpage}

% stile di pagina:
\pagestyle{fancy}

% solo se si vuole eliminare l'indentazione ad ogni paragrafo:
\setlength{\parindent}{0pt}

% intestazione:
\lhead{\Large{\proj}}
\rhead{\includegraphics[keepaspectratio=true,width=50px]{../../util/hivex_logo2.png}}
\renewcommand{\headrulewidth}{0.4pt}

% pie' di pagina:
\lfoot{\email}
\rfoot{\thepage}
\cfoot{}
\renewcommand{\footrulewidth}{0.4pt}

% spazio verticale tra le celle di una tabella:
\renewcommand{\arraystretch}{1.5}

% profondità di indicizzazione:
\setcounter{tocdepth}{4}
\setcounter{secnumdepth}{4}

% numerazione innestata per elenchi numerati:
\renewcommand{\labelenumii}{\theenumii}
\renewcommand{\theenumii}{\theenumi.\arabic{enumii}.}


\documentclass[a4paper,titlepage]{article}
\usepackage[T1]{fontenc}
\usepackage[utf8]{inputenc}
\usepackage[english,italian]{babel}
\usepackage{microtype}
\usepackage{lmodern}
\usepackage{underscore}
\usepackage{graphicx}
\usepackage{eurosym}
\usepackage{float}
\usepackage{fancyhdr}
\usepackage[table,dvipsnames]{xcolor}
\usepackage{multirow}
\usepackage{longtable}
\usepackage{chngpage}
\usepackage{grffile}
\usepackage[titles]{tocloft}
\usepackage{hyperref}
\hypersetup{hidelinks}

\usepackage{../../util/hx-vers}
\usepackage{../../util/hx-macro}
\usepackage{../../util/hx-front}

% solo se si vuole una nuova pagina ad ogni \section:
\usepackage{titlesec}
\newcommand{\sectionbreak}{\clearpage}

% stile di pagina:
\pagestyle{fancy}

% solo se si vuole eliminare l'indentazione ad ogni paragrafo:
\setlength{\parindent}{0pt}

% intestazione:
\lhead{\Large{\proj}}
\rhead{\includegraphics[keepaspectratio=true,width=50px]{../../util/hivex_logo2.png}}
\renewcommand{\headrulewidth}{0.4pt}

% pie' di pagina:
\lfoot{\email}
\rfoot{\thepage}
\cfoot{}
\renewcommand{\footrulewidth}{0.4pt}

% spazio verticale tra le celle di una tabella:
\renewcommand{\arraystretch}{1.5}

% profondità di indicizzazione:
\setcounter{tocdepth}{4}
\setcounter{secnumdepth}{4}

% numerazione innestata per elenchi numerati:
\renewcommand{\labelenumii}{\theenumii}
\renewcommand{\theenumii}{\theenumi.\arabic{enumii}.}

\usepackage{float}
\usepackage{varwidth}

\version{4.0.0}
\creaz{28 dicembre 2016}
\author{\MM}
\supervisor{\PB}
\uso{esterno}
\dest{\TV, \RC, \ZU}
\title{Piano di Qualifica}

% per andare a capo in una cella "multirow":
\newcommand\ambito[2]{%
	\multirow{#1}*{%
		\begin{varwidth}{3cm}%
		#2%
		\end{varwidth}}}

% per neutralizzare \nogloxy:
\newcommand{\nogloxy}[1]{#1}



\begin{document}
\maketitle
% diario delle modifiche per l'analisi dei requisiti
% da includere con % diario delle modifiche per l'analisi dei requisiti
% da includere con % diario delle modifiche per l'analisi dei requisiti
% da includere con \include{diario}

\begin{diario}
	4.0.0 & {\LB} (Responsabile) & 02/05/2017 & Approvazione del documento \\ \hline
	3.1.0 & {\PB} (Verificatore) & 02/05/2017 & Verifica del documento \\ \hline
	3.0.1 & {\MM} (Analista) & 01/05/2017 & 
	\begin{itemize}
	\item Inserimento UC5.35 e relativo requisito;
	\item Inserimento UC8 e relativo requisito;
	\item Inserimento tabella Requisiti Implementati come appendice.
\end{itemize}\\ \hline
	3.0.0 & {\AZ} (Responsabile) & 19/03/2017 & Approvazione del documento \\ \hline
	2.1.0 & {\MM} (Verificatore) & 19/03/2017 & Verifica del documento \\ \hline
	2.0.3 & {\PB} (Progettista) & 18/03/2017 &  
\begin{itemize}
	\item Modifica tabella Tracciamento Fonti-Requisiti;
	\item Modifica tabella Requisiti-Fonti;
	\item Modifica Estensione UC7.
\end{itemize}\\ \hline
	2.0.2 & {\PB} (Progettista) & 17/03/2017 &  Ristrutturato UC5 e relativi requisiti\\ \hline
	2.0.1 & {\PB} (Progettista) & 16/03/2017 &  Ristrutturato UC4 e relativi requisiti\\ \hline
	2.0.0 & {\LS} (Responsabile) & 01/02/2017 & Approvazione del documento \\ \hline
	1.1.0 & {\GG} (Verificatore) & 01/02/2017 & Verifica del documento \\ \hline
	1.0.4 & {\AZ} (Analista) & 31/01/2017 & Inserito UC5.26 con relativo requisito e tracciamento nelle tabelle e inseriti i requisiti RFO7, RFO8, RFO8.1, RFO8.2, RFO9, RFO10 e RFO11\\ \hline
	1.0.3 & {\AZ} (Analista) & 29/01/2017 & Corretta la descrizione dello UC5 e approfondita la descrizione dello UC7 \\ \hline
	1.0.2 & {\AZ} (Analista) & 28/01/2017 & Corretti UC4.1.6.3.2, UC4.2.1 e inserito perimetro sistema del UC5\\ \hline
	1.0.1 & {\AZ} (Analista) & 26/01/2017 & Inserimento scenario alternativo allo UC2, creazione UC3.1 con relativo requisito e tracciamento nelle tabelle e corrette alcune postcondizioni \\ \hline
	1.0.0 & {\LB} (Responsabile) & 09/01/2017 & Approvazione documento \\ \hline
	0.4.0 & {\LS} (Verificatore) & 06/01/2017 & Verifica introduzione, descrizione generale e requisiti \\ \hline
	0.3.0 & {\MM} (Verificatore) & 06/01/2017 & Verifica UC5.3-UC7 \\ \hline
	0.2.0 & {\LB} (Verificatore) & 06/01/2017 & Verifica UC4.2-UC5.2 \\ \hline
	0.1.0 & {\AZ} (Verificatore) & 06/01/2017 & Verifica UC1-4.1.8 \\ \hline
	0.0.11 & {\LS} (Analista) & 04/01/2017 & Stesura UC6-UC7 \\ \hline
	0.0.10 & {\GG} (Analista) & 03/01/2017 & Stesura UC5.6-UC5.18 \\ \hline
	0.0.9 & {\LS} (Analista) & 03/01/2017 & Stesura UC5.3-UC5.5.6.1 \\ \hline
	0.0.8 & {\PB} (Analista) & 02/01/2017 & Stesura UC5-UC5.2 \\ \hline
	0.0.7 & {\AZ} (Analista) & 02/01/2017 & Stesura UC4.3.3.1-UC4.11 \\ \hline
	0.0.6 & {\MM} (Analista) & 30/12/2016 & Stesura UC4.2-UC4.3.3.1 \\ \hline
	0.0.5 & {\GG} (Analista) & 29/12/2016 & Stesura UC4.1.6-UC4.1.8 \\ \hline
	0.0.4 & {\PB} (Analista) & 29/12/2016 & Stesura UC4-UC4.1.5 \\ \hline
	0.0.3 & {\LB} (Analista) & 28/12/2016 & Stesura UC1-UC2-UC3 \\ \hline
	0.0.2 & {\LS} (Analista) & 27/12/2016 & Stesura introduzione e descrizione generale \\ \hline
	0.0.1 & {\AZ} (Analista) & 27/12/2016 & Stesura scheletro \\ \hline
\end{diario}


\begin{diario}
	4.0.0 & {\LB} (Responsabile) & 02/05/2017 & Approvazione del documento \\ \hline
	3.1.0 & {\PB} (Verificatore) & 02/05/2017 & Verifica del documento \\ \hline
	3.0.1 & {\MM} (Analista) & 01/05/2017 & 
	\begin{itemize}
	\item Inserimento UC5.35 e relativo requisito;
	\item Inserimento UC8 e relativo requisito;
	\item Inserimento tabella Requisiti Implementati come appendice.
\end{itemize}\\ \hline
	3.0.0 & {\AZ} (Responsabile) & 19/03/2017 & Approvazione del documento \\ \hline
	2.1.0 & {\MM} (Verificatore) & 19/03/2017 & Verifica del documento \\ \hline
	2.0.3 & {\PB} (Progettista) & 18/03/2017 &  
\begin{itemize}
	\item Modifica tabella Tracciamento Fonti-Requisiti;
	\item Modifica tabella Requisiti-Fonti;
	\item Modifica Estensione UC7.
\end{itemize}\\ \hline
	2.0.2 & {\PB} (Progettista) & 17/03/2017 &  Ristrutturato UC5 e relativi requisiti\\ \hline
	2.0.1 & {\PB} (Progettista) & 16/03/2017 &  Ristrutturato UC4 e relativi requisiti\\ \hline
	2.0.0 & {\LS} (Responsabile) & 01/02/2017 & Approvazione del documento \\ \hline
	1.1.0 & {\GG} (Verificatore) & 01/02/2017 & Verifica del documento \\ \hline
	1.0.4 & {\AZ} (Analista) & 31/01/2017 & Inserito UC5.26 con relativo requisito e tracciamento nelle tabelle e inseriti i requisiti RFO7, RFO8, RFO8.1, RFO8.2, RFO9, RFO10 e RFO11\\ \hline
	1.0.3 & {\AZ} (Analista) & 29/01/2017 & Corretta la descrizione dello UC5 e approfondita la descrizione dello UC7 \\ \hline
	1.0.2 & {\AZ} (Analista) & 28/01/2017 & Corretti UC4.1.6.3.2, UC4.2.1 e inserito perimetro sistema del UC5\\ \hline
	1.0.1 & {\AZ} (Analista) & 26/01/2017 & Inserimento scenario alternativo allo UC2, creazione UC3.1 con relativo requisito e tracciamento nelle tabelle e corrette alcune postcondizioni \\ \hline
	1.0.0 & {\LB} (Responsabile) & 09/01/2017 & Approvazione documento \\ \hline
	0.4.0 & {\LS} (Verificatore) & 06/01/2017 & Verifica introduzione, descrizione generale e requisiti \\ \hline
	0.3.0 & {\MM} (Verificatore) & 06/01/2017 & Verifica UC5.3-UC7 \\ \hline
	0.2.0 & {\LB} (Verificatore) & 06/01/2017 & Verifica UC4.2-UC5.2 \\ \hline
	0.1.0 & {\AZ} (Verificatore) & 06/01/2017 & Verifica UC1-4.1.8 \\ \hline
	0.0.11 & {\LS} (Analista) & 04/01/2017 & Stesura UC6-UC7 \\ \hline
	0.0.10 & {\GG} (Analista) & 03/01/2017 & Stesura UC5.6-UC5.18 \\ \hline
	0.0.9 & {\LS} (Analista) & 03/01/2017 & Stesura UC5.3-UC5.5.6.1 \\ \hline
	0.0.8 & {\PB} (Analista) & 02/01/2017 & Stesura UC5-UC5.2 \\ \hline
	0.0.7 & {\AZ} (Analista) & 02/01/2017 & Stesura UC4.3.3.1-UC4.11 \\ \hline
	0.0.6 & {\MM} (Analista) & 30/12/2016 & Stesura UC4.2-UC4.3.3.1 \\ \hline
	0.0.5 & {\GG} (Analista) & 29/12/2016 & Stesura UC4.1.6-UC4.1.8 \\ \hline
	0.0.4 & {\PB} (Analista) & 29/12/2016 & Stesura UC4-UC4.1.5 \\ \hline
	0.0.3 & {\LB} (Analista) & 28/12/2016 & Stesura UC1-UC2-UC3 \\ \hline
	0.0.2 & {\LS} (Analista) & 27/12/2016 & Stesura introduzione e descrizione generale \\ \hline
	0.0.1 & {\AZ} (Analista) & 27/12/2016 & Stesura scheletro \\ \hline
\end{diario}


\begin{diario}
	4.0.0 & {\LB} (Responsabile) & 02/05/2017 & Approvazione del documento \\ \hline
	3.1.0 & {\PB} (Verificatore) & 02/05/2017 & Verifica del documento \\ \hline
	3.0.1 & {\MM} (Analista) & 01/05/2017 & 
	\begin{itemize}
	\item Inserimento UC5.35 e relativo requisito;
	\item Inserimento UC8 e relativo requisito;
	\item Inserimento tabella Requisiti Implementati come appendice.
\end{itemize}\\ \hline
	3.0.0 & {\AZ} (Responsabile) & 19/03/2017 & Approvazione del documento \\ \hline
	2.1.0 & {\MM} (Verificatore) & 19/03/2017 & Verifica del documento \\ \hline
	2.0.3 & {\PB} (Progettista) & 18/03/2017 &  
\begin{itemize}
	\item Modifica tabella Tracciamento Fonti-Requisiti;
	\item Modifica tabella Requisiti-Fonti;
	\item Modifica Estensione UC7.
\end{itemize}\\ \hline
	2.0.2 & {\PB} (Progettista) & 17/03/2017 &  Ristrutturato UC5 e relativi requisiti\\ \hline
	2.0.1 & {\PB} (Progettista) & 16/03/2017 &  Ristrutturato UC4 e relativi requisiti\\ \hline
	2.0.0 & {\LS} (Responsabile) & 01/02/2017 & Approvazione del documento \\ \hline
	1.1.0 & {\GG} (Verificatore) & 01/02/2017 & Verifica del documento \\ \hline
	1.0.4 & {\AZ} (Analista) & 31/01/2017 & Inserito UC5.26 con relativo requisito e tracciamento nelle tabelle e inseriti i requisiti RFO7, RFO8, RFO8.1, RFO8.2, RFO9, RFO10 e RFO11\\ \hline
	1.0.3 & {\AZ} (Analista) & 29/01/2017 & Corretta la descrizione dello UC5 e approfondita la descrizione dello UC7 \\ \hline
	1.0.2 & {\AZ} (Analista) & 28/01/2017 & Corretti UC4.1.6.3.2, UC4.2.1 e inserito perimetro sistema del UC5\\ \hline
	1.0.1 & {\AZ} (Analista) & 26/01/2017 & Inserimento scenario alternativo allo UC2, creazione UC3.1 con relativo requisito e tracciamento nelle tabelle e corrette alcune postcondizioni \\ \hline
	1.0.0 & {\LB} (Responsabile) & 09/01/2017 & Approvazione documento \\ \hline
	0.4.0 & {\LS} (Verificatore) & 06/01/2017 & Verifica introduzione, descrizione generale e requisiti \\ \hline
	0.3.0 & {\MM} (Verificatore) & 06/01/2017 & Verifica UC5.3-UC7 \\ \hline
	0.2.0 & {\LB} (Verificatore) & 06/01/2017 & Verifica UC4.2-UC5.2 \\ \hline
	0.1.0 & {\AZ} (Verificatore) & 06/01/2017 & Verifica UC1-4.1.8 \\ \hline
	0.0.11 & {\LS} (Analista) & 04/01/2017 & Stesura UC6-UC7 \\ \hline
	0.0.10 & {\GG} (Analista) & 03/01/2017 & Stesura UC5.6-UC5.18 \\ \hline
	0.0.9 & {\LS} (Analista) & 03/01/2017 & Stesura UC5.3-UC5.5.6.1 \\ \hline
	0.0.8 & {\PB} (Analista) & 02/01/2017 & Stesura UC5-UC5.2 \\ \hline
	0.0.7 & {\AZ} (Analista) & 02/01/2017 & Stesura UC4.3.3.1-UC4.11 \\ \hline
	0.0.6 & {\MM} (Analista) & 30/12/2016 & Stesura UC4.2-UC4.3.3.1 \\ \hline
	0.0.5 & {\GG} (Analista) & 29/12/2016 & Stesura UC4.1.6-UC4.1.8 \\ \hline
	0.0.4 & {\PB} (Analista) & 29/12/2016 & Stesura UC4-UC4.1.5 \\ \hline
	0.0.3 & {\LB} (Analista) & 28/12/2016 & Stesura UC1-UC2-UC3 \\ \hline
	0.0.2 & {\LS} (Analista) & 27/12/2016 & Stesura introduzione e descrizione generale \\ \hline
	0.0.1 & {\AZ} (Analista) & 27/12/2016 & Stesura scheletro \\ \hline
\end{diario}

\tableofcontents



%%%%%%%%%%%%%%%%
%%  Introduzione
%%%%%%%%%%%%%%%%

\section{Introduzione}
	\subsection{Scopo del documento}
	Questo documento ha lo scopo di fissare le strategie di verifica e validazione che il gruppo \hx{} ha deciso di adottare per perseguire gli obiettivi di qualità di processo e di prodotto relativi al progetto \proj. Il presente documento si propone di descrivere l'approccio del gruppo al processo di verifica per poter ottenere il miglior risultato auspicabile in termini di qualità. Per perseguire tali obiettivi e risultati occorre verificare continuamente le attività svolte in modo da individuare e correggere velocemente le anomalie, minimizzando così l'utilizzo di risorse e allo stesso tempo mantenendo la correttezza del prodotto.

	\subsection{Scopo del prodotto}
	\scopo
	
	\subsection{Glossario}
	\presgloss
	
	\subsection{Riferimenti}
		\subsubsection{Riferimenti normativi}
		\begin{itemize}
			\item \NdP;
			\item \textbf{Capitolato d'Appalto C6, \proj}: \\
			\url{http://www.math.unipd.it/~tullio/IS-1/2016/Progetto/C6.pdf}, visitato il 08/02/2017;
			\item \textbf{Standard ISO/IEC 12207:2008}: \\
			\url{ieeexplore.ieee.org/document/4475826/}, visitato il 08/02/2017;
			\item \textbf{Standard ISO/IEC 15504}: \\
			\url{en.wikipedia.org/wiki/ISO/IEC_15504}, visitato il 08/02/2017;
			\item \textbf{Standard ISO/IEC 9126}: \\
			\url{it.wikipedia.org/wiki/ISO/IEC_9126}, visitato il 08/02/2017;
			\item \textbf{PDCA (\emph{Plan, Do, Check, Act})}: \\
			\url{it.wikipedia.org/wiki/Ciclo_di_Deming}, visitato il 08/02/2017.
		\end{itemize}
		
		\subsubsection{Riferimenti informativi}
		\begin{itemize}
			\item \textbf{Analisi dei Requisiti}: \AdR;
			\item \textbf{Piano di Progetto}: \PdP;
			\item \textbf{Glossario}: \Glossario;
			\item \textbf{Indice di Gulpease}: \\
			\url{it.wikipedia.org/wiki/Indice_Gulpease}, visitato il 08/02/2017;
			\item \textbf{Regole del Progetto Didattico}: \\
			\url{www.math.unipd.it/~tullio/IS-1/2016/Dispense/L09.pdf}, visitato il 08/02/2017;
			\item \textbf{Slide Qualità del Software}: \\
			\url{www.math.unipd.it/~tullio/IS-1/2016/Dispense/L10.pdf}, visitato il 08/02/2017;
			\item \textbf{Slide Qualità del Prodotto}: \\
			\url{www.math.unipd.it/~tullio/IS-1/2016/Dispense/L11.pdf}, visitato il 08/02/2017.
		\end{itemize}





%%%%%%%%%%%%%%%%%%%%
%%  Visione generale
%%%%%%%%%%%%%%%%%%%%

\section{Visione generale della strategia di gestione della qualità}
	\subsection{Obiettivi di qualità}
	In questa sezione vengono illustrati in modo quanto più completo ed esaustivo gli obiettivi di qualità che il gruppo intende perseguire nel corso 	dello svolgimento del progetto, opportunamente declinati nelle due sottosezioni \nameref{sec:qualprocesso} e \nameref{sec:qualprodotto}.

	\subsection{Qualità di processo} \label{sec:qualprocesso}
	È indubbio che la qualità di un prodotto software non possa in alcun modo prescindere dalla qualità dei diversi processi che concorrono nel definirlo e 	che, anzi, ne sia intrinsecamente dipendente. Con tale consapevolezza il gruppo ha scelto come riferimento per la valutazione della qualità dei propri processi lo standard ISO/IEC 15504, altresì conosciuto sotto l'acronimo di \gloss{SPICE}.
		\subsubsection{Procedure per il controllo della qualità di processo}
		L'adozione dello standard SPICE, in combinazione con l'approccio automigliorativo del ciclo PDCA, è alla base della strategia prevista dal gruppo per garantire una sempre maggiore qualità dei processi in atto e dei prodotti risultanti.

		Il documento \PdP{} stabilisce gli obiettivi del gruppo in relazione alla pianificazione in dettaglio dei vari processi e alla ripartizione delle risorse ad essi assegnate per un efficace svolgimento. Altri obiettivi di qualità sono illustrati successivamente in questo documento.

		Ad ogni obiettivo di qualità sono associate una o più metriche allo scopo di rendere misurabile la valutazione del suo raggiungimento o meno; ad ogni metrica, inoltre, si riferisce un valore o un range che stabilisce in senso strategico l'obiettivo qualitativo che il team si prefigge di raggiungere. Da sottolineare che anche le metriche relative alla qualità del prodotto software costituiscono un'indicazione preziosa per valutare la qualità dei processi in atto: infatti un prodotto di bassa qualità è indicativo di un processo da migliorarsi.

		\subsubsection{Obiettivi di qualità di processo} \label{sec:ob_qual_proc}
		In questa sezione vengono elencati i principali processi software identificati dal team sulla base dello standard ISO/IEC 12207:2008 e, per ognuno di essi, gli obiettivi di qualità perseguiti.

\paragraph{Project Planning Process \& Process Assessment and Control Process}
Questo macro-processo, derivato dall'unione dei processi 6.3.1 e 6.3.2 previsti dallo standard ISO/IEC 12207:2008, ha lo scopo di pianificare le attività richieste dal progetto.

Gli obiettivi di qualità stabiliti per questo processo sono i seguenti:
\begin{center}
\begin{tabular}{| l | p{6cm} | r |}
	\hline
	\textbf{ambito} & \textbf{metrica} & \textbf{obiettivo} \\
	\hline
	rispetto della pianificazione & schedule variance & $0$ \\
	\hline
	rispetto della pianificazione & budget variance & $0$ \\
	\hline
\end{tabular}
\end{center}

\paragraph{Software Documentation Management Process}
Il processo, corrispondente al processo 7.2.1 definito in ISO/IEC 12207:2008, ha lo scopo di produrre e gestire la documentazione relativa alle funzionalità e alle caratteristiche del sistema software prodotto.

Gli obiettivi di pianificazione e le metriche ad essi associate sono i seguenti:
\begin{center}
\begin{tabular}{| l | l | r |}
	\hline
	\textbf{ambito} & \textbf{metrica} & \textbf{obiettivo} \\
	\hline
	leggibilità della documentazione & indice di Gulpease & $[60, 100]$ \\
	\hline
\end{tabular}
\end{center}

\paragraph{Software Architectural Design Process}
Il processo, corrispondente al processo 7.1.3 di ISO/IEC 12207:2008, ha lo scopo di definire un'architettura software che implementi i requisiti corrispondenti e identifichi le diverse componenti del sistema.

Gli obiettivi di qualità stabiliti per questo processo sono i seguenti:
\begin{center}
\begin{tabular}{| l | p{6cm} | r |}
	\hline
	\textbf{ambito} & \textbf{metrica} & \textbf{obiettivo} \\
	\hline
	\multirow{3}{*}{completezza} & numero di violazioni della completezza architetturale di alta importanza & $0$ \\
	& numero di violazioni della completezza architetturale di media importanza & $0$ \\
	& numero di violazioni della completezza architetturale di bassa importanza & $0$ \\
	\hline
	\multirow{3}{*}{correttezza} & numero di violazioni della correttezza architetturale di alta importanza & $0$ \\
	& numero di violazioni della correttezza architetturale di media importanza & $0$ \\
	& numero di violazioni della correttezza architetturale di bassa importanza & $0$ \\
	\hline
	\multirow{3}{*}{stile} & numero di violazioni dello stile architetturale di alta importanza & $0$ \\
	& numero di violazioni dello stile architetturale di media importanza & $0$ \\
	& numero di violazioni dello stile architetturale di bassa importanza & $0$ \\
	\hline
\end{tabular}
\end{center}

\paragraph{Software Construction Process}
Il processo, corrispondente al processo 7.1.5 di ISO/IEC 12207:2008, definisce le attività principali volte alla produzione di unità software eseguibili che riflettano quanto identificato a livello di progettazione.

Gli obiettivi di qualità stabiliti per questo processo sono i seguenti:
\begin{center}
\begin{tabular}{| l | p{6cm} | r |}
	\hline
	\textbf{ambito} & \textbf{metrica} & \textbf{obiettivo} \\
	\hline
	\ambito{3}{rispetto delle norme di codifica} & numero di violazioni di alta importanza delle norme di codifica & $0$ \\
	& numero di violazioni di media importanza delle norme di codifica & $0$ \\
	& numero di violazioni di bassa importanza delle norme di codifica & $0$ \\
	\hline
\end{tabular}
\end{center}

\paragraph{System Qualification Testing Process \& Software Qualification Testing Process}
Il macro-processo, derivante dall'unione dei processi 6.4.6 e 7.1.7 di ISO/IEC 12207:2008, ha lo scopo che ogni requisito individuato sia stato implementato nel prodotto.

Gli obiettivi di qualità stabiliti per questo processo sono i seguenti:
\begin{center}
\begin{tabular}{| l | p{6cm} | r |}
	\hline
	\textbf{ambito} & \textbf{metrica} & \textbf{obiettivo} \\
	\hline
	\ambito{3}{corretto funzionamento del sistema e integrazione delle componenenti} & percentuale di test di unità eseguiti & $100$ \\
	& percentuale di test di integrazione eseguiti & $100$ \\
	& percentuale di test di sistema eseguiti & $100$ \\
	& percentuale di test di validazione eseguiti & $100$ \\
	\hline
\end{tabular}
\end{center}

	\subsection{Qualità di prodotto} \label{sec:qualprodotto}
	Per garantire una buona qualità di prodotto, il gruppo \hx{} ha individuato dallo standard \gloss{ISO/IEC 9126:2001} le qualità che ritiene di maggior importanza durante il ciclo di vita del prodotto e ha individuato gli obiettivi e le metriche coerenti con i livelli di qualità stabiliti.
		\subsubsection{Procedure per il controllo delle qualità di prodotto}
		Il controllo di qualità del prodotto verrà garantito da:
		\begin{itemize}
			\item \textbf{quality assurance}: l'insieme di attività realizzate per garantire il raggiungimento degli obiettivi di qualità. Tali attività prevedono la realizzazione di tecniche di analisi statica e dinamica.
			\item \textbf{verifica}: il processo che stabilisce se il prodotto in uscita da una fase è consistente, corretto e completo. Per tutta la durata del progetto verranno svolte attività di verifica.
			\item \textbf{validazione}: la conferma oggettiva che il sistema soddisfa i requisiti.
		\end{itemize}
		\subsubsection{Obiettivi di qualità del prodotto} \label{sec:ob_qual_sw}
		Gli obiettivi di qualità del software che il gruppo \hx{} desidera raggiungere nell'arco del progetto sono un sottoinsieme di quelli enunciati nello standard ISO/IEC 9126:2001.

	\paragraph{Funzionalità}
	Intendiamo far sì che il prodotto possieda tutte le funzionalità descritte dai requisiti obbligatori e gran parte di quelle descritte dai requisiti desiderabili. I nostri obiettivi di funzionalità del prodotto e le metriche ad essi associate sono i seguenti:
	\begin{center}
	\begin{tabular}{| l | r |}
		\hline
		\textbf{metrica} & \textbf{obiettivo} \\
		\hline
		copertura dei requisiti obbligatori & $100$ \\
		\hline
		copertura dei requisiti desiderabili & $[80, 100]$ \\
		\hline
	\end{tabular}
	\end{center}

	\paragraph{Affidabilità}
	Intendiamo testare il prodotto negli aspetti più importanti e in determinate situazioni nelle quali esso si può trovare. I nostri obiettivi di affidabilità e le metriche ad essi associate sono i seguenti:
	\begin{center}
	\begin{tabular}{| l | r |}
		\hline
		\textbf{metrica} & \textbf{obiettivo} \\
		\hline
		percentuale di test superati & $100$ \\
		\hline
	\end{tabular}
	\end{center}

	\paragraph{Efficienza}
	Il codice sorgente di \proj{} non presenterà alto grado di complessità. A tal fine, i nostri obiettivi di efficienza e le metriche ad essi associate sono i seguenti:
	\begin{center}
	\begin{tabular}{| l | r |}
		\hline
		\textbf{metrica} & \textbf{obiettivo} \\
		\hline
		Profondità di annidamento dei blocchi & $[0, 4]$ \\
		\hline
	\end{tabular}
	\end{center}

	\paragraph{Manutenibilità}
	Il codice sorgente di \proj{} risulterà manutenibile e facilmente comprensibile. I nostri obiettivi di manutenibilità e le metriche ad essi associate sono i seguenti:
	\begin{center}
	\begin{tabular}{| l | r |}
		\hline
		\textbf{metrica} & \textbf{obiettivo} \\
		\hline
		Numero di linee di codice per metodo & $[0, 20]$ \\
		\hline
		Numero di parametri per metodo & $[0, 4]$ \\
		\hline
		Numero di campi dati per classe & $[0, 10]$ \\
		\hline
		Numero di metodi per classe & $[0, 10]$ \\
		\hline
		Grado di accoppiamento afferente per package & $[0, 7]$ \\
		\hline
		Grado di accoppiamento efferente per package & $[0, 6]$ \\
		\hline
		Complessità ciclomatica per metodo & $[0, 8]$ \\
		\hline
		Numero di tipi per package & $[0, 20]$ \\
		\hline
		Distanza dalla sequenza principale normalizzata & $[0, 0.5]$ \\
		\hline
		Instabilità & $[0, 0.3]$ e $[0.7, 1.0]$ \\
		\hline
		Percentuale linee di commento su linee di codice & $[20, 40]$ \\
		\hline
		Numero di figli diretti & $[0, 2]$ \\
		\hline
		Profondità nella gerarchia & $[1, 2]$ \\
		\hline
	\end{tabular}
	\end{center}

	\paragraph{Portabilità}
	\proj{} dovrà poter garantire le sue funzionalità in browser differenti. Gli obiettivi di portabilità e le metriche ad essi associate sono i seguenti:
	\begin{center}
	\begin{tabular}{| l | r |}
		\hline
		\textbf{metrica} & \textbf{obiettivo} \\
		\hline
		Numero di errori segnalati dalla validazione \gloss{W3C} & $0$ \\
		\hline
	\end{tabular}
	\end{center}

		
		 % 	\paragraph{Funzionalità}
		 % 	Il prodotto possiede tutte le funzionalità descritte all'interno dei requisiti obbligatori e gran parte delle funzionalità descritte all'interno dei requisiti desiderabili.
		 % 	Gli obiettivi di funzionalità e le metriche ad essi associate sono i seguenti:
		 % 	\begin{itemize}
			% 	\item \textbf{copertura dei requisiti obbligatori}: l'obiettivo stabilito per la metrica è il valore $100$;
			% 	\item \textbf{copertura dei requisiti desiderabili}: l'obiettivo stabilito per la metrica è l'intervallo $[80, 100]$.
			% \end{itemize}
		 % 	\paragraph{Affidabilità}
		 % 	Il prodotto è testato negli aspetti più importanti e in determinate situazioni nelle quali esso si può trovare.
		 % 	Gli obiettivi di affidabilità e le metriche ad essi associate sono i seguenti:
		 % 	\begin{itemize}
			% 	\item \textbf{percentuale di test superati}: l'obiettivo stabilito per la metrica è il valore $100$.
			% \end{itemize}
		 % 	\paragraph{Efficienza}
		 % 	Il prodotto presenta codice senza elevati gradi di complessità relativamente ad alcuni vincoli definiti.
		 % 	Gli obiettivi di efficienza e le metriche ad essi associate sono i seguenti:
		 % 	\begin{itemize}
			% 	\item \textbf{Profondità di annidamento dei blocchi}: l'obiettivo stabilito per la metrica è l'intervallo $[0, 4]$;
			% \end{itemize}
		 % 	\paragraph{Manutenibilità}
		 % 	Il codice risulta manutenibile e facilmente comprensibile.
		 % 	Gli obiettivi di manutenibilità e le metriche ad essi associate sono i seguenti:
		 % 	\begin{itemize}
			% 	\item \textbf{Numero di linee di codice per metodo}: l'obiettivo stabilito per la metrica è l'intervallo $[0, 20]$; 
			% 	\item \textbf{Numero di parametri per metodo}: l'obiettivo stabilito per la metrica è l'intervallo $[0, 4]$;
			% 	\item \textbf{Numero di campi dati per classe}: l'obiettivo stabilito per la metrica è l'intervallo $[0, 10]$;
			% 	\item \textbf{Numero di metodi per classe}: l'obiettivo stabilito per la metrica è l'intervallo $[0, 10]$;
			% 	\item \textbf{Grado di accoppiamento afferente per package}: l'obiettivo stabilito per la metrica è l'intervallo $[0, 7]$;
			% 	\item \textbf{Grado di accoppiamento efferente per package}: l'obiettivo stabilito per la metrica è l'intervallo $[0, 6]$;
			% 	\item \textbf{Complessità ciclomatica per metodo}: l'obiettivo stabilito per la metrica è l'intervallo $[0, 8]$;
			% 	\item \textbf{Numero di tipi per package}: l'obiettivo stabilito per la metrica è l'intervallo $[0, 20]$;
			% 	\item \textbf{Distanza dalla sequenza principale normalizzata}: l'obiettivo stabilito per la metrica è l'intervallo $[0, 0.5]$; 
			% 	\item \textbf{Instabilità}: l'obiettivo stabilito per la metrica sono gli intervalli $[0, 0.3]$ e $[0.7, 1.0]$; 
			% 	\item \textbf{Percentuale linee di commento su linee di codice}: l'obiettivo stabilito per la metrica è l'intervallo $[20, 40]$;	
			% 	\item \textbf{Numero di figli diretti}: l'obiettivo stabilito per la metrica è l'intervallo $[0, 2]$;
			% 	\item \textbf{Profondità nella gerarchia}: l'obiettivo stabilito per la metrica è l'intervallo $[1, 2]$;
			% \end{itemize}
			% \paragraph{Portabilità}
		 % 	Il prodotto deve poter garantire le sue funzionalità in browser differenti.
		 % 	Gli obiettivi di portabilità e le metriche ad essi associate sono i seguenti:
		 % 	\begin{itemize}
			% 	\item \textbf{Numero di errori segnalati dalla validazione \gloss{W3C}}: l'obiettivo stabilito per la metrica è il valore $0$.
			% \end{itemize}





%%%%%%%%
%%  Test
%%%%%%%%

\section{Test}

\subsection{Tipi di test}
Individuiamo quattro tipologie di test:
\begin{itemize}
	\item \textbf{Test di unità [TU]}: test con i quali si cerca di verificare la più piccola parte di lavoro prodotta da un programmatore, quali metodi e funzioni scritte;
	\item \textbf{Test di integrazione [TI]}: test per verificare le componenti di sistema con l'obiettivo di verificare il funzionamento dei vari package prodotti, singolarmente o nel loro insieme;
	\item \textbf{Test di sistema [TS]}: test con i quali si tenta di verificare che il funzionamento e il comportamento dell'architettura siano corretti;
	\item \textbf{Test di validazione [TV]}: test per verificare che il prodotto soddisfi le richieste del proponente.
\end{itemize}



\subsection{Test di validazione}
I test di validazione vengono eseguiti con il proponente per collaudare il prodotto e hanno lo scopo di accertare che esso sia conforme alle attese. Per ogni test viene riportata una descrizione contenente i passi che l'utente deve seguire per verificare che i requisiti siano soddisfatti.

\subsubsection{Specifica dei test di validazione}
\normalsize
\begin{longtable}{|c|>{}m{8cm}|c|}
\hline 
\textbf{Id Test} & \textbf{Descrizione} & \textbf{Stato}\\
\hline
\endhead
\hypertarget{TV1}{TV1} & L'utente intende caricare un progetto precedentemente salvato. All'utente è richiesto di:
\begin{itemize}
	\item premere sul pulsante menu;
	\item premere sul pulsante di caricamento del progetto;
	\item selezionare un progetto valido;
	\item confermare il caricamento.
\end{itemize}
 & \textcolor{Green}{\textit{Superato}}\\ \hline
 
\hypertarget{TV2}{TV2} & L'utente intende creare un  nuovo progetto.
All'utente è richiesto di:
\begin{itemize}
	\item premere sul pulsante menu;
	\item premere sul pulsante di creazione nuovo progetto.
\end{itemize} & \textcolor{Green}{\textit{Superato}}\\ \hline

\hypertarget{TV3}{TV3} & L'utente intende inserire una classe.
All'utente è richiesto di:
\begin{itemize}
	\item trovarsi nella schermata di gestione del \gloss{diagramma delle classi} del progetto;
	\item premere il pulsante di inserimento nuova classe;
	\item indicare dove inserire la nuova classe nell'apposita area.
\end{itemize} & \textcolor{Green}{\textit{Superato}}\\ \hline

\hypertarget{TV4}{TV4} & L'utente intende inserire il nome di una classe.
All'utente è richiesto di:
\begin{itemize}
	\item trovarsi nella schermata di gestione del diagramma delle classi del progetto;
	\item selezionare la classe;
	\item inserire il nome della classe.
\end{itemize} & \textcolor{Green}{\textit{Superato}}\\ \hline

\hypertarget{TV5}{TV5} & L'utente intende inserire un attributo di una classe.
All'utente è richiesto di:
\begin{itemize}
	\item trovarsi nella schermata di gestione del diagramma delle classi del progetto;
	\item selezionare la classe;
	\item selezionare l'opzione di aggiunta di un attributo.
\end{itemize} & \textcolor{Green}{\textit{Superato}}\\ \hline

\hypertarget{TV6}{TV6} & L'utente intende inserire la visibilità di un attributo.
All'utente è richiesto di:
\begin{itemize}
	\item trovarsi nella schermata di gestione del diagramma delle classi del progetto;
	\item selezionare la classe;
	\item selezionare l'attributo;
	\item scegliere la visibilità dell'attributo.
\end{itemize} & \textcolor{Green}{\textit{Superato}}\\ \hline

\hypertarget{TV7}{TV7} & L'utente intende inserire il nome di un attributo.
All'utente è richiesto di:
\begin{itemize}
	\item trovarsi nella schermata di gestione del diagramma delle classi del progetto;
	\item selezionare la classe;
	\item spostarsi sull'apposita area di modifica dati; 
	\item inserire il nome dell'attributo.
\end{itemize} & \textcolor{Green}{\textit{Superato}}\\ \hline

\hypertarget{TV3.1.4.3}{TV8} & L'utente intende inserire il tipo di un attributo.
All'utente è richiesto di:
\begin{itemize}
	\item trovarsi nella schermata di gestione del diagramma delle classi del progetto;
	\item selezionare la classe;
	\item spostarsi sull'apposita area di modifica dati;
	\item inserire il tipo dell'attributo.
\end{itemize} & \textcolor{Green}{\textit{Superato}}\\ \hline

\hypertarget{TV3.1.4.5}{TV9} & L'utente intende inserire il valore di default  di un attributo.
All'utente è richiesto di:
\begin{itemize}
	\item trovarsi nella schermata di gestione del diagramma delle classi del progetto;
	\item selezionare la classe;
	\item spostarsi sull'apposita area di modifica dati;
	\item inserire il valore di default  dell'attributo.
\end{itemize} & \textcolor{Green}{\textit{Superato}}\\ \hline

\hypertarget{TV3.1.5}{TV10} & L'utente intende rimuovere un attributo.
All'utente è richiesto di:
\begin{itemize}
	\item trovarsi nella schermata di gestione del diagramma delle classi del progetto;
	\item selezionare la classe;
	\item spostarsi sull'apposita area di modifica dati;
	\item scegliere l'opzione di rimozione dell'attributo.
\end{itemize} & \textcolor{Green}{\textit{Superato}}\\ \hline

\hypertarget{TV3.1.6}{TV11} & L'utente intende inserire un metodo di una classe.
All'utente è richiesto di:
\begin{itemize}
	\item trovarsi nella schermata di gestione del diagramma delle classi del progetto;
	\item selezionare la classe;
	\item selezionare l'opzione di aggiunta di un metodo.
\end{itemize} & \textcolor{Green}{\textit{Superato}}\\ \hline

\hypertarget{TV3.1.6.1}{TV12} & L'utente intende inserire la visibilità di un metodo di una classe.
All'utente è richiesto di:
\begin{itemize}
	\item trovarsi nella schermata di gestione del diagramma delle classi del progetto;
	\item selezionare la classe;
	\item spostarsi sull'apposita area di modifica dati del metodo;
	\item selezionare la visibilità del metodo.
\end{itemize} & \textcolor{Green}{\textit{Superato}}\\ \hline

\hypertarget{TV3.1.6.2}{TV13} & L'utente intende inserire il nome di un metodo di una classe.
All'utente è richiesto di:
\begin{itemize}
	\item trovarsi nella schermata di gestione del diagramma delle classi del progetto;
	\item selezionare la classe;
	\item spostarsi sull'apposita area di modifica dati del metodo;
	\item inserire il nome del metodo.
\end{itemize} & \textcolor{Green}{\textit{Superato}}\\ \hline

\hypertarget{TV3.1.6.3}{TV14} & L'utente intende inserire un parametro di un metodo di una classe.
All'utente è richiesto di:
\begin{itemize}
	\item trovarsi nella schermata di gestione del diagramma delle classi del progetto;
	\item selezionare la classe;
	\item spostarsi sull'apposita area di modifica dati del metodo;
	\item selezionare l'opzione di aggiunta di un parametro.
\end{itemize} & \textcolor{Green}{\textit{Superato}}\\ \hline

\hypertarget{TV3.1.6.3.1}{TV15} & L'utente intende inserire il nome di un parametro di un metodo di una classe.
All'utente è richiesto di:
\begin{itemize}
	\item trovarsi nella schermata di gestione del diagramma delle classi del progetto;
	\item selezionare la classe;
	\item spostarsi sull'apposita area di modifica dati del metodo;
	\item inserire il nome del parametro.
\end{itemize} & \textcolor{Green}{\textit{Superato}}\\ \hline

\hypertarget{TV3.1.6.3.2}{TV16} & L'utente intende inserire il tipo di un parametro di un metodo di una classe.
All'utente è richiesto di:
\begin{itemize}
	\item trovarsi nella schermata di gestione del diagramma delle classi del progetto;
	\item selezionare la classe;
	\item spostarsi sull'apposita area di modifica dati del metodo;
	\item inserire il tipo del parametro.
\end{itemize} & \textcolor{Green}{\textit{Superato}}\\ \hline

\hypertarget{TV3.1.6.3.3}{TV17} & L'utente intende inserire il valore di default di un parametro di un metodo di una classe.
All'utente è richiesto di:
\begin{itemize}
	\item trovarsi nella schermata di gestione del diagramma delle classi del progetto;
	\item selezionare la classe;
	\item spostarsi sull'apposita area di modifica dati del metodo;
	\item inserire il valore di default del parametro.
\end{itemize} & \textcolor{Green}{\textit{Superato}}\\ \hline

\hypertarget{TV3.1.6.4}{TV18} & L'utente intende inserire il tipo di ritorno di un metodo di una classe.
All'utente è richiesto di:
\begin{itemize}
	\item trovarsi nella schermata di gestione del diagramma delle classi del progetto;
	\item selezionare la classe;
	\item spostarsi sull'apposita area di modifica dati del metodo;
	\item inserire il tipo di ritorno del metodo.
\end{itemize} & \textcolor{Green}{\textit{Superato}}\\ \hline

\hypertarget{TV3.1.6.6}{TV19} & L'utente intende rimuovere un parametro di un metodo di una classe.
All'utente è richiesto di:
\begin{itemize}
	\item trovarsi nella schermata di gestione del diagramma delle classi del progetto;
	\item selezionare una classe;
	\item spostarsi sull'apposita area di modifica dati del metodo;
	\item scegliere l'opzione di rimozione del parametro.
\end{itemize} & \textcolor{Green}{\textit{Superato}}\\ \hline

\hypertarget{TV3.1.7}{TV20} & L'utente intende rimuovere un metodo di una classe.
All'utente è richiesto di:
\begin{itemize}
	\item trovarsi nella schermata di gestione del diagramma delle classi del progetto;
	\item selezionare la classe;
	\item spostarsi sull'apposita area di modifica dati del metodo;
	\item scegliere l'opzione di rimozione del metodo.
\end{itemize} & \textcolor{Green}{\textit{Superato}}\\ \hline

\hypertarget{TV3.2}{TV21} & L'utente intende inserire una relazione.
All'utente è richiesto di:
\begin{itemize}
	\item trovarsi nella schermata di gestione del diagramma delle classi del progetto;
	\item premere il pulsante di inserimento di una nuova relazione.
	\item selezionare la classe di partenza della relazione;
	\item selezionare la classe di destinazione della relazione.
\end{itemize} & \textcolor{Green}{\textit{Superato}}\\ \hline

\hypertarget{TV3.2.1}{TV22} & L'utente intende inserire il nome di una associazione.
All'utente è richiesto di:
\begin{itemize}
	\item trovarsi nella schermata di gestione del diagramma delle classi del progetto;
	\item selezionare un'associazione;
	\item inserire il nome dell'associazione.
\end{itemize} & \textcolor{Green}{\textit{Superato}}\\ \hline

\hypertarget{TV3.2.7}{TV23} & L'utente intende inserire la cardinalità di un'associazione.
All'utente è richiesto di:
\begin{itemize}
	\item trovarsi nella schermata di gestione del diagramma delle classi del progetto;
	\item selezionare una associazione;
	\item inserire la cardinalità dell'associazione.
\end{itemize} & \textcolor{Green}{\textit{Superato}}\\ \hline

\hypertarget{TV3.3}{TV24} & L'utente intende inserire un commento. 
All'utente è richiesto di: 
\begin{itemize} 
	\item trovarsi nella schermata di gestione del diagramma delle classi del progetto;
	\item premere il pulsante di inserimento di un nuovo commento;
	\item indicare la posizione di inserimento del commento nell'apposita area.
	
\end{itemize} & \textcolor{Green}{\textit{Superato}}\\ \hline
\hypertarget{TV3.3.1}{TV25} & L'utente intende inserire il testo di un commento. 
All'utente è richiesto di: 
\begin{itemize} 
	\item trovarsi nella schermata di gestione del diagramma delle classi del progetto;
	\item selezionare un commento;
	\item inserire il testo del commento.
\end{itemize} & \textcolor{Green}{\textit{Superato}}\\ \hline

\hypertarget{TV3.4}{TV26} & L'utente intende rimuovere una classe.
All'utente è richiesto di:
\begin{itemize}
	\item trovarsi nella schermata di gestione del diagramma delle classi del progetto;
	\item selezionare una classe;
	\item scegliere l'opzione di rimozione della classe.
\end{itemize} & \textcolor{Green}{\textit{Superato}}\\ \hline

\hypertarget{TV3.5}{TV27} & L'utente intende rimuovere una relazione.
All'utente è richiesto di:
\begin{itemize}
	\item trovarsi nella schermata di gestione del diagramma delle classi del progetto;
	\item selezionare una relazione; 
	\item scegliere l'opzione di rimozione della relazione.
\end{itemize} & \textcolor{Green}{\textit{Superato}}\\ \hline

\hypertarget{TV3.6}{TV28} & L'utente intende rimuovere un commento.
All'utente è richiesto di:
\begin{itemize}
	\item trovarsi nella schermata di gestione del diagramma delle classi del progetto;
	\item selezionare un commento;
	\item scegliere l'opzione di rimozione del commento.
\end{itemize} & \textcolor{Green}{\textit{Superato}}\\ \hline

\hypertarget{TV3.11.2}{TV29} & L'utente intende modificare il nome di una classe.
All'utente è richiesto di:
\begin{itemize}
	\item trovarsi nella schermata di gestione del diagramma delle classi del progetto;
	\item selezionare una classe;
	\item modificare il nome della classe.
\end{itemize} & \textcolor{Green}{\textit{Superato}}\\ \hline

\hypertarget{TV3.11.3.1}{TV30} & L'utente intende modificare la visibilità di un attributo.
All'utente è richiesto di:
\begin{itemize}
	\item trovarsi nella schermata di gestione del diagramma delle classi del progetto;
	\item selezionare una classe;
	\item selezionare un attributo;
	\item modificare la visibilità dell'attributo.
\end{itemize} & \textcolor{Green}{\textit{Superato}}\\ \hline

\hypertarget{TV3.11.3.2}{TV31} & L'utente intende modificare il nome di un attributo.
All'utente è richiesto di:
\begin{itemize}
	\item trovarsi nella schermata di gestione del diagramma delle classi del progetto;
	\item selezionare una classe;
	\item selezionare un attributo;
	\item modificare il nome dell'attributo.
\end{itemize} & \textcolor{Green}{\textit{Superato}}\\ \hline

\hypertarget{TV3.11.3.3}{TV32} & L'utente intende modificare il tipo di un attributo.
All'utente è richiesto di:
\begin{itemize}
	\item trovarsi nella schermata di gestione del diagramma delle classi del progetto;
	\item selezionare una classe;
	\item selezionare un attributo;
	\item modificare il tipo dell'attributo.
\end{itemize} & \textcolor{Green}{\textit{Superato}}\\ \hline

\hypertarget{TV3.11.3.5}{TV33} & L'utente intende modificare il valore di default di un attributo.
All'utente è richiesto di:
\begin{itemize}
	\item trovarsi nella schermata di gestione del diagramma delle classi del progetto;
	\item selezionare una classe;
	\item selezionare un attributo;
	\item modificare il valore di default dell'attributo.
\end{itemize} & \textcolor{Green}{\textit{Superato}}\\ \hline

\hypertarget{TV3.11.4.1}{TV34} & L'utente intende modificare la visibilità di un metodo di una classe.
All'utente è richiesto di:
\begin{itemize}
	\item trovarsi nella schermata di gestione del diagramma delle classi del progetto;
	\item selezionare una classe;
	\item selezionare un metodo;
	\item modificare la visibilità del metodo.
\end{itemize} & \textcolor{Green}{\textit{Superato}}\\ \hline

\hypertarget{TV3.11.4.2}{TV35} & L'utente intende modificare il nome di un metodo di una classe.
All'utente è richiesto di:
\begin{itemize}
	\item trovarsi nella schermata di gestione del diagramma delle classi del progetto;
	\item selezionare una classe;
	\item selezionare un metodo;
	\item modificare il nome del metodo.
\end{itemize} & \textcolor{Green}{\textit{Superato}}\\ \hline

\hypertarget{TV3.11.4.3.1}{TV36} & L'utente intende modificare il nome di un parametro di un metodo di una classe.
All'utente è richiesto di:
\begin{itemize}
	\item trovarsi nella schermata di gestione del diagramma delle classi del progetto;
	\item selezionare una classe; 
	\item selezionare un metodo; 
	\item selezionare un parametro;
	\item modificare il nome del parametro.
\end{itemize} & \textcolor{Green}{\textit{Superato}}\\ \hline

\hypertarget{TV3.11.4.3.2}{TV37} & L'utente intende modificare il tipo di un parametro di un metodo di una classe.
All'utente è richiesto di:
\begin{itemize}
	\item trovarsi nella schermata di gestione del diagramma delle classi del progetto;
	\item selezionare una classe; 
	\item selezionare un metodo;
	\item selezionare un parametro;
	\item modificare il tipo del parametro.
\end{itemize} & \textcolor{Green}{\textit{Superato}}\\ \hline

\hypertarget{TV3.11.4.3.3}{TV38} & L'utente intende modificare il valore di default un parametro di un metodo di una classe.
All'utente è richiesto di:
\begin{itemize}
	\item trovarsi nella schermata di gestione del diagramma delle classi del progetto;
	\item selezionare una classe; 
	\item selezionare un metodo;
	\item selezionare un parametro;
	\item modificare il valore di default del parametro.
\end{itemize} & \textcolor{Green}{\textit{Superato}}\\ \hline

\hypertarget{TV3.11.4.4}{TV39} & L'utente intende modificare il tipo di ritorno di un metodo di una classe.
All'utente è richiesto di:
\begin{itemize}
	\item trovarsi nella schermata di gestione del diagramma delle classi del progetto;
	\item selezionare una classe;
	\item selezionare un metodo;
	\item modificare il tipo di ritorno del metodo.
\end{itemize} & \textcolor{Green}{\textit{Superato}}\\ \hline

\hypertarget{TV3.12.2}{TV40} & L'utente intende modificare il nome di un'associazione.
All'utente è richiesto di:
\begin{itemize}
	\item trovarsi nella schermata di gestione del diagramma delle classi del progetto;
	\item selezionare un'associazione;
	\item modificare il nome dell'associazione.
\end{itemize} & \textcolor{Green}{\textit{Superato}}\\ \hline

\hypertarget{TV3.12.3}{TV41} & L'utente intende modificare la classe di partenza di una relazione.
All'utente è richiesto di:
\begin{itemize}
	\item trovarsi nella schermata di gestione del diagramma delle classi del progetto;
	\item selezionare una relazione;
	\item modificare la classe di partenza della relazione.
\end{itemize} & \textcolor{Green}{\textit{Superato}}\\ \hline

\hypertarget{TV3.12.4}{TV42} & L'utente intende modificare la classe di destinazione di una relazione.
All'utente è richiesto di:
\begin{itemize}
	\item trovarsi nella schermata di gestione del diagramma delle classi del progetto;
	\item selezionare una relazione;
	\item modificare la classe di destinazione della relazione.
\end{itemize} & \textcolor{Green}{\textit{Superato}}\\ \hline

\hypertarget{TV3.12.5}{TV43} & L'utente intende modificare la cardinalità di un'associazione.
All'utente è richiesto di:
\begin{itemize}
	\item trovarsi nella schermata di gestione del diagramma delle classi del progetto;
	\item selezionare un'associazione;
	\item modificare la cardinalità dell'associazione.
\end{itemize} & \textcolor{Green}{\textit{Superato}}\\ \hline

\hypertarget{TV3.13.1}{TV44} & L'utente intende modificare il testo di un commento. 
All'utente è richiesto di: 
\begin{itemize}
	\item trovarsi nella schermata di gestione del diagramma delle classi del progetto;
	\item selezionare un commento;
	\item modificare il testo del commento.
\end{itemize} & \textcolor{Green}{\textit{Superato}}\\ \hline

\hypertarget{TV3.1.8}{TV45} & L'utente intende specificare se una classe è astratta.
All'utente è richiesto di:
\begin{itemize}
	\item trovarsi nella schermata di gestione del diagramma delle classi del progetto;
	\item selezionare una classe;
	\item selezionare il campo "Abstract".
\end{itemize} & \textcolor{Green}{\textit{Superato}}\\ \hline


\hypertarget{TV3.1.8}{TV46} & L'utente intende specificare se una classe è statica.
All'utente è richiesto di:
\begin{itemize}
	\item trovarsi nella schermata di gestione del diagramma delle classi del progetto;
	\item selezionare una classe;
	\item selezionare il campo "Static".
\end{itemize} & \textcolor{Green}{\textit{Superato}}\\ \hline

\hypertarget{TV3.1.10}{TV47} & L'utente intende togliere la specifica "Abstract" di una classe.
All'utente è richiesto di:
\begin{itemize}
	\item trovarsi nella schermata di gestione del diagramma delle classi del progetto;
	\item selezionare una classe;
	\item selezionare il campo "Abstract" se questo risulta già evidenziato.
\end{itemize} & \textcolor{Green}{\textit{Superato}}\\ \hline

\hypertarget{TV3.1.11}{TV48} & L'utente intende togliere la specifica "Static" di una classe.
All'utente è richiesto di:
\begin{itemize}
	\item trovarsi nella schermata di gestione del diagramma delle classi del progetto;
	\item selezionare una classe;
	\item selezionare il campo "Static" se questo risulta già evidenziato.
\end{itemize} & \textcolor{Green}{\textit{Superato}}\\ \hline

\hypertarget{TV3.1.12}{TV49} & L'utente intende specificare se un metodo è astratto.
All'utente è richiesto di:
\begin{itemize}
	\item trovarsi nella schermata di gestione del diagramma delle classi del progetto;
	\item selezionare una classe;
	\item spostarsi nell'apposita area di modifica;
	\item selezionare il campo "Abstract" del metodo desiderato.
\end{itemize} & \textcolor{Green}{\textit{Superato}}\\ \hline

\hypertarget{TV3.1.13}{TV50} & L'utente intende specificare se un metodo è statico.
All'utente è richiesto di:
\begin{itemize}
	\item trovarsi nella schermata di gestione del diagramma delle classi del progetto;
	\item selezionare una classe;
	\item spostarsi nell'apposita area di modifica;
	\item selezionare il campo "Static" del metodo desiderato.
\end{itemize} & \textcolor{Green}{\textit{Superato}}\\ \hline

\hypertarget{TV3.1.14}{TV51} & L'utente intende togliere la specifica "Abstract" di un metodo.
All'utente è richiesto di:
\begin{itemize}
	\item trovarsi nella schermata di gestione del diagramma delle classi del progetto;
	\item selezionare una classe;
	\item spostarsi nell'apposita area di modifica;
	\item selezionare il campo "Abstract" del metodo desiderato se questo è già evidenziato.
\end{itemize} & \textcolor{Green}{\textit{Superato}}\\ \hline

\hypertarget{TV3.1.15}{TV52} & L'utente intende togliere la specifica "Static" di un metodo.
All'utente è richiesto di:
\begin{itemize}
	\item trovarsi nella schermata di gestione del diagramma delle classi del progetto;
	\item selezionare una classe;
	\item spostarsi nell'apposita area di modifica;
	\item selezionare il campo "Static" del metodo desiderato se questo è già evidenziato.
\end{itemize} & \textcolor{Green}{\textit{Superato}}\\ \hline

\hypertarget{TV4}{TV53} & L'utente intende visualizzare il \gloss{diagramma delle attività} di un particolare metodo.
All'utente è richiesto di:
\begin{itemize}
	\item trovarsi nella schermata di gestione del diagramma delle classi del progetto;
	\item selezionare una classe;
	\item selezionare l'opzione di visualizzazione del diagramma delle attività di un metodo della classe.
\end{itemize} & \textcolor{Green}{\textit{Superato}}\\ \hline

\hypertarget{TV4.1}{TV54} & L'utente intende inserire un blocco variabile.
All'utente è richiesto di:
\begin{itemize}
	\item trovarsi nella schermata di gestione del diagramma delle attività di un particolare metodo;
	\item premere sul pulsante di inserimento del blocco variabile;
	\item indicare dove inserire il blocco variabile nell'apposita area.
\end{itemize} & \textcolor{Green}{\textit{Superato}}\\ \hline

\hypertarget{TV4.1.1}{TV55} & L'utente intende inserire il nome di una variabile.
All'utente è richiesto di:
\begin{itemize}
	\item trovarsi nella schermata di gestione del diagramma delle attività di un particolare metodo;
	\item selezionare un blocco variabile già inserito;
	\item indicare il nome della variabile.
\end{itemize} & \textcolor{Green}{\textit{Superato}}\\ \hline

\hypertarget{TV4.1.2}{TV56} & L'utente intende inserire il tipo di una variabile.
All'utente è richiesto di:
\begin{itemize}
	\item trovarsi nella schermata di gestione del diagramma delle attività di un particolare metodo;
	\item selezionare un blocco variabile già inserito;
	\item indicare il tipo della variabile.
\end{itemize} & \textcolor{Green}{\textit{Superato}}\\ \hline

\hypertarget{TV4.1.3}{TV57} & L'utente intende inserire un'operazione per una variabile.
All'utente è richiesto di:
\begin{itemize}
	\item trovarsi nella schermata di gestione del diagramma delle attività di un particolare metodo;
	\item selezionare un blocco variabile già inserito;
	\item indicare l'operazione per la variabile.
\end{itemize} & \textcolor{Green}{\textit{Superato}}\\ \hline

\hypertarget{TV4.1.4}{TV58} & L'utente intende inserire un valore per una variabile.
All'utente è richiesto di:
\begin{itemize}
	\item trovarsi nella schermata di gestione del diagramma delle attività di un particolare metodo;
	\item selezionare un blocco variabile già inserito;
	\item indicare il valore per la variabile.
\end{itemize} & \textcolor{Green}{\textit{Superato}}\\ \hline


\hypertarget{TV4.1.3}{TV59} & L'utente intende inserire un commento per un blocco variabile.
All'utente è richiesto di:
\begin{itemize}
	\item trovarsi nella schermata di gestione del diagramma delle attività di un particolare metodo;
	\item selezionare un blocco variabile già inserito;
	\item inserire il commento relativo al blocco variabile.
\end{itemize} & \textcolor{Green}{\textit{Superato}}\\ \hline

\hypertarget{TV4.3}{TV60} & L'utente intende inserire un blocco if.
All'utente è richiesto di:
\begin{itemize}
	\item trovarsi nella schermata di gestione del diagramma delle attività di un particolare metodo;
	\item premere sul pulsante di inserimento del blocco if.
	\item indicare dove inserire il blocco if nell'apposita area.
\end{itemize} & \textcolor{Green}{\textit{Superato}}\\ \hline

\hypertarget{TV4.3.1}{TV61} & L'utente intende inserire la condizione di un blocco if.
All'utente è richiesto di:
\begin{itemize}
	\item trovarsi nella schermata di gestione del diagramma delle attività di un particolare metodo;
	\item selezionare un blocco if già esistente;
	\item inserire la condizione del blocco if.
\end{itemize} & \textcolor{Green}{\textit{Superato}}\\ \hline

\hypertarget{TV4.3.2}{TV62} & L'utente intende completare il corpo di un blocco if inserendo una serie di blocchi tra quelli resi disponibili dall'editor.
All'utente è richiesto di:
\begin{itemize}
	\item trovarsi nella schermata di gestione del diagramma delle attività di un particolare metodo;
	\item selezionare un blocco da inserire;
	\item premere nell'area di lavoro per inserirlo in coda;
	\item trascinare il blocco appena inserito nel corpo del blocco if;
	\item eventualmente ripetere le precedenti tre istruzioni per inserire altri blocchi.
\end{itemize} & \textcolor{Green}{\textit{Superato}}\\ \hline

\hypertarget{TV4.3.3}{TV63} & L'utente intende completare il corpo di un blocco else inserendo una serie di blocchi tra quelli resi disponibili dall'editor.
All'utente è richiesto di:
\begin{itemize}
	\item trovarsi nella schermata di gestione del diagramma delle attività di un particolare metodo;
	\item selezionare un blocco da inserire;
	\item premere nell'area di lavoro per inserirlo in coda;
	\item trascinare il blocco appena inserito nel corpo del blocco else;
	\item eventualmente ripetere le precedenti tre istruzioni per inserire altri blocchi.
\end{itemize} & \textcolor{Green}{\textit{Superato}}\\ \hline

\hypertarget{TV4.3.4}{TV64} & L'utente intende inserire un commento per un blocco if.
All'utente è richiesto di:
\begin{itemize}
	\item trovarsi nella schermata di gestione del diagramma delle attività di un particolare metodo;
	\item selezionare un blocco if;
	\item inserire il commento relativo al blocco if.
\end{itemize} & \textcolor{Green}{\textit{Superato}}\\ \hline

\hypertarget{TV4.4}{TV65} & L'utente intende inserire un blocco while.
All'utente è richiesto di:
\begin{itemize}
	\item trovarsi nella schermata di gestione del diagramma delle attività di un particolare metodo;
	\item premere sul pulsante di inserimento del blocco while;
	\item indicare dove inserire il blocco while nell'apposita area.
\end{itemize} & \textcolor{Green}{\textit{Superato}}\\ \hline

\hypertarget{TV4.4.1}{TV66} & L'utente intende inserire la condizione da verificare di un blocco while.
All'utente è richiesto di:
\begin{itemize}
	\item trovarsi nella schermata di gestione del diagramma delle attività di un particolare metodo;
	\item selezionare un blocco while;
	\item inserire la condizione da verificare del blocco while.
\end{itemize} & \textcolor{Green}{\textit{Superato}}\\ \hline

\hypertarget{TV4.4.2}{TV67} & L'utente intende completare il corpo di un blocco while inserendo una serie di blocchi tra quelli resi disponibili dall'editor.
All'utente è richiesto di:
\begin{itemize}
	\item trovarsi nella schermata di gestione del diagramma delle attività di un particolare metodo;
	\item selezionare un blocco da inserire;
	\item premere nell'area di lavoro per inserirlo in coda;
	\item trascinare il blocco appena inserito nel corpo del blocco while;
	\item eventualmente ripetere le precedenti tre istruzioni per inserire altri blocchi.
\end{itemize} & \textcolor{Green}{\textit{Superato}}\\ \hline

\hypertarget{TV4.4.3}{TV68} & L'utente intende inserire un commento per un blocco while.
All'utente è richiesto di:
\begin{itemize}
	\item trovarsi nella schermata di gestione del diagramma delle attività di un particolare metodo;
	\item selezionare un blocco while;
	\item inserire il commento relativo al blocco while.
\end{itemize} & \textcolor{Green}{\textit{Superato}}\\ \hline

\hypertarget{TV4.5}{TV69} & L'utente intende inserire un blocco r.
All'utente è richiesto di:
\begin{itemize}
	\item trovarsi nella schermata di gestione del diagramma delle attività di un particolare metodo;
	\item premere sul pulsante di inserimento del blocco for;
	\item indicare dove inserire il blocco for nell'apposita area.
\end{itemize} & \textcolor{Green}{\textit{Superato}}\\ \hline

\hypertarget{TV4.5.1}{TV70} & L'utente intende inserire l'inizializzazione di un blocco for .
All'utente è richiesto di:
\begin{itemize}
	\item trovarsi nella schermata di gestione del diagramma delle attività di un particolare metodo;
	\item selezionare un blocco for;
	\item inserire l'inizializzazione.
\end{itemize} & \textcolor{Green}{\textit{Superato}}\\ \hline

\hypertarget{TV4.5.2}{TV71} & L'utente intende inserire la condizione da verificare di un blocco for.
All'utente è richiesto di:
\begin{itemize}
	\item trovarsi nella schermata di gestione del diagramma delle attività di un particolare metodo;
	\item selezionare un blocco for;
	\item inserire la condizione da verificare del blocco for.
\end{itemize} & \textcolor{Green}{\textit{Superato}}\\ \hline

\hypertarget{TV4.5.3}{TV72} & L'utente intende inserire l'incremento-decremento di un blocco for .
All'utente è richiesto di:
\begin{itemize}
	\item trovarsi nella schermata di gestione del diagramma delle attività di un particolare metodo;
	\item selezionare un blocco for;
	\item inserire l'incremento-decremento del blocco for.
\end{itemize} & \textcolor{Green}{\textit{Superato}}\\ \hline

\hypertarget{TV4.5.4}{TV73} & L'utente intende completare il corpo di un blocco for inserendo una serie di blocchi tra quelli resi disponibili dall'editor.
All'utente è richiesto di:
\begin{itemize}
	\item trovarsi nella schermata di gestione del diagramma delle attività di un particolare metodo;
	\item selezionare un blocco da inserire;
	\item premere nell'area di lavoro per inserirlo in coda;
	\item trascinare il blocco appena inserito nel corpo del blocco for;
	\item eventualmente ripetere le precedenti tre istruzioni per inserire altri blocchi.
\end{itemize} & \textcolor{Green}{\textit{Superato}}\\ \hline

\hypertarget{TV4.5.5}{TV74} & L'utente intende inserire un commento per un blocco for.
All'utente è richiesto di:
\begin{itemize}
	\item trovarsi nella schermata di gestione del diagramma delle attività di un particolare metodo;
	\item selezionare un blocco for;
	\item inserire il commento relativo al blocco for.
\end{itemize} & \textcolor{Green}{\textit{Superato}}\\ \hline

\hypertarget{TV4.6}{TV75} & L'utente intende inserire un blocco custom.
All'utente è richiesto di:
\begin{itemize}
	\item trovarsi nella schermata di gestione del diagramma delle attività di un particolare metodo;
	\item premere sul pulsante di inserimento del blocco custom di codice;
	\item indicare dove inserire il blocco custom nell'apposita area.
\end{itemize} & \textcolor{Green}{\textit{Superato}}\\ \hline

\hypertarget{TV4.6.1}{TV76} & L'utente intende inserire il contenuto di un blocco custom di codice.
All'utente è richiesto di:
\begin{itemize}
	\item trovarsi nella schermata di gestione del diagramma delle attività di un particolare metodo;
	\item selezionare un blocco custom di codice;
	\item inserire il contenuto del blocco custom di codice.
\end{itemize} & \textcolor{Green}{\textit{Superato}}\\ \hline

\hypertarget{TV4.6.2}{TV77} & L'utente intende inserire un commento per un blocco custom di codice.
All'utente è richiesto di:
\begin{itemize}
	\item trovarsi nella schermata di gestione del diagramma delle attività di un particolare metodo;
	\item selezionare un blocco custom di codice;
	\item inserire il commento relativo al blocco custom di codice.
\end{itemize} & \textcolor{Green}{\textit{Superato}}\\ \hline

\hypertarget{TV4.7}{TV78} & L'utente intende rimuovere un blocco variabile esistente.
All'utente è richiesto di:
\begin{itemize}
	\item trovarsi nella schermata di gestione del diagramma delle attività di un particolare metodo;
	\item selezionare il blocco variabile da eliminare;
	\item premere sul pulsante di cancellazione.
\end{itemize} & \textcolor{Green}{\textit{Superato}}\\ \hline

\hypertarget{TV4.9}{TV79} & L'utente intende rimuovere un blocco if esistente.
All'utente è richiesto di:
\begin{itemize}
	\item trovarsi nella schermata di gestione del diagramma delle attività di un particolare metodo;
	\item selezionare il blocco if da eliminare;
	\item premere sul pulsante di cancellazione.
\end{itemize} & \textcolor{Green}{\textit{Superato}}\\ \hline

\hypertarget{TV4.10}{TV80} & L'utente intende rimuovere un blocco while esistente.
All'utente è richiesto di:
\begin{itemize}
	\item trovarsi nella schermata di gestione del diagramma delle attività di un particolare metodo;
	\item selezionare il blocco while da eliminare;
	\item premere sul pulsante di cancellazione.
\end{itemize} & \textcolor{Green}{\textit{Superato}}\\ \hline

\hypertarget{TV4.11}{TV81} & L'utente intende rimuovere un blocco for esistente.
All'utente è richiesto di:
\begin{itemize}
	\item trovarsi nella schermata di gestione del diagramma delle attività di un particolare metodo;
	\item selezionare il blocco for da eliminare;
	\item premere sul pulsante di cancellazione.
\end{itemize} & \textcolor{Green}{\textit{Superato}}\\ \hline

\hypertarget{TV4.12}{TV82} & L'utente intende rimuovere un blocco custom di codice esistente.
All'utente è richiesto di:
\begin{itemize}
	\item trovarsi nella schermata di gestione del diagramma delle attività di un particolare metodo;
	\item selezionare il blocco custom di codice da eliminare;
	\item premere sul pulsante di cancellazione.
\end{itemize} & \textcolor{Green}{\textit{Superato}}\\ \hline

\hypertarget{TV4.13}{TV83} & L'utente intende ridurre la dimensione di un blocco if.
All'utente è richiesto di:
\begin{itemize}
	\item trovarsi nella schermata di gestione del diagramma delle classi del progetto;
	\item selezionare un blocco if a visualizzazione espansa;
	\item selezionare l'opzione di riduzione dimensione.
\end{itemize} & \textcolor{Green}{\textit{Superato}}\\ \hline

\hypertarget{TV4.14}{TV84} & L'utente intende espandere la dimensione di un blocco if.
All'utente è richiesto di:
\begin{itemize}
	\item trovarsi nella schermata di gestione del diagramma delle classi del progetto;
	\item selezionare un blocco if a visualizzazione ridotta;
	\item selezionare l'opzione di espansione dimensione.
\end{itemize} & \textcolor{Green}{\textit{Superato}}\\ \hline

\hypertarget{TV4.15}{TV85} & L'utente intende ridurre la dimensione di un blocco while.
All'utente è richiesto di:
\begin{itemize}
	\item trovarsi nella schermata di gestione del diagramma delle classi del progetto;
	\item selezionare blocco while a visualizzazione espansa;
	\item selezionare l'opzione di riduzione dimensione.
\end{itemize} & \textcolor{Green}{\textit{Superato}}\\ \hline

\hypertarget{TV4.16}{TV86} & L'utente intende espandere la dimensione di un blocco while.
All'utente è richiesto di:
\begin{itemize}
	\item trovarsi nella schermata di gestione del diagramma delle classi del progetto;
	\item selezionare blocco while a visualizzazione ridotta;
	\item selezionare l'opzione di espansione dimensione.
\end{itemize} & \textcolor{Green}{\textit{Superato}}\\ \hline

\hypertarget{TV4.17}{TV87} & L'utente intende ridurre la dimensione di un blocco for.
All'utente è richiesto di:
\begin{itemize}
	\item trovarsi nella schermata di gestione del diagramma delle classi del progetto;
	\item selezionare blocco for a visualizzazione espansa;
	\item selezionare l'opzione di riduzione dimensione.
\end{itemize} & \textcolor{Green}{\textit{Superato}}\\ \hline

\hypertarget{TV4.18}{TV88} & L'utente intende espandere la dimensione di un blocco for.
All'utente è richiesto di:
\begin{itemize}
	\item trovarsi nella schermata di gestione del diagramma delle classi del progetto;
	\item selezionare blocco for a visualizzazione ridotta;
	\item selezionare l'opzione di espansione dimensione.
\end{itemize} & \textcolor{Green}{\textit{Superato}}\\ \hline

\hypertarget{TV4.19}{TV89} & L'utente intende spostare uno dei blocchi inseriti all’interno del diagramma delle attività gestito in una nuova posizione.
All'utente è richiesto di:
\begin{itemize}
	\item trovarsi nella schermata di gestione del diagramma delle classi del progetto;
	\item selezionare uno dei blocchi esistenti;
	\item trascinarlo in basso od in alto fino ad inserirlo nella nuova posizione desiderata.
\end{itemize} & \textcolor{Green}{\textit{Superato}}\\ \hline

\hypertarget{TV4.20.1}{TV90} & L'utente intende modificare il nome di una variabile relativa ad un blocco variabile esistente.
All'utente è richiesto di:
\begin{itemize}
	\item trovarsi nella schermata di gestione del diagramma delle classi del progetto;
	\item selezionare uno dei blocchi variabili esistenti;
	\item modificare il nome della variabile.
\end{itemize} & \textcolor{Green}{\textit{Superato}}\\ \hline

\hypertarget{TV4.20.2}{TV91} & L'utente intende modificare il tipo di una variabile relativa ad un blocco variabile esistente.
All'utente è richiesto di:
\begin{itemize}
	\item trovarsi nella schermata di gestione del diagramma delle classi del progetto;
	\item selezionare uno dei blocchi variabili esistenti;
	\item modificare il tipo della variabile.
\end{itemize} & \textcolor{Green}{\textit{Superato}}\\ \hline

\hypertarget{TV4.20.3}{TV92} & L'utente intende modificare il valore di una variabile relativa ad un blocco variabile esistente.
All'utente è richiesto di:
\begin{itemize}
	\item trovarsi nella schermata di gestione del diagramma delle classi del progetto;
	\item selezionare uno dei blocchi variabili esistenti;
	\item modificare il valore della variabile.
\end{itemize} & \textcolor{Green}{\textit{Superato}}\\ \hline

\hypertarget{TV4.20.4}{TV93} & L'utente intende modificare l'operazione per una variabile relativa ad un blocco variabile esistente.
All'utente è richiesto di:
\begin{itemize}
	\item trovarsi nella schermata di gestione del diagramma delle classi del progetto;
	\item selezionare uno dei blocchi variabili esistenti;
	\item modificare l'operazione per la variabile.
\end{itemize} & \textcolor{Green}{\textit{Superato}}\\ \hline

\hypertarget{TV4.22.1}{TV94} & L'utente intende modificare la condizione di un blocco if esistente.
All'utente è richiesto di:
\begin{itemize}
	\item trovarsi nella schermata di gestione del diagramma delle attività di un particolare metodo;
	\item selezionare un blocco if;
	\item modificare la condizione del blocco if.
\end{itemize} & \textcolor{Green}{\textit{Superato}}\\ \hline

\hypertarget{TV4.22.2}{TV95} & L'utente intende modificare il commento relativo ad un blocco if esistente.
All'utente è richiesto di:
\begin{itemize}
	\item trovarsi nella schermata di gestione del diagramma delle attività di un particolare metodo;
	\item selezionare un blocco if;
	\item modificare il commento relativo al blocco if.
\end{itemize} & \textcolor{Green}{\textit{Superato}}\\ \hline

\hypertarget{TV4.23.1}{TV96} & L'utente intende modificare la condizione di un blocco while esistente.
All'utente è richiesto di:
\begin{itemize}
\item trovarsi nella schermata di gestione del diagramma delle attività di un particolare metodo;
\item selezionare un blocco while;
\item modificare la condizione del blocco while.
\end{itemize} & \textcolor{Green}{\textit{Superato}}\\ \hline

\hypertarget{TV4.23.2}{TV97} & L'utente intende modificare il commento relativo ad un blocco while esistente.
All'utente è richiesto di:
\begin{itemize}
\item trovarsi nella schermata di gestione del diagramma delle attività di un particolare metodo;
\item selezionare un blocco while;
\item modificare il commento relativo al blocco while.
\end{itemize} & \textcolor{Green}{\textit{Superato}}\\ \hline

\hypertarget{TV4.24.1}{TV98} & L'utente intende modificare l'inizializzazione di un blocco for esistente.
All'utente è richiesto di:
\begin{itemize}
	\item trovarsi nella schermata di gestione del diagramma delle attività di un particolare metodo; 
	\item selezionare un blocco for; 
	\item modificare l'inizializzazione del blocco for. 
\end{itemize} & \textcolor{Green}{\textit{Superato}}\\ \hline

\hypertarget{TV4.24.2}{TV99} & L'utente intende modificare la condizione da verificare di un blocco for esistente.
All'utente è richiesto di:
\begin{itemize}
	\item trovarsi nella schermata di gestione del diagramma delle attività di un particolare metodo;
	\item selezionare un blocco for;
	\item modificare la condizione da verificare del blocco for.
\end{itemize} & \textcolor{Green}{\textit{Superato}}\\ \hline

\hypertarget{TV4.24.3}{TV100} & L'utente intende modificare l'incremento-decremento di un blocco for .
All'utente è richiesto di:
\begin{itemize}
	\item trovarsi nella schermata di gestione del diagramma delle attività di un particolare metodo;
	\item selezionare un blocco for esistente;
	\item modificare l'incremento-decremento del blocco for.
\end{itemize} & \textcolor{Green}{\textit{Superato}}\\ \hline

\hypertarget{TV4.24.4}{TV101} & L'utente intende modificare il commento relativo ad un blocco for esistente.
All'utente è richiesto di:
\begin{itemize}
	\item trovarsi nella schermata di gestione del diagramma delle attività di un particolare metodo;
	\item selezionare un blocco for;
	\item modificare il commento relativo al blocco for.
\end{itemize} & \textcolor{Green}{\textit{Superato}}\\ \hline

\hypertarget{TV4.25.1}{TV102} & L'utente modificare il contenuto di un blocco custom di codice esistente.
All'utente è richiesto di:
\begin{itemize}
	\item trovarsi nella schermata di gestione del diagramma delle attività di un particolare metodo;
	\item selezionare un blocco custom di codice esistente;
	\item modificare il contenuto del blocco custom di codice.
\end{itemize} & \textcolor{Green}{\textit{Superato}}\\ \hline

\hypertarget{TV4.25.2}{TV103} & L'utente intende modificare il commento relativo ad un blocco custom di codice esistente.
All'utente è richiesto di:
\begin{itemize}
	\item trovarsi nella schermata di gestione del diagramma delle attività di un particolare metodo;
	\item selezionare un blocco custom di codice esistente;
	\item modificare il commento relativo al blocco custom di codice.
\end{itemize} & \textcolor{Green}{\textit{Superato}}\\ \hline

\hypertarget{TV4.26}{TV104} & L'utente intende inserire un blocco return.
All'utente è richiesto di:
\begin{itemize}
	\item trovarsi nella schermata di gestione del diagramma delle attività di un particolare metodo;
	\item premere sul pulsante di inserimento del blocco return;
	\item indicare dove inserire il blocco return nell'apposita area.
\end{itemize} & \textcolor{Green}{\textit{Superato}}\\ \hline

\hypertarget{TV4.26.1}{TV105} & L'utente intende inserire un valore per il blocco return.
All'utente è richiesto di:
\begin{itemize}
	\item trovarsi nella schermata di gestione del diagramma delle attività di un particolare metodo;
	\item selezionare un blocco return;
	\item inserire il valore del blocco return.
\end{itemize} & \textcolor{Green}{\textit{Superato}}\\ \hline

\hypertarget{TV4.26.2}{TV106} & L'utente intende inserire un commento per un blocco return.
All'utente è richiesto di:
\begin{itemize}
	\item trovarsi nella schermata di gestione del diagramma delle attività di un particolare metodo;
	\item selezionare un blocco return;
	\item inserire il commento relativo al blocco return.
\end{itemize} & \textcolor{Green}{\textit{Superato}}\\ \hline

\hypertarget{TV4.27}{TV107} & L'utente intende rimuovere un blocco return esistente.
All'utente è richiesto di:
\begin{itemize}
	\item trovarsi nella schermata di gestione del diagramma delle attività di un particolare metodo;
	\item selezionare il blocco return da eliminare;
	\item premere sul pulsante di cancellazione.
\end{itemize} & \textcolor{Green}{\textit{Superato}}\\ \hline

\hypertarget{TV4.28}{TV108} & L'utente intende ridurre la dimensione di un blocco return.
All'utente è richiesto di:
\begin{itemize}
	\item trovarsi nella schermata di gestione del diagramma delle classi del progetto;
	\item selezionare blocco return a visualizzazione espansa;
	\item selezionare l'opzione di riduzione dimensione.
\end{itemize} & \textcolor{Green}{\textit{Superato}}\\ \hline

\hypertarget{TV4.29}{TV109} & L'utente intende espandere la dimensione di un blocco return.
All'utente è richiesto di:
\begin{itemize}
	\item trovarsi nella schermata di gestione del diagramma delle classi del progetto;
	\item selezionare blocco return a visualizzazione ridotta;
	\item selezionare l'opzione di espansione dimensione.
\end{itemize} & \textcolor{Green}{\textit{Superato}}\\ \hline

\hypertarget{TV4.30.1}{TV110} & L'utente intende modificare il valore di un blocco return esistente.
All'utente è richiesto di:
\begin{itemize}
	\item trovarsi nella schermata di gestione del diagramma delle attività di un particolare metodo;
	\item selezionare un blocco return;
	\item modificare il valore del blocco return.
\end{itemize} & \textcolor{Green}{\textit{Superato}}\\ \hline

\hypertarget{TV4.30.2}{TV111} & L'utente intende modificare il commento relativo ad un blocco return esistente.
All'utente è richiesto di:
\begin{itemize}
	\item trovarsi nella schermata di gestione del diagramma delle attività di un particolare metodo;
	\item selezionare un blocco return;
	\item modificare il commento relativo al blocco return.
\end{itemize} & \textcolor{Green}{\textit{Superato}}\\ \hline

\hypertarget{TV4.31}{TV112} & L'utente intende inserire un blocco else.
All'utente è richiesto di:
\begin{itemize}
	\item trovarsi nella schermata di gestione del diagramma delle attività di un particolare metodo;
	\item premere sul pulsante di inserimento del blocco else.
	\item indicare dove inserire il blocco else nell'apposita area.
\end{itemize} & \textcolor{Green}{\textit{Superato}}\\ \hline

\hypertarget{TV4.31.1}{TV113} & L'utente intende inserire un commento per un blocco else.
All'utente è richiesto di:
\begin{itemize}
	\item trovarsi nella schermata di gestione del diagramma delle attività di un particolare metodo;
	\item selezionare un blocco else;
	\item inserire il commento relativo al blocco else.
\end{itemize} & \textcolor{Green}{\textit{Superato}}\\ \hline

\hypertarget{TV4.32}{TV114} & L'utente intende rimuovere un blocco else esistente.
All'utente è richiesto di:
\begin{itemize}
	\item trovarsi nella schermata di gestione del diagramma delle attività di un particolare metodo;
	\item selezionare il blocco else da eliminare;
	\item premere sul pulsante di cancellazione.
\end{itemize} & \textcolor{Green}{\textit{Superato}}\\ \hline

\hypertarget{TV4.33}{TV115} & L'utente intende ridurre la dimensione di un blocco else.
All'utente è richiesto di:
\begin{itemize}
	\item trovarsi nella schermata di gestione del diagramma delle classi del progetto;
	\item selezionare blocco else a visualizzazione espansa;
	\item selezionare l'opzione di riduzione dimensione.
\end{itemize} & \textcolor{Green}{\textit{Superato}}\\ \hline

\hypertarget{TV4.34}{TV116} & L'utente intende espandere la dimensione di un blocco else.
All'utente è richiesto di:
\begin{itemize}
	\item trovarsi nella schermata di gestione del diagramma delle classi del progetto;
	\item selezionare blocco else a visualizzazione ridotta;
	\item selezionare l'opzione di espansione dimensione.
\end{itemize} & \textcolor{Green}{\textit{Superato}}\\ \hline

\hypertarget{TV4.35.1}{TV117} & L'utente intende modificare il commento relativo ad un blocco else esistente.
All'utente è richiesto di:
\begin{itemize}
	\item trovarsi nella schermata di gestione del diagramma delle attività di un particolare metodo;
	\item selezionare un blocco else;
	\item modificare il commento relativo al blocco else.
\end{itemize} & \textcolor{Green}{\textit{Superato}}\\ \hline

\hypertarget{TV4.36}{TV118} & L'utente intende ridurre la dimensione di un blocco variabile.
All'utente è richiesto di:
\begin{itemize}
	\item trovarsi nella schermata di gestione del diagramma delle classi del progetto;
	\item selezionare blocco variabile a visualizzazione espansa;
	\item selezionare l'opzione di riduzione dimensione.
\end{itemize} & \textcolor{Green}{\textit{Superato}}\\ \hline

\hypertarget{TV4.37}{TV119} & L'utente intende espandere la dimensione di un blocco variabile.
All'utente è richiesto di:
\begin{itemize}
	\item trovarsi nella schermata di gestione del diagramma delle classi del progetto;
	\item selezionare blocco variabile a visualizzazione ridotta;
	\item selezionare l'opzione di espansione dimensione.
\end{itemize} & \textcolor{Green}{\textit{Superato}}\\ \hline

\hypertarget{TV4.38}{TV120} & L'utente intende ridurre la dimensione di un blocco custom.
All'utente è richiesto di:
\begin{itemize}
	\item trovarsi nella schermata di gestione del diagramma delle classi del progetto;
	\item selezionare blocco variabile a visualizzazione espansa;
	\item selezionare l'opzione di riduzione dimensione.
\end{itemize} & \textcolor{Green}{\textit{Superato}}\\ \hline

\hypertarget{TV4.39}{TV121} & L'utente intende espandere la dimensione di un blocco custom.
All'utente è richiesto di:
\begin{itemize}
	\item trovarsi nella schermata di gestione del diagramma delle classi del progetto;
	\item selezionare blocco variabile a visualizzazione ridotta;
	\item selezionare l'opzione di espansione dimensione.
\end{itemize} & \textcolor{Green}{\textit{Superato}}\\ \hline

\hypertarget{TV5}{TV122} & L'utente intende salvare il progetto corrente.
All'utente è richiesto di:
\begin{itemize}
	\item premere sul pulsante di salvataggio.
\end{itemize} & \textcolor{Green}{\textit{Superato}}\\ \hline

\hypertarget{TV6}{TV123} & L'utente intende scaricare il codice relativo al progetto corrente.
All'utente è richiesto di:
\begin{itemize}
	\item cliccare sul pulsante genera codice;
	\item attendere lo scaricamento della cartella.
\end{itemize} & \textcolor{Green}{\textit{Superato}}\\ \hline
\caption[Test di validazione]{Test di validazione}
\label{tab:valid}
\end{longtable}
\clearpage


\subsection{Test di sistema}
I test di sistema vengono eseguiti sul prodotto una volta che tutte le sue componenti sono completamente integrate. Questi test hanno lo scopo di verificare che il prodotto soddisfi i requisiti, come definiti nel documento \AdR.

\subsubsection{Specifica dei test di sistema}
\normalsize
\begin{longtable}{|c|>{}m{8cm}|c|}
\hline 
\textbf{Id Test} & \textbf{Descrizione} & \textbf{Stato}\\
\hline
\endhead
\hypertarget{TS1}{TS1} & Verificare che il sistema permetta all'utente di caricare un progetto. & \textcolor{Green}{\textit{Superato}}\\ \hline
\hypertarget{TS2}{TS2} & Verificare che il sistema visualizzi un messaggio d'errore se l'attore cerca di caricare come progetto un file incompatibile con l'editor stesso. & \textcolor{Green}{\textit{Superato}}\\ \hline
\hypertarget{TS3}{TS3} & Verificare che il sistema permetta all'utente di creare un nuovo progetto. & \textcolor{Green}{\textit{Superato}}\\ \hline
\hypertarget{TS4}{TS4} & Verificare che il sistema permetta all'utente di inserire una classe nel diagramma delle classi. & \textcolor{Green}{\textit{Superato}}\\ \hline
\hypertarget{TS5}{TS5} & Verificare che il sistema permetta all'utente di inserire il nome di una classe & \textcolor{Green}{\textit{Superato}}\\ \hline
\hypertarget{TS6}{TS6} & Verificare che il sistema permetta all'utente di inserire un attributo per la classe. & \textcolor{Green}{\textit{Superato}}\\ \hline
\hypertarget{TS7}{TS7} & Verificare che il sistema permetta all'utente di scegliere la visibilità per l'attributo & \textcolor{Green}{\textit{Superato}}\\ \hline
\hypertarget{TS8}{TS8} & Verificare che il sistema permetta all'utente di inserire il nome dell'attributo & \textcolor{Green}{\textit{Superato}}\\ \hline
\hypertarget{TS9}{TS9} & Verificare che il sistema permetta all'utente di inserire il tipo dell'attributo & \textcolor{Green}{\textit{Superato}}\\ \hline
\hypertarget{TS10}{TS10} & Verificare che il sistema permetta all'utente di inserire il valore di default dell'attributo & \textcolor{Green}{\textit{Superato}}\\ \hline
\hypertarget{TS11}{TS11} & Verificare che il sistema permetta all'utente di rimuovere un attributo precedentemente inserito & \textcolor{Green}{\textit{Superato}}\\ \hline
\hypertarget{TS12}{TS12} & Verificare che il sistema permetta all'utente di inserire un metodo per una classe & \textcolor{Green}{\textit{Superato}}\\ \hline
\hypertarget{TS13}{TS13} & Verificare che il sistema permetta all'utente di scegliere la visibilità per un metodo & \textcolor{Green}{\textit{Superato}}\\ \hline
\hypertarget{TS14}{TS14} & Verificare che il sistema permetta all'utente di inserire il nome di un metodo & \textcolor{Green}{\textit{Superato}}\\ \hline
\hypertarget{TS15}{TS15} & Verificare che il sistema permetta all'utente di inserire un parametro per un metodo & \textcolor{Green}{\textit{Superato}}\\ \hline
\hypertarget{TS16}{TS16} & Verificare che il sistema permetta all'utente di inserire il nome di un parametro del metodo & \textcolor{Green}{\textit{Superato}}\\ \hline
\hypertarget{TS17}{TS17} & Verificare che il sistema permetta all'utente di inserire il tipo del parametro di un metodo & \textcolor{Green}{\textit{Superato}}\\ \hline
\hypertarget{TS18}{TS18} & Verificare che il sistema permetta all'utente di inserire il valore di default di un parametro di un metodo & \textcolor{Green}{\textit{Superato}}\\ \hline
\hypertarget{TS19}{TS19} & Verificare che il sistema permetta all'utente di inserire il tipo di ritorno di un metodo & \textcolor{Green}{\textit{Superato}}\\ \hline
\hypertarget{TS20}{TS20} & Verificare che il sistema permetta all'utente di rimuovere un parametro di un metodo precedentemente inseirito & \textcolor{Green}{\textit{Superato}}\\ \hline
\hypertarget{TS21}{TS21} & Verificare che il sistema permetta all'utente di rimuovere un metodo precedentemente inserito & \textcolor{Green}{\textit{Superato}}\\ \hline
\hypertarget{TS22}{TS22} & Verificare che il sistema permetta all'utente di inserire una relazione & \textcolor{Green}{\textit{Superato}}\\ \hline
\hypertarget{TS23}{TS23} & Verificare che il sistema permetta all'utente di scegliere il nome di un'associazione & \textcolor{Green}{\textit{Superato}}\\ \hline
\hypertarget{TS24}{TS24} & Verificare che il sistema permetta all'utente di inserire la classe di partenza della relazione & \textcolor{Green}{\textit{Superato}}\\ \hline
\hypertarget{TS25}{TS25} & Verificare che il sistema permetta all'utente di scegliere la classe di destinazione della relazione & \textcolor{Green}{\textit{Superato}}\\ \hline
\hypertarget{TS26}{TS26} & Verificare che il sistema permetta all'utente di scegliere la cardinalità di un'associazione & \textcolor{Green}{\textit{Superato}}\\ \hline
\hypertarget{TS27}{TS27} & Verificare che il sistema permetta all'utente di inserire un commento & \textcolor{Green}{\textit{Superato}}\\ \hline
\hypertarget{TS28}{TS28} & Verificare che il sistema permetta all'utente di inserire il testo di un commento & \textcolor{Green}{\textit{Superato}}\\ \hline

\hypertarget{TS29}{TS29} & Verificare che il sistema permetta all'utente di rimuovere una classe & \textcolor{Green}{\textit{Superato}}\\ \hline
\hypertarget{TS30}{TS30} & Verificare che il sistema permetta all'utente di rimuovere una relazione & \textcolor{Green}{\textit{Superato}}\\ \hline
\hypertarget{TS31}{TS31} & Verificare che il sistema permetta all'utente di rimuovere un commento & \textcolor{Green}{\textit{Superato}}\\ \hline

\hypertarget{TS32}{TS32} & Verificare che il sistema permetta all'utente di modificare il nome di una classe & \textcolor{Green}{\textit{Superato}}\\ \hline
\hypertarget{TS33}{TS33} & Verificare che il sistema permetta all'utente di modificare la visibilità di un attributo & \textcolor{Green}{\textit{Superato}}\\ \hline
\hypertarget{TS34}{TS34} & Verificare che il sistema permetta all'utente di modificare il nome di un attributo & \textcolor{Green}{\textit{Superato}}\\ \hline
\hypertarget{TS35}{TS35} & Verificare che il sistema permetta all'utente di modificare il tipo dell'attributo & \textcolor{Green}{\textit{Superato}}\\ \hline
\hypertarget{TS36}{TS36} & Verificare che il sistema permetta all'utente di modificare il valore di default dell'attributo & \textcolor{Green}{\textit{Superato}}\\ \hline
\hypertarget{TS37}{TS37} & Verificare che il sistema permetta all'utente di modificare la visibilità di un metodo & \textcolor{Green}{\textit{Superato}}\\ \hline
\hypertarget{TS38}{TS38} & Verificare che il sistema permetta all'utente di modificare il nome di un metodo & \textcolor{Green}{\textit{Superato}}\\ \hline
\hypertarget{TS39}{TS39} & Verificare che il sistema permetta all'utente di modificare il nome del parametro di un metodo & \textcolor{Green}{\textit{Superato}}\\ \hline
\hypertarget{TS40}{T40} & Verificare che il sistema permetta all'utente di modificare il tipo del parametro di un metodo & \textcolor{Green}{\textit{Superato}}\\ \hline
\hypertarget{TS41}{TS41} & Verificare che il sistema permetta all'utente di modificare il valore di default di un parametro di un metodo & \textcolor{Green}{\textit{Superato}}\\ \hline
\hypertarget{TS42}{TS42} & Verificare che il sistema permetta all'utente di modificare il tipo di ritorno di un metodo & \textcolor{Green}{\textit{Superato}}\\ \hline
\hypertarget{TS43}{TS43} & Verificare che il sistema permetta all'utente di modificare il nome di un'associazione & \textcolor{Green}{\textit{Superato}}\\ \hline
\hypertarget{TS44}{TS44} & Verificare che il sistema permetta all'utente di modificare la classe1 di una relazione & \textcolor{Green}{\textit{Superato}}\\ \hline
\hypertarget{TS45}{TS45} & Verificare che il sistema permetta all'utente di modificare la classe2 di una relazione & \textcolor{Green}{\textit{Superato}}\\ \hline
\hypertarget{TS46}{TS46} & Verificare che il sistema permetta all'utente di modificare la cardinalità di un'associazione & \textcolor{Green}{\textit{Superato}}\\ \hline
\hypertarget{TS47}{TS47} & Verificare che il sistema permetta all'utente di modificare il testo di un commento & \textcolor{Green}{\textit{Superato}}\\ \hline
\hypertarget{TS48}{TS48} & Verificare che il sistema permetta all'utente di specificare una classe come statica & \textcolor{Green}{\textit{Superato}}\\ \hline
\hypertarget{TS49}{TS49} & Verificare che il sistema permetta all'utente di specificare una classe come astratta & \textcolor{Green}{\textit{Superato}}\\ \hline
\hypertarget{TS50}{TS50} & Verificare che il sistema permetta all'utente di rimuovere la specifica "Static" di una classe & \textcolor{Green}{\textit{Superato}}\\ \hline
\hypertarget{TS51}{TS51} & Verificare che il sistema permetta all'utente di rimuovere la specifica "Abstract" di una classe & \textcolor{Green}{\textit{Superato}}\\ \hline
\hypertarget{TS52}{TS52} & Verificare che il sistema permetta all'utente di specificare un metodo come statico & \textcolor{Green}{\textit{Superato}}\\ \hline
\hypertarget{TS53}{TS53} & Verificare che il sistema permetta all'utente di specificare un metodo come astratto & \textcolor{Green}{\textit{Superato}}\\ \hline
\hypertarget{TS54}{TS54} & Verificare che il sistema permetta all'utente di rimuovere la specifica "Static" di un metodo & \textcolor{Green}{\textit{Superato}}\\ \hline
\hypertarget{TS55}{TS55} & Verificare che il sistema permetta all'utente di rimuovere la specifica "Abstract" di un metodo & \textcolor{Green}{\textit{Superato}}\\ \hline
\hypertarget{TS56}{TS56} & Verificare che il sistema permetta all'utente di cliccare su un metodo di una classe ed aprire il relativo diagramma delle attività per quel metodo & \textcolor{Green}{\textit{Superato}}\\ \hline
\hypertarget{TS57}{TS57} & Verificare che il sistema permetta all'utente di inserire un blocco variabile & \textcolor{Green}{\textit{Superato}}\\ \hline
\hypertarget{TS58}{TS58} & Verificare che il sistema permetta all'utente di inserire il nome di una variabile & \textcolor{Green}{\textit{Superato}}\\ \hline
\hypertarget{TS59}{TS59} & Verificare che il sistema permetta all'utente di inserire il tipo di una variabile& \textcolor{Green}{\textit{Superato}}\\ \hline
\hypertarget{TS60}{TS60} & Verificare che il sistema permetta all'utente di inserire l'operazione per una variabile& \textcolor{Green}{\textit{Superato}}\\ \hline
\hypertarget{TS61}{TS61} & Verificare che il sistema permetta all'utente di inserire il valore per una variabile& \textcolor{Green}{\textit{Superato}}\\ \hline
\hypertarget{TS62}{TS62} & Verificare che il sistema permetta all'utente di inserire un commento per il blocco variabile & \textcolor{Green}{\textit{Superato}}\\ \hline
\hypertarget{TS63}{TS63} & Verificare che il sistema permetta all'utente di inserire un blocco if & \textcolor{Green}{\textit{Superato}}\\ \hline
\hypertarget{TS64}{TS64} & Verificare che il sistema permetta all'utente di inserire la condizione da verificare del blocco if & \textcolor{Green}{\textit{Superato}}\\ \hline
\hypertarget{TS4.3.2}{TS65} & Verificare che il sistema permetta all'utente di completare il corpo del blocco if scegliendo tra una serie di blocchi tra quelli resi disponibili dall'editor & \textcolor{Green}{\textit{Superato}}\\ \hline
\hypertarget{TS4.3.3}{TS66} & Verificare che il sistema permetta all'utente di completare il corpo del blocco else inserendo una serie di blocchi tra quelli resi disponibili dall'editor & \textcolor{Green}{\textit{Superato}}\\ \hline
\hypertarget{TS4.3.4}{TS67} & Verificare che il sistema permetta all'utente di inserire un commento per il blocco if & \textcolor{Green}{\textit{Superato}}\\ \hline
\hypertarget{TS4.4}{TS68} & Verificare che il sistema permetta all'utente di inserire un blocco while & \textcolor{Green}{\textit{Superato}}\\ \hline
\hypertarget{TS4.4.1}{TS69} & Verificare che il sistema permetta all'utente di inserire la condizione da verificare del blocco while & \textcolor{Green}{\textit{Superato}}\\ \hline
\hypertarget{TS4.4.2}{TS70} & Verificare che il sistema permetta all'utente di completare il corpo del blocco while inserendo una serie di blocchi tra quelli resi disponibili dall'editor & \textcolor{Green}{\textit{Superato}}\\ \hline
\hypertarget{TS4.4.3}{TS71} & Verificare che il sistema permetta all'utente di inserire un commento per il blocco while & \textcolor{Green}{\textit{Superato}}\\ \hline
\hypertarget{TS4.5}{TS72} & Verificare che il sistema permetta all'utente di inserire un blocco for & \textcolor{Green}{\textit{Superato}}\\ \hline
\hypertarget{TS4.5.1}{TS73} & Verificare che il sistema permetta all'utente di inserire l'inizializzazione del blocco for & \textcolor{Green}{\textit{Superato}}\\ \hline
\hypertarget{TS4.5.2}{TS74} & Verificare che il sistema permetta all'utente di inserire la condizione da verificare del blocco for & \textcolor{Green}{\textit{Superato}}\\ \hline
\hypertarget{TS4.5.3}{TS75} & Verificare che il sistema permetta all'utente di inserire l'incremento/decremento del blocco for & \textcolor{Green}{\textit{Superato}}\\ \hline
\hypertarget{TS4.5.4}{TS76} & Verificare che il sistema permetta all'utente di completare il corpo del blocco for inserendo una serie di blocchi tra quelli resi disponibili dall'editor & \textcolor{Green}{\textit{Superato}}\\ \hline
\hypertarget{TS4.5.5}{TS77} & Verificare che il sistema permetta all'utente di inserire un commento per il blocco for & \textcolor{Green}{\textit{Superato}}\\ \hline


\hypertarget{TS4.6}{TS78} & Verificare che il sistema permetta all'utente di inserire un blocco custom di codice & \textcolor{Green}{\textit{Superato}}\\ \hline
\hypertarget{TS4.6.1}{TS79} & Verificare che il sistema permetta all'utente di inserire il contenuto del blocco custom & \textcolor{Green}{\textit{Superato}}\\ \hline
\hypertarget{TS4.6.2}{TS80} & Verificare che il sistema permetta all'utente di inserire un commento per il blocco custom & \textcolor{Green}{\textit{Superato}}\\ \hline
\hypertarget{TS4.7}{TS81} & Verificare che il sistema permetta all'utente di rimuovere un blocco variabile & \textcolor{Green}{\textit{Superato}}\\ \hline
\hypertarget{TS4.9}{TS82} & Verificare che il sistema permetta all'utente di rimuovere un blocco if & \textcolor{Green}{\textit{Superato}}\\ \hline
\hypertarget{TS4.10}{TS83} & Verificare che il sistema permetta all'utente di rimuovere un blocco while & \textcolor{Green}{\textit{Superato}}\\ \hline
\hypertarget{TS4.11}{TS84} & Verificare che il sistema permetta all'utente di rimuovere un blocco for & \textcolor{Green}{\textit{Superato}}\\ \hline
\hypertarget{TS4.12}{TS85} & Verificare che il sistema permetta all'utente di rimuovere un blocco custom & \textcolor{Green}{\textit{Superato}}\\ \hline
\hypertarget{TS4.13}{TS86} & Verificare che il sistema permetta all'utente di ridurre il blocco if & \textcolor{Green}{\textit{Superato}}\\ \hline
\hypertarget{TS4.14}{TS87} & Verificare che il sistema permetta all'utente di espandere il blocco if & \textcolor{Green}{\textit{Superato}}\\ \hline
\hypertarget{TS4.15}{TS88} & Verificare che il sistema permetta all'utente di ridurre il blocco while & \textcolor{Green}{\textit{Superato}}\\ \hline
\hypertarget{TS4.16}{TS89} & Verificare che il sistema permetta all'utente di espandere il blocco while & \textcolor{Green}{\textit{Superato}}\\ \hline
\hypertarget{TS4.17}{TS90} & Verificare che il sistema permetta all'utente di ridurre il blocco for & \textcolor{Green}{\textit{Superato}}\\ \hline
\hypertarget{TS4.18}{TS91} & Verificare che il sistema permetta all'utente di espandere il blocco for & \textcolor{Green}{\textit{Superato}}\\ \hline
\hypertarget{TS4.19}{TS92} & Verificare che il sistema permetta all'utente di spostare uno dei blocchi inseriti all'interno del diagramma delle attività gestito in una nuova posizione & \textcolor{Green}{\textit{Superato}}\\ \hline
\hypertarget{TS4.20.1}{TS93} & Verificare che il sistema permetta all'utente di modificare il nome della variabile relativa al blocco variabile & \textcolor{Green}{\textit{Superato}}\\ \hline
\hypertarget{TS4.20.2}{TS94} & Verificare che il sistema permetta all'utente di modificare il tipo della variabile relativa al blocco variabile & \textcolor{Green}{\textit{Superato}}\\ \hline
\hypertarget{TS4.20.3}{TS95} & Verificare che il sistema permetta all'utente di modificare il valore della variabile relativa al blocco variabile & \textcolor{Green}{\textit{Superato}}\\ \hline
\hypertarget{TS4.20.4}{TS96} & Verificare che il sistema permetta all'utente di modificare l'operazione relativa al blocco variabile & \textcolor{Green}{\textit{Superato}}\\ \hline
\hypertarget{TS4.22.1}{TS97} & Verificare che il sistema permetta all'utente di modificare la condizione relativo al blocco if/else & \textcolor{Green}{\textit{Superato}}\\ \hline
\hypertarget{TS4.22.2}{TS98} & Verificare che il sistema permetta all'utente di modificare il commento relativo al blocco if/else & \textcolor{Green}{\textit{Superato}}\\ \hline
\hypertarget{TS4.23.1}{TS99} & Verificare che il sistema permetta all'utente di modificare la condizione relativa al blocco while & \textcolor{Green}{\textit{Superato}}\\ \hline
\hypertarget{TS4.23.2}{TS100} & Verificare che il sistema permetta all'utente di modificare il commento relativo al blocco while & \textcolor{Green}{\textit{Superato}}\\ \hline
\hypertarget{TS4.24.1}{TS101} & Verificare che il sistema permetta all'utente di modificare l'inizializzazione relativa al blocco for & \textcolor{Green}{\textit{Superato}}\\ \hline
\hypertarget{TS4.24.2}{TS102} & Verificare che il sistema permetta all'utente di modificare la condizione relativa al blocco for & \textcolor{Green}{\textit{Superato}}\\ \hline
\hypertarget{TS4.24.3}{TS103} & Verificare che il sistema permetta all'utente di modificare il passo d'incremento/decremento relativo al blocco for & \textcolor{Green}{\textit{Superato}}\\ \hline
\hypertarget{TS4.24.4}{TS104} & Verificare che il sistema permetta all'utente di modificare il commento relativo al blocco for & \textcolor{Green}{\textit{Superato}}\\ \hline
\hypertarget{TS4.25.1}{TS105} & Verificare che il sistema permetta all'utente di modificare il contenuto in codice relativo al blocco custom & \textcolor{Green}{\textit{Superato}}\\ \hline
\hypertarget{TS4.25.2}{TS106} & Verificare che il sistema permetta all'utente di modificare il commento relativo al blocco custom & \textcolor{Green}{\textit{Superato}}\\ \hline
\hypertarget{TS4.26}{TS107} & Verificare che il sistema permetta all'utente di inserire un blocco return all'interno del diagramma delle attività & \textcolor{Green}{\textit{Superato}}\\ \hline
\hypertarget{TS4.26.1}{TS108} & Verificare che il sistema permetta all'utente di inserire il valore di un blocco return & \textcolor{Green}{\textit{Superato}}\\ \hline
\hypertarget{TS4.26.2}{TS109} & Verificare che il sistema permetta all'utente di inserire il commento di un blocco return & \textcolor{Green}{\textit{Superato}}\\ \hline
\hypertarget{TS4.27.1}{TS110} & Verificare che il sistema permetta all'utente di modificare il valore di un blocco return & \textcolor{Green}{\textit{Superato}}\\ \hline
\hypertarget{TS4.27.2}{TS111} & Verificare che il sistema permetta all'utente di modificare il commento di un blocco return & \textcolor{Green}{\textit{Superato}}\\ \hline
\hypertarget{TS4.28}{TS112} & Verificare che il sistema permetta all'utente di rimuovere un blocco return & \textcolor{Green}{\textit{Superato}}\\ \hline
\hypertarget{TS4.29}{TS113} & Verificare che il sistema permetta all'utente di espandere un blocco return & \textcolor{Green}{\textit{Superato}}\\ \hline
\hypertarget{TS4.30}{TS114} & Verificare che il sistema permetta all'utente di ridurre un blocco return & \textcolor{Green}{\textit{Superato}}\\ \hline
\hypertarget{TS4.31}{TS115} & Verificare che il sistema permetta all'utente di inserire un blocco else all'interno del diagramma delle attività & \textcolor{Green}{\textit{Superato}}\\ \hline
\hypertarget{TS4.32.1}{TS116} & Verificare che il sistema permetta all'utente di inserire il commento di un blocco else & \textcolor{Green}{\textit{Superato}}\\ \hline
\hypertarget{TS4.33.1}{TS117} & Verificare che il sistema permetta all'utente di modificare il commento di un blocco else & \textcolor{Green}{\textit{Superato}}\\ \hline
\hypertarget{TS4.34}{TS118} & Verificare che il sistema permetta all'utente di rimuovere un blocco else & \textcolor{Green}{\textit{Superato}}\\ \hline
\hypertarget{TS4.35}{TS119} & Verificare che il sistema permetta all'utente di espandere un blocco else & \textcolor{Green}{\textit{Superato}}\\ \hline
\hypertarget{TS4.36}{TS120} & Verificare che il sistema permetta all'utente di ridurre un blocco else & \textcolor{Green}{\textit{Superato}}\\ \hline
\hypertarget{TS4.37}{TS121} & Verificare che il sistema permetta all'utente di espandere un blocco variabile & \textcolor{Green}{\textit{Superato}}\\ \hline
\hypertarget{TS4.38}{TS122} & Verificare che il sistema permetta all'utente di ridurre un blocco variabile & \textcolor{Green}{\textit{Superato}}\\ \hline
\hypertarget{TS4.39}{TS123} & Verificare che il sistema permetta all'utente di espandere un blocco custom & \textcolor{Green}{\textit{Superato}}\\ \hline
\hypertarget{TS4.40}{TS124} & Verificare che il sistema permetta all'utente di ridurre un blocco custom & \textcolor{Green}{\textit{Superato}}\\ \hline

\hypertarget{TS5}{TS125} & Verificare che il sistema permetta all'utente di salvare il progretto & \textcolor{Green}{\textit{Superato}}\\ \hline
\hypertarget{TS6}{TS126} & Verificare che il sistema permetta all'utente di ottenere automaticamente una cartella compressa contenente i diagrammi disegnati da SWEDesigner, l'applicativo generato tramite i diagrammi forniti dall'utente ed il codice sorgente nel linguaggio target & \textcolor{Green}{\textit{Superato}}\\ \hline
\hypertarget{TS7}{TS127} & Verificare che il sistema riceva una richiesta http post contenente il file \gloss{JSON} generato dal programma utente & \textcolor{Green}{\textit{Superato}}\\ \hline
\hypertarget{TS8.1}{TS128} & Verificare che il sistema traduca il file JSON in un oggetto Java corrispondente & \textcolor{Green}{\textit{Superato}}\\ \hline
\hypertarget{TS8.2}{TS129} & Verificare che il sistema traduca l'oggetto Java nel corrispondente codice sorgente Java & \textcolor{Green}{\textit{Superato}}\\ \hline
\hypertarget{TS9}{TS130} & Verificare che il sistema compili il codice sorgente Java generato & \textcolor{Green}{\textit{Superato}}\\ \hline
\hypertarget{TS10}{TS131} & Verificare che il sistema crei una cartella compressa contenente il file JSON, il codice sorgente Java e il programmma Java compilato  & \textcolor{Green}{\textit{Superato}}\\ \hline
\hypertarget{TS11}{TS132} & Verificare che il sistema elabori una risposta http contenente l'indirizzo della cartella compressa creata & \textcolor{Green}{\textit{Superato}}\\ \hline
\hypertarget{TS1.1}{TS133} & Verificare che l'applicativo lato \gloss{client} sia realizzato in HTML5, CSS3 e JavaScript & \textcolor{Green}{\textit{Superato}}\\ \hline
\hypertarget{TS1.2}{TS134} & verificare che l’applicativo lato \gloss{server} sia realizzato in Java con server \gloss{Tomcat} & \textcolor{Green}{\textit{Superato}}\\ \hline
\hypertarget{TS1.3}{TS135} & Verificare che l’applicativo funziona su Mozilla Firefox versione 43 o superiore & \textcolor{Green}{\textit{Superato}}\\ \hline
\hypertarget{TS1.4}{TS136} & Verificare che l’applicativo funzioni su Google Chrome versione 47 o superiore & \textcolor{Green}{\textit{Superato}}\\ \hline
\hypertarget{TS1.5}{TS137} & Verificare che l’applicativo funzioni su Internet Explorer versione 11 o superiore & \textcolor{Green}{\textit{Superato}}\\ \hline
\hypertarget{TS1.6}{TS138} & Verificare che l’applicativo funzioni su Safari versione 9 o superiore & \textcolor{Green}{\textit{Superato}}\\ \hline
\hypertarget{TS1.7}{TS138} & Verificare che l’applicativo funzioni su Microsoft Edge versione 25 o superiore & \textcolor{Green}{\textit{Superato}}\\ \hline
\hypertarget{TSO2}{TS140} & Verificare che il progetto sia sviluppato sulla piattaforma \gloss{GitHub} in modalità pubblica & \textcolor{Green}{\textit{Superato}}\\ \hline
\caption[Test di sistema]{Test di sistema}
\label{tab:sys}
\end{longtable}
\clearpage

\clearpage



\subsection{Test di integrazione}
I test di integrazione consentono di controllare che più moduli funzionino assieme in modo corretto.
	
\subsubsection{Specifica dei test di integrazione}
\normalsize
\begin{longtable}{|c|>{}m{8cm}|c|}
\hline
\textbf{Id Test} & \textbf{Descrizione} & \textbf{Stato}\\
\hline
\endhead
\hypertarget{TI1}{TI1} & La richiesta di un client alla risorsa "/" oppure "/index.html" provoca una risposta del server con stato HTTP "200 OK" recante la pagina \emph{index.html}. & \textit{Non Implementato}\\ \hline
\hypertarget{TI2}{TI2} & La richiesta di un client alla risorsa "/generate" (che alleghi un file JSON valido) provoca una risposta del server con stato HTTP "200 OK" recante un archivio ZIP. & \textit{Non Implementato}\\ \hline
\hypertarget{TI3}{TI3} & Il sistema gestisce correttamente l'interazione tra un \texttt{RequestHandlerController} e un \texttt{Compiler} di \texttt{swedesigner::server}. & \textcolor{Green}{\textit{Superato}}\\ \hline
\hypertarget{TI4}{TI4} & Il sistema gestisce correttamente l'interazione tra un \texttt{RequestHandlerController} e i suoi membri di tipo \texttt{Parser}, \texttt{Generator}, \texttt{Compiler} e \texttt{Compressor} (del \texttt{swedesigner::server}). & \textcolor{Green}{\textit{Superato}}\\ \hline
\hypertarget{TI5}{TI5} & Il sistema gestisce correttamente le componenti relative al package \texttt{generator}; in particolare, gestisce correttamente l'interazione tra un \texttt{Generator} e un \texttt{Template} di \texttt{swedesigner::server}. & \textcolor{Green}{\textit{Superato}}\\ \hline
\hypertarget{TI6}{TI6} & Il sistema gestisce correttamente le componenti relative al package \texttt{parser}; in particolare, gestisce correttamente l'interazione tra un \texttt{Parser} e un \texttt{ParsedProgram} di \texttt{swedesigner::server}. & \textcolor{Green}{\textit{Superato}}\\ \hline
\hypertarget{TI7}{TI7} & Il sistema gestisce correttamente le componenti relative al package \texttt{project}; in particolare, gestisce correttamente l'interazione tra un \texttt{ParsedElement} e uno \texttt{Stereotype} di \texttt{swedesigner::server}. & \textcolor{Green}{\textit{Superato}}\\ \hline
\hypertarget{TI8}{TI8} & Il sistema gestisce correttamente le componenti relative al package \texttt{utility}; in particolare, gestisce correttamente l'interazione tra un \texttt{RequestHandlerController} e un \texttt{Compressor} di \texttt{swedesigner::server}. & \textcolor{Green}{\textit{Superato}}\\ \hline
\hypertarget{TI9}{TI9} & Le componenti dei package \texttt{models} e \texttt{views} di \texttt{swedesigner::client} interagiscono correttamente tra loro e con la libreria esterna \emph{JointJS}. & \textcolor{Green}{\textit{Superato}}\\ \hline
\hypertarget{TI10}{TI10} & Il sistema gestisce correttamente le componenti relative al package \texttt{models::celltypes}; in particolare, gestisce correttamente l'interazione con il package \texttt{models} di \texttt{swedesigner::client} e la libreria esterna \emph{JointJS}. & \textcolor{Green}{\textit{Superato}}\\ \hline
\caption[Test di integrazione]{Test di integrazione}
\label{tab:integr}
\end{longtable}
\clearpage

\subsubsection{Tracciamento test di integrazione-componenti}
\normalsize
\begin{longtable}{|>{\centering}m{3cm}|m{9cm}<{\centering}|}
\hline 
\textbf{Test} & \textbf{Componente}\\
\hline
\endhead
\hyperlink{TI1}{TI1} & \nogloxy{\texttt{swedesigner}}\\ \hline
\hyperlink{TI2}{TI2} & \nogloxy{\texttt{swedesigner::server}}\\ \hline
\hyperlink{TI3}{TI3} & \nogloxy{\texttt{swedesigner::server::compiler}}\\ \hline
\hyperlink{TI4}{TI4} & \nogloxy{\texttt{swedesigner::server::controller}}\\ \hline
\hyperlink{TI5}{TI5} & \nogloxy{\texttt{swedesigner::server::generator}}\\ \hline
\hyperlink{TI6}{TI6} & \nogloxy{\texttt{swedesigner::server::parser}}\\ \hline
\hyperlink{TI7}{TI7} & \nogloxy{\texttt{swedesigner::server::project}}\\ \hline
\hyperlink{TI8}{TI8} & \nogloxy{\texttt{swedesigner::server::stereotype}}\\ \hline
\hyperlink{TI9}{TI9} & \nogloxy{\texttt{swedesigner::server::template}}\\ \hline
\hyperlink{TI10}{TI10} & \nogloxy{\texttt{swedesigner::server::utility}}\\ \hline
\hyperlink{TI11}{TI11} & \nogloxy{\texttt{swedesigner::server::compiler::java}}\\ \hline
\hyperlink{TI12}{TI12} & \nogloxy{\texttt{swedesigner::server::generator::java}}\\ \hline
\hyperlink{TI13}{TI13} & \nogloxy{\texttt{swedesigner::server::template::java}}\\ \hline
\hyperlink{TI14}{TI14} & \nogloxy{\texttt{swedesigner::client}}\\ \hline
\hyperlink{TI15}{TI15} & \nogloxy{\texttt{swedesigner::client::model::celltypes}}\\ \hline
\hyperlink{TI16}{TI16} & \nogloxy{\texttt{swedesigner::client::collection}}\\ \hline
\hyperlink{TI17}{TI17} & \nogloxy{\texttt{swedesigner::client::model}}\\ \hline
\hyperlink{TI18}{TI18} & \nogloxy{\texttt{swedesigner::client::model::utility}}\\ \hline
\hyperlink{TI19}{TI19} & \nogloxy{\texttt{swedesigner::client::view}}\\ \hline
\caption[Tracciamento test di integrazione-componenti]{Tracciamento test di integrazione-componenti}
\label{tab:ti-pkg}
\end{longtable}
\clearpage

\subsubsection{Tracciamento componenti-test di integrazione}
\normalsize
\begin{longtable}{|>{\centering}m{9cm}|m{3cm}<{\centering}|}
\hline 
\textbf{Componente} & \textbf{Test}\\
\hline
\endhead
\nogloxy{\texttt{swedesigner}} & \hyperlink{TI1}{TI1}\\ \hline
\nogloxy{\texttt{swedesigner::client}} & \hyperlink{TI9}{TI9}\\ \hline
\nogloxy{\texttt{swedesigner::client::model::celltypes}} & \hyperlink{TI10}{TI10}\\ \hline
\nogloxy{\texttt{swedesigner::server}} & \hyperlink{TI2}{TI2}\\ \hline
\nogloxy{\texttt{swedesigner::server::compiler}} & \hyperlink{TI3}{TI3}\\ \hline
\nogloxy{\texttt{swedesigner::server::controller}} & \hyperlink{TI4}{TI4}\\ \hline
\nogloxy{\texttt{swedesigner::server::generator}} & \hyperlink{TI5}{TI5}\\ \hline
\nogloxy{\texttt{swedesigner::server::parser}} & \hyperlink{TI6}{TI6}\\ \hline
\nogloxy{\texttt{swedesigner::server::project}} & \hyperlink{TI7}{TI7}\\ \hline
\nogloxy{\texttt{swedesigner::server::utility}} & \hyperlink{TI8}{TI8}\\ \hline
\caption[Tracciamento componenti-test di integrazione]{Tracciamento componenti-test di integrazione}
\label{tab:pkg-ti}
\end{longtable}
\clearpage



\subsection{Test di unità}
I test di unità verificano che ogni singola unità (parte di una componente software) funzioni correttamente; definiamo un'unità come la più piccola quantità di software che conviene verificare da sola. Solitamente l'unità è il metodo di una classe.

\subsubsection{Specifica dei test di unità} \label{sec:spec_tu}
\normalsize
\begin{longtable}{|c|>{}m{8cm}|c|}
\hline
\textbf{Id Test} & \textbf{Descrizione} & \textbf{Stato}\\
\hline
\endhead
\hypertarget{TU1}{TU1} & Un \texttt{JavaCompiler} è in grado di compilare dei file sorgente Java (validi per la compilazione e posti in una stessa directory), senza generare errori di compilazione. & \textcolor{Green}{\textit{Superato}}\\ \hline
\hypertarget{TU2}{TU2} & Dato il nome di una directory esistente e un \texttt{ParsedProgram} valido, un \texttt{JavaGenerator} crea nella directory un file sorgente per ogni tipo appartenente al programma. & \textcolor{Green}{\textit{Superato}}\\ \hline
\hypertarget{TU3}{TU3} & Dato un \texttt{ParsedProgram} vuoto, un \texttt{JavaGenerator} risponde a una richiesta \texttt{generate} senza lanciare eccezioni. & \textcolor{Green}{\textit{Superato}}\\ \hline
\hypertarget{TU4}{TU4} & Dato un file JSON contenente informazioni per generare un programma (quindi un array di tipi), un \texttt{Parser} è in grado di ricavare un \texttt{ParsedProgram} con tanti tipi quanti dichiarati nel file JSON. & \textcolor{Green}{\textit{Superato}}\\ \hline
\hypertarget{TU5}{TU5} & Costruito un \texttt{ParsedAttribute}, questo è in grado di generare una stringa Java contenente tipo e nome passatigli nel costruttore. & \textcolor{Green}{\textit{Superato}}\\ \hline
\hypertarget{TU6}{TU6} & Costruito un \texttt{ParsedAttribute} statico, questo è in grado di generare una stringa Java contenente la keyword ``static'' seguita dalla dichiarazione dell'attributo. & \textcolor{Green}{\textit{Superato}}\\ \hline
\hypertarget{TU7}{TU7} & Costruito un \texttt{ParsedAttribute} con nome composto di sole maiuscole, questo è in grado di generare una stringa Java contenente la keyword ``final'' seguita dalla dichiarazione dell'attributo. & \textcolor{Green}{\textit{Superato}}\\ \hline
\hypertarget{TU8}{TU8} & Costruito un \texttt{ParsedAttribute}, questo è in grado di generare una stringa Java contenente visibilità, tipo, nome e valore iniziale passatigli nel costruttore. & \textcolor{Green}{\textit{Superato}}\\ \hline
\hypertarget{TU9}{TU9} & Costruita una \texttt{ParsedClass}, questa è in grado di generare una stringa Java contenente contenente la keyword ``class'' seguita dal nome della \texttt{ParsedClass}. & \textcolor{Green}{\textit{Superato}}\\ \hline
\hypertarget{TU10}{TU10} & Costruita una \texttt{ParsedClass} astratta, questa è in grado di generare una stringa Java contenente contenente le keyword ``abstract class'' seguite dal nome della \texttt{ParsedClass}. & \textcolor{Green}{\textit{Superato}}\\ \hline
\hypertarget{TU11}{TU11} & Costruita una \texttt{ParsedClass} e aggiuntole un attributo, questa è in grado di generare una stringa Java contenente la dichiarazione dell'attributo. & \textcolor{Green}{\textit{Superato}}\\ \hline
\hypertarget{TU12}{TU12} & Costruita una \texttt{ParsedClass}, è possibile aggiungerle delle interfacce come supertipi. & \textcolor{Green}{\textit{Superato}}\\ \hline
\hypertarget{TU13}{TU13} & Costruita una \texttt{ParsedClass}, è possibile aggiungerle una classe come supertipo. & \textcolor{Green}{\textit{Superato}}\\ \hline
\hypertarget{TU14}{TU14} & Costruita una \texttt{ParsedClass}, l'aggiunta di un supertipo che non sia né classe né interfaccia lancia una \texttt{ParsedException}. & \textcolor{Green}{\textit{Superato}}\\ \hline
\hypertarget{TU15}{TU15} & Costruito un \texttt{ParsedCustom}, questo è in grado di generare la stringa passatagli nel costruttore. & \textcolor{Green}{\textit{Superato}}\\ \hline
\hypertarget{TU16}{TU16} & Costruito un \texttt{ParsedElse}, questo è in grado di generare una stringa Java contenente la keyword ``else'' e l'insieme di istruzioni passatogli nel costruttore. & \textcolor{Green}{\textit{Superato}}\\ \hline
\hypertarget{TU17}{TU17} & Costruito un \texttt{ParsedFor}, questo è in grado di generare una stringa Java contenente ``for(A; B; C)'', dove A, B e C sono rispettivamente l'inizializzazione, la condizione e l'aggiornamento passatigli nel costruttore. & \textcolor{Green}{\textit{Superato}}\\ \hline
\hypertarget{TU18}{TU18} & Costruito un \texttt{ParsedFor}, questo è in grado di generare una stringa Java contenente l'insieme di istruzioni passatogli nel costruttore. & \textcolor{Green}{\textit{Superato}}\\ \hline
\hypertarget{TU19}{TU19} & Costruito un \texttt{ParsedIf}, questo è in grado di generare una stringa Java contenente la keyword ``if'' con la condizione e l'insieme di istruzioni passatigli nel costruttore. & \textcolor{Green}{\textit{Superato}}\\ \hline
\hypertarget{TU20}{TU20} & Costruita una \texttt{ParsedInterface}, questa è in grado di generare una stringa Java contenente la keyword ``interface'' seguita dal nome della \texttt{ParsedInterface}. & \textcolor{Green}{\textit{Superato}}\\ \hline
\hypertarget{TU21}{TU21} & Costruita una \texttt{ParsedInterface} e aggiuntole un attributo finale, statico e pubblico, questa è in grado di generare una stringa Java contenente la dichiarazione dell'attributo aggiunto. & \textcolor{Green}{\textit{Superato}}\\ \hline
\hypertarget{TU22}{TU22} & Costruita una \texttt{ParsedInterface} e aggiuntole un metodo non implementato, questa è in grado di generare una stringa Java contenente la dichiarazione del metodo aggiunto. & \textcolor{Green}{\textit{Superato}}\\ \hline
\hypertarget{TU23}{TU23} & Costruita una \texttt{ParsedInterface} e aggiuntole un supertipo, questa è in grado di generare una stringa Java contenente la keyword ``interface'' seguita dal nome della \texttt{ParsedInterface}, poi dalla keyword ``extends'' e infine dal nome del supertipo. & \textcolor{Green}{\textit{Superato}}\\ \hline
\hypertarget{TU24}{TU24} & Costruita una \texttt{ParsedInterface}, l'aggiunta di un attributo privato lancia una \texttt{ParsedException}. & \textcolor{Green}{\textit{Superato}}\\ \hline
\hypertarget{TU25}{TU25} & Costruita una \texttt{ParsedInterface}, l'aggiunta di un attributo non statico lancia una \texttt{ParsedException}. & \textcolor{Green}{\textit{Superato}}\\ \hline
\hypertarget{TU26}{TU26} & Costruita una \texttt{ParsedInterface}, l'aggiunta di un attributo non finale lancia una \texttt{ParsedException}. & \textcolor{Green}{\textit{Superato}}\\ \hline
\hypertarget{TU27}{TU27} & Costruita una \texttt{ParsedInterface}, l'aggiunta di un metodo privato lancia una \texttt{ParsedException}. & \textcolor{Green}{\textit{Superato}}\\ \hline
\hypertarget{TU28}{TU28} & Costruita una \texttt{ParsedInterface}, l'aggiunta di un metodo implementato lancia una \texttt{ParsedException}. & \textcolor{Green}{\textit{Superato}}\\ \hline
\hypertarget{TU29}{TU29} & Costruita una \texttt{ParsedInterface}, l'aggiunta di una classe come suo supertipo lancia una \texttt{ParsedException}. & \textcolor{Green}{\textit{Superato}}\\ \hline
\hypertarget{TU30}{TU30} & Costruito un \texttt{ParsedMethod}, questo è in grado di generare una stringa Java contenente la segnatura passatagli nel costruttore. & \textcolor{Green}{\textit{Superato}}\\ \hline
\hypertarget{TU31}{TU31} & Costruito un \texttt{ParsedMethod}, questo è in grado di generare una stringa Java contenente la sequenza di istruzioni passatagli nel costruttore. & \textcolor{Green}{\textit{Superato}}\\ \hline
\hypertarget{TU32}{TU32} & Costruito un \texttt{ParsedProgram} e aggiuntigli dei \texttt{ParsedType}, un \texttt{JavaGenerator} è in grado di generare una stringa Java contenente i tipi aggiunti al \texttt{ParsedProgram}. & \textcolor{Green}{\textit{Superato}}\\ \hline
\hypertarget{TU33}{TU33} & Costruito un \texttt{ParsedReturn}, questo è in grado di generare una stringa Java contenente la keyword ``return'' seguita dal valore passatogli nel costruttore e da un punto e virgola. & \textcolor{Green}{\textit{Superato}}\\ \hline
\hypertarget{TU34}{TU34} & Costruito un \texttt{ParsedWhile}, questo è in grado di generare una stringa Java contenente la keyword ``while'' seguita dalla condizione e dalla sequenza di istruzioni passategli nel costruttore. & \textcolor{Green}{\textit{Superato}}\\ \hline
\hypertarget{TU35}{TU35} & Un \texttt{JavaTemplate} è in grado di fornire un oggetto \texttt{ST} che generi una stringa Java contenente la dichiarazione di un attributo. & \textcolor{Green}{\textit{Superato}}\\ \hline
\hypertarget{TU36}{TU36} & Un \texttt{JavaTemplate} è in grado di fornire un oggetto \texttt{ST} che generi una stringa Java contenente la dichiarazione di una classe. & \textcolor{Green}{\textit{Superato}}\\ \hline
\hypertarget{TU37}{TU37} & Un \texttt{JavaTemplate} è in grado di fornire un oggetto \texttt{ST} che generi una stringa Java contenente la keyword ``else'' e il corpo del ramo else. & \textcolor{Green}{\textit{Superato}}\\ \hline
\hypertarget{TU38}{TU38} & Un \texttt{JavaTemplate} è in grado di fornire un oggetto \texttt{ST} che generi una stringa Java contenente la dichiarazione di un ciclo for e il corpo del ciclo. & \textcolor{Green}{\textit{Superato}}\\ \hline
\hypertarget{TU39}{TU39} & Un \texttt{JavaTemplate} è in grado di fornire un oggetto \texttt{ST} che generi una stringa Java contenente la dichiarazione di un blocco if e il corpo del blocco. & \textcolor{Green}{\textit{Superato}}\\ \hline
\hypertarget{TU40}{TU40} & Un \texttt{JavaTemplate} è in grado di fornire un oggetto \texttt{ST} che generi una stringa Java contenente un'istruzione. & \textcolor{Green}{\textit{Superato}}\\ \hline
\hypertarget{TU41}{TU41} & Un \texttt{JavaTemplate} è in grado di fornire un oggetto \texttt{ST} che generi una stringa Java contenente la dichiarazione di un'interfaccia. & \textcolor{Green}{\textit{Superato}}\\ \hline
\hypertarget{TU42}{TU42} & Un \texttt{JavaTemplate} è in grado di fornire un oggetto \texttt{ST} che generi una stringa Java contenente la dichiarazione di un metodo. & \textcolor{Green}{\textit{Superato}}\\ \hline
\hypertarget{TU43}{TU43} & Un \texttt{JavaTemplate} è in grado di fornire un oggetto \texttt{ST} che generi una stringa Java contenente la keyword ``return'' seguita dal valore da ritornare e da un punto e virgola. & \textcolor{Green}{\textit{Superato}}\\ \hline
\hypertarget{TU44}{TU44} & Un \texttt{JavaTemplate} è in grado di fornire un oggetto \texttt{ST} che generi una stringa Java contenente la dichiarazione di un ciclo while e il corpo del ciclo. & \textcolor{Green}{\textit{Superato}}\\ \hline
\hypertarget{TU45}{TU45} & Un \texttt{Compressor} è in grado di comprimere in formato ZIP una directory nel filesystem del server. & \textcolor{Green}{\textit{Superato}}\\ \hline
\hypertarget{TU46}{TU46} & Un \texttt{RequestHandlerController} è in grado di rispondere a una richiesta \texttt{handleGeneratorRequest} senza lanciare eccezioni. & \textcolor{Green}{\textit{Superato}}\\ \hline
\hypertarget{TU47}{TU47} & Un \texttt{JavaCompiler} a cui venga chiesto di compilare i file di una directory inesistente ritorna una lista di errori contenente l'errore ``[cartella] is not a valid directory''. & \textcolor{Green}{\textit{Superato}}\\ \hline
\hypertarget{TU48}{TU48} & Una \texttt{AppView} è una \texttt{backbone.View}. & \textcolor{Green}{\textit{Superato}}\\ \hline
\hypertarget{TU49}{TU49} & Una \texttt{AppView} possiede un oggetto \texttt{views}. & \textcolor{Green}{\textit{Superato}}\\ \hline
\hypertarget{TU50}{TU50} & Una \texttt{ProjectView} è una \texttt{backbone.View}. & \textcolor{Green}{\textit{Superato}}\\ \hline
\hypertarget{TU51}{TU51} & Una \texttt{ProjectView} è in grado di disegnare un diagramma di attività. & \textcolor{Green}{\textit{Superato}}\\ \hline
\hypertarget{TU52}{TU52} & Una \texttt{ProjectView} possiede un \texttt{joint.dia.Paper}. & \textcolor{Green}{\textit{Superato}}\\ \hline
\hypertarget{TU53}{TU53} & Una \texttt{DetailsView} è una \texttt{backbone.View}. & \textcolor{Green}{\textit{Superato}}\\ \hline
\hypertarget{TU54}{TU54} & Una \texttt{DetailsView} è in grado di disegnare il riquadro dei dettagli. & \textcolor{Green}{\textit{Superato}}\\ \hline
\hypertarget{TU55}{TU55} & Una \texttt{NewCellView} è una \texttt{backbone.View}. & \textcolor{Green}{\textit{Superato}}\\ \hline
\hypertarget{TU56}{TU56} & Una \texttt{NewCellView} è in grado di disegnare il riquadro di creazione. & \textcolor{Green}{\textit{Superato}}\\ \hline
\hypertarget{TU57}{TU57} & Un \texttt{ProjectModel} è un \texttt{backbone.Model}. & \textcolor{Green}{\textit{Superato}}\\ \hline
\hypertarget{TU58}{TU58} & Un \texttt{ProjectModel} è in grado di creare un nuovo progetto. & \textcolor{Green}{\textit{Superato}}\\ \hline
\hypertarget{TU59}{TU59} & Un \texttt{NewCellModel} è un \texttt{backbone.Model}. & \textcolor{Green}{\textit{Superato}}\\ \hline
\hypertarget{TU60}{TU60} & Una \texttt{NewCellFactory} è in grado di istanziare una \texttt{HxClass}. & \textcolor{Green}{\textit{Superato}}\\ \hline
\hypertarget{TU61}{TU61} & Una \texttt{NewCellFactory} è in grado di istanziare una \texttt{HxInterface}. & \textcolor{Green}{\textit{Superato}}\\ \hline
\hypertarget{TU62}{TU62} & Una \texttt{NewCellFactory} è in grado di istanziare un \texttt{HxComment}. & \textcolor{Green}{\textit{Superato}}\\ \hline
\hypertarget{TU63}{TU63} & Una \texttt{NewCellFactory} è in grado di istanziare una \texttt{HxGeneralization}. & \textcolor{Green}{\textit{Superato}}\\ \hline
\hypertarget{TU64}{TU64} & Una \texttt{NewCellFactory} è in grado di istanziare una \texttt{HxAssociation}. & \textcolor{Green}{\textit{Superato}}\\ \hline
\hypertarget{TU65}{TU65} & Una \texttt{NewCellFactory} è in grado di istanziare una \texttt{HxImplementation}. & \textcolor{Green}{\textit{Superato}}\\ \hline
\hypertarget{TU66}{TU66} & Una \texttt{NewCellFactory} è in grado di istanziare un \texttt{HxCustom}. & \textcolor{Green}{\textit{Superato}}\\ \hline
\hypertarget{TU67}{TU67} & Una \texttt{NewCellFactory} è in grado di istanziare un \texttt{HxElse}. & \textcolor{Green}{\textit{Superato}}\\ \hline
\hypertarget{TU68}{TU68} & Una \texttt{NewCellFactory} è in grado di istanziare un \texttt{HxFor}. & \textcolor{Green}{\textit{Superato}}\\ \hline
\hypertarget{TU69}{TU69} & Una \texttt{NewCellFactory} è in grado di istanziare un \texttt{HxIf}. & \textcolor{Green}{\textit{Superato}}\\ \hline
\hypertarget{TU70}{TU70} & Una \texttt{NewCellFactory} è in grado di istanziare una \texttt{HxVariable}. & \textcolor{Green}{\textit{Superato}}\\ \hline
\hypertarget{TU71}{TU71} & Una \texttt{NewCellFactory} è in grado di istanziare un \texttt{HxReturn}. & \textcolor{Green}{\textit{Superato}}\\ \hline
\hypertarget{TU72}{TU72} & Una \texttt{NewCellFactory} è in grado di istanziare un \texttt{HxWhile}. & \textcolor{Green}{\textit{Superato}}\\ \hline
\hypertarget{TU73}{TU73} & Un \texttt{Command} è in grado di creare un nuovo progetto. & \textcolor{Green}{\textit{Superato}}\\ \hline
\hypertarget{TU74}{TU74} & Una \texttt{HxClass} possiede nome, attributi, metodi, astrazione e staticità. & \textcolor{Green}{\textit{Superato}}\\ \hline
\hypertarget{TU75}{TU75} & Una \texttt{HxInterface} possiede nome e metodi. & \textcolor{Green}{\textit{Superato}}\\ \hline
\hypertarget{TU76}{TU76} & Un \texttt{HxComment} possiede un campo di testo. & \textcolor{Green}{\textit{Superato}}\\ \hline
\hypertarget{TU77}{TU77} & Un \texttt{HxGeneralization} possiede un meta-tipo. & \textcolor{Green}{\textit{Superato}}\\ \hline
\hypertarget{TU78}{TU78} & Un \texttt{HxAssociation} possiede un meta-tipo. & \textcolor{Green}{\textit{Superato}}\\ \hline
\hypertarget{TU79}{TU79} & Un \texttt{HxImplementation} possiede un meta-tipo. & \textcolor{Green}{\textit{Superato}}\\ \hline
\hypertarget{TU80}{TU80} & Un \texttt{HxCustom} possiede meta-tipo, commento e una stringa. & \textcolor{Green}{\textit{Superato}}\\ \hline
\hypertarget{TU81}{TU81} & Un \texttt{HxElse} possiede meta-tipo e commento. & \textcolor{Green}{\textit{Superato}}\\ \hline
\hypertarget{TU82}{TU82} & Un \texttt{HxFor} possiede meta-tipo, commento, inizializzazione, terminazione e incremento. & \textcolor{Green}{\textit{Superato}}\\ \hline
\hypertarget{TU83}{TU83} & Un \texttt{HxIf} possiede meta-tipo, commento e condizione. & \textcolor{Green}{\textit{Superato}}\\ \hline
\hypertarget{TU84}{TU84} & Una \texttt{HxVariable} possiede meta-tipo, commento, nome, tipo, operazione e valore. & \textcolor{Green}{\textit{Superato}}\\ \hline
\hypertarget{TU85}{TU85} & Un \texttt{HxReturn} possiede meta-tipo, commento e valore. & \textcolor{Green}{\textit{Superato}}\\ \hline
\hypertarget{TU86}{TU86} & Un \texttt{HxWhile} possiede meta-tipo, commento e condizione. & \textcolor{Green}{\textit{Superato}}\\ \hline
\caption[Test di unità]{Test di unità}
\label{tab:unit}
\end{longtable}
\clearpage


\subsubsection{Tracciamento test di unità-metodi} \label{sec:tracc_tu}
\normalsize
\begin{longtable}{|>{\centering}m{1cm}|m{12cm}<{\centering}|}
\hline
\textbf{Test} & \textbf{Metodi}\\
\hline
\endhead
\hyperlink{TU1}{TU1} & \nogloxy{\texttt{swedesigner::server::compiler::java::-\linebreak JavaCompiler::compile()}}\\ \hline

\hyperlink{TU2}{TU2} & \nogloxy{\texttt{swedesigner::server::generator::java::-\linebreak JavaGenerator::generate()}}\\ \hline

\hyperlink{TU3}{TU3} & \nogloxy{\texttt{swedesigner::server::generator::java::-\linebreak JavaGenerator::generate()}}\\ \hline

\hyperlink{TU4}{TU4} & \nogloxy{\texttt{swedesigner::server::parser::Parser::-\linebreak createParsedProgram()}}\\ \hline

\hyperlink{TU5}{TU5} & \nogloxy{\texttt{swedesigner::server::project::ParsedAttribute::-\linebreak renderTemplate()}}\\ \hline

\hyperlink{TU6}{TU6} & \nogloxy{\texttt{swedesigner::server::project::ParsedAttribute::-\linebreak renderTemplate()}}\\ \hline

\hyperlink{TU7}{TU7} & \nogloxy{\texttt{swedesigner::server::project::ParsedAttribute::-\linebreak renderTemplate()}}\\ \hline

\hyperlink{TU8}{TU8} & \nogloxy{\texttt{swedesigner::server::project::ParsedAttribute::-\linebreak renderTemplate()}}\\ \hline

\hyperlink{TU9}{TU9} & \nogloxy{\texttt{swedesigner::server::project::ParsedClass::-\linebreak renderTemplate()}}\\ \hline

\hyperlink{TU10}{TU10} & \nogloxy{\texttt{swedesigner::server::project::ParsedClass::-\linebreak renderTemplate()}}\\ \hline

\hyperlink{TU11}{TU11} & \nogloxy{\texttt{swedesigner::server::project::ParsedClass::-\linebreak addField()}}\\ & \nogloxy{\texttt{swedesigner::server::project::ParsedClass::-\linebreak renderTemplate()}}\\ \hline

\hyperlink{TU12}{TU12} & \nogloxy{\texttt{swedesigner::server::project::ParsedClass::-\linebreak addSupertype()}}\\ & \nogloxy{\texttt{swedesigner::server::project::ParsedClass::-\linebreak renderTemplate()}}\\ \hline

\hyperlink{TU13}{TU13} & \nogloxy{\texttt{swedesigner::server::project::ParsedClass::-\linebreak addSupertype()}}\\ & \nogloxy{\texttt{swedesigner::server::project::ParsedClass::-\linebreak renderTemplate()}}\\ \hline

\hyperlink{TU14}{TU14} & \nogloxy{\texttt{swedesigner::server::project::ParsedClass::-\linebreak addSupertype()}}\\ & \nogloxy{\texttt{swedesigner::server::project::ParsedClass::-\linebreak renderTemplate()}}\\ \hline

\hyperlink{TU15}{TU15} & \nogloxy{\texttt{swedesigner::server::project::ParsedCustom::-\linebreak renderTemplate()}}\\ \hline

\hyperlink{TU16}{TU16} & \nogloxy{\texttt{swedesigner::server::project::ParsedElse::-\linebreak renderTemplate()}}\\ \hline

\hyperlink{TU17}{TU17} & \nogloxy{\texttt{swedesigner::server::project::ParsedFor::-\linebreak renderTemplate()}}\\ \hline

\hyperlink{TU18}{TU18} & \nogloxy{\texttt{swedesigner::server::project::ParsedFor::-\linebreak renderTemplate()}}\\ \hline

\hyperlink{TU19}{TU19} & \nogloxy{\texttt{swedesigner::server::project::ParsedIf::-\linebreak renderTemplate()}}\\ \hline

\hyperlink{TU20}{TU20} & \nogloxy{\texttt{swedesigner::server::project::ParsedInterface::-\linebreak renderTemplate()}}\\ \hline

\hyperlink{TU21}{TU21} & \nogloxy{\texttt{swedesigner::server::project::ParsedInterface::-\linebreak renderTemplate()}}\\ \hline

\hyperlink{TU22}{TU22} & \nogloxy{\texttt{swedesigner::server::project::ParsedInterface::-\linebreak addMethod()}}\\ & \nogloxy{\texttt{swedesigner::server::project::ParsedInterface::-\linebreak renderTemplate()}}\\ \hline

\hyperlink{TU23}{TU23} & \nogloxy{\texttt{swedesigner::server::project::ParsedInterface::-\linebreak addSupertype()}}\\ & \nogloxy{\texttt{swedesigner::server::project::ParsedInterface::-\linebreak renderTemplate()}}\\ \hline

\hyperlink{TU24}{TU24} & \nogloxy{\texttt{swedesigner::server::project::ParsedInterface::-\linebreak addField()}}\\ \hline

\hyperlink{TU25}{TU25} & \nogloxy{\texttt{swedesigner::server::project::ParsedInterface::-\linebreak addField()}}\\ \hline

\hyperlink{TU26}{TU26} & \nogloxy{\texttt{swedesigner::server::project::ParsedInterface::-\linebreak addField()}}\\ \hline

\hyperlink{TU27}{TU27} & \nogloxy{\texttt{swedesigner::server::project::ParsedInterface::-\linebreak addMethod()}}\\ \hline

\hyperlink{TU28}{TU28} & \nogloxy{\texttt{swedesigner::server::project::ParsedInterface::-\linebreak addMethod()}}\\ \hline

\hyperlink{TU29}{TU29} & \nogloxy{\texttt{swedesigner::server::project::ParsedInterface::-\linebreak addSupertype()}}\\ \hline

\hyperlink{TU30}{TU30} & \nogloxy{\texttt{swedesigner::server::project::ParsedMethod::-\linebreak renderTemplate()}}\\ \hline

\hyperlink{TU31}{TU31} & \nogloxy{\texttt{swedesigner::server::project::ParsedMethod::-\linebreak renderTemplate()}}\\ \hline

\hyperlink{TU32}{TU32} & \nogloxy{\texttt{swedesigner::server::generator::java::-\linebreak JavaGenerator::generate()}}\\ & \nogloxy{\texttt{swedesigner::server::project::ParsedProgram::-\linebreak addType()}}\\ \hline

\hyperlink{TU33}{TU33} & \nogloxy{\texttt{swedesigner::server::project::ParsedReturn::-\linebreak renderTemplate()}}\\ \hline

\hyperlink{TU34}{TU34} & \nogloxy{\texttt{swedesigner::server::project::ParsedWhile::-\linebreak renderTemplate()}}\\ \hline

\hyperlink{TU35}{TU35} & \nogloxy{\texttt{swedesigner::server::template::java::-\linebreak JavaTemplate::getAttributeTemplate()}}\\ \hline

\hyperlink{TU36}{TU36} & \nogloxy{\texttt{swedesigner::server::template::java::-\linebreak JavaTemplate::getClassTemplate()}}\\ \hline

\hyperlink{TU37}{TU37} & \nogloxy{\texttt{swedesigner::server::template::java::-\linebreak JavaTemplate::getElseTemplate()}}\\ \hline

\hyperlink{TU38}{TU38} & \nogloxy{\texttt{swedesigner::server::template::java::-\linebreak JavaTemplate::getForTemplate()}}\\ \hline

\hyperlink{TU39}{TU39} & \nogloxy{\texttt{swedesigner::server::template::java::-\linebreak JavaTemplate::getIfTemplate()}}\\ \hline

\hyperlink{TU40}{TU40} & \nogloxy{\texttt{swedesigner::server::template::java::-\linebreak JavaTemplate::getStatementTemplate()}}\\ \hline

\hyperlink{TU41}{TU41} & \nogloxy{\texttt{swedesigner::server::template::java::-\linebreak JavaTemplate::getInterfaceTemplate()}}\\ \hline

\hyperlink{TU42}{TU42} & \nogloxy{\texttt{swedesigner::server::template::Template::-\linebreak getMethodTemplate()}}\\ \hline

\hyperlink{TU43}{TU43} & \nogloxy{\texttt{swedesigner::server::template::Template::-\linebreak getReturnTemplate()}}\\ \hline

\hyperlink{TU44}{TU44} & \nogloxy{\texttt{swedesigner::server::template::java::-\linebreak JavaTemplate::getWhileTemplate()}}\\ \hline

\hyperlink{TU45}{TU45} & \nogloxy{\texttt{swedesigner::server::utility::Compressor::-\linebreak zip()}}\\ \hline

\hyperlink{TU46}{TU46} & \nogloxy{\texttt{swedesigner::server::controller::-\linebreak RequestHandlerController::HandleGenerationRequest()}}\\ \hline

\hyperlink{TU47}{TU47} & \nogloxy{\texttt{swedesigner::server::compiler::java::-\linebreak JavaCompiler::compile()}}\\ \hline

\hyperlink{TU48}{TU48} & \nogloxy{\texttt{swedesigner::client::view::AppView::-\linebreak initialize()}}\\ \hline

\hyperlink{TU49}{TU49} & \nogloxy{\texttt{swedesigner::client::view::AppView::-\linebreak initialize()}}\\ \hline

\hyperlink{TU50}{TU50} & \nogloxy{\texttt{swedesigner::client::view::ProjectView::-\linebreak initialize()}}\\ \hline

\hyperlink{TU51}{TU51} & \nogloxy{\texttt{swedesigner::client::view::ProjectView::-\linebreak renderActivity()}}\\ \hline

\hyperlink{TU52}{TU52} & \nogloxy{\texttt{swedesigner::client::view::ProjectView::-\linebreak initialize()}}\\ \hline

\hyperlink{TU53}{TU53} & \nogloxy{\texttt{swedesigner::client::view::DetailsView::-\linebreak initialize()}}\\ \hline

\hyperlink{TU54}{TU54} & \nogloxy{\texttt{swedesigner::client::view::DetailsView::-\linebreak render()}}\\ \hline

\hyperlink{TU55}{TU55} & \nogloxy{\texttt{swedesigner::client::view::NewCellView::-\linebreak initialize()}}\\ \hline

\hyperlink{TU56}{TU56} & \nogloxy{\texttt{swedesigner::client::view::NewCellView::-\linebreak render()}}\\ \hline

\hyperlink{TU57}{TU57} & \nogloxy{\texttt{swedesigner::client::model::ProjectModel::-\linebreak initialize()}}\\ \hline

\hyperlink{TU58}{TU58} & \nogloxy{\texttt{swedesigner::client::model::ProjectModel::-\linebreak initialize()}}\\ \hline

\hyperlink{TU59}{TU59} & \nogloxy{\texttt{swedesigner::client::model::NewCellModel::-\linebreak initialize()}}\\ \hline

\hyperlink{TU60}{TU60} & \nogloxy{\texttt{swedesigner::client::model::NewCellFactory::-\linebreak getCell()}}\\ \hline

\hyperlink{TU61}{TU61} & \nogloxy{\texttt{swedesigner::client::model::NewCellFactory::-\linebreak getCell()}}\\ \hline

\hyperlink{TU62}{TU62} & \nogloxy{\texttt{swedesigner::client::model::NewCellFactory::-\linebreak getCell()}}\\ \hline

\hyperlink{TU63}{TU63} & \nogloxy{\texttt{swedesigner::client::model::NewCellFactory::-\linebreak getCell()}}\\ \hline

\hyperlink{TU64}{TU64} & \nogloxy{\texttt{swedesigner::client::model::NewCellFactory::-\linebreak getCell()}}\\ \hline

\hyperlink{TU65}{TU65} & \nogloxy{\texttt{swedesigner::client::model::NewCellFactory::-\linebreak getCell()}}\\ \hline

\hyperlink{TU66}{TU66} & \nogloxy{\texttt{swedesigner::client::model::NewCellFactory::-\linebreak getCell()}}\\ \hline

\hyperlink{TU67}{TU67} & \nogloxy{\texttt{swedesigner::client::model::NewCellFactory::-\linebreak getCell()}}\\ \hline

\hyperlink{TU68}{TU68} & \nogloxy{\texttt{swedesigner::client::model::NewCellFactory::-\linebreak getCell()}}\\ \hline

\hyperlink{TU69}{TU69} & \nogloxy{\texttt{swedesigner::client::model::NewCellFactory::-\linebreak getCell()}}\\ \hline

\hyperlink{TU70}{TU70} & \nogloxy{\texttt{swedesigner::client::model::NewCellFactory::-\linebreak getCell()}}\\ \hline

\hyperlink{TU71}{TU71} & \nogloxy{\texttt{swedesigner::client::model::NewCellFactory::-\linebreak getCell()}}\\ \hline

\hyperlink{TU72}{TU72} & \nogloxy{\texttt{swedesigner::client::model::NewCellFactory::-\linebreak getCell()}}\\ \hline

\hyperlink{TU73}{TU73} & \nogloxy{\texttt{swedesigner::client::model::ProjectCommand::-\linebreak execute()}}\\ \hline

\hyperlink{TU74}{TU74} & \nogloxy{\texttt{swedesigner::client::model::celltypes::-\linebreak class::ClassDiagramElement::getValues()}}\\ \hline

\hyperlink{TU75}{TU75} & \nogloxy{\texttt{swedesigner::client::model::celltypes::-\linebreak class::ClassDiagramElement::getValues()}}\\ \hline

\hyperlink{TU76}{TU76} & \nogloxy{\texttt{swedesigner::client::model::celltypes::-\linebreak class::ClassDiagramElement::getValues()}}\\ \hline

\hyperlink{TU77}{TU77} & \nogloxy{\texttt{swedesigner::client::model::celltypes::-\linebreak class::ClassDiagramLink::getValues()}}\\ \hline

\hyperlink{TU78}{TU78} & \nogloxy{\texttt{swedesigner::client::model::celltypes::-\linebreak class::ClassDiagramLink::getValues()}}\\ \hline

\hyperlink{TU79}{TU79} & \nogloxy{\texttt{swedesigner::client::model::celltypes::-\linebreak class::ClassDiagramLink::getValues()}}\\ \hline

\hyperlink{TU80}{TU80} & \nogloxy{\texttt{swedesigner::client::model::celltypes::-\linebreak activity::ActivityDiagramElement::getValues()}}\\ \hline

\hyperlink{TU81}{TU81} & \nogloxy{\texttt{swedesigner::client::model::celltypes::-\linebreak activity::ActivityDiagramElement::getValues()}}\\ \hline

\hyperlink{TU82}{TU82} & \nogloxy{\texttt{swedesigner::client::model::celltypes::-\linebreak activity::ActivityDiagramElement::getValues()}}\\ \hline

\hyperlink{TU83}{TU83} & \nogloxy{\texttt{swedesigner::client::model::celltypes::-\linebreak activity::ActivityDiagramElement::getValues()}}\\ \hline

\hyperlink{TU84}{TU84} & \nogloxy{\texttt{swedesigner::client::model::celltypes::-\linebreak activity::ActivityDiagramElement::getValues()}}\\ \hline

\hyperlink{TU85}{TU85} & \nogloxy{\texttt{swedesigner::client::model::celltypes::-\linebreak activity::ActivityDiagramElement::getValues()}}\\ \hline

\hyperlink{TU86}{TU86} & \nogloxy{\texttt{swedesigner::client::model::celltypes::-\linebreak activity::ActivityDiagramElement::getValues()}}\\ \hline

\caption[Tracciamento test di unità-metodi]{Tracciamento test di unità-metodi}
\label{tab:tu-met}
\end{longtable}





\appendix

%%%%%%%%%%%%%%%%%%%%%%%%%%%%%%%%%%
%%  Resoconto attività di verifica
%%%%%%%%%%%%%%%%%%%%%%%%%%%%%%%%%%

\section{Resoconto attività di verifica}
All'interno di questa sezione vengono riportati gli esiti delle attività di verifica svolte sui processi attivati e sui relativi prodotti secondo quanto stabilito nel \PdP.
	\subsection{Revisione dei requisiti}
		\subsubsection{Verifica Software Documentation Management Process}
		Le attività di verifica svolte sui documenti prodotti sono state di due tipi:
		\begin{itemize}		
			\item attività di verifica manuali;
			\item attività di verifica automatizzate.
		\end{itemize}
		
		Le prime sono state svolte dai verificatori assegnati ad ogni documento utilizzando la tecnica di 					analisi statica walkthrough. Grazie ad essa è stato possibile correggere una discreta quantità di errori e imprecisioni presenti all'interno dei documenti prodotti fra cui: 
		\begin{itemize}	
			\item nuovi termini da inserire nel glossario;
			\item termini presenti nel glossario, ma non correttamente segnati;
			\item violazioni delle norme tipografiche e grammaticali come definite nel documento \NdP;
			\item refusi ed errori grammaticali;
			\item periodi troppo lunghi e complessi da spezzare, ridurre e semplificare.
		\end{itemize}
		A partire dalla natura e frequenza degli errori identificati il gruppo ha iniziato a stilare una lista
		di controllo da applicare nei successivi momenti di verifica nel contesto della strategia di verifica
		inspection.

		Le attività di verifica automatizzate, invece, sono state svolte calcolando l'indice Gulpease dei 	
		diversi documenti prodotti attraverso l'uso di appositi strumenti web automatici. 
		\\I risultati ottenuti sono elencati nella seguente tabella:
		\begin{table}[H]
		\begin{tabular}{|l|l|l|}
		\hline
		\textbf{Documento} 		&\textbf{Valutazione} &\textbf{~~~~~~Esito~~~~~~} \\
		\hline
		\PdQ 					&51		&~~~~~~Superato~~~~~~\\
		\NdP 					&41		&~~~~~~Superato~~~~~~\\
		\SdF 					&49		&~~~~~~Superato~~~~~~\\	
		\AdR 					&72		&~~~~~~Superato~~~~~~\\
		\PdP 					&55		&~~~~~~Superato~~~~~~\\
		\Glossario 				&48		&~~~~~~Superato~~~~~~\\
		\textit{Verbali.pdf} 		&49		&~~~~~~Superato~~~~~~\\
		\hline
		\end{tabular}
		\caption{Esiti del calcolo dell'indice di Gulpease dei documenti consegnati}
		\end{table}
		
		\subsubsection{Verifica Project Planning Process \& Process Assessment and Control Process}
		Per verificare il soddisfacimento degli obiettivi di qualità definiti per tale processo, a partire dal 				consultivo redatto nel \PdP{} sono stati calcolati gli indici di budget variance e schedule variance per ognuna delle attività previste.
		
		I risultati ottenuti sono elencati nella seguente tabella:	
		\begin{table}[H]
		\begin{tabular}{|l|l|l|}
		\hline
		\textbf{Documento} 		&\textbf{Schedule variance} &\textbf{Budget variance} \\
		\hline
		\PdQ 					&0\%		&0\%\\
		\NdP 					&0\%		&0\%\\
		\SdF 					&0\%		&0\%\\
		\AdR 					&0\%		&0\%\\
		\PdP 					&-25\%		&0\%\\
		\Glossario 				&0\%		&0\%\\
		\textit{Verbali.pdf} 	&0\%		&0\%\\
		\hline
		\end{tabular}
		\caption{Esiti del calcolo degli indici di schedule e budget variance}
		\end{table}


	\subsection{Revisione dei progettazione}
		\subsubsection{Verifica Software Documentation Management Process}
		Le attività di verifica svolte sui documenti prodotti durante la progettazione sono state di due tipi:
		\begin{itemize}		
			\item attività di verifica manuali;
			\item attività di verifica automatizzate.
		\end{itemize}
		
		Le prime sono state svolte dai verificatori assegnati ad ogni documento utilizzando la tecnica di 					analisi statica walkthrough. Grazie ad essa è stato possibile correggere una discreta quantità di errori e imprecisioni presenti all'interno dei documenti prodotti fra cui: 
		\begin{itemize}	
			\item nuovi termini da inserire nel glossario;
			\item termini presenti nel glossario, ma non correttamente segnati;
			\item violazioni delle norme tipografiche e grammaticali come definite nel documento \NdP;
			\item refusi ed errori grammaticali;
			\item periodi troppo lunghi e complessi da spezzare, ridurre e semplificare.
		\end{itemize}
		A partire dalla natura e frequenza degli errori identificati il gruppo ha integrato le liste di controllo precedentemente stilate.

		Le attività di verifica automatizzate, invece, sono state svolte utilizzando gli strumenti indicati nel documento \NdP. 
		
		I risultati ottenuti sono elencati nella seguente tabella:
		\begin{table}[H]
		\begin{tabular}{|l|l|l|}
		\hline
		\textbf{Documento} 		&\textbf{Valutazione} &\textbf{~~~~~~Esito~~~~~~} \\
		\hline
		\PdQ 					&61		&~~~~~~Superato~~~~~~\\
		\NdP 					&49		&~~~~~~Superato~~~~~~\\
		\AdR 					&71		&~~~~~~Superato~~~~~~\\
		\PdP 					&55		&~~~~~~Superato~~~~~~\\
		\ST 					&69		&~~~~~~Superato~~~~~~\\	
		\Glossario 				&51		&~~~~~~Superato~~~~~~\\
		\textit{VI_17-02-16.pdf} 		&57		&~~~~~~Superato~~~~~~\\
		\textit{VI_17-02-22.pdf} 		&59		&~~~~~~Superato~~~~~~\\
		\textit{VI_17-02-24.pdf} 		&62		&~~~~~~Superato~~~~~~\\
		\hline
		\end{tabular}
		\caption{Esiti del calcolo dell'indice di Gulpease dei documenti consegnati}
		\end{table}
		
		\subsubsection{Verifica Project Planning Process \& Process Assessment and Control Process}
		Per verificare il soddisfacimento degli obiettivi di qualità definiti per tale processo, a partire dal consultivo redatto nel \PdP{} sono stati calcolati gli indici di budget variance e schedule variance per ognuna delle attività previste.
		
		I risultati ottenuti sono elencati nella seguente tabella:	
		\begin{table}[H]
		\begin{tabular}{|l|l|l|}
		\hline
		\textbf{Documento} 		&\textbf{Schedule variance} &\textbf{Budget variance} 		\\
		\hline
		\PdQ 					&0\%		&-105*\%\\
		\NdP 					&0\%		&0\%\\
		\AdR 					&0\%		&0\%\\
		\PdP 					&0\%		&0\%\\
		\ST					&0\%		&+11*\%\\
		\Glossario 				&0\%		&0\%\\
		\textit{VI_17-02-16.pdf} 		&0\%		&0\%\\
		\textit{VI_17-02-22.pdf} 		&0\%		&0\%\\
		\textit{VI_17-02-24.pdf} 		&0\%		&0\%\\
		\hline
		\end{tabular}
		\caption{Esiti del calcolo degli indici di schedule e budget variance}
		\end{table}
		* è stato necessario allocare un numero maggiore di ore per il piano di qualifica rispetto a quanto preventivato alla luce della valutazione negativa ottenuta in sede di revisione RR. Tali ore sono state sottratte all'attività di progettazione e quindi alla redazione del documento \ST.
		
		\subsubsection{Verifica Software Architectural Design Process}
		La verifica del rispetto degli obiettivi stabiliti per tale processo è stata effettuata calcolando una serie di metriche specificate nel documento \NdP. Queste metriche sono d'aiuto alla comprensione del grado di completezza, correttezza e stile dell'architettura software prodotta.
		
		I risultati ottenuti sono elencati nella seguente tabella:
		\begin{table}[H]
		\begin{tabular}{|l|p{2.25cm}|p{2.25cm}|p{2.25cm}|}
		\hline
		\textbf{Metrica} & \textbf{Valore/range accettabile} & \textbf{Valore/range obiettivo} & \textbf{Valore/range effettivo} \\
		\hline
		Completezza & ~ & ~ & ~ \\
		Numero di violazioni di alta importanza & 0 & 0 & $27$* \\
		Numero di violazioni di media importanza & $[0, 5]$ & 0 & 0 \\
		Numero di violazioni di bassa importanza & $[0, 10]$ & 0 & 0 \\
		\hline
		Correttezza & ~ & ~ & ~ \\
		Numero di violazioni di alta importanza & 0 & 0 \\
		Numero di violazioni di media importanza & $[0, 5]$ & 0 & 0 \\
		Numero di violazioni di bassa importanza & $[0, 10]$ & 0 & 0 \\
		\hline
		Stile & ~ & ~ & ~ \\
		Numero di violazioni di alta importanza & 0 & 0 & 0 \\
		Numero di violazioni di media importanza & $[0, 5]$ & 0 & $13$** \\
		Numero di violazioni di bassa importanza & $[0, 10]$ & 0 & 0 \\
		\hline
		\end{tabular}
		\caption{Esiti del calcolo delle metriche relative alla progettazione}
		\end{table}
		
		* l'alto valore del numero di violazioni di alta importanza relative alla completezza è dovuto al fatto che SDMetrics indica come non usate le classi che nell'attuale diagramma dei package implementano interfacce o classi astratte e che quindi a differenza della loro superclasse diretta non sono esplicitamente coinvolte in alcun tipo di associazione di dipendenza o riferimento.
		
		** l'alto valore del numero di violazione di media importanza relative allo stile è dovuto al fatto che nel server in alcune occasioni viene violato lo Stable-Dependencies Principle perchè alcuni package dipendono da altri package contenenti una singola classe concreta. Basta quindi che i primi abbiano una singola interfaccia affinchè la violazione venga segnalata.

		\subsubsection{Verifica della qualità dell'architettura}
		Per verificare l'osservanza degli obiettivi riguardanti la qualità dell'architettura prodotta, sono stati calcolati i valori delle metriche relative ai diagrammi dei package.
		
		I risultati ottenuti sono elencati nella seguente tabella:
		\begin{longtable}{|p{5.5cm}|p{2.25cm}|p{2.25cm}|p{2.25cm}|}
		\hline
		\textbf{Metrica} 		 					&\textbf{Valore/range accettabile}	&\textbf{Valore/range obiettivo}	&\textbf{Valore/range effettivo}\\
		\hline
		\textbf{Distanza dalla sequenza principale normalizzata} & & &\\
		client.model.utility &0.0 - 0.7 &0.0 - 0.5 &0.5\\
		client.model.celltypes &0.0 - 0.7 &0.0 - 0.5 &0.5\\
		client.model &0.0 - 0.7 &0.0 - 0.5 &0.625\\
		client.collection &0.0 - 0.7 &0.0 - 0.5 &0.5\\
		client.view &0.0 - 0.7 &0.0 - 0.5 &0\\
		server.controller &0.0 - 0.7 &0.0 - 0.5 &0.39\\
		server.parser  &0.0 - 0.7 &0.0 - 0.5 &0\\
		server.generator.java  &0.0 - 0.7 &0.0 - 0.5 &0\\
		server.generator &0.0 - 0.7 &0.0 - 0.5 &0.5\\
		server.stereotype  &0.0 - 0.7 &0.0 - 0.5 &0.33\\
		server.project  &0.0 - 0.7 &0.0 - 0.5 &0.39\\
		server.template.java  &0.0 - 0.7 &0.0 - 0.5 &0.63\\
		server.template  &0.0 - 0.7 &0.0 - 0.5 &0.16\\
		server.compiler.java  &0.0 - 0.7 &0.0 - 0.5 &0.63\\
		server.compiler  &0.0 - 0.7 &0.0 - 0.5 &0.5\\
		server.utility  &0.0 - 0.7 &0.0 - 0.5 &0.5\\
		\hline
		Numero di figli diretti &0 - 4 &0 - 2 &min0 max8* **\\
		\hline
		\textbf{Numero di tipi per package} & & &\\
		client.model.utility &0 - 30 &0 - 20 &6\\
		client.model.celltypes &0 - 30 &0 - 20 &17\\
		client.model &0 - 30 &0 - 20 &3\\
		client.collection &0 - 30 &0 - 20 &1\\
		client.view &0 - 30 &0 - 20 &4\\
		server.controller &0 - 30 &0 - 20 &1\\
		server.parser &0 - 30 &0 - 20 &1\\
		server.generator.java &0 - 30 &0 - 20 &1\\
		server.generator &0 - 30 &0 - 20 &2\\
		server.stereotype &0 - 30 &0 - 20 &1\\
		server.project &0 - 30 &0 - 20 &16\\
		server.template.java &0 - 30 &0 - 20 &1\\
		server.template &0 - 30 &0 - 20 &2\\
		server.compiler.java &0 - 30 &0 - 20 &1\\
		server.compiler &0 - 30 &0 - 20 &2\\
		server.utility &0 - 30 &0 - 20 &1\\
		\hline
		Profondità della gerarchia &0 - 4 &1 - 2 &min 0 max 2*\\
		\hline
		\caption{Esiti del calcolo delle metriche relative all'architettura}
		\end{longtable}
	* abbiamo deciso di specificare solamente i valori massimi e minimi della metrica per evitare di creare tabelle troppo prolisse e poco leggibili, a fronte di un contenuto equivalente a quello proposto.
	
	** l'unica classe che ha un numero di figli diretti al di fuori del range d'accettazione è la classe ActivityDiagramElement.

\subsection{Revisione di qualifica}
		\subsubsection{Verifica Software Documentation Management Process}
		Le attività di verifica svolte sui documenti prodotti durante il periodo precedente la revisione di qualifica sono state di due tipi:
		\begin{itemize}		
			\item attività di verifica manuali;
			\item attività di verifica automatizzate.
		\end{itemize}
		
		Le prime sono state svolte dai verificatori assegnati ad ogni documento utilizzando la tecnica di 					analisi statica walkthrough. Grazie ad essa è stato possibile correggere una discreta quantità di errori e imprecisioni presenti all'interno dei documenti prodotti fra cui: 
		\begin{itemize}	
			\item nuovi termini da inserire nel glossario;
			\item termini presenti nel glossario, ma non correttamente segnati;
			\item violazioni delle norme tipografiche e grammaticali come definite nel documento \NdP;
			\item refusi ed errori grammaticali;
			\item periodi troppo lunghi e complessi da spezzare, ridurre e semplificare.
		\end{itemize}
		A partire dalla natura e frequenza degli errori identificati il gruppo ha integrato le liste di controllo precedentemente stilate.

		Le attività di verifica automatizzate, invece, sono state svolte utilizzando gli strumenti indicati nel documento \NdP. 
		
		I risultati ottenuti sono elencati nella seguente tabella:
		\begin{table}[H]
		\begin{tabular}{|l|l|l|}
		\hline
		\textbf{Documento} 		&\textbf{Valutazione} &\textbf{~~~~~~Esito~~~~~~} \\
		\hline
		\PdQ 					&72		&~~~~~~Superato~~~~~~\\
		\NdP 					&68		&~~~~~~Superato~~~~~~\\
		\AdR 					&71		&~~~~~~Superato~~~~~~\\
		\PdP 					&59		&~~~~~~Superato~~~~~~\\
		\ST 					&75		&~~~~~~Superato~~~~~~\\
		\DP 					&70		&~~~~~~Superato~~~~~~\\	
		\Glossario 				&51		&~~~~~~Superato~~~~~~\\
		\textit{VI_17-03-17.pdf} 		&61		&~~~~~~Superato~~~~~~\\
		\textit{VE_17-03-20.pdf} 		&70		&~~~~~~Superato~~~~~~\\
		\textit{VE_17-03-23.pdf} 		&65		&~~~~~~Superato~~~~~~\\
		\hline
		\end{tabular}
		\caption{Esiti del calcolo dell'indice di Gulpease dei documenti consegnati}
		\end{table}
		
		\subsubsection{Verifica Project Planning Process \& Process Assessment and Control Process}
		Per verificare il soddisfacimento degli obiettivi di qualità definiti per tale processo, a partire dal consultivo redatto nel \PdP{} sono stati calcolati gli indici di budget variance e schedule variance per ognuna delle attività previste.
		
		I risultati ottenuti sono elencati nella seguente tabella:	
		\begin{table}[H]
		\begin{tabular}{|l|l|l|}
		\hline
		\textbf{Documento} 		&\textbf{Schedule variance} &\textbf{Budget variance} 		\\
		\hline
		\PdQ 					&0\%		&0\%\\
		\NdP 					&0\%		&0\%\\
		\AdR 					&0\%		&0\%\\
		\PdP 					&0\%		&0\%\\
		\ST						&0\%		&0\%\\
		\DP						&-38\%*		&-35\%*\\
		\Glossario 				&0\%		&0\%\\
		\textit{VI_17-03-17.pdf} 		&0\%		&0\%\\
		\textit{VE_17-03-20.pdf} 		&0\%		&0\%\\
		\textit{VE_17-03-23.pdf} 		&0\%		&0\%\\
		\hline
		\end{tabular}
		\caption{Esiti del calcolo degli indici di schedule e budget variance}
		\end{table}
		*Per maggiori dettagli relativi alla riduzione del tempo e del budget dedicato al documento \DP si consulti il \PdP.

		\subsubsection{Verifica Software Architectural Design Process}
		La verifica del rispetto degli obiettivi stabiliti per tale processo è stata effettuata calcolando una serie di metriche specificate nel documento \NdP. Queste metriche sono d'aiuto alla comprensione del grado di completezza, correttezza e stile dell'architettura software prodotta.
		
		I risultati ottenuti sono elencati nella seguente tabella:
		\begin{table}[H]
		\begin{tabular}{|l|p{2.25cm}|p{2.25cm}|p{2.25cm}|}
		\hline
		\textbf{Metrica} & \textbf{Valore/range accettabile} & \textbf{Valore/range obiettivo} & \textbf{Valore/range effettivo} \\
		\hline
		Completezza & ~ & ~ & ~ \\
		Numero di violazioni di alta importanza &0 & 0 &0 \\
		Numero di violazioni di media importanza &0-5 & 0 & 0 \\
		Numero di violazioni di bassa importanza &0-10 & 0 & 0 \\
		\hline
		Correttezza & ~ & ~ & ~ \\
		Numero di violazioni di alta importanza &0 & 0 & 3 \\
		Numero di violazioni di media importanza &0-5 & 0 & 0 \\
		Numero di violazioni di bassa importanza &0-10 & 0 & 0 \\
		\hline
		Stile & ~ & ~ & ~ \\
		Numero di violazioni di alta importanza &0 & 0 & 0 \\
		Numero di violazioni di media importanza &0-5 & 0 & 3 \\
		Numero di violazioni di bassa importanza &0-10 & 0 & 0 \\
		\hline
		\end{tabular}
		\caption{Esiti del calcolo delle metriche relative alla progettazione}
		\end{table}
		
		\subsubsection{Verifica della qualità dell'architettura}
		Per verificare l'osservanza degli obiettivi riguardanti la qualità dell'architettura prodotta, sono stati calcolati i valori delle metriche relative ai diagrammi dei package.
		
		I risultati ottenuti sono elencati nella seguente tabella:
		\begin{longtable}{|p{5.5cm}|p{2.25cm}|p{2.25cm}|p{2.25cm}|}
		\hline
		\textbf{Metrica} 		 					&\textbf{Valore/range accettabile}	&\textbf{Valore/range obiettivo}	&\textbf{Valore/range effettivo}\\
		\hline
		\textbf{Distanza dalla sequenza principale normalizzata} & & &\\
		client.model.utility &0.0 - 0.7 &0.0 - 0.5 &0.5\\
		client.model.celltypes &0.0 - 0.7 &0.0 - 0.5 &0.45\\
		client.model &0.0 - 0.7 &0.0 - 0.5 &0.55\\
		client.view &0.0 - 0.7 &0.0 - 0.5 &0.67\\
		server.controller &0.0 - 0.7 &0.0 - 0.5 &0.39\\
		server.parser  &0.0 - 0.7 &0.0 - 0.5 &0.09\\
		server.generator.java  &0.0 - 0.7 &0.0 - 0.5 &0.25\\
		server.generator &0.0 - 0.7 &0.0 - 0.5 &0.21\\
		server.stereotype  &0.0 - 0.7 &0.0 - 0.5 &0.33\\
		server.project  &0.0 - 0.7 &0.0 - 0.5 &0.11\\
		server.template.java  &0.0 - 0.7 &0.0 - 0.5 &0.33\\
		server.template  &0.0 - 0.7 &0.0 - 0.5 &0.32\\
		server.compiler.java  &0.0 - 0.7 &0.0 - 0.5 &0.33\\
		server.compiler  &0.0 - 0.7 &0.0 - 0.5 &0.30\\
		server.utility  &0.0 - 0.7 &0.0 - 0.5 &0.5\\
		\hline
		\textbf{Grado di accoppiamento afferente per package} & & &\\
		client.model.utility &0-7 &0-39 &5\\
		client.model.celltypes &0-7 &0-39 &4\\
		client.model &0-7 &0-39 &2\\
		client.view &0-7 &0-39 &4\\
		server.controller &0-7 &0-39 &4\\
		server.parser  &0-7 &0-39 &1\\
		server.generator.java  &0-7 &0-39 &1\\
		server.generator &0-7 &0-39 &2\\
		server.stereotype  &0-7 &0-39 &1\\
		server.project  &0-7 &0-39 &4\\
		server.template.java  &0-7 &0-39 &1\\
		server.template  &0-7 &0-39 &14\\
		server.compiler.java  &0-7 &0-39 &1\\
		server.compiler  &0-7 &0-39 &1\\
		server.utility  &0-7 &0-39 &1\\
		\hline
		\textbf{Grado di accoppiamento efferente per package} & & &\\
		client.model.utility &0-6 &0-16 &4\\
		client.model.celltypes &0-6 &0-16 &3\\
		client.model &0-6 &0-16 &5\\
		client.view &0-6 &0-16 &2\\
		server.controller &0-6 &0-16 &6\\
		server.parser &0-6 &0-16 &10\\
		server.generator.java &0-6 &0-16 &3\\
		server.generator &0-6 &0-16 &5\\
		server.stereotype &0-6 &0-16 &2\\
		server.project &0-6 &0-16 &10\\
		server.template.java &0-6 &0-16 &2\\
		server.template &0-6 &0-16 &3\\
		server.compiler.java &0-6 &0-16 &2\\
		server.compiler &0-6 &0-16 &4\\
		server.utility &0-6 &0-16 &1\\
		\hline
		\textbf{Instabilità} & & &\\
		client.model.utility &0.0-0.3,0.7-1.0 &0.0-0.4,0.6-1.0 &0.5\\
		client.model.celltypes &0.0-0.3,0.7-1.0 &0.0-0.4,0.6-1.0 &0.33\\
		client.model &0.0-0.3,0.7-1.0 &0.0-0.4,0.6-1.0 &0.22\\
		client.view &0.0-0.3,0.7-1.0 &0.0-0.4,0.6-1.0 &1\\
		server.controller &0.0-0.3,0.7-1.0 &0.0-0.4,0.6-1.0 &0.6\\
		server.parser &0.0-0.3,0.7-1.0 &0.0-0.4,0.6-1.0 &0.9\\
		server.generator.java &0.0-0.3,0.7-1.0 &0.0-0.4,0.6-1.0 &0.75\\
		server.generator &0.0-0.3,0.7-1.0 &0.0-0.4,0.6-1.0 &0.71\\
		server.stereotype &0.0-0.3,0.7-1.0 &0.0-0.4,0.6-1.0 &0.66\\
		server.project &0.0-0.3,0.7-1.0 &0.0-0.4,0.6-1.0 &0.71\\
		server.template.java &0.0-0.3,0.7-1.0 &0.0-0.4,0.6-1.0 &0.66\\
		server.template &0.0-0.3,0.7-1.0 &0.0-0.4,0.6-1.0 &0.17\\
		server.compiler.java &0.0-0.3,0.7-1.0 &0.0-0.4,0.6-1.0 &0.66\\
		server.compiler &0.0-0.3,0.7-1.0 &0.0-0.4,0.6-1.0 &0.8\\
		server.utility &0.0-0.3,0.7-1.0 &0.0-0.4,0.6-1.0 &0.5\\
		\hline
		\textbf{Numero di campi dati per classe} & & &\\
		Classi client &0-10 &0-20 &min0 max7*\\
		Classi server &0-10 &0-20 &min0 max7*\\
		\hline
		\textbf{Numero di figli diretti} & & &\\
		Classi client &0 - 4 &0 - 2 &min0 max7* **\\
		Classi server &0 - 4 &0 - 2 &min0 max8* **\\
		\hline
		\textbf{Numero di metodi per classe} & & &\\
		Classi client &0-10 &0-20 &min0 max11*\\
		Classi server &0-10 &0-20 &min1 max10*\\
		\hline
		\textbf{Numero di tipi per package} & & &\\
		client.model.utility &0-20 &0-30 &5\\
		client.model.celltypes &0-20 &0-30 &9\\
		client.model &0-20 &0-30 &31***\\
		client.view &0-20 &0-30 &4\\
		server.controller &0-20 &0-30 &1\\
		server.parser &0-20 &0-30 &1\\
		server.generator.java &0-20 &0-30 &1\\
		server.generator &0-20 &0-30 &2\\
		server.stereotype &0-20 &0-30 &1\\
		server.project &0-20 &0-30 &16\\
		server.template.java &0-20 &0-30 &1\\
		server.template &0-20 &0-30 &2\\
		server.compiler.java &0-20 &0-30 &1\\
		server.compiler &0-20 &0-30 &2\\
		server.utility &0-20 &0-30 &1\\
		\hline
		\textbf{Profondità della gerarchia} & & &\\
		Classi client &0 - 4 &1 - 2 &min 0 max 2*\\
		Classi server &0 - 4 &1 - 2 &min 0 max 1*\\
		\hline
		\textbf{Profondità della gerarchia} & & &\\
		Classi client &0-4 &0 &min0 max2*\\
		Classi server &0-4 &0 &min0 max1*\\
		\hline
		
		\caption{Esiti del calcolo delle metriche relative all'architettura}
		\end{longtable}
	* abbiamo deciso di specificare solamente i valori massimi e minimi della metrica per evitare di creare tabelle troppo prolisse e poco leggibili, a fronte di un contenuto equivalente a quello proposto.
	
	** le uniche due classi che hanno un numero di figli diretti al di fuori del range d'accettazione sono le classi ActivityDiagramElement e ParsedInstruction.
	
	*** l'unico package contenente un numero di tipi al di fuori del range di accettazione è il package client.model.
	
		\subsubsection{Verifica della qualità del software}
		Per verificare l'osservanza degli obiettivi riguardanti la qualità del software prodotto, sono stati calcolati i valori delle metriche relative al linguaggio Java utilizzato nel back-end.
		
		I risultati ottenuti sono elencati nella seguente tabella:
		\begin{longtable}{|p{5.5cm}|p{2.25cm}|p{2.25cm}|p{2.25cm}|}
		\hline
		\textbf{Metrica} &\textbf{Valore/range accettabile}	&\textbf{Valore/range obiettivo}	&\textbf{Valore/range effettivo}\\
		\hline
		\textbf{Complessita ciclomatica per metodo} &0-8 &0-10 &min1 max20***\\
		\hline
		\textbf{Grado di accoppiamento afferente per package} & & &\\
		server.controller &0-7 &0-39 &4\\
		server.parser  &0-7 &0-39 &1\\
		server.generator.java  &0-7 &0-39 &1\\
		server.generator &0-7 &0-39 &2\\
		server.stereotype  &0-7 &0-39 &1\\
		server.project  &0-7 &0-39 &4\\
		server.template.java  &0-7 &0-39 &1\\
		server.template  &0-7 &0-39 &14\\
		server.compiler.java  &0-7 &0-39 &1\\
		server.compiler  &0-7 &0-39 &1\\
		server.utility  &0-7 &0-39 &1\\
		\hline
		\textbf{Grado di accoppiamento efferente per package} & & &\\
		server.controller &0-6 &0-16 &6\\
		server.parser &0-6 &0-16 &10\\
		server.generator.java &0-6 &0-16 &3\\
		server.generator &0-6 &0-16 &5\\
		server.stereotype &0-6 &0-16 &2\\
		server.project &0-6 &0-16 &10\\
		server.template.java &0-6 &0-16 &2\\
		server.template &0-6 &0-16 &3\\
		server.compiler.java &0-6 &0-16 &2\\
		server.compiler &0-6 &0-16 &4\\
		server.utility &0-6 &0-16 &1\\
		\hline
		\textbf{Instabilità} & & &\\
		server.controller &0.0-0.3,0.7-1.0 &0.0-0.4,0.6-1.0 &1\\
		server.parser &0.0-0.3,0.7-1.0 &0.0-0.4,0.6-1.0 &1\\
		server.generator.java &0.0-0.3,0.7-1.0 &0.0-0.4,0.6-1.0 &0.25\\
		server.generator &0.0-0.3,0.7-1.0 &0.0-0.4,0.6-1.0 &0.20\\
		server.stereotype &0.0-0.3,0.7-1.0 &0.0-0.4,0.6-1.0 &1\\
		server.project &0.0-0.3,0.7-1.0 &0.0-0.4,0.6-1.0 &0.43\\
		server.template.java &0.0-0.3,0.7-1.0 &0.0-0.4,0.6-1.0 &0.07\\
		server.template &0.0-0.3,0.7-1.0 &0.0-0.4,0.6-1.0 &1\\
		server.compiler.java &0.0-0.3,0.7-1.0 &0.0-0.4,0.6-1.0 &0.33\\
		server.compiler &0.0-0.3,0.7-1.0 &0.0-0.4,0.6-1.0 &0.20\\
		server.utility &0.0-0.3,0.7-1.0 &0.0-0.4,0.6-1.0 &0.33\\
		\hline
		\textbf{Numero di campi dati per classe} &0-10 &0-20 &min1 max7*\\
		\hline
		\textbf{Numero di figli diretti} &0 - 4 &0 - 2 &min0 max8* **\\
		\hline
		\textbf{Numero di metodi per classe} &0-10 &0-20 &min1 max11*\\
		\hline
		\textbf{Numero di tipi per package} & & &\\
		server.controller &0-20 &0-30 &1\\
		server.parser &0-20 &0-30 &1\\
		server.generator.java &0-20 &0-30 &1\\
		server.generator &0-20 &0-30 &1\\
		server.stereotype &0-20 &0-30 &1\\
		server.project &0-20 &0-30 &16\\
		server.template.java &0-20 &0-30 &1\\
		server.template &0-20 &0-30 &1\\
		server.compiler.java &0-20 &0-30 &2\\
		server.compiler &0-20 &0-30 &1\\
		server.utility &0-20 &0-30 &1\\
		\hline
		\textbf{Profondità della gerarchia} &0 - 4 &1 - 2 &min1 max4*\\
		&0 &0 &0\\
		\hline
		\textbf{Numero di linee di codice per metodo} &0-20 &0-40 &min1 max66* ****\\
		\hline
		\textbf{Numero di parametri per metodo} &0-4 &0-8 &min1 max6*\\
		\hline
		\textbf{Percentuale linee di commento su linee di codice} &20-40 &20-40 &11\\
		\hline
		\textbf{Profondità annidamento blocchi} &0-4 &0-8 &min0 max6*\\
		\hline
		
		\caption{Esiti del calcolo delle metriche relative al codice}
		\end{longtable}
	* abbiamo deciso di specificare solamente i valori massimi e minimi della metrica per evitare di creare tabelle troppo prolisse e poco leggibili, a fronte di un contenuto equivalente a quello proposto.
	
	** l'unica classe che ha un numero di figli diretti al di fuori del range d'accettazione è la classe e ParsedInstruction.
	
	*** i metodi che presentano una complessità cicolomatica al di fuori del valore accettabile sono relBuilder, typeBuilder, recursiveBuilder e createActivity.
	
	**** la classe che presenta metodi con una lunghezza superiore a quella accettabile è la classe Parser.
	
	
	\subsection{System Qualification Testing Process \& Software Qualification Testing Process}
		Per verificare l'osservanza degli obiettivi di qualità riguardanti la produzione di unità software eseguibili, sono stati calcolati i valori delle seguenti metriche.
		
	\begin{center}
\begin{tabular}{| p{6cm} | p{2.5cm} | p{2.5cm} | p{2.5cm} |}
	\hline
	\textbf{Metrica} & \textbf{Valore/range obiettivo} & \textbf{Valore/range accettabile} & \textbf{Valore/range effettivo} \\
	\hline
	percentuale test unità eseguiti & $100$ & $90-100$ & $97$\\
	\hline
\end{tabular}
\end{center}
	
	L'implementazione dei test di validazione, integrazione e sistema avverrà nel corso della fase di revisione di accettazione.
	
	
\subsection{Revisione di accettazione}
		\subsubsection{Verifica Software Documentation Management Process}
		Le attività di verifica svolte sui documenti prodotti durante il periodo precedente la revisione di accettazione sono state di due tipi:
		\begin{itemize}		
			\item attività di verifica manuali;
			\item attività di verifica automatizzate.
		\end{itemize}
		
		Le prime sono state svolte dai verificatori assegnati ad ogni documento utilizzando come guida le liste di controllo stilate durante le precedenti attività di verifica con walkthrough.

		Le attività di verifica automatizzate, invece, sono state svolte utilizzando gli strumenti indicati nel documento \NdP. 
		
		I risultati ottenuti sono elencati nella seguente tabella:
		\begin{table}[H]
		\begin{tabular}{|l|l|l|}
		\hline
		\textbf{Documento} 		&\textbf{Valutazione} &\textbf{~~~~~~Esito~~~~~~} \\
		\hline
		\PdQ 					&75		&~~~~~~Superato~~~~~~\\
		\NdP 					&70		&~~~~~~Superato~~~~~~\\
		\AdR 					&77		&~~~~~~Superato~~~~~~\\
		\PdP 					&64		&~~~~~~Superato~~~~~~\\
		\ST 					&77		&~~~~~~Superato~~~~~~\\
		\DP 					&72		&~~~~~~Superato~~~~~~\\	
		\Glossario 				&51		&~~~~~~Superato~~~~~~\\
		\textit{VI_17-05-01.pdf} 		&66	&~~~~~~Superato~~~~~~\\
		\textit{VE_17-05-04.pdf} 		&68		&~~~~~~Superato~~~~~~\\
			\hline
		\end{tabular}
		\caption{Esiti del calcolo dell'indice di Gulpease dei documenti consegnati}
		\end{table}
		
		\subsubsection{Verifica Project Planning Process \& Process Assessment and Control Process}
		Per verificare il soddisfacimento degli obiettivi di qualità definiti per tale processo, a partire dal consultivo redatto nel \PdP{} sono stati calcolati gli indici di budget variance e schedule variance per ognuna delle attività previste.
		
		I risultati ottenuti sono elencati nella seguente tabella:	
		\begin{table}[H]
		\begin{tabular}{|l|l|l|}
		\hline
		\textbf{Documento} 		&\textbf{Schedule variance} &\textbf{Budget variance} 		\\
		\hline
		\PdQ 					&0\%		&0\%\\
		\NdP 					&0\%		&0\%\\
		\AdR 					&0\%		&0\%\\
		\PdP 					&0\%		&0\%\\
		\ST						&0\%		&0\%\\
		\DP						&0\%		&0\%\\
		\Glossario 				&0\%		&0\%\\
		\textit{VI_17-05-04.pdf} 		&0\%		&0\%\\
		\textit{VE_17-05-01.pdf} 		&0\%		&0\%\\
		\hline
		\end{tabular}
		\caption{Esiti del calcolo degli indici di schedule e budget variance}
		\end{table}

		\subsubsection{Verifica Software Architectural Design Process}
		La verifica del rispetto degli obiettivi stabiliti per tale processo è stata effettuata calcolando una serie di metriche specificate nel documento \NdP. Queste metriche sono d'aiuto alla comprensione del grado di completezza, correttezza e stile dell'architettura software prodotta.
		
		I risultati ottenuti sono elencati nella seguente tabella:
		\begin{table}[H]
		\begin{tabular}{|l|p{2.25cm}|p{2.25cm}|p{2.25cm}|}
		\hline
		\textbf{Metrica} & \textbf{Valore/range accettabile} & \textbf{Valore/range obiettivo} & \textbf{Valore/range effettivo} \\
		\hline
		Completezza & ~ & ~ & ~ \\
		Numero di violazioni di alta importanza &0 & 0 &0 \\
		Numero di violazioni di media importanza &0-5 & 0 & 0 \\
		Numero di violazioni di bassa importanza &0-10 & 0 & 0 \\
		\hline
		Correttezza & ~ & ~ & ~ \\
		Numero di violazioni di alta importanza &0 & 0 & 0 \\
		Numero di violazioni di media importanza &0-5 & 0 & 0 \\
		Numero di violazioni di bassa importanza &0-10 & 0 & 0 \\
		\hline
		Stile & ~ & ~ & ~ \\
		Numero di violazioni di alta importanza &0 & 0 & 0 \\
		Numero di violazioni di media importanza &0-5 & 0 & 0 \\
		Numero di violazioni di bassa importanza &0-10 & 0 & 0 \\
		\hline
		\end{tabular}
		\caption{Esiti del calcolo delle metriche relative alla progettazione}
		\end{table}
		
		\subsubsection{Verifica della qualità dell'architettura}
		Per verificare l'osservanza degli obiettivi riguardanti la qualità dell'architettura prodotta, sono stati calcolati i valori delle metriche relative ai diagrammi dei package.
		
		I risultati ottenuti sono elencati nella seguente tabella:
		\begin{longtable}{|p{5.5cm}|p{2.25cm}|p{2.25cm}|p{2.25cm}|}
		\hline
		\textbf{Metrica} 		 					&\textbf{Valore/range accettabile}	&\textbf{Valore/range obiettivo}	&\textbf{Valore/range effettivo}\\
		\hline
		\textbf{Distanza dalla sequenza principale normalizzata} & & &\\
		client.model.utility &0.0 - 0.7 &0.0 - 0.5 &0.5\\
		client.model.celltypes &0.0 - 0.7 &0.0 - 0.5 &0.45\\
		client.model &0.0 - 0.7 &0.0 - 0.5 &0.55\\
		client.view &0.0 - 0.7 &0.0 - 0.5 &0.67\\
		server.controller &0.0 - 0.7 &0.0 - 0.5 &0.39\\
		server.parser  &0.0 - 0.7 &0.0 - 0.5 &0.09\\
		server.generator.java  &0.0 - 0.7 &0.0 - 0.5 &0.25\\
		server.generator &0.0 - 0.7 &0.0 - 0.5 &0.21\\
		server.stereotype  &0.0 - 0.7 &0.0 - 0.5 &0.33\\
		server.project  &0.0 - 0.7 &0.0 - 0.5 &0.11\\
		server.template.java  &0.0 - 0.7 &0.0 - 0.5 &0.33\\
		server.template  &0.0 - 0.7 &0.0 - 0.5 &0.32\\
		server.compiler.java  &0.0 - 0.7 &0.0 - 0.5 &0.33\\
		server.compiler  &0.0 - 0.7 &0.0 - 0.5 &0.30\\
		server.utility  &0.0 - 0.7 &0.0 - 0.5 &0.5\\
		\hline
		\textbf{Grado di accoppiamento afferente per package} & & &\\
		client.model.utility &0-7 &0-39 &5\\
		client.model.celltypes &0-7 &0-39 &4\\
		client.model &0-7 &0-39 &2\\
		client.view &0-7 &0-39 &4\\
		server.controller &0-7 &0-39 &4\\
		server.parser  &0-7 &0-39 &1\\
		server.generator.java  &0-7 &0-39 &1\\
		server.generator &0-7 &0-39 &2\\
		server.stereotype  &0-7 &0-39 &1\\
		server.project  &0-7 &0-39 &4\\
		server.template.java  &0-7 &0-39 &1\\
		server.template  &0-7 &0-39 &14\\
		server.compiler.java  &0-7 &0-39 &1\\
		server.compiler  &0-7 &0-39 &1\\
		server.utility  &0-7 &0-39 &1\\
		\hline
		\textbf{Grado di accoppiamento efferente per package} & & &\\
		client.model.utility &0-6 &0-16 &4\\
		client.model.celltypes &0-6 &0-16 &3\\
		client.model &0-6 &0-16 &5\\
		client.view &0-6 &0-16 &2\\
		server.controller &0-6 &0-16 &6\\
		server.parser &0-6 &0-16 &10\\
		server.generator.java &0-6 &0-16 &3\\
		server.generator &0-6 &0-16 &5\\
		server.stereotype &0-6 &0-16 &2\\
		server.project &0-6 &0-16 &10\\
		server.template.java &0-6 &0-16 &2\\
		server.template &0-6 &0-16 &3\\
		server.compiler.java &0-6 &0-16 &2\\
		server.compiler &0-6 &0-16 &4\\
		server.utility &0-6 &0-16 &1\\
		\hline
		\textbf{Instabilità} & & &\\
		client.model.utility &0.0-0.3,0.7-1.0 &0.0-0.4,0.6-1.0 &0.5\\
		client.model.celltypes &0.0-0.3,0.7-1.0 &0.0-0.4,0.6-1.0 &0.33\\
		client.model &0.0-0.3,0.7-1.0 &0.0-0.4,0.6-1.0 &0.22\\
		client.view &0.0-0.3,0.7-1.0 &0.0-0.4,0.6-1.0 &1\\
		server.controller &0.0-0.3,0.7-1.0 &0.0-0.4,0.6-1.0 &0.6\\
		server.parser &0.0-0.3,0.7-1.0 &0.0-0.4,0.6-1.0 &0.9\\
		server.generator.java &0.0-0.3,0.7-1.0 &0.0-0.4,0.6-1.0 &0.75\\
		server.generator &0.0-0.3,0.7-1.0 &0.0-0.4,0.6-1.0 &0.71\\
		server.stereotype &0.0-0.3,0.7-1.0 &0.0-0.4,0.6-1.0 &0.66\\
		server.project &0.0-0.3,0.7-1.0 &0.0-0.4,0.6-1.0 &0.71\\
		server.template.java &0.0-0.3,0.7-1.0 &0.0-0.4,0.6-1.0 &0.66\\
		server.template &0.0-0.3,0.7-1.0 &0.0-0.4,0.6-1.0 &0.17\\
		server.compiler.java &0.0-0.3,0.7-1.0 &0.0-0.4,0.6-1.0 &0.66\\
		server.compiler &0.0-0.3,0.7-1.0 &0.0-0.4,0.6-1.0 &0.8\\
		server.utility &0.0-0.3,0.7-1.0 &0.0-0.4,0.6-1.0 &0.5\\
		\hline
		\textbf{Numero di campi dati per classe} & & &\\
		Classi client &0-10 &0-20 &min0 max7*\\
		Classi server &0-10 &0-20 &min0 max7*\\
		\hline
		\textbf{Numero di figli diretti} & & &\\
		Classi client &0 - 4 &0 - 2 &min0 max7* **\\
		Classi server &0 - 4 &0 - 2 &min0 max8* **\\
		\hline
		\textbf{Numero di metodi per classe} & & &\\
		Classi client &0-10 &0-20 &min0 max11*\\
		Classi server &0-10 &0-20 &min1 max10*\\
		\hline
		\textbf{Numero di tipi per package} & & &\\
		client.model.utility &0-20 &0-30 &5\\
		client.model.celltypes &0-20 &0-30 &9\\
		client.model &0-20 &0-30 &31***\\
		client.view &0-20 &0-30 &4\\
		server.controller &0-20 &0-30 &1\\
		server.parser &0-20 &0-30 &1\\
		server.generator.java &0-20 &0-30 &1\\
		server.generator &0-20 &0-30 &2\\
		server.stereotype &0-20 &0-30 &1\\
		server.project &0-20 &0-30 &16\\
		server.template.java &0-20 &0-30 &1\\
		server.template &0-20 &0-30 &2\\
		server.compiler.java &0-20 &0-30 &1\\
		server.compiler &0-20 &0-30 &2\\
		server.utility &0-20 &0-30 &1\\
		\hline
		\textbf{Profondità della gerarchia} & & &\\
		Classi client &0 - 4 &1 - 2 &min 0 max 2*\\
		Classi server &0 - 4 &1 - 2 &min 0 max 1*\\
		\hline
		\textbf{Profondità della gerarchia} & & &\\
		Classi client &0-4 &0 &min0 max2*\\
		Classi server &0-4 &0 &min0 max1*\\
		\hline
		
		\caption{Esiti del calcolo delle metriche relative all'architettura}
		\end{longtable}
	* abbiamo deciso di specificare solamente i valori massimi e minimi della metrica per evitare di creare tabelle troppo prolisse e poco leggibili, a fronte di un contenuto equivalente a quello proposto.
	
	** le uniche due classi che hanno un numero di figli diretti al di fuori del range d'accettazione sono le classi ActivityDiagramElement e ParsedInstruction.
	
	*** l'unico package contenente un numero di tipi al di fuori del range di accettazione è il package client.model.
	
		\subsubsection{Verifica della qualità del software}
		Per verificare l'osservanza degli obiettivi riguardanti la qualità del software prodotto, sono stati calcolati i valori delle metriche relative al linguaggio Java utilizzato nel back-end.
		
		I risultati ottenuti sono elencati nella seguente tabella:
		\begin{longtable}{|p{5.5cm}|p{2.25cm}|p{2.25cm}|p{2.25cm}|}
		\hline
		\textbf{Metrica} &\textbf{Valore/range accettabile}	&\textbf{Valore/range obiettivo}	&\textbf{Valore/range effettivo}\\
		\hline
		\textbf{Complessita ciclomatica per metodo} &0-8 &0-10 &min1 max20***\\
		\hline
		\textbf{Grado di accoppiamento afferente per package} & & &\\
		server.controller &0-7 &0-39 &4\\
		server.parser  &0-7 &0-39 &1\\
		server.generator.java  &0-7 &0-39 &1\\
		server.generator &0-7 &0-39 &2\\
		server.stereotype  &0-7 &0-39 &1\\
		server.project  &0-7 &0-39 &4\\
		server.template.java  &0-7 &0-39 &1\\
		server.template  &0-7 &0-39 &14\\
		server.compiler.java  &0-7 &0-39 &1\\
		server.compiler  &0-7 &0-39 &1\\
		server.utility  &0-7 &0-39 &1\\
		\hline
		\textbf{Grado di accoppiamento efferente per package} & & &\\
		server.controller &0-6 &0-16 &6\\
		server.parser &0-6 &0-16 &10\\
		server.generator.java &0-6 &0-16 &3\\
		server.generator &0-6 &0-16 &5\\
		server.stereotype &0-6 &0-16 &2\\
		server.project &0-6 &0-16 &10\\
		server.template.java &0-6 &0-16 &2\\
		server.template &0-6 &0-16 &3\\
		server.compiler.java &0-6 &0-16 &2\\
		server.compiler &0-6 &0-16 &4\\
		server.utility &0-6 &0-16 &1\\
		\hline
		\textbf{Instabilità} & & &\\
		server.controller &0.0-0.3,0.7-1.0 &0.0-0.4,0.6-1.0 &1\\
		server.parser &0.0-0.3,0.7-1.0 &0.0-0.4,0.6-1.0 &1\\
		server.generator.java &0.0-0.3,0.7-1.0 &0.0-0.4,0.6-1.0 &0.25\\
		server.generator &0.0-0.3,0.7-1.0 &0.0-0.4,0.6-1.0 &0.20\\
		server.stereotype &0.0-0.3,0.7-1.0 &0.0-0.4,0.6-1.0 &1\\
		server.project &0.0-0.3,0.7-1.0 &0.0-0.4,0.6-1.0 &0.43\\
		server.template.java &0.0-0.3,0.7-1.0 &0.0-0.4,0.6-1.0 &0.07\\
		server.template &0.0-0.3,0.7-1.0 &0.0-0.4,0.6-1.0 &1\\
		server.compiler.java &0.0-0.3,0.7-1.0 &0.0-0.4,0.6-1.0 &0.33\\
		server.compiler &0.0-0.3,0.7-1.0 &0.0-0.4,0.6-1.0 &0.20\\
		server.utility &0.0-0.3,0.7-1.0 &0.0-0.4,0.6-1.0 &0.33\\
		\hline
		\textbf{Numero di campi dati per classe} &0-10 &0-20 &min1 max7*\\
		\hline
		\textbf{Numero di figli diretti} &0 - 4 &0 - 2 &min0 max8* **\\
		\hline
		\textbf{Numero di metodi per classe} &0-10 &0-20 &min1 max11*\\
		\hline
		\textbf{Numero di tipi per package} & & &\\
		server.controller &0-20 &0-30 &1\\
		server.parser &0-20 &0-30 &1\\
		server.generator.java &0-20 &0-30 &1\\
		server.generator &0-20 &0-30 &1\\
		server.stereotype &0-20 &0-30 &1\\
		server.project &0-20 &0-30 &16\\
		server.template.java &0-20 &0-30 &1\\
		server.template &0-20 &0-30 &1\\
		server.compiler.java &0-20 &0-30 &2\\
		server.compiler &0-20 &0-30 &1\\
		server.utility &0-20 &0-30 &1\\
		\hline
		\textbf{Profondità della gerarchia} &0 - 4 &1 - 2 &min1 max4*\\
		&0 &0 &0\\
		\hline
		\textbf{Numero di linee di codice per metodo} &0-20 &0-40 &min1 max66* ****\\
		\hline
		\textbf{Numero di parametri per metodo} &0-4 &0-8 &min1 max6*\\
		\hline
		\textbf{Percentuale linee di commento su linee di codice} &20-40 &20-40 &23\\
		\hline
		\textbf{Profondità annidamento blocchi} &0-4 &0-8 &min0 max6*\\
		\hline
		\textbf{Copertura requisiti obbligatori} &100 &100 &0\\
		\hline
		\textbf{Copertura requisiti desiderabili} &80-100 &50-100 &0\\
		\hline
		\textbf{Percentuale test superati} &100 &90-100 &100\\
		\hline
		\textbf{Validazione W3C} &0 &0-20 &1\\
		\hline
		
		\caption{Esiti del calcolo delle metriche relative al codice}
		\end{longtable}
	* abbiamo deciso di specificare solamente i valori massimi e minimi della metrica per evitare di creare tabelle troppo prolisse e poco leggibili, a fronte di un contenuto equivalente a quello proposto.
	
	** l'unica classe che ha un numero di figli diretti al di fuori del range d'accettazione è la classe e ParsedInstruction.
	
	*** i metodi che presentano una complessità cicolomatica al di fuori del valore accettabile sono relBuilder, typeBuilder, recursiveBuilder e createActivity.
	
	**** la classe che presenta metodi con una lunghezza superiore a quella accettabile è la classe Parser.
	
	
	\subsection{System Qualification Testing Process \& Software Qualification Testing Process}
		Per verificare l'osservanza degli obiettivi di qualità riguardanti la produzione di unità software eseguibili, sono stati calcolati i valori delle seguenti metriche.
		
	\begin{center}
\begin{tabular}{| p{6cm} | p{2.5cm} | p{2.5cm} | p{2.5cm} |}
	\hline
	\textbf{Metrica} & \textbf{Valore/range obiettivo} & \textbf{Valore/range accettabile} & \textbf{Valore/range effettivo} \\
	\hline
	percentuale test unità eseguiti & $100$ & $90-100$ & $100$\\
	\hline
	percentuale test integrazione eseguiti & $100$ & $90-100$ & $100$\\
	\hline
	percentuale test sistema eseguiti & $100$ & $90-100$ & $100$\\
	\hline
	percentuale test validazione eseguiti & $100$ & $90-100$ & $100$\\
	\hline
\end{tabular}
\end{center}
	
	



%%%%%%%%%%%%%%%%%%%%%
%%  Standard adottati
%%%%%%%%%%%%%%%%%%%%%

\section{Standard adottati}
	\subsection{Qualità di processo - SPICE}
	Lo standard prevede sei diversi livelli di maturità (o capacità), e 9 attributi di processo che, se posseduti, lo portano ad un certo livello di capacità.
	\begin{itemize}
	\item \textbf{Livello 0: Incompleto} il processo non è implementato o non raggiunge il suo obiettivo; non esiste evidenza di esecuzione sistematica delle attività che lo compongono.
	 \item \textbf{Livello 1: Attuato} il processo è implementato e raggiunge il suo obiettivo; esiste evidenza di semplice attuazione delle attività che lo compongono.
	Il relativo attributo di processo:
		\begin{itemize}
			\item \emph{Esecuzione del Processo}: la misura in cui il processo raggiunge i propri obiettivi trasformando   prodotti in ingresso identificabili in prodotti in uscita identificabili.
		\end{itemize}
	\item \textbf{Livello 2: Gestito} il processo è gestito e i suoi prodotti sono stabiliti, controllati e manutenuti; le attività sono pianificate e 			controllate e il loro svolgimento risulta documentato.
	I relativi attributi di processo:
		\begin{itemize}
			\item \emph{Gestione delle Prestazioni}: la misura in cui il processo produce un risultato coerente con gli obiettivi attesi.
			\item \emph{Gestione dei Prodotti}: la misura in cui il processo viene gestito per elaborare prodotti documentati, controllati e verificati in modo appropriato.
		\end{itemize}
	
	\item \textbf{Livello 3: Definito} il processo viene eseguito in base ai principi dell'ingegneria del software. Le procedure sono definite e 				adattate ai progetti e ruoli, competenze e responsabilità sono definiti e controllati.
	I relativi attributi di processo:
		\begin{itemize}
			\item \emph{Definizione del Processo}: la misura in cui il processo raggiunge i risultati attesi aderendo ad un particolare standard di processo.
			\item \emph{Utilizzo del Processo}: la misura in cui il processo attinge alle risorse allocate per la sua esecuzione.	
		\end{itemize}
		
	\item \textbf{Livello 4: Predicibile} il processo è messo in atto costantemente entro limiti definiti. Le attività che lo compongono, la loro gestione e i relativi risultati sono controllati quantitativamente.
	I relativi attributi di processo:
		\begin{itemize}
			\item \emph{Misurazione del Processo}: la misura in cui il processo utilizza i risultati raggiunti e le misure ricavate durante l'esecuzione per garantire il raggiungimento dei traguardi definiti.
			\item \emph{Controllo del Processo}: la misura in cui il processo viene controllato tramite la raccolta, l'analisi e la messa in uso di misurazioni di prodotto e processo allo scopo di correggere, ove necessario, la sua esecuzione per raggiungere i risultati attesi.	
		\end{itemize}

	\item \textbf{Livello 5: Ottimizzante} il processo è continuativamente migliorato per soddisfare i rilevanti traguardi di business attuali e 				previsti. I cambiamenti del processo sono valutati e lo studio per il miglioramento è un'attività costante.
	I relativi attributi di processo:
		\begin{itemize}
			\item \emph{Innovazione del Processo}: la misura in cui cambiamenti relativi alla definizione, alla gestione e all'esecuzione del processo sono controllati per raggiungere gli obiettivi di business dell'organizzazione.
			\item \emph{Ottimizzazione del Processo}: la misura in cui vengono identificati e implementati cambiamenti relativamente all'esecuzione del processo in modo tale da assicurare un miglioramento continuo nel raggiungimento degli obiettivi rilevanti dell'organizzazione.
		\end{itemize}
	\end{itemize}

Ogni attributo di processo è misurabile e sono definiti dallo standard 4 diversi gradi di possesso:
\begin{itemize}
	\item \textbf{N}: non posseduto (0\% - 15\%);
	\item \textbf{P}: parzialmente posseduto (16\% - 50\%);
	\item \textbf{L}: largamente posseduto (51\% - 85\%);
	\item \textbf{F}: completamente posseduto (86\% - 100\%).
\end{itemize}

Le misurazioni e le valutazioni risultanti dal monitoraggio dei diversi processi sono usate nel contesto di una strategia di miglioramento continuo della qualità, realizzata attraverso il ciclo PDCA.
\\Il ciclo PDCA, altresì noto come ciclo di Deming, definisce un'organizzazione interna dei processi incentrata sul principio del miglioramento continuo allo scopo di renderli automigliorativi.
\\I passi in cui esso si articola sono quattro:
	\begin{itemize}
		\item \textbf{Pianificare (Plan)}: vengono definite attività, scadenze, responsabilità, risorse utili a raggiungere specifici obiettivi di miglioramento opportunamente pianificati.
		\item \textbf{Eseguire (Do)}: vengono attuate le azioni migliorative pianificate al passo precedente. Si procede inoltre ad eseguire misurazioni e raccogliere dati utili per i successivi passi di analisi e controllo.
		\item \textbf{Valutare (Check)}: l'esito delle azioni di miglioramento viene confrontato rispetto alle attese e agli obiettivi pianificati.
		\item \textbf{Agire (Act)}: se l'esito delle valutazioni effettuate al passo precedente risulta positivo, i cambiamenti introdotti nell'esecuzione del processo vengono incorporati stabilmente in esso e standardizzati.
	\end{itemize}
	\subsection{Qualità di prodotto - ISO/IEC:9126}
	Lo Standard ISO/IEC 9126:2001 si articola in quattro parti:
	\begin{enumerate}
		\item Modello della qualità del software (9126-1);
		\item Metriche per la qualità esterna (9126-2);
		\item Metriche per la qualità interna (9126-3);
		\item Metriche per la qualità in uso (9126-4).
	\end{enumerate}
	Lo standard analizza la qualità del software sotto tre diversi punti di vista:
	\begin{itemize}
		\item \textbf{Qualità interna}: è la qualità del prodotto software che fa riferimento alle caratteristiche implementative del software come l'architettura e il codice derivante da quest'ultima.
		\item \textbf{Qualità esterna}: è la qualità del prodotto software relativa a quando esso viene eseguito e testato in un ambiente di prova. 
		\item \textbf{Qualità in uso}: è la qualità del prodotto software dal lato di chi utilizza tale prodotto all'interno di uno specifico sistema.
	\end{itemize}
		\subsubsection{Modello della qualità del software}
		Nella prima parte dello standard vengono presentati i modelli per la qualità esterna, interna ed in uso.
			\paragraph{Modello della qualità esterna ed interna}
			Il modello di qualità esterna ed interna sancito nella prima parte dello standard è suddiviso nelle seguenti sei caratteristiche generali misurabili attraverso delle metriche:
			\begin{itemize}
				\item \textbf{funzionalità}: è la capacità del software di fornire le funzioni che soddisfano determinate esigenze, necessarie per operare in determinate condizioni. 
				\item \textbf{affidabilità}: rappresenta la capacità del prodotto software di mantenere uno specifico livello di prestazioni quando viene usato in certe condizioni e per un periodo di tempo determinato.
				\item \textbf{usabilità}: è la capacità di un prodotto software di essere facilmente comprensibile e attraente in ogni sua parte per un utente qualsiasi. Un software è considerato usabile proporzionalmente alla facilità con cui un utente opera per sfruttare al massimo le funzionalità che il software mette a disposizione.
				\item \textbf{efficienza}: è la capacità di un prodotto di eseguire le funzioni richieste nel minor tempo possibile ed utilizzando le risorse necessarie nel modo migliore.
				\item \textbf{manutenibilità}: rappresenta la capacità di un prodotto software di essere modificato in tempi rapidi e a costi accessibili. Le modifiche possono riguardare correzioni o adattamenti del prodotto a variazioni negli ambienti, nei requisiti e nelle specifiche funzionali.
				\item \textbf{portabilità}: è la capacità di un prodotto software di poter essere spostato da un ambiente all'altro velocemente. L'ambiente include sia aspetti hardware che software.
			\end{itemize}

			\paragraph{Modello della qualità in uso}
			Gli attributi presenti nel modello relativo alla qualità del software in uso sono rappresentati da quattro grandi categorie:
			\begin{itemize}
				\item \textbf{efficacia}: è la capacità di consentire all'utente di raggiungere obiettivi specifici con precisione e completezza.
				\item \textbf{produttività}: rappresenta la capacità di permettere all'utente di utilizzare un numero stabilito di risorse, in relazione all'efficienza raggiunta in uno specifico contesto di utilizzo.
				\item \textbf{sicurezza fisica}: è la capacità di raggiungere un livello accettabile di rischio di danni a dati, persone, proprietà o ambienti.
				\item \textbf{soddisfazione}: rappresenta la capacità di soddisfare gli utenti.
			\end{itemize}
		\subsubsection{Metriche per la qualità del software}
		Nelle restanti tre parti vengono trattate le metriche per la qualità esterna, interna e in uso.
			\paragraph{Metriche per la qualità esterna}
			Le metriche esterne misurano i comportamenti del software che si possono rilevare dai test, dall'operatività e dall'osservazione durante la sua esecuzione sulla base degli obiettivi stabiliti. Le metriche esterne sono scelte in base alle caratteristiche che il prodotto finale dovrà dimostrare una volta utilizzato.
			\paragraph{Metriche per la qualità interna}
			Le metriche interne si applicano al software non eseguibile (un esempio è il codice sorgente) e alla documentazione. Le misure effettuate permettono di prevedere il livello di qualità esterna ed in uso del prodotto finale poiché gli attributi interni influenzano le caratteristiche esterne e quelle in uso.
			\paragraph{Metriche per la qualità in uso}
			Le metriche della qualità in uso valutano il livello con cui il software consente agli utenti di svolgere le proprie attività con efficacia, produttività, sicurezza e soddisfazione nel contesto operativo previsto.
	
\end{document}
