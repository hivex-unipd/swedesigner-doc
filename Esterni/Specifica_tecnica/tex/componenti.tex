%%%%%%%%%%%%%%
%%  Componenti
%%%%%%%%%%%%%%



% [Breve introduzione...]



\subsection{Client}

\subsubsection{MainApp}
% ...

\subsubsection{AppView}
% ...

% eccetera...




\subsection{Server}
% server==package principale
%Controller== classe che fa da Front Controller
%generator== package dentro server che fa il lavoro di creare il zip
%Project==classe (forse interface o abstract) dentro generator che fa il lavoro di creare il zip
\subsubsection{server}
--immagine package server con la classe controller che comunica con la classe Project contenuta nel package generator--
\begin{description}
\item[Descrizione] è il package che racchiude tutti i componenti del back-end che si occupano di soddisfare le richieste provenienti dal front-end ed elaborarle;
\item[Package contenuti] 
	\begin{itemize}
	\item server::endpoints (in teoria nn c'è);
	\item server::generator;
	\end{itemize}
\item[Interazione con altri componenti] interagisce con il client definito nella sez xxx tramite i servizi REST offerti. (in teoria non c'è)
\end{description}

\paragraph{Classi}
\subparagraph{server::Controller}
\begin{description}
\item[Descrizione] è la classe che implementa il pattern Front Controller e si occupa di raccogliere le richieste dal client (anche Singleton??)
\item[Utilizzo] viene invocata ""dal client"" e trasferisce la richiesta alla classe Project contenuta nel package server::generator che si occuperà di invocare il relativo comando per soddisfare la richiesta; 
\item[Relazioni con altre classi] 
	\begin{itemize}
	\item server::generator::Project.
	\end{itemize}
\end{description}

\subsubsection{server::generator}
--immagine--
\begin{description}
\item[Descrizione] è il package che contiene tutti i componenti che definiscono la logica necessaria per soddisfare le richieste il arrivo dal client;
\item[Padre] server; 
\item[Package contenuti]
	\begin{itemize}
	...
	\end{itemize}
\item[Interazione con altri componenti] 
	\begin{itemize}
	\item server::Controller.
	\end{itemize}
\end{description}
