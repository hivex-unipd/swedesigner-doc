%%%%%%%%%%%%%%%%%%%%%%%%%%%%%%
%%  Design pattern creazionali
%%%%%%%%%%%%%%%%%%%%%%%%%%%%%%

\subsection{Dependency injection}

\begin{figure}[H] \label{fig:injector}
	\includegraphics[scale=0.8]{img/injector.png}
	\caption{Dependency Injection.}
\end{figure}

\subsubsection{Scopo} Permette di disaccoppiare il comportamento di una componente dalla risoluzione delle sue dipendenze, semplificando perciò lo sviluppo di un software di grandi dimensioni e allo stesso tempo migliorandone la testabilità.

\subsubsection{Motivazione} Grazie a questo pattern è possibile separare la responsabilità di uso e creazione di un oggetto. Il componente dipendente dovrà solo sapere come usare un servizio richiesto, mentre il compito di creare ed iniettare quest'ultimo spetta ad un injector. Questo permette al componente dipendente di essere altamente configurabile in quanto è fisso solo il suo comportamento.

\subsubsection{Struttura} Sono coinvolte 3 componenti nella dependency injection:
\begin{enumerate}
	\item gli oggetti servizi(ossia un qualsiasi oggetto che potrebbe essere usato);
	\item l'oggetto dipendente da questi servizi;
	\item un injector responsabile di creare ed iniettare i servizi.
\end{enumerate}

\subsubsection{Applicabilità} È possibile applicare il pattern in tre modi differenti:
\begin{itemize}
	\item \textbf{setter injection:} la dipendenza viene iniettata tramite dei metodi setter del component dipendente;
	\item \textbf{costruction injection:} la dipendenza viene iniettata tramite un paramento del costruttore;
	\item \textbf{interface injection:} l'iniezione viene eseguita attraverso l'interfaccia che fornirà un setter a chiunque la implementa.
\end{itemize}

\subsubsection{Utilizzo}
Il pattern dependency injection viene applicato ai vari servizi del server in modo da poter richiedere un'istanza del servizio in base alla configurazione che cambia per linguaggio target di generazione.
Questi includono Parser(il quale è configurato per accettare file json) e Generator,Compiler,Template(i quali sono configurati per il linguaggio Java).
Viene implementato tramite il framework \emph{Spring}, il quale fornisce dei componenti altamente versatili(chiamati bean) che permettono di creare e risolvere le dipendenze.
Un esempio concreto é la richesta di un oggetto Generator alla quale Spring fornirà un'istanza di tipo JavaGenerator.

%\begin{figure}[H] \label{fig:injector}
%	\includegraphics[scale=0.8]{img/depExample.png}
%	\caption{Dependency Injection.}
%\end{figure}

\subsection{Command}

\begin{figure}[H] \label{fig:command}
	\includegraphics[scale=0.6]{img/command.png}
	\caption{Command.}
\end{figure}

\subsubsection{Scopo} Permette di isolare la porzione di codice che effettua un'azione dal codice che ne richiede l'esecuzione. Tale azione è incapsulata nell'oggetto Command.

\subsubsection{Struttura} Sono coinvolte le seguenti componenti:
\begin{enumerate}
	\item Client: colui che richiede il comando  ed imposta il Receiver;
	\item Invoker: colui che effettua l'invocazione del comando;
	\item Command: interfaccia generica per l'esecuzione del comando;
	\item ConcreteCommand: implementazione del comando che consente di collegare l'invoker con il Receiver;
	\item Receiver: colui che riceve il comando e sa come eseguirlo.
\end{enumerate}

\subsubsection{Applicabilità} È possibile applicare il pattern per:
\begin{itemize}
	\item parametrizzare gli oggetti sull'azione da eseguire;
	\item specificare,accodare ed eseguire richieste molteplici volte;
	\item supportare operazioni di Undo e Redo;
	\item supportare la transazione: un comando equivale ad un'operazione atomica.
\end{itemize}
\subsubsection{Utiizzo}
Il pattern command viene sfruttato dal client per chiamare funzioni come il salvataggio del progetto e la generazione del codice. È implementato seguendo le linee guida descritte da Addy Osmani nel libro \emph{Learning JavaScript Design Patterns} nel capitolo \emph{JavaScript Design Patterns-Command Pattern}. Il \texttt{ProjectCommand} è un oggetto javascript il quale contiene una funzione per ogni comando. Inoltre offre il metodo execute che permette di incapsulare le invocazioni dei metodi e permette la parametrizzazione di essi. In questo modo è possibile avere degli ggetti disccopiati e lascia libertà di cambiare l'imlementazione di ProjectCommand non avendo classi che chiama direttamente una funzione di \texttt{ProjectCommand}.
\begin{figure}[H] \label{fig:command}
	\includegraphics[scale=0.6]{img/commandExample.png}
	\caption{Command.}
\end{figure}

\subsection{Factory}

\begin{figure}[H] \label{fig:factory}
	\includegraphics[scale=0.8]{img/factory2.png}
	\caption{Factory.}
\end{figure}

\subsubsection{Scopo} Indirizza il problema della creazione di oggetti senza specificarne la classe esatta fornendo un'interfaccia per creare l'oggetto, ma lascia che le sottoclassidecidano quale oggetto istanziare.

\subsubsection{Struttura} Sono coinvolte le seguenti componenti:
\begin{enumerate}
	\item Creator (\texttt{VehicleFactory}): dichiara la Factory che avrà il compito di ritornare l'oggetto appropriato;
	\item Product (\texttt{Vehicle}): definisce l'interfaccia dell'oggetto che deve essere creato dalla Factory;
	\item ConcreteProduct (\texttt{Truck}, \texttt{Car}): implementa l'oggetto in base ai metodi  definiti dall'interfaccia Product.
\end{enumerate}

\subsubsection{Applicabilità} È possibile applicare il pattern quando:
\begin{itemize}
	\item la creazione di un oggetto preclude il suo riuso senza una duplicazione di codice;
	\item la creazione di un oggetto richide l'accesso ad informazioni o risorse che non dovrebbero essere contenute nella classe di composizione;
	\item la gestione del ciclo di vita degli oggetti gestiti deve essere centralizzata in modo da assicurare un comportamento coerente all'interno dell'applicazione.
\end{itemize}

\subsubsection{Utilizzo}
Il pattern viene utilizzato dal client per richiedere un'istanza di una nuova cella da aggiungere al diagramma corrente. L'oggetto \texttt{NewCellFactory} permette di registrare i vari tipi di celle e di richiedere un istanza del tipo desiderato descritto da una stringa. La richiesta delle celle viene effettuata da \texttt{NewCellModel} in base alla scelta dell'utente.
L'imlpementazione di questo pattern segue le linee guida descritte da Addy Osmani nel libro \emph{Learning JavaScript Design Patterns} nel capitolo \emph{JavaScript Design Patterns-Factory Pattern}.
\begin{figure}[H] \label{fig:factory}
	\includegraphics[scale=0.8]{img/factoryExample.png}
	\caption{Factory.}
\end{figure}

\subsection{Observer}

\begin{figure}[H] \label{fig:observer}
	\includegraphics[scale=0.6]{img/observer.png}
	\caption{Observer.}
\end{figure}

\subsubsection{Scopo} Questo pattern viene utilizzato quando si vuole realizzare una dipendenza uno-a-molti in cui il cambiamento di stato di un soggetto venga notiifcato a tutui i soggetti che si sono mostrati interessati.

\subsubsection{Struttura} Sono coinvolte le seguenti componenti:
\begin{enumerate}
	\item Subject: espone l’interfaccia che consente agli osservatori di iscriversi e cancellarsi mantenendi una reference a tutti gli osservatori iscritti;
	\item Observer: espone l’interfaccia che consente di aggiornare gli osservatori in caso di cambio di stato del soggetto osservato;
	\item ConcreteSubject: mantiene lo stato del soggetto osservato e notifica gli osservatori in caso di un cambio di stato;
	\item ConcreteObserver: implementa l’interfaccia dell’Observer definendo il comportamento in caso di cambio di stato del soggetto osservato.
\end{enumerate}

\subsubsection{Applicabilità} È possibile applicare il pattern quando:
\begin{itemize}
	\item in un problema ci sono due aspetti tra loro dipendenti, che possono essere rappresentati come classi che possono essere usati indipendentemente tra loro;
	\item quando il cambiamento di un oggetto provoca un cambiamento in un altro oggetto;
	\item quando un oggetto ha la necessità di comunicare con altri oggetti, senza fare assunzioni sugli altri oggetti.
\end{itemize}

\subsubsection{Utilizzo}
Il pattern Observer è utilizzato per permettere alle classi di rimanere in ascolto di eventi scatenati da altre classi e di conseguenza richiamare i propri metodi.
L'implementazione di questo pattern è fornita da \emph{Backbone} che mette a dispozione di chi estende modelli e viste di esso i metodi \emph{trigger} e \emph{listenTo}. Oltre all'utilizzo base di ogni view che resta in ascolto di cambiamenti dei valori del proprio model e quindi l'esecuzione del metodo render di esse viene sfruttato anche ad esempio da \texttt{ProjectView} che notifica \texttt{DetailsView} che l'utente ha selzionato una cella del diagramma, permettendo quindi a \texttt{DetailsView} di visualizzare all'utente le proprietà specifiche di quella cella.
\begin{figure}[H] \label{fig:observer}
	\includegraphics[scale=0.6]{img/observerExample.png}
	\caption{Observer.}
\end{figure}