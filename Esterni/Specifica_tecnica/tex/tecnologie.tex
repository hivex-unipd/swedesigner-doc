%%%%%%%%%%%%%%%%%%%%%%%%%
%%  Tecnologie utilizzate
%%%%%%%%%%%%%%%%%%%%%%%%%



\subsection{Client} \label{sec:tech_client}
Il \gloss{client} di \proj{} sarà implementato con le tecnologie web richieste dal capitolato d'appalto: HTML5, CSS3 e JavaScript. Di seguito sono elencate e descritte le librerie JavaScript di cui il client necessita. Tutte le librerie sono open-source, come richiesto dal capitolato.


\subsubsection{JointJS}
La libreria \jointjs{} (\url{www.jointjs.com/opensource}) è una libreria open-source per realizzare editor di diagrammi interattivi, in maniera altamente personalizzabile.

La libreria sfrutta le seguenti tecnologie per il suo funzionamento:
\begin{itemize}
	\item \html: \jointjs{} richiede che una pagina \html{} sia popolata con un tag \texttt{<svg>}, che conterrà un diagramma.
	\item Le seguenti librerie \js:
	\begin{itemize}
		\item \jquery;
		\item \lodash;
		\item \backbonejs.
	\end{itemize}
\end{itemize}

L'appendice \ref{sec:app_jointjs} riporta un approfondimento su \jointjs.


\subsubsection{\backbonejs}
La libreria \backbonejs{} permette di strutturare applicazioni web single page (\gloss{SPA}) fornendo \textbf{modelli} con binding di chiave-valore, eventi, \textbf{collezioni} e \textbf{viste} con una gestione degli eventi dichiarativa. Essa offre inoltre una interfaccia \gloss{RESTful}.

La struttura di \jointjs{} \`e sviluppata tramite il modello MVC offerto da \backbonejs. Al fine di ridurre il numero di librerie necessarie allo sviluppo del progetto, si costruirà il \emph{front end} di \proj{} estendendo le funzionalità di base offerte da \jointjs{}, per avvalerci cos\`i della struttura MVC di \backbonejs.

Alcuni pro di \backbonejs{} sono i seguenti:
\begin{itemize}
	\item permette una facile implementazione lato client del pattern architetturale MVC;
	\item \`e semplice, flessibile e offre un gran numero di plug-in ed estensioni;
	\item fornisce un modello di ereditarietà semplice ma allo stesso tempo potente, basato sul metodo \texttt{extend()}.
\end{itemize}
Tra i contro, invece, citiamo:
\begin{itemize}
	\item fare test di unità sulle \emph{View} di \backbonejs{} può essere molto complicato e richiedere una gran quantità di codice di \emph{mock};
	\item creaz di tag div ridondanti nel codice HTML generato;
	\item rende difficile la comunicazione con un \emph{back end} non RESTful;
	\item mancanza di una netta separazione tra \emph{View} e \emph{Controller} nell'interpretazione del pattern MVC che esso propone.
\end{itemize}


\subsubsection{\jquery}
La libreria \jquery{} è sfruttata da \backbonejs{} (e quindi da \jointjs); semplifica varie operazioni di basso livello a livello grafico e di comunicazione asincrona con il server.

I vantaggi più rilevanti di \jquery{} sono i seguenti:
\begin{itemize}
	\item grande supporto da parte della \emph{community};
	\item permette di manipolare facilmente gli elementi del DOM;
	\item agevola la realizzazione di animazioni elementari;
	\item presenta un gran numero di plug-in.
\end{itemize}
Alcuni suoi contro sono:
\begin{itemize}
	\item lentezza nelle prestazioni, nel caso di complesse manipolazioni del DOM;
	\item un uso scorretto può portare alla creazione di grandi moli di codice difficile da manutenere.
\end{itemize} 


\subsubsection{\requirejs}
La libreria \requirejs{} è un \emph{loader} di moduli e file \js. Nel nostro prodotto, si occuperà principalmente di risolvere dipendenze delle librerie \js{} utilizzate.

I vantaggi principali sono:
\begin{itemize}
	\item esplicita le dipendenze tra i diversi moduli;
	\item \`e in grado di caricare i moduli in modo asincrono e ``su richiesta''.
\end{itemize}
Tra gli svantaggi principali, invece:
\begin{itemize}
	\item necessità di uno studio approfondito e ragionato prima di un utilizzo efficace;
	\item curva di apprendimento ripida;
	\item maggiore complessità di integrazione con i test, rispetto alla semplice inclusione di script con tag HTML. 
\end{itemize} 


\subsubsection{\lodash}
La libreria \lodash{} fornisce metodi di utilità non offerti da \js{} puro. 

Rispetto alla simile libreria \emph{Underscore}, questa fornisce più performance, più features e miglior documentazione. Essa inoltre è usata da \jointjs.



\subsection{Server}

\subsubsection{Apache Tomcat}
Il \emph{back end} della nostra applicazione è ospitato su un \gloss{server} Apache Tomcat, come richiesto dal capitolato d'appalto nel caso il \emph{back end} fosse scritto in Java.

Tomcat fornisce numerosi vantaggi:
\begin{itemize}
	\item \`e facile da installare;
	\item richiede poche impostazioni personalizzate per poter esssere pienamente operativo;
	\item comporta un consumo di memoria molto basso rispetto ad applicazioni simili.
\end{itemize}
Allo stesso tempo, manca della funzionalità che permette il recupero automatico delle funzionalità in caso di \emph{failure}.


\subsubsection{Spring Boot}
Le richieste HTTP inviate dal client al server vengono gestite tramite la libreria open-source \emph{Spring Boot} (disponibile all'indirizzo \url{spring.io}), utile per aderire ai princìpi dello stile architetturale REST.

La scelta di Spring è stata guidata dalle seguenti caratteristiche proprie di questa libreria:
\begin{itemize}
	\item \`e modulare: rende possibile integrare solo alcuni moduli all'interno del progetto;
	\item \`e integrabile: rende facile l'utilizzazione di altri framework oper source;
	\item \`e testabile: permette di scrivere software facile da testare.
\end{itemize}
Tuttavia Spring possiede alcuni difetti:
\begin{itemize}
	\item \`e pesante: l'ambiente di Spring è tutt'altro che leggero;
	\item \`e complesso: sviluppare in Spring presenta una curva di apprendimento ripida;
\end{itemize}

In particolare, abbiamo preferito la libreria derivata ``Spring Boot'' per le seguenti caratteristiche:
\begin{itemize}
	\item semplicità: l'ambiente di Spring Boot fornisce una suite preconfezionata e preconfigurata \emph{out-of-the-box}, in modo da semplificare l'attività di codifica;
	\item (relativa) leggerezza: rispetto all'ambiente di Spring, Spring Boot risulta sensibilmente più leggero.
	% \item altro?
\end{itemize}

Secondo noi l'utilizzo di questa tecnologia, rispetto a \emph{Spring}, è preferibile e l'abbiamo valutata positivamente dopo un breve periodo di prova di entrambe le tecnologie.

% Si rimanda all'appendice %\ref{sec:app_spring}
% per ulteriori considerazioni sull'uso di Spring Boot nella nostra applicazione.
Futuri riferimenti alla libreria Spring dovranno essere considerati come riferiti alla libreria Spring Boot, qualora non diversamente specificato.


\subsubsection{StringTemplate}
StringTemplate è una libreria open-source scritta da Terence Parr e disponibile all'indirizzo \url{stringtemplate.org}. Come indicato sul sito ufficiale, questa libreria è particolarmente adatta alla generazione di codice. 

Essa viene utilizzata per:
\begin{itemize}
	\item creare dei template di codice Java, inframmezzati da speciali marcatori definiti da StringTemplate;
	\item popolare tali template con i dati ricevuti dal client.
\end{itemize}
I marcatori definiti dalla libreria servono da segnaposto per ospitare i dati ottenuti analizzando il documento JSON generato dal client.

Le principali funzionalit\`a utili al progetto sono per esempio le seguenti:
\begin{itemize}
	\item espressioni semplici, espressioni condizionali;
	\item template anonimi;
	\item funzioni;
	\item valutazione lazy;
	\item auto-indentazione;
	\item iteratori;
\end{itemize}

Molto importante è notare la filosofia sfruttata dalla libreria (consultabile al seguente indirizzo: \url{https://github.com/antlr/stringtemplate4/blob/master/doc/motivation.md}): StringTemplate, non permettendo certe operazioni come assegnazioni rende questo sistema di templating privo di \emph{side-effects}. Malgrado questo può portare a rendere la soluzione finale più complessa da sviluppare, ciò porta molti vantaggi:
\begin{itemize}
	\item si favorisce il riuso di componenti;
	\item i template sono sostituibili più facilmente;
	\item i template sono più facilmente testabili;
	\item la parte del linguaggio specifico viene isolata totalmente dalla logica di gestione del template, astratta per ogni linguaggio. 
\end{itemize}
