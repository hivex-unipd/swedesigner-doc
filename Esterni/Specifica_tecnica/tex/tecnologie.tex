%%%%%%%%%%%%%%%%%%%%%%%%%
%%  Tecnologie utilizzate
%%%%%%%%%%%%%%%%%%%%%%%%%



\subsection{Client} \label{sec:tech_client}
Il \gloss{client} di \proj{} sarà implementato con le tecnologie web richieste dal capitolato d'appalto: HTML5, CSS3 e JavaScript. Di seguito sono elencate e descritte le librerie JavaScript di cui il client necessita. Tutte le librerie sono open-source, come richiesto dal capitolato.


\subsubsection{JointJS}
La libreria \jointjs{} (\url{www.jointjs.com/opensource}) è una libreria open-source per realizzare editor di diagrammi interattivi, in maniera altamente personalizzabile.

La libreria sfrutta le seguenti tecnologie per il suo funzionamento:
\begin{itemize}
	\item \html: \jointjs{} richiede che una pagina \html{} sia popolata con un tag \texttt{<svg>}, che conterrà un diagramma.
	\item Le seguenti librerie \js:
	\begin{itemize}
		\item \jquery;
		\item \lodash;
		\item \backbonejs.
	\end{itemize}
\end{itemize}

% È reperibile un \emph{plug-in} di \jointjs{} che sarà sfruttato per realizzare diagrammi UML. Tale \emph{plug-in} offre delle forme geometriche e delle frecce che ci servono per il disegno del diagramma delle classi e dei diagrammi a blocchi di ogni metodo.

L'appendice \ref{sec:app_jointjs} riporta un approfondimento su \jointjs.

\subsubsection{\backbonejs}
La libreria \backbonejs{} permette di strutturare applicazioni web single page (\gloss{SPA}) fornendo \textbf{modelli} con binding di chiave-valore, eventi, \textbf{collezioni} e \textbf{viste} con una gestione degli eventi dichiarativa. Essa offre inoltre una interfaccia \gloss{RESTful}.

A causa della struttura data alla libreria \jointjs{}, sviluppata tramite MVC, al fine di ridurre il numero di librerie necessarie allo sviluppo del progetto, si costruirà il lato client di \proj{} estendendo le funzionalità di base offerte da \jointjs{} usando il modello MVC offerto da \backbonejs.

Tra i pro di \backbonejs{}:
\begin{itemize}
	\item permette una facile implementazione lato client del pattern architetturale MVC;
	\item semplice, flessibile e con un gran numero di plug-in ed estensioni disponibili;
	\item fornisce un modello di ereditarietà semplice, ma allo stesso tempo potente basato su metodo emph{extend}.
\end{itemize}

Tra i contro di \backbonejs{}:
\begin{itemize}
	\item fare test di unità sulle Views di \backbonejs{} può essere molto complicato e richiedere una gran quantità di codice di mock;
	\item creazione di tag div ridondanti nel codice HTML generato;
	\item difficoltà nel comunicare con un back-end non RESTful;
	\item mancanza di una netta separazione tra View e Controller nell'interpretazione del pattern MVC che esso propone.
\end{itemize}

\subsubsection{\jquery}
La libreria \jquery{} è sfruttata da \jointjs{} ed è necessaria per semplificare varie operazioni di basso livello.

I vantaggi più rilevanti sono:
\begin{itemize}
	\item il grande supporto da parte della community;
	\item permette di malipolare gli elementi del DOM facilmente;
	\item agevola la realizzazione di animazioni elementari;
	\item presenta un gran numero di plug-in.
\end{itemize}

Tra i contro dell'utilizzo di \jquery{}:
\begin{itemize}
	\item lentezza nelle prestazioni in caso di complesse manipolazioni del DOM e manipolazioni;
	\item un uso scorretto può portare alla creazione di grandi moli di codice difficile da mantenere;
\end{itemize} 

\subsubsection{\requirejs}
La libreria \requirejs{} è un loader di moduli e file \js{}. Esso si occuperà principalmente di risolvere dipendenze delle librerie \js{} utilizzate.

I vantaggi principali:
\begin{itemize}
	\item esplicita le dipendenze tra i diversi moduli;
	\item caricamento dei moduli asincrono e "su richiesta".
\end{itemize}

Tra gli svantaggi principali:
\begin{itemize}
\item necessità di uno studio approfondito e ragionato prima di un utilizzo efficace;
\item curva di apprendimento ripida;
\item complessità maggiore di integrazione con i test; 
\end{itemize} 

\subsubsection{\lodash}
La libreria \lodash{} fornisce metodi di utilità non offerti da \js{} puro. 

Rispetto alla simile libreria \emph{Underscore}, questa fornisce più performance, più features e miglior documentazione. Essa inoltre è usata da \jointjs.

% \subsubsection{qunit}
% assertions, test di regressione

% \subsubsection{sinonjs}
% stubs, mocks, (test spies??)

\subsection{Server}

\subsubsection{Apache Tomcat}
Il \emph{back end} della nostra applicazione è ospitato su un \gloss{server} Apache Tomcat, come richiesto dal capitolato d'appalto nel caso il \emph{back end} fosse scritto in Java.

Tomcat fornisce numerosi vantaggi:
\begin{itemize}
	\item facile da installare;
	\item richiede poche impostazione personalizzate per poter esssere pienamente operativo;
	\item consumo di memoria molto basso rispetto alle applicazioni concorrenti;
	\item tempo di avvio immediato. 
\end{itemize}

Allo stesso tempo manca della funzionalità che permette il recupero automatico delle funzionalità in caso di \emph{failure}.

\subsubsection{Spring Boot}
Le richieste HTTP inviate dal client al server vengono gestite tramite la libreria open-source \emph{Spring Boot} (disponibile all'indirizzo \url{spring.io}), utile per aderire ai princìpi dello stile architetturale REST.

La scelta di Spring è stata guidata dalle seguenti caratteristiche proprie di questa libreria:
\begin{itemize}
	\item modulare: rende possibile integrare solo alcuni moduli all'interno del progetto;
	\item integrabile: rende facile l'utilizzazione di altri framework oper source;
	\item testabile: permette di scrivere software facile da testare.
\end{itemize}

Tuttavia Spring possiede vari difetti:
\begin{itemize}
	\item pesantezza: l'ambiente di Spring è tutt'altro che leggero;
	\item complessità: sviluppare in Spring presenta una curva di apprendimento ripida;
\end{itemize}

In particolare la libreria derivata chiamata Spring Boot è stata preferita per le seguenti caratteristiche:
\begin{itemize}
	\item semplicità: l'ambiente di Spring Boot fornisce una suite preconfezionata e preconfigurata \emph{out-of-the-box}, in modo da semplificare l'attività di codifica;
	\item (relativa) leggerezza: rispetto all'ambiente di Spring, Spring Boot risulta sensibilmente più leggero.
	% \item altro?
\end{itemize}

L'utilizzo di questa tecnologia rispetto a \emph{Spring} secondo il team è preferibile e valutata positivamente dopo un breve periodo di prova di entrambe le tecnologie.

Si rimanda all'appendice %\ref{sec:app_spring}
per ulteriori considerazioni sull'uso di Spring Boot nella nostra applicazione. Futuri riferimenti alla libreria Spring dovranno essere considerati come riferiti alla libreria Spring Boot, qualora non diversamente specificato.

\subsubsection{StringTemplate}
StringTemplate è una libreria open-source scritta da Terence Parr e disponibile all'indirizzo \url{stringtemplate.org}. Come indicato sul sito ufficiale, questa libreria è particolarmente adatta alla generazione di codice. 

Essa viene utilizzata per:
\begin{itemize}
	\item creare dei template di codice Java, inframmezzati da speciali marcatori definiti da StringTemplate;
	\item popolare tali template con i dati ricevuti dal client.
\end{itemize}
I marcatori definiti dalla libreria servono da segnaposto per ospitare i dati ottenuti analizzando il documento JSON generato dal client.

Le principali feature utili al progetto sono per esempio le seguenti:

\begin{itemize}
\item espressioni semplici, espressioni condizionali;
\item template anonimi;
\item funzioni;
\item valutazione lazy;
\item auto-indentazione;
\item iteratori;
\end{itemize}

Molto importante è notare la filosofia sfruttata dalla libreria (consultabile al seguente indirizzo: \url{https://github.com/antlr/stringtemplate4/blob/master/doc/motivation.md}): StringTemplate, non permettendo certe operazioni come assegnazioni rende questo sistema di templating privo di \emph{side-effects}. Malgrado questo può portare a rendere la soluzione finale più complessa da sviluppare, ciò porta molti vantaggi:

\begin{itemize}
\item si favorisce il riuso di componenti;
\item i template sono sostituibili più facilmente;
\item i template sono più facilmente testabili;
\item la parte del linguaggio specifico viene isolata totalmente dalla logica di gestione del template, astratta per ogni linguaggio. 
\end{itemize}

