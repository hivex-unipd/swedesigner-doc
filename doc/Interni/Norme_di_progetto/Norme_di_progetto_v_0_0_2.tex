% norme di progetto
% da compilare con il comando pdflatex Interni/Norme_di_progetto/Norme_di_progetto_v_x.x.x.tex

% Dichiarazioni di ambiente e inclusione di pacchetti
% da usare tramite il comando % Dichiarazioni di ambiente e inclusione di pacchetti
% da usare tramite il comando % Dichiarazioni di ambiente e inclusione di pacchetti
% da usare tramite il comando \input{../../util/hx-ambiente}

\documentclass[a4paper,titlepage]{article}
\usepackage[T1]{fontenc}
\usepackage[utf8]{inputenc}
\usepackage[english,italian]{babel}
\usepackage{microtype}
\usepackage{lmodern}
\usepackage{underscore}
\usepackage{graphicx}
\usepackage{eurosym}
\usepackage{float}
\usepackage{fancyhdr}
\usepackage[table,dvipsnames]{xcolor}
\usepackage{multirow}
\usepackage{longtable}
\usepackage{chngpage}
\usepackage{grffile}
\usepackage[titles]{tocloft}
\usepackage{hyperref}
\hypersetup{hidelinks}

\usepackage{../../util/hx-vers}
\usepackage{../../util/hx-macro}
\usepackage{../../util/hx-front}

% solo se si vuole una nuova pagina ad ogni \section:
\usepackage{titlesec}
\newcommand{\sectionbreak}{\clearpage}

% stile di pagina:
\pagestyle{fancy}

% solo se si vuole eliminare l'indentazione ad ogni paragrafo:
\setlength{\parindent}{0pt}

% intestazione:
\lhead{\Large{\proj}}
\rhead{\includegraphics[keepaspectratio=true,width=50px]{../../util/hivex_logo2.png}}
\renewcommand{\headrulewidth}{0.4pt}

% pie' di pagina:
\lfoot{\email}
\rfoot{\thepage}
\cfoot{}
\renewcommand{\footrulewidth}{0.4pt}

% spazio verticale tra le celle di una tabella:
\renewcommand{\arraystretch}{1.5}

% profondità di indicizzazione:
\setcounter{tocdepth}{4}
\setcounter{secnumdepth}{4}

% numerazione innestata per elenchi numerati:
\renewcommand{\labelenumii}{\theenumii}
\renewcommand{\theenumii}{\theenumi.\arabic{enumii}.}


\documentclass[a4paper,titlepage]{article}
\usepackage[T1]{fontenc}
\usepackage[utf8]{inputenc}
\usepackage[english,italian]{babel}
\usepackage{microtype}
\usepackage{lmodern}
\usepackage{underscore}
\usepackage{graphicx}
\usepackage{eurosym}
\usepackage{float}
\usepackage{fancyhdr}
\usepackage[table,dvipsnames]{xcolor}
\usepackage{multirow}
\usepackage{longtable}
\usepackage{chngpage}
\usepackage{grffile}
\usepackage[titles]{tocloft}
\usepackage{hyperref}
\hypersetup{hidelinks}

\usepackage{../../util/hx-vers}
\usepackage{../../util/hx-macro}
\usepackage{../../util/hx-front}

% solo se si vuole una nuova pagina ad ogni \section:
\usepackage{titlesec}
\newcommand{\sectionbreak}{\clearpage}

% stile di pagina:
\pagestyle{fancy}

% solo se si vuole eliminare l'indentazione ad ogni paragrafo:
\setlength{\parindent}{0pt}

% intestazione:
\lhead{\Large{\proj}}
\rhead{\includegraphics[keepaspectratio=true,width=50px]{../../util/hivex_logo2.png}}
\renewcommand{\headrulewidth}{0.4pt}

% pie' di pagina:
\lfoot{\email}
\rfoot{\thepage}
\cfoot{}
\renewcommand{\footrulewidth}{0.4pt}

% spazio verticale tra le celle di una tabella:
\renewcommand{\arraystretch}{1.5}

% profondità di indicizzazione:
\setcounter{tocdepth}{4}
\setcounter{secnumdepth}{4}

% numerazione innestata per elenchi numerati:
\renewcommand{\labelenumii}{\theenumii}
\renewcommand{\theenumii}{\theenumi.\arabic{enumii}.}


\documentclass[a4paper,titlepage]{article}
\usepackage[T1]{fontenc}
\usepackage[utf8]{inputenc}
\usepackage[english,italian]{babel}
\usepackage{microtype}
\usepackage{lmodern}
\usepackage{underscore}
\usepackage{graphicx}
\usepackage{eurosym}
\usepackage{float}
\usepackage{fancyhdr}
\usepackage[table,dvipsnames]{xcolor}
\usepackage{multirow}
\usepackage{longtable}
\usepackage{chngpage}
\usepackage{grffile}
\usepackage[titles]{tocloft}
\usepackage{hyperref}
\hypersetup{hidelinks}

\usepackage{../../util/hx-vers}
\usepackage{../../util/hx-macro}
\usepackage{../../util/hx-front}

% solo se si vuole una nuova pagina ad ogni \section:
\usepackage{titlesec}
\newcommand{\sectionbreak}{\clearpage}

% stile di pagina:
\pagestyle{fancy}

% solo se si vuole eliminare l'indentazione ad ogni paragrafo:
\setlength{\parindent}{0pt}

% intestazione:
\lhead{\Large{\proj}}
\rhead{\includegraphics[keepaspectratio=true,width=50px]{../../util/hivex_logo2.png}}
\renewcommand{\headrulewidth}{0.4pt}

% pie' di pagina:
\lfoot{\email}
\rfoot{\thepage}
\cfoot{}
\renewcommand{\footrulewidth}{0.4pt}

% spazio verticale tra le celle di una tabella:
\renewcommand{\arraystretch}{1.5}

% profondità di indicizzazione:
\setcounter{tocdepth}{4}
\setcounter{secnumdepth}{4}

% numerazione innestata per elenchi numerati:
\renewcommand{\labelenumii}{\theenumii}
\renewcommand{\theenumii}{\theenumi.\arabic{enumii}.}

\version{0.0.1}
\creaz{24 dicembre 2016}
\author{\GG, \MM}
\supervisor{\LB, \AZ}
\uso{interno}
\dest{Tutti i membri del gruppo}
\title{Norme di progetto}

\renewcommand{\arraystretch}{1.5}
\setcounter{tocdepth}{4}
\setcounter{secnumdepth}{4}

\begin{document}
\maketitle
% diario delle modifiche per l'analisi dei requisiti
% da includere con % diario delle modifiche per l'analisi dei requisiti
% da includere con % diario delle modifiche per l'analisi dei requisiti
% da includere con \include{diario}

\begin{diario}
	4.0.0 & {\LB} (Responsabile) & 02/05/2017 & Approvazione del documento \\ \hline
	3.1.0 & {\PB} (Verificatore) & 02/05/2017 & Verifica del documento \\ \hline
	3.0.1 & {\MM} (Analista) & 01/05/2017 & 
	\begin{itemize}
	\item Inserimento UC5.35 e relativo requisito;
	\item Inserimento UC8 e relativo requisito;
	\item Inserimento tabella Requisiti Implementati come appendice.
\end{itemize}\\ \hline
	3.0.0 & {\AZ} (Responsabile) & 19/03/2017 & Approvazione del documento \\ \hline
	2.1.0 & {\MM} (Verificatore) & 19/03/2017 & Verifica del documento \\ \hline
	2.0.3 & {\PB} (Progettista) & 18/03/2017 &  
\begin{itemize}
	\item Modifica tabella Tracciamento Fonti-Requisiti;
	\item Modifica tabella Requisiti-Fonti;
	\item Modifica Estensione UC7.
\end{itemize}\\ \hline
	2.0.2 & {\PB} (Progettista) & 17/03/2017 &  Ristrutturato UC5 e relativi requisiti\\ \hline
	2.0.1 & {\PB} (Progettista) & 16/03/2017 &  Ristrutturato UC4 e relativi requisiti\\ \hline
	2.0.0 & {\LS} (Responsabile) & 01/02/2017 & Approvazione del documento \\ \hline
	1.1.0 & {\GG} (Verificatore) & 01/02/2017 & Verifica del documento \\ \hline
	1.0.4 & {\AZ} (Analista) & 31/01/2017 & Inserito UC5.26 con relativo requisito e tracciamento nelle tabelle e inseriti i requisiti RFO7, RFO8, RFO8.1, RFO8.2, RFO9, RFO10 e RFO11\\ \hline
	1.0.3 & {\AZ} (Analista) & 29/01/2017 & Corretta la descrizione dello UC5 e approfondita la descrizione dello UC7 \\ \hline
	1.0.2 & {\AZ} (Analista) & 28/01/2017 & Corretti UC4.1.6.3.2, UC4.2.1 e inserito perimetro sistema del UC5\\ \hline
	1.0.1 & {\AZ} (Analista) & 26/01/2017 & Inserimento scenario alternativo allo UC2, creazione UC3.1 con relativo requisito e tracciamento nelle tabelle e corrette alcune postcondizioni \\ \hline
	1.0.0 & {\LB} (Responsabile) & 09/01/2017 & Approvazione documento \\ \hline
	0.4.0 & {\LS} (Verificatore) & 06/01/2017 & Verifica introduzione, descrizione generale e requisiti \\ \hline
	0.3.0 & {\MM} (Verificatore) & 06/01/2017 & Verifica UC5.3-UC7 \\ \hline
	0.2.0 & {\LB} (Verificatore) & 06/01/2017 & Verifica UC4.2-UC5.2 \\ \hline
	0.1.0 & {\AZ} (Verificatore) & 06/01/2017 & Verifica UC1-4.1.8 \\ \hline
	0.0.11 & {\LS} (Analista) & 04/01/2017 & Stesura UC6-UC7 \\ \hline
	0.0.10 & {\GG} (Analista) & 03/01/2017 & Stesura UC5.6-UC5.18 \\ \hline
	0.0.9 & {\LS} (Analista) & 03/01/2017 & Stesura UC5.3-UC5.5.6.1 \\ \hline
	0.0.8 & {\PB} (Analista) & 02/01/2017 & Stesura UC5-UC5.2 \\ \hline
	0.0.7 & {\AZ} (Analista) & 02/01/2017 & Stesura UC4.3.3.1-UC4.11 \\ \hline
	0.0.6 & {\MM} (Analista) & 30/12/2016 & Stesura UC4.2-UC4.3.3.1 \\ \hline
	0.0.5 & {\GG} (Analista) & 29/12/2016 & Stesura UC4.1.6-UC4.1.8 \\ \hline
	0.0.4 & {\PB} (Analista) & 29/12/2016 & Stesura UC4-UC4.1.5 \\ \hline
	0.0.3 & {\LB} (Analista) & 28/12/2016 & Stesura UC1-UC2-UC3 \\ \hline
	0.0.2 & {\LS} (Analista) & 27/12/2016 & Stesura introduzione e descrizione generale \\ \hline
	0.0.1 & {\AZ} (Analista) & 27/12/2016 & Stesura scheletro \\ \hline
\end{diario}


\begin{diario}
	4.0.0 & {\LB} (Responsabile) & 02/05/2017 & Approvazione del documento \\ \hline
	3.1.0 & {\PB} (Verificatore) & 02/05/2017 & Verifica del documento \\ \hline
	3.0.1 & {\MM} (Analista) & 01/05/2017 & 
	\begin{itemize}
	\item Inserimento UC5.35 e relativo requisito;
	\item Inserimento UC8 e relativo requisito;
	\item Inserimento tabella Requisiti Implementati come appendice.
\end{itemize}\\ \hline
	3.0.0 & {\AZ} (Responsabile) & 19/03/2017 & Approvazione del documento \\ \hline
	2.1.0 & {\MM} (Verificatore) & 19/03/2017 & Verifica del documento \\ \hline
	2.0.3 & {\PB} (Progettista) & 18/03/2017 &  
\begin{itemize}
	\item Modifica tabella Tracciamento Fonti-Requisiti;
	\item Modifica tabella Requisiti-Fonti;
	\item Modifica Estensione UC7.
\end{itemize}\\ \hline
	2.0.2 & {\PB} (Progettista) & 17/03/2017 &  Ristrutturato UC5 e relativi requisiti\\ \hline
	2.0.1 & {\PB} (Progettista) & 16/03/2017 &  Ristrutturato UC4 e relativi requisiti\\ \hline
	2.0.0 & {\LS} (Responsabile) & 01/02/2017 & Approvazione del documento \\ \hline
	1.1.0 & {\GG} (Verificatore) & 01/02/2017 & Verifica del documento \\ \hline
	1.0.4 & {\AZ} (Analista) & 31/01/2017 & Inserito UC5.26 con relativo requisito e tracciamento nelle tabelle e inseriti i requisiti RFO7, RFO8, RFO8.1, RFO8.2, RFO9, RFO10 e RFO11\\ \hline
	1.0.3 & {\AZ} (Analista) & 29/01/2017 & Corretta la descrizione dello UC5 e approfondita la descrizione dello UC7 \\ \hline
	1.0.2 & {\AZ} (Analista) & 28/01/2017 & Corretti UC4.1.6.3.2, UC4.2.1 e inserito perimetro sistema del UC5\\ \hline
	1.0.1 & {\AZ} (Analista) & 26/01/2017 & Inserimento scenario alternativo allo UC2, creazione UC3.1 con relativo requisito e tracciamento nelle tabelle e corrette alcune postcondizioni \\ \hline
	1.0.0 & {\LB} (Responsabile) & 09/01/2017 & Approvazione documento \\ \hline
	0.4.0 & {\LS} (Verificatore) & 06/01/2017 & Verifica introduzione, descrizione generale e requisiti \\ \hline
	0.3.0 & {\MM} (Verificatore) & 06/01/2017 & Verifica UC5.3-UC7 \\ \hline
	0.2.0 & {\LB} (Verificatore) & 06/01/2017 & Verifica UC4.2-UC5.2 \\ \hline
	0.1.0 & {\AZ} (Verificatore) & 06/01/2017 & Verifica UC1-4.1.8 \\ \hline
	0.0.11 & {\LS} (Analista) & 04/01/2017 & Stesura UC6-UC7 \\ \hline
	0.0.10 & {\GG} (Analista) & 03/01/2017 & Stesura UC5.6-UC5.18 \\ \hline
	0.0.9 & {\LS} (Analista) & 03/01/2017 & Stesura UC5.3-UC5.5.6.1 \\ \hline
	0.0.8 & {\PB} (Analista) & 02/01/2017 & Stesura UC5-UC5.2 \\ \hline
	0.0.7 & {\AZ} (Analista) & 02/01/2017 & Stesura UC4.3.3.1-UC4.11 \\ \hline
	0.0.6 & {\MM} (Analista) & 30/12/2016 & Stesura UC4.2-UC4.3.3.1 \\ \hline
	0.0.5 & {\GG} (Analista) & 29/12/2016 & Stesura UC4.1.6-UC4.1.8 \\ \hline
	0.0.4 & {\PB} (Analista) & 29/12/2016 & Stesura UC4-UC4.1.5 \\ \hline
	0.0.3 & {\LB} (Analista) & 28/12/2016 & Stesura UC1-UC2-UC3 \\ \hline
	0.0.2 & {\LS} (Analista) & 27/12/2016 & Stesura introduzione e descrizione generale \\ \hline
	0.0.1 & {\AZ} (Analista) & 27/12/2016 & Stesura scheletro \\ \hline
\end{diario}


\begin{diario}
	4.0.0 & {\LB} (Responsabile) & 02/05/2017 & Approvazione del documento \\ \hline
	3.1.0 & {\PB} (Verificatore) & 02/05/2017 & Verifica del documento \\ \hline
	3.0.1 & {\MM} (Analista) & 01/05/2017 & 
	\begin{itemize}
	\item Inserimento UC5.35 e relativo requisito;
	\item Inserimento UC8 e relativo requisito;
	\item Inserimento tabella Requisiti Implementati come appendice.
\end{itemize}\\ \hline
	3.0.0 & {\AZ} (Responsabile) & 19/03/2017 & Approvazione del documento \\ \hline
	2.1.0 & {\MM} (Verificatore) & 19/03/2017 & Verifica del documento \\ \hline
	2.0.3 & {\PB} (Progettista) & 18/03/2017 &  
\begin{itemize}
	\item Modifica tabella Tracciamento Fonti-Requisiti;
	\item Modifica tabella Requisiti-Fonti;
	\item Modifica Estensione UC7.
\end{itemize}\\ \hline
	2.0.2 & {\PB} (Progettista) & 17/03/2017 &  Ristrutturato UC5 e relativi requisiti\\ \hline
	2.0.1 & {\PB} (Progettista) & 16/03/2017 &  Ristrutturato UC4 e relativi requisiti\\ \hline
	2.0.0 & {\LS} (Responsabile) & 01/02/2017 & Approvazione del documento \\ \hline
	1.1.0 & {\GG} (Verificatore) & 01/02/2017 & Verifica del documento \\ \hline
	1.0.4 & {\AZ} (Analista) & 31/01/2017 & Inserito UC5.26 con relativo requisito e tracciamento nelle tabelle e inseriti i requisiti RFO7, RFO8, RFO8.1, RFO8.2, RFO9, RFO10 e RFO11\\ \hline
	1.0.3 & {\AZ} (Analista) & 29/01/2017 & Corretta la descrizione dello UC5 e approfondita la descrizione dello UC7 \\ \hline
	1.0.2 & {\AZ} (Analista) & 28/01/2017 & Corretti UC4.1.6.3.2, UC4.2.1 e inserito perimetro sistema del UC5\\ \hline
	1.0.1 & {\AZ} (Analista) & 26/01/2017 & Inserimento scenario alternativo allo UC2, creazione UC3.1 con relativo requisito e tracciamento nelle tabelle e corrette alcune postcondizioni \\ \hline
	1.0.0 & {\LB} (Responsabile) & 09/01/2017 & Approvazione documento \\ \hline
	0.4.0 & {\LS} (Verificatore) & 06/01/2017 & Verifica introduzione, descrizione generale e requisiti \\ \hline
	0.3.0 & {\MM} (Verificatore) & 06/01/2017 & Verifica UC5.3-UC7 \\ \hline
	0.2.0 & {\LB} (Verificatore) & 06/01/2017 & Verifica UC4.2-UC5.2 \\ \hline
	0.1.0 & {\AZ} (Verificatore) & 06/01/2017 & Verifica UC1-4.1.8 \\ \hline
	0.0.11 & {\LS} (Analista) & 04/01/2017 & Stesura UC6-UC7 \\ \hline
	0.0.10 & {\GG} (Analista) & 03/01/2017 & Stesura UC5.6-UC5.18 \\ \hline
	0.0.9 & {\LS} (Analista) & 03/01/2017 & Stesura UC5.3-UC5.5.6.1 \\ \hline
	0.0.8 & {\PB} (Analista) & 02/01/2017 & Stesura UC5-UC5.2 \\ \hline
	0.0.7 & {\AZ} (Analista) & 02/01/2017 & Stesura UC4.3.3.1-UC4.11 \\ \hline
	0.0.6 & {\MM} (Analista) & 30/12/2016 & Stesura UC4.2-UC4.3.3.1 \\ \hline
	0.0.5 & {\GG} (Analista) & 29/12/2016 & Stesura UC4.1.6-UC4.1.8 \\ \hline
	0.0.4 & {\PB} (Analista) & 29/12/2016 & Stesura UC4-UC4.1.5 \\ \hline
	0.0.3 & {\LB} (Analista) & 28/12/2016 & Stesura UC1-UC2-UC3 \\ \hline
	0.0.2 & {\LS} (Analista) & 27/12/2016 & Stesura introduzione e descrizione generale \\ \hline
	0.0.1 & {\AZ} (Analista) & 27/12/2016 & Stesura scheletro \\ \hline
\end{diario}

\tableofcontents

\section{Introduzione}

\subsection{Scopo del documento}
Questo documento (interno al gruppo) regolamenta i processi del progetto didattico e va quindi letto da ciascun membro del gruppo. Le convenzioni qui prescritte servono a:
\begin{itemize}
	\item garantire ordine all'interno dei documenti e delle varie parti della configurazione del prodotto;
	\item mantenere coerenza nelle notazioni e nelle procedure;
	\item minimizzare i conflitti tra i vari ruoli;
	\item garantire che l'infrastruttura di lavoro sia il più possibile semplice e gestibile --- quindi fruibile;
\end{itemize}

\subsection{Scopo del prodotto}
\scopo % da util/hx-macro.sty

	\subsection{Glossario}
	\subsection{Riferimenti}
		\subsubsection{Normativi}
\section{Processi di sviluppo}
	\subsection{Studio di fattibilità}
	\subsection{Analisi dei requisiti}
		\subsubsection{Requisiti}
		\subsubsection{Casi d'uso}
	\subsection{Progettazione}
	\subsection{Codifica}
		\subsubsection{Formattazione del codice}
	\subsection{Strumenti}
		\subsubsection{PragmaDB}
		\subsubsection{Astah}
		\subsubsection{Ide ancora da scegliere}
	\subsection{Tecnologie utilizzate}
\section{Processi di supporto}

\subsection{Processi di documentazione}
Per gestire la documentazione di progetto si è pensato, innanzitutto, di creare un ramo a parte (intitolato \texttt{doc}) a partire dal ramo \texttt{master} del nostro repository; questa scelta è stata dettata dal basso grado di dipendenza tra la documentazione il resto delle parti della configurazione del prodotto.

Enumeriamo qui di seguito le norme che riguardano lo sviluppo della documentazione; quelle enunciate con il condizionale vanno interpretate come consigli:
\begin{enumerate}
	\item I documenti andranno mantenuti in una struttura ad albero ben organizzata ma non troppo profonda; il miglior compromesso tra organizzazione e profondità dell'albero è risultata essere quella in figura [figura...].
	\item Nome ed estensione dei file sono regolati nel seguente modo:
	\begin{enumerate}
		\item Il nome di un file o di una directory non dovrebbe contenere spazi, al fine di facilitare operazioni da riga di comando.
		\item Ogni documento dev'essere in formato \LaTeX (esportabile in PDF su foglio A4); la prima versione di un documento può essere in altri formati ma va presto sostituita da una versione in \LaTeX.
		\item Ogni versione di un documento è identificata univocamente dal nome del file, composto di una prima stringa esprimente la sua funzione (ad esempio “Piano\_di\_progetto”) e una seconda stringa con il numero di versionamento (ad esempio “\_0.0.2”).
	\end{enumerate}
	\item Il numero di versionamento di un documento si compone di tre numeri: il primo viene incremementato ad ogni \textbf{approvazione} del documento; il secondo ad ogni \textbf{revisione}; il terzo ad ogni \textbf{aggiunta o modifica} sostanziale.
	\item Ogni documento, anche se breve, deve avere un frontespizio recante: titolo del documento, versione, nome del nostro gruppo, nome del progetto, data, autori, revisori, uso (interno o esterno), lista di distribuzione. % più il nostro logo, magari
	\item Ogni documento deve includere un “diario delle modifiche” che elenca i cambiamenti che ogni versione ha apportato rispetto alla precedente, assieme all'autore di tali cambiamenti e alla data.
	\item Ogni pagina dev'essere numerata e deve riportare il titolo del documento e della sezione che ospita. % e magari il nome/logo del gruppo e il nome del progetto
	\item Dev'esserci un'unica struttura di sezioni, sottosezioni e paragrafi, comune a tutti i documenti dello stesso tipo.
	\item Documenti con più di tre o quattro sezioni devono riportare, all'inizio, un indice dei contenuti.
	\item Andrebbe minimizzato l'uso di colori, al fine di minimizzare le ambiguità nel caso che un documento venga stampato su carta o letto da una persona daltonica.
	\item All'interno di una sezione, la prima occorrenza di un termine che si trovi nel glossario dev'essere segnata in modo da far capire che si trova nel glossario.
\end{enumerate}

		\subsubsection{Strumenti}
		\subsubsection{Ciclo di vita di un documento}
		\subsubsection{Documenti finali}
			\paragraph{Studio di fattibilità}
			\paragraph{Norme di progetto}
			\paragraph{Piano di progetto}
			\paragraph{Piano di qualifica}
			\paragraph{Analisi dei requisiti}
			\paragraph{Specifica tecnica}
			\paragraph{Definizione di prodotto}
			\paragraph{Glossario}
			\paragraph{Manuale utente}
			\paragraph{Verbali}
	\subsection{Processi di verifica}
	\subsection{Strumenti}
\section{Processi organizzativi}
	\subsection{Strumenti}
\end{document}
