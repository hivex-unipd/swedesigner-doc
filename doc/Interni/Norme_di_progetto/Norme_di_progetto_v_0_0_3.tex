% norme di progetto
% da compilare con il comando pdflatex Norme_di_progetto_v_x.x.x.tex

% Dichiarazioni di ambiente e inclusione di pacchetti
% da usare tramite il comando % Dichiarazioni di ambiente e inclusione di pacchetti
% da usare tramite il comando % Dichiarazioni di ambiente e inclusione di pacchetti
% da usare tramite il comando \input{../../util/hx-ambiente}

\documentclass[a4paper,titlepage]{article}
\usepackage[T1]{fontenc}
\usepackage[utf8]{inputenc}
\usepackage[english,italian]{babel}
\usepackage{microtype}
\usepackage{lmodern}
\usepackage{underscore}
\usepackage{graphicx}
\usepackage{eurosym}
\usepackage{float}
\usepackage{fancyhdr}
\usepackage[table,dvipsnames]{xcolor}
\usepackage{multirow}
\usepackage{longtable}
\usepackage{chngpage}
\usepackage{grffile}
\usepackage[titles]{tocloft}
\usepackage{hyperref}
\hypersetup{hidelinks}

\usepackage{../../util/hx-vers}
\usepackage{../../util/hx-macro}
\usepackage{../../util/hx-front}

% solo se si vuole una nuova pagina ad ogni \section:
\usepackage{titlesec}
\newcommand{\sectionbreak}{\clearpage}

% stile di pagina:
\pagestyle{fancy}

% solo se si vuole eliminare l'indentazione ad ogni paragrafo:
\setlength{\parindent}{0pt}

% intestazione:
\lhead{\Large{\proj}}
\rhead{\includegraphics[keepaspectratio=true,width=50px]{../../util/hivex_logo2.png}}
\renewcommand{\headrulewidth}{0.4pt}

% pie' di pagina:
\lfoot{\email}
\rfoot{\thepage}
\cfoot{}
\renewcommand{\footrulewidth}{0.4pt}

% spazio verticale tra le celle di una tabella:
\renewcommand{\arraystretch}{1.5}

% profondità di indicizzazione:
\setcounter{tocdepth}{4}
\setcounter{secnumdepth}{4}

% numerazione innestata per elenchi numerati:
\renewcommand{\labelenumii}{\theenumii}
\renewcommand{\theenumii}{\theenumi.\arabic{enumii}.}


\documentclass[a4paper,titlepage]{article}
\usepackage[T1]{fontenc}
\usepackage[utf8]{inputenc}
\usepackage[english,italian]{babel}
\usepackage{microtype}
\usepackage{lmodern}
\usepackage{underscore}
\usepackage{graphicx}
\usepackage{eurosym}
\usepackage{float}
\usepackage{fancyhdr}
\usepackage[table,dvipsnames]{xcolor}
\usepackage{multirow}
\usepackage{longtable}
\usepackage{chngpage}
\usepackage{grffile}
\usepackage[titles]{tocloft}
\usepackage{hyperref}
\hypersetup{hidelinks}

\usepackage{../../util/hx-vers}
\usepackage{../../util/hx-macro}
\usepackage{../../util/hx-front}

% solo se si vuole una nuova pagina ad ogni \section:
\usepackage{titlesec}
\newcommand{\sectionbreak}{\clearpage}

% stile di pagina:
\pagestyle{fancy}

% solo se si vuole eliminare l'indentazione ad ogni paragrafo:
\setlength{\parindent}{0pt}

% intestazione:
\lhead{\Large{\proj}}
\rhead{\includegraphics[keepaspectratio=true,width=50px]{../../util/hivex_logo2.png}}
\renewcommand{\headrulewidth}{0.4pt}

% pie' di pagina:
\lfoot{\email}
\rfoot{\thepage}
\cfoot{}
\renewcommand{\footrulewidth}{0.4pt}

% spazio verticale tra le celle di una tabella:
\renewcommand{\arraystretch}{1.5}

% profondità di indicizzazione:
\setcounter{tocdepth}{4}
\setcounter{secnumdepth}{4}

% numerazione innestata per elenchi numerati:
\renewcommand{\labelenumii}{\theenumii}
\renewcommand{\theenumii}{\theenumi.\arabic{enumii}.}


\documentclass[a4paper,titlepage]{article}
\usepackage[T1]{fontenc}
\usepackage[utf8]{inputenc}
\usepackage[english,italian]{babel}
\usepackage{microtype}
\usepackage{lmodern}
\usepackage{underscore}
\usepackage{graphicx}
\usepackage{eurosym}
\usepackage{float}
\usepackage{fancyhdr}
\usepackage[table,dvipsnames]{xcolor}
\usepackage{multirow}
\usepackage{longtable}
\usepackage{chngpage}
\usepackage{grffile}
\usepackage[titles]{tocloft}
\usepackage{hyperref}
\hypersetup{hidelinks}

\usepackage{../../util/hx-vers}
\usepackage{../../util/hx-macro}
\usepackage{../../util/hx-front}

% solo se si vuole una nuova pagina ad ogni \section:
\usepackage{titlesec}
\newcommand{\sectionbreak}{\clearpage}

% stile di pagina:
\pagestyle{fancy}

% solo se si vuole eliminare l'indentazione ad ogni paragrafo:
\setlength{\parindent}{0pt}

% intestazione:
\lhead{\Large{\proj}}
\rhead{\includegraphics[keepaspectratio=true,width=50px]{../../util/hivex_logo2.png}}
\renewcommand{\headrulewidth}{0.4pt}

% pie' di pagina:
\lfoot{\email}
\rfoot{\thepage}
\cfoot{}
\renewcommand{\footrulewidth}{0.4pt}

% spazio verticale tra le celle di una tabella:
\renewcommand{\arraystretch}{1.5}

% profondità di indicizzazione:
\setcounter{tocdepth}{4}
\setcounter{secnumdepth}{4}

% numerazione innestata per elenchi numerati:
\renewcommand{\labelenumii}{\theenumii}
\renewcommand{\theenumii}{\theenumi.\arabic{enumii}.}


\version{0.0.3}
\creaz{24 dicembre 2016}
\author{\GG, \MM}
\supervisor{\LB, \AZ}
\uso{interno}
\dest{Tutti i membri del gruppo}
\title{Norme di progetto}

\begin{document}
\maketitle
% diario delle modifiche per l'analisi dei requisiti
% da includere con % diario delle modifiche per l'analisi dei requisiti
% da includere con % diario delle modifiche per l'analisi dei requisiti
% da includere con \include{diario}

\begin{diario}
	4.0.0 & {\LB} (Responsabile) & 02/05/2017 & Approvazione del documento \\ \hline
	3.1.0 & {\PB} (Verificatore) & 02/05/2017 & Verifica del documento \\ \hline
	3.0.1 & {\MM} (Analista) & 01/05/2017 & 
	\begin{itemize}
	\item Inserimento UC5.35 e relativo requisito;
	\item Inserimento UC8 e relativo requisito;
	\item Inserimento tabella Requisiti Implementati come appendice.
\end{itemize}\\ \hline
	3.0.0 & {\AZ} (Responsabile) & 19/03/2017 & Approvazione del documento \\ \hline
	2.1.0 & {\MM} (Verificatore) & 19/03/2017 & Verifica del documento \\ \hline
	2.0.3 & {\PB} (Progettista) & 18/03/2017 &  
\begin{itemize}
	\item Modifica tabella Tracciamento Fonti-Requisiti;
	\item Modifica tabella Requisiti-Fonti;
	\item Modifica Estensione UC7.
\end{itemize}\\ \hline
	2.0.2 & {\PB} (Progettista) & 17/03/2017 &  Ristrutturato UC5 e relativi requisiti\\ \hline
	2.0.1 & {\PB} (Progettista) & 16/03/2017 &  Ristrutturato UC4 e relativi requisiti\\ \hline
	2.0.0 & {\LS} (Responsabile) & 01/02/2017 & Approvazione del documento \\ \hline
	1.1.0 & {\GG} (Verificatore) & 01/02/2017 & Verifica del documento \\ \hline
	1.0.4 & {\AZ} (Analista) & 31/01/2017 & Inserito UC5.26 con relativo requisito e tracciamento nelle tabelle e inseriti i requisiti RFO7, RFO8, RFO8.1, RFO8.2, RFO9, RFO10 e RFO11\\ \hline
	1.0.3 & {\AZ} (Analista) & 29/01/2017 & Corretta la descrizione dello UC5 e approfondita la descrizione dello UC7 \\ \hline
	1.0.2 & {\AZ} (Analista) & 28/01/2017 & Corretti UC4.1.6.3.2, UC4.2.1 e inserito perimetro sistema del UC5\\ \hline
	1.0.1 & {\AZ} (Analista) & 26/01/2017 & Inserimento scenario alternativo allo UC2, creazione UC3.1 con relativo requisito e tracciamento nelle tabelle e corrette alcune postcondizioni \\ \hline
	1.0.0 & {\LB} (Responsabile) & 09/01/2017 & Approvazione documento \\ \hline
	0.4.0 & {\LS} (Verificatore) & 06/01/2017 & Verifica introduzione, descrizione generale e requisiti \\ \hline
	0.3.0 & {\MM} (Verificatore) & 06/01/2017 & Verifica UC5.3-UC7 \\ \hline
	0.2.0 & {\LB} (Verificatore) & 06/01/2017 & Verifica UC4.2-UC5.2 \\ \hline
	0.1.0 & {\AZ} (Verificatore) & 06/01/2017 & Verifica UC1-4.1.8 \\ \hline
	0.0.11 & {\LS} (Analista) & 04/01/2017 & Stesura UC6-UC7 \\ \hline
	0.0.10 & {\GG} (Analista) & 03/01/2017 & Stesura UC5.6-UC5.18 \\ \hline
	0.0.9 & {\LS} (Analista) & 03/01/2017 & Stesura UC5.3-UC5.5.6.1 \\ \hline
	0.0.8 & {\PB} (Analista) & 02/01/2017 & Stesura UC5-UC5.2 \\ \hline
	0.0.7 & {\AZ} (Analista) & 02/01/2017 & Stesura UC4.3.3.1-UC4.11 \\ \hline
	0.0.6 & {\MM} (Analista) & 30/12/2016 & Stesura UC4.2-UC4.3.3.1 \\ \hline
	0.0.5 & {\GG} (Analista) & 29/12/2016 & Stesura UC4.1.6-UC4.1.8 \\ \hline
	0.0.4 & {\PB} (Analista) & 29/12/2016 & Stesura UC4-UC4.1.5 \\ \hline
	0.0.3 & {\LB} (Analista) & 28/12/2016 & Stesura UC1-UC2-UC3 \\ \hline
	0.0.2 & {\LS} (Analista) & 27/12/2016 & Stesura introduzione e descrizione generale \\ \hline
	0.0.1 & {\AZ} (Analista) & 27/12/2016 & Stesura scheletro \\ \hline
\end{diario}


\begin{diario}
	4.0.0 & {\LB} (Responsabile) & 02/05/2017 & Approvazione del documento \\ \hline
	3.1.0 & {\PB} (Verificatore) & 02/05/2017 & Verifica del documento \\ \hline
	3.0.1 & {\MM} (Analista) & 01/05/2017 & 
	\begin{itemize}
	\item Inserimento UC5.35 e relativo requisito;
	\item Inserimento UC8 e relativo requisito;
	\item Inserimento tabella Requisiti Implementati come appendice.
\end{itemize}\\ \hline
	3.0.0 & {\AZ} (Responsabile) & 19/03/2017 & Approvazione del documento \\ \hline
	2.1.0 & {\MM} (Verificatore) & 19/03/2017 & Verifica del documento \\ \hline
	2.0.3 & {\PB} (Progettista) & 18/03/2017 &  
\begin{itemize}
	\item Modifica tabella Tracciamento Fonti-Requisiti;
	\item Modifica tabella Requisiti-Fonti;
	\item Modifica Estensione UC7.
\end{itemize}\\ \hline
	2.0.2 & {\PB} (Progettista) & 17/03/2017 &  Ristrutturato UC5 e relativi requisiti\\ \hline
	2.0.1 & {\PB} (Progettista) & 16/03/2017 &  Ristrutturato UC4 e relativi requisiti\\ \hline
	2.0.0 & {\LS} (Responsabile) & 01/02/2017 & Approvazione del documento \\ \hline
	1.1.0 & {\GG} (Verificatore) & 01/02/2017 & Verifica del documento \\ \hline
	1.0.4 & {\AZ} (Analista) & 31/01/2017 & Inserito UC5.26 con relativo requisito e tracciamento nelle tabelle e inseriti i requisiti RFO7, RFO8, RFO8.1, RFO8.2, RFO9, RFO10 e RFO11\\ \hline
	1.0.3 & {\AZ} (Analista) & 29/01/2017 & Corretta la descrizione dello UC5 e approfondita la descrizione dello UC7 \\ \hline
	1.0.2 & {\AZ} (Analista) & 28/01/2017 & Corretti UC4.1.6.3.2, UC4.2.1 e inserito perimetro sistema del UC5\\ \hline
	1.0.1 & {\AZ} (Analista) & 26/01/2017 & Inserimento scenario alternativo allo UC2, creazione UC3.1 con relativo requisito e tracciamento nelle tabelle e corrette alcune postcondizioni \\ \hline
	1.0.0 & {\LB} (Responsabile) & 09/01/2017 & Approvazione documento \\ \hline
	0.4.0 & {\LS} (Verificatore) & 06/01/2017 & Verifica introduzione, descrizione generale e requisiti \\ \hline
	0.3.0 & {\MM} (Verificatore) & 06/01/2017 & Verifica UC5.3-UC7 \\ \hline
	0.2.0 & {\LB} (Verificatore) & 06/01/2017 & Verifica UC4.2-UC5.2 \\ \hline
	0.1.0 & {\AZ} (Verificatore) & 06/01/2017 & Verifica UC1-4.1.8 \\ \hline
	0.0.11 & {\LS} (Analista) & 04/01/2017 & Stesura UC6-UC7 \\ \hline
	0.0.10 & {\GG} (Analista) & 03/01/2017 & Stesura UC5.6-UC5.18 \\ \hline
	0.0.9 & {\LS} (Analista) & 03/01/2017 & Stesura UC5.3-UC5.5.6.1 \\ \hline
	0.0.8 & {\PB} (Analista) & 02/01/2017 & Stesura UC5-UC5.2 \\ \hline
	0.0.7 & {\AZ} (Analista) & 02/01/2017 & Stesura UC4.3.3.1-UC4.11 \\ \hline
	0.0.6 & {\MM} (Analista) & 30/12/2016 & Stesura UC4.2-UC4.3.3.1 \\ \hline
	0.0.5 & {\GG} (Analista) & 29/12/2016 & Stesura UC4.1.6-UC4.1.8 \\ \hline
	0.0.4 & {\PB} (Analista) & 29/12/2016 & Stesura UC4-UC4.1.5 \\ \hline
	0.0.3 & {\LB} (Analista) & 28/12/2016 & Stesura UC1-UC2-UC3 \\ \hline
	0.0.2 & {\LS} (Analista) & 27/12/2016 & Stesura introduzione e descrizione generale \\ \hline
	0.0.1 & {\AZ} (Analista) & 27/12/2016 & Stesura scheletro \\ \hline
\end{diario}


\begin{diario}
	4.0.0 & {\LB} (Responsabile) & 02/05/2017 & Approvazione del documento \\ \hline
	3.1.0 & {\PB} (Verificatore) & 02/05/2017 & Verifica del documento \\ \hline
	3.0.1 & {\MM} (Analista) & 01/05/2017 & 
	\begin{itemize}
	\item Inserimento UC5.35 e relativo requisito;
	\item Inserimento UC8 e relativo requisito;
	\item Inserimento tabella Requisiti Implementati come appendice.
\end{itemize}\\ \hline
	3.0.0 & {\AZ} (Responsabile) & 19/03/2017 & Approvazione del documento \\ \hline
	2.1.0 & {\MM} (Verificatore) & 19/03/2017 & Verifica del documento \\ \hline
	2.0.3 & {\PB} (Progettista) & 18/03/2017 &  
\begin{itemize}
	\item Modifica tabella Tracciamento Fonti-Requisiti;
	\item Modifica tabella Requisiti-Fonti;
	\item Modifica Estensione UC7.
\end{itemize}\\ \hline
	2.0.2 & {\PB} (Progettista) & 17/03/2017 &  Ristrutturato UC5 e relativi requisiti\\ \hline
	2.0.1 & {\PB} (Progettista) & 16/03/2017 &  Ristrutturato UC4 e relativi requisiti\\ \hline
	2.0.0 & {\LS} (Responsabile) & 01/02/2017 & Approvazione del documento \\ \hline
	1.1.0 & {\GG} (Verificatore) & 01/02/2017 & Verifica del documento \\ \hline
	1.0.4 & {\AZ} (Analista) & 31/01/2017 & Inserito UC5.26 con relativo requisito e tracciamento nelle tabelle e inseriti i requisiti RFO7, RFO8, RFO8.1, RFO8.2, RFO9, RFO10 e RFO11\\ \hline
	1.0.3 & {\AZ} (Analista) & 29/01/2017 & Corretta la descrizione dello UC5 e approfondita la descrizione dello UC7 \\ \hline
	1.0.2 & {\AZ} (Analista) & 28/01/2017 & Corretti UC4.1.6.3.2, UC4.2.1 e inserito perimetro sistema del UC5\\ \hline
	1.0.1 & {\AZ} (Analista) & 26/01/2017 & Inserimento scenario alternativo allo UC2, creazione UC3.1 con relativo requisito e tracciamento nelle tabelle e corrette alcune postcondizioni \\ \hline
	1.0.0 & {\LB} (Responsabile) & 09/01/2017 & Approvazione documento \\ \hline
	0.4.0 & {\LS} (Verificatore) & 06/01/2017 & Verifica introduzione, descrizione generale e requisiti \\ \hline
	0.3.0 & {\MM} (Verificatore) & 06/01/2017 & Verifica UC5.3-UC7 \\ \hline
	0.2.0 & {\LB} (Verificatore) & 06/01/2017 & Verifica UC4.2-UC5.2 \\ \hline
	0.1.0 & {\AZ} (Verificatore) & 06/01/2017 & Verifica UC1-4.1.8 \\ \hline
	0.0.11 & {\LS} (Analista) & 04/01/2017 & Stesura UC6-UC7 \\ \hline
	0.0.10 & {\GG} (Analista) & 03/01/2017 & Stesura UC5.6-UC5.18 \\ \hline
	0.0.9 & {\LS} (Analista) & 03/01/2017 & Stesura UC5.3-UC5.5.6.1 \\ \hline
	0.0.8 & {\PB} (Analista) & 02/01/2017 & Stesura UC5-UC5.2 \\ \hline
	0.0.7 & {\AZ} (Analista) & 02/01/2017 & Stesura UC4.3.3.1-UC4.11 \\ \hline
	0.0.6 & {\MM} (Analista) & 30/12/2016 & Stesura UC4.2-UC4.3.3.1 \\ \hline
	0.0.5 & {\GG} (Analista) & 29/12/2016 & Stesura UC4.1.6-UC4.1.8 \\ \hline
	0.0.4 & {\PB} (Analista) & 29/12/2016 & Stesura UC4-UC4.1.5 \\ \hline
	0.0.3 & {\LB} (Analista) & 28/12/2016 & Stesura UC1-UC2-UC3 \\ \hline
	0.0.2 & {\LS} (Analista) & 27/12/2016 & Stesura introduzione e descrizione generale \\ \hline
	0.0.1 & {\AZ} (Analista) & 27/12/2016 & Stesura scheletro \\ \hline
\end{diario}

\tableofcontents
\newpage



%%% Introduzione

\section{Introduzione}

\subsection{Scopo del documento}
Questo documento (interno al gruppo) regolamenta i processi del progetto didattico e va quindi letto da ciascun membro del gruppo. Le convenzioni qui prescritte servono a:
\begin{itemize}
	\item garantire ordine all'interno dei documenti e delle varie parti della configurazione del prodotto;
	\item mantenere coerenza nelle notazioni e nelle procedure;
	\item minimizzare i conflitti tra i vari ruoli;
	\item garantire che l'infrastruttura di lavoro sia il più possibile semplice e gestibile --- quindi fruibile;
\end{itemize}

\subsection{Scopo del prodotto}
\scopo % da ../../util/hx-macro.sty

	\subsection{Glossario}
	\subsection{Riferimenti}
		\subsubsection{Normativi}
		\begin{itemize}
		\item ISO/IEC 12207-1995: \url{http://www.math.unipd.it/\~tullio/IS-1/2009/Approfondimenti/ISO_12207-1995.pdf}
		\end{itemize}



%%% Processi primari

\section{Processi primari}

\subsection{Fornitura}

	\subsubsection{Studio di fattibilità}
	La fase antecedente alla realizzazione del progetto\ped{g} consiste nel fissare delle riunioni in modo che il team\ped{g} possa discutere di esso. Sarà poi compito degli analisti redigere lo \texttt{studio di fattibilità},la cui struttura è specificata a seguire: %inserire struttura studio di fattibilità

\subsection{Sviluppo}
	\subsubsection{Analisi dei requisiti}
	Sarà sempre compito degli analisti al termine dello studio di fattibilità di produrre l'\texttt{analisi dei requisiti}.Questo documento conterrà i requisiti raccolti ed i casi uso individuati nelle riunioni svolte.
	Viene elencata nelle sezioni a seguire la struttura obbligatoria di requisiti e casi d'uso.
		\paragraph{Requisiti}
		\paragraph{Casi d'uso}
		Ogni caso d'uso viene identificato nel seguente modo:
		\centerline{UCx}
		Ogni macro caso d'uso sarà poi analizzato in modo più specifico ed identificato da UC[p].[f] dove p indica il il caso d'uso padre e f il figlio(ogni figlio potrà a sua volta avere altri figli).
		Inoltre ogni caso'uso dovrà avere le seguenti proprietà:
		\begin{itemize}
		\item Nome: Nome identificativo del caso d'uso;
		\item Attori: attori coinvolti nel caso d'uso;
		\item Descrizione: Descrizione del caso d'uso;
		\item Precondizioni: Condizioni da rispettare prima di eseguire il caso d'uso;%facoltative?
		\item Postcondizioni:  Condizioni da rispettare dopo di eseguire il caso d'uso;%facoltative?
		\item Scenario principale: descrizione del caso d'uso tramite i casi d'uso figli;
		\item Inclusioni:Eventuali inclusioni se specificate;
		\item Estensioni:Eventuali estensioni se specificate;
		\item Scenari alternativi:descrizione tramite casi d'uso non appartenenti al flusso principale.
		\end{itemize}
	\subsubsection{Progettazione}
	Il compito dei progettisti consiste di progettare l'architettura del software. % da completare.
	La fase di progettazione deve utilizzare come diagrammi \gloss{UML} quelli sotto descritti,in riferimento allo standard uml 2.0:
	\begin{itemize}
		\item \gloss{Diagrammi delle classi}:desccrivono i tipi di entità(classi) e le relazioni tra loro.
		\item \gloss{Diagrammi dei package}:raggruppa un numero di elementi uml in una sola unità di livello più alto.
		\item \gloss{Diagrammi di attività}:descrive in dettaglio un algoritmo
		\item \gloss{Diagrammi di sequenza}:descrive uno scenario,dove le azioni sono disposte in sequenza e le varie scelte sono già state prese.
		\end{itemize}
	\subsubsection{Codifica}
	Lo scopo dell'attività di codifica è di implementare quanto descritto nei documenti di definizione di prodotto a livello di codice. Per mantenere un alto grado di comprensione del codice scritto è quindi utile seguire delle linee guida che definiscono lo stanrd di scrittura e documentazione di esso.
		\paragraph{Formattazione del codice}
		Per standardizzare il codice scritto dai vari programmatori è necessario rispettare i seguenti punti:
			\begin{itemize}
				\item i nomi di variabili,classi,funzioni e metodi sono scritti in inglese.
				\item I nomi composti dovranno avere la prima lettera minuscola e la lettera maiuscola ad ogni iniziale della nuova parloa che compone il nome.
			\end{itemize}
		\paragraph{Commenti}
		Per facilitare la comprensione del codice è fondamentale l'uso di commenti. A seguire dei suggerimenti per il loro inserimento:
		\begin{itemize}
			\item È preferibile commentare ogni funzione in modo da poter poi creare una documentazione.
			\item Ogni commento dovrà essere relativo alla parte di codice interessata.
			\item Ogni commetno dovrà decrivere in modo breve ma preciso la parte di codice interessata
		\end{itemize}
	\subsubsection{Strumenti}
		\paragraph{PragmaDB}
		Per tenere traccia dei casi d'uso e dei requisiti il team\ped{g} ha scelto di usare PragmaDB(https://github.com/StefanoMunari/PragmaDB) creato da studenti degli anni passati. Il programma è accessibile via broswer ed è possibile usarlo collegandosi al server creato dal team tramite autenticazione. Inoltre permette l'esportazione in \LaTeX{} delle varie tabelle(requisiti,glossario,...) seguendo le linee guida di struttura descritte in questo documento.
		\paragraph{Astah}
		\paragraph{Ide ancora da scegliere}
	\subsubsection{Tecnologie utilizzate}
	\paragraph{HTML5}
	L'\gloss{html5} è un linguaggio di markup utilizzato per strutturare pagine web. Porta con se molte caratteristiche nuove rispetto al suo predecessore volte soprattutto a separare struttra,stile e contenuti,oltre a nuovi elementi che permettono di creare un sito web molto potente.
	Esso verrà usato per la creazione della parte parte client del programma.
	\paragraph{CSS3}
	\gloss{Css3} è un linguaggio usato per la formattazione di documenti \gloss{html} e \gloss{xml}. Esso consente di definire l'aspetto e la formattazione della pagina. Viene utilizzato nella parte client per arricchire l'aspetto dell'applicativo client.
	\paragraph{Javascript}
	\gloss{Javascript} è un linguaggio orientato agli oggetti e agli eventi utilizzato comunemente per la parte cleint di un sito web. Esso permette di aggiungere effetti dinamici interattivi in base a eventi.
	Viene utilizzato nella parte client per poter avere un edito interattivo utilizzabile.%da riscrivere quest'ultima frase 
	\paragraph{Java tomcat o node.js}



%%% Processi di supporto

\section{Processi di supporto}
I processi normati in questa sezione hanno la funzione di aiutare gli altri processi.

\subsection{Documentazione}
La \gloss{documentazione} di progetto va sviluppata nel \gloss{branch} \texttt{doc} del nostro \gloss{repository}, come motivato nella sezione \ref{sec:config}. I documenti che svilupperemo sono i seguenti:
\begin{itemize}
	\item Documenti interni al gruppo:
	\begin{description}
		\item[studio di fattibilità] vaglia i \emph{pro} e i \emph{contra} di ogni capitolato d'appalto;
		\item[norme di progetto] norma i processi del progetto.
	\end{description}
	\item Documenti rivolti non solo al gruppo ma anche ai committenti e/o ai docenti:
	\begin{description}
		\item[piano di progetto] dichiara come il gruppo intende gestire le risorse umane e temporali nel corso del progetto;
		\item[piano di qualifica] descrive come il gruppo affronta il problema di garantire la qualità del prodotto da fornire;
		\item[analisi dei requisiti] elenca, descrive e traccia i requisiti e i casi d'uso del prodotto da fornire;
		\item[specifica tecnica] descrive l'architettura logica del sistema da fornire, mostrando ciò che il sistema deve fare senza però fissarne i dettagli implementativi (descrive l'interfaccia di ogni componente del sistema, attraverso più livelli gerarchici di decomposizione);
		\item[definizione di prodotto] rispetto alla specifica tecnica, decompone l'architettura in moduli a granda più fine, finché ogni modulo ha dimensione, coesione, complessità e accoppiamento tali per cui i moduli possano essere sviluppati in parallelo dai programmatori;
		\item[glossario] riporta specifici termini che il gruppo ha ritenuto opportuno definire. I motivi che portano ad inserire un termine nel glossario sono la sua potenziale ambiguità e/o la sua natura “tecnica”;
		\item[manuale utente] fornisce all'utente una guida completa ed esaustiva di tutte le funzionalità del prodotto;
		\item[verbali] riporta i verbali delle riunioni del gruppo (sia quelle interne al gruppo sia quelle in presenza del committente), ognuna identificata univocamente dalla data.
	\end{description}
\end{itemize}
Riportiamo qui di seguito le norme che riguardano lo sviluppo della documentazione; quelle enunciate con il condizionale vanno interpretate come consigli.

\subsubsection{Ambiente di lavoro} I documenti andranno mantenuti in una struttura ad albero ben organizzata ma non troppo profonda; il miglior compromesso tra organizzazione e profondità dell'albero è risultato essere quello in figura [figura...].

\subsubsection{Nomi, formati e identificazione} Nome ed estensione dei file sono regolati nel seguente modo:
\paragraph{Nome} Il nome di un file o di una directory non dovrebbe contenere spazi, al fine di facilitare operazioni da riga di comando.
\paragraph{Formato} Ogni documento dev'essere in formato gloss{Latex} con estensione \texttt{.tex} (esportabile in formato PDF su foglio A4); la prima \gloss{versione} di un documento può essere in altri formati ma va presto sostituita da una versione in Latex. I file Latex che definiscano delle \gloss{macro} o diano istruzioni di stile iniziano con la sigla “hx-” (per rimarcare che sono librerie create dal gruppo), hanno estensione \texttt{.sty} (tranne “hx-ambiente”, per permetterne l'inclusione prima di dichiarare la classe del documento) e vanno inclusi con il comando \texttt{\textbackslash usepackage\{\}}.
\paragraph{Identificazione} Ogni versione di un documento è identificata univocamente dal nome del file, composto di una prima stringa esprimente la sua funzione (ad esempio “Piano\_di\_progetto”) e una seconda stringa con il numero di versionamento (ad esempio “\_0.0.2”).
\paragraph{Numero di versionamento} Il numero di versionamento di un documento si compone di tre numeri separati da un punto: il primo viene incremementato ad ogni \textbf{approvazione} del documento da parte del \gloss{responsabile di progetto}; il secondo ad ogni sua \textbf{revisione} da parte di un \gloss{verificatore}; il terzo ad ogni \textbf{aggiunta o modifica} sostanziale apportata da un redattore. L'incremento di un numero azzera i numeri alla destra di esso.

\subsubsection{Struttura di un documento}
\paragraph{Frontespizio} Ogni documento che sia più lungo di una pagina deve iniziare con un frontespizio recante: titolo del documento, versione, nome del nostro gruppo, nome del progetto, data, autori, revisori, uso (interno o esterno), lista di distribuzione. Per fare questo, il redattore deve inserire \texttt{\textbackslash input\{../../util/hx-ambiente\}} all'inizio del documento e aggiustare i parametri con i comandi \texttt{\textbackslash version\{\}}, \texttt{\textbackslash creaz\{\}}, \texttt{\textbackslash author\{\}}, \texttt{\textbackslash supervisor\{\}}, \texttt{\textbackslash uso\{\}}, \texttt{\textbackslash dest\{\}} e \texttt{\textbackslash title\{\}}; per generare il frontespizio, va usato il comando \texttt{\textbackslash maketitle}.
\paragraph{Diario delle modifiche} Ogni documento deve includere un diario delle modifiche che elenca, dal più recente al più vecchio, i cambiamenti apportati a ogni versione rispetto alla precedente, assieme all'autore e alla data di tali cambiamenti. Per questo, il redattore deve includere (dopo il frontespizio) un file intitolato \texttt{diario.tex} nella stessa cartella del documento; al suo interno, deve usare le macro \texttt{\textbackslash begin\{diario\}} e \texttt{\textbackslash end\{diario\}}.
\paragraph{Indice dei contenuti} Documenti con più di tre o quattro sezioni devono riportare, all'inizio, un indice dei contenuti. Per questo, dopo l'inclusione del diario delle modifiche va posto il comando \texttt{\textbackslash tableofcontents}.
\paragraph{Sezionamento} Dev'esserci un'unica struttura gerarchica di sezioni comune a tutta la documentazione: il sezionamento principale si attua con \texttt{\textbackslash section\{\}}; poi si scende nello specifico con \texttt{\textbackslash subsection\{\}}, \texttt{\textbackslash subsubsection\{\}} e infine \texttt{\textbackslash paragraph\{\}}. Questi quattro livelli di sezionamento generano parti di testo che vengono tracciate dall'indice generale e sono quindi atti a contenere parti di testo leggibili “ad accesso casuale” (cioè anche senza dover rileggere parti di testo precedenti); per quelle parti di testo che, pur avendo una propria strutturazione, vanno lette in modo “sequenziale”, si usano invece gli elenchi; questi non vengono riportati nell'indice generale e possono essere di tre tipi:
\begin{description}
	\item[elenchi puntati] ambiente \texttt{itemize} --- elencano gli elementi di un insieme;
	\item[elenchi numerati] ambiente \texttt{enumerate} --- elencano gli elementi di una sequenza;
	\item[elenchi descrittivi] ambiente \texttt{description} --- elencano gli elementi di un dizionario di termini.
\end{description}
\paragraph{Intestazione} Ogni pagina dev'essere numerata e deve riportare in testata il titolo del documento e della sezione che ospita. % e magari il nome/logo del gruppo e il nome del progetto

\subsubsection{Ciclo di vita di un documento} Per ogni documento creato, i redattori stendono una bozza (in Latex) che dev'essere poi controllata dai verificatori; se questi rilevano errori o possibili miglioramenti da apportare, segnalano il fatto al \gloss{responsabile di progetto} --- che provvede a rendicontare le ore di lavoro aggiuntive --- e modificano il documento dopo aver avvisato i redattori di tale documento.

\subsubsection{Termini del glossario} All'interno di una sezione, la prima occorrenza di un termine che si trovi nel glossario dev'essere evidenziata e segnata con un pedice (\gloss{[termine in glossario]}) tramite la macro \texttt{\textbackslash gloss\{termine\}}, in modo da far capire che si trova nel glossario.

\subsubsection{Grammatica e tipografia} Tutti i documenti vanno redatti in italiano. Il manuale utente viene anche tradotto in inglese, data la natura open-source del progetto. I documenti in italiano osservano le usuali regole grammaticali italiane, tra cui evidenziamo:
\begin{itemize}
	\item La punteggiatura è seguita ma non preceduta da uno spazio.
	\item Le parole straniere di uso comune (computer, radar, web\dots) non vanno evidenziate, per non appesantire il testo; vanno invece evidenziate quelle straniere di uso \emph{non} comune, come \emph{framework} o \emph{way of working}. Quando un termine straniero ha una traduzione italiana che non sia troppo strana o artificiosa, va preferita la traduzione in italiano.
	\item Gli elementi di un elenco sono separati da un punto e virgola (quindi iniziano con lettera minuscola), tranne l'ultimo che si conclude con un punto. Nel caso che anche un solo elemento dell'elenco si componga di più frasi separate da segno di punteggiatura forte, tutti gli elementi dell'elenco iniziano per maiuscola e finiscono con un punto. In entrambi i casi, ogni elenco è introdotto dai due punti.
\end{itemize}
L'evidenziazione di un termine va fatta con \texttt{\textbackslash emph\{\}}. Ogni altro dubbio su grammatica o tipografia va chiarito consultando l'appendice A del testo open-source \url{http://www.lorenzopantieri.net/LaTeX_files/ArteLaTeX.pdf}.

\subsubsection{Colori} Andrebbe minimizzato l'uso di colori, al fine di evitare ambiguità nel caso un documento venga stampato in bassa qualità.

\subsection{Configurazione} \label{sec:config}
Per gestire la configurazione del software si è scelto di utilizzare il sistema \gloss{Git}, appoggiandoci per questo alla piattaforma \gloss{Github}. Abbiamo scelto Github non solo per la qualità del servizio fornito ma anche per la natura open-source del progetto \proj, che ci abilitava ad usare questa piattaforma gratuitamente. Il \gloss{repository} creato all'indirizzo \repo{} si divide nelle seguenti directory:
\begin{description}
	\item[\texttt{doc}] per la documentazione;
	\item[\texttt{src}] per il codice sorgente.
\end{description}

\subsubsection{Branching} La vita di un \gloss{branch} (ramo) sottosta alle seguenti regole:
\begin{itemize}
	\item Il ramo \texttt{master} contiene solo elementi della configurazione approvati dal responsabile di progetto.
	\item Il ramo \texttt{doc}, figlio di \texttt{master}, contiene documentazione non ancora approvata dal responsabile di progetto.
	\item Il ramo \texttt{develop}, figlio di \texttt{master}, contiene codice sorgente non ancora approvato dal responsabile di progetto.
	\item Il numero di rami va contenuto, affinché Git rimanga uno strumento e non diventi un peso per i membri del gruppo: la creazione di un nuovo ramo può avvenire solo previa autorizzazione del responsabile di progetto, che dovrà valutarne l'utilità.
	\item Il \emph{merge} di un ramo va sottoposto ad un verificatore (tramite una \emph{merge request}) e va approvata dal responsabile di progetto.
\end{itemize}

\subsection{Qualità} \dots

\subsection{Verifica} \dots

\subsection{Validazione} \dots



%%% Processi organizzativi

\section{Processi organizzativi}

\subsection{Gestione organizzativa}

\subsection{Formazione}

\end{document}
