% Studio di Fattibilita
% da compilare con il comando pdflatex Studio_di_Fattibilita_x.x.x.tex

% Dichiarazioni di ambiente e inclusione di pacchetti
% da usare tramite il comando % Dichiarazioni di ambiente e inclusione di pacchetti
% da usare tramite il comando % Dichiarazioni di ambiente e inclusione di pacchetti
% da usare tramite il comando \input{../../util/hx-ambiente}

\documentclass[a4paper,titlepage]{article}
\usepackage[T1]{fontenc}
\usepackage[utf8]{inputenc}
\usepackage[english,italian]{babel}
\usepackage{microtype}
\usepackage{lmodern}
\usepackage{underscore}
\usepackage{graphicx}
\usepackage{eurosym}
\usepackage{float}
\usepackage{fancyhdr}
\usepackage[table,dvipsnames]{xcolor}
\usepackage{multirow}
\usepackage{longtable}
\usepackage{chngpage}
\usepackage{grffile}
\usepackage[titles]{tocloft}
\usepackage{hyperref}
\hypersetup{hidelinks}

\usepackage{../../util/hx-vers}
\usepackage{../../util/hx-macro}
\usepackage{../../util/hx-front}

% solo se si vuole una nuova pagina ad ogni \section:
\usepackage{titlesec}
\newcommand{\sectionbreak}{\clearpage}

% stile di pagina:
\pagestyle{fancy}

% solo se si vuole eliminare l'indentazione ad ogni paragrafo:
\setlength{\parindent}{0pt}

% intestazione:
\lhead{\Large{\proj}}
\rhead{\includegraphics[keepaspectratio=true,width=50px]{../../util/hivex_logo2.png}}
\renewcommand{\headrulewidth}{0.4pt}

% pie' di pagina:
\lfoot{\email}
\rfoot{\thepage}
\cfoot{}
\renewcommand{\footrulewidth}{0.4pt}

% spazio verticale tra le celle di una tabella:
\renewcommand{\arraystretch}{1.5}

% profondità di indicizzazione:
\setcounter{tocdepth}{4}
\setcounter{secnumdepth}{4}

% numerazione innestata per elenchi numerati:
\renewcommand{\labelenumii}{\theenumii}
\renewcommand{\theenumii}{\theenumi.\arabic{enumii}.}


\documentclass[a4paper,titlepage]{article}
\usepackage[T1]{fontenc}
\usepackage[utf8]{inputenc}
\usepackage[english,italian]{babel}
\usepackage{microtype}
\usepackage{lmodern}
\usepackage{underscore}
\usepackage{graphicx}
\usepackage{eurosym}
\usepackage{float}
\usepackage{fancyhdr}
\usepackage[table,dvipsnames]{xcolor}
\usepackage{multirow}
\usepackage{longtable}
\usepackage{chngpage}
\usepackage{grffile}
\usepackage[titles]{tocloft}
\usepackage{hyperref}
\hypersetup{hidelinks}

\usepackage{../../util/hx-vers}
\usepackage{../../util/hx-macro}
\usepackage{../../util/hx-front}

% solo se si vuole una nuova pagina ad ogni \section:
\usepackage{titlesec}
\newcommand{\sectionbreak}{\clearpage}

% stile di pagina:
\pagestyle{fancy}

% solo se si vuole eliminare l'indentazione ad ogni paragrafo:
\setlength{\parindent}{0pt}

% intestazione:
\lhead{\Large{\proj}}
\rhead{\includegraphics[keepaspectratio=true,width=50px]{../../util/hivex_logo2.png}}
\renewcommand{\headrulewidth}{0.4pt}

% pie' di pagina:
\lfoot{\email}
\rfoot{\thepage}
\cfoot{}
\renewcommand{\footrulewidth}{0.4pt}

% spazio verticale tra le celle di una tabella:
\renewcommand{\arraystretch}{1.5}

% profondità di indicizzazione:
\setcounter{tocdepth}{4}
\setcounter{secnumdepth}{4}

% numerazione innestata per elenchi numerati:
\renewcommand{\labelenumii}{\theenumii}
\renewcommand{\theenumii}{\theenumi.\arabic{enumii}.}


\documentclass[a4paper,titlepage]{article}
\usepackage[T1]{fontenc}
\usepackage[utf8]{inputenc}
\usepackage[english,italian]{babel}
\usepackage{microtype}
\usepackage{lmodern}
\usepackage{underscore}
\usepackage{graphicx}
\usepackage{eurosym}
\usepackage{float}
\usepackage{fancyhdr}
\usepackage[table,dvipsnames]{xcolor}
\usepackage{multirow}
\usepackage{longtable}
\usepackage{chngpage}
\usepackage{grffile}
\usepackage[titles]{tocloft}
\usepackage{hyperref}
\hypersetup{hidelinks}

\usepackage{../../util/hx-vers}
\usepackage{../../util/hx-macro}
\usepackage{../../util/hx-front}

% solo se si vuole una nuova pagina ad ogni \section:
\usepackage{titlesec}
\newcommand{\sectionbreak}{\clearpage}

% stile di pagina:
\pagestyle{fancy}

% solo se si vuole eliminare l'indentazione ad ogni paragrafo:
\setlength{\parindent}{0pt}

% intestazione:
\lhead{\Large{\proj}}
\rhead{\includegraphics[keepaspectratio=true,width=50px]{../../util/hivex_logo2.png}}
\renewcommand{\headrulewidth}{0.4pt}

% pie' di pagina:
\lfoot{\email}
\rfoot{\thepage}
\cfoot{}
\renewcommand{\footrulewidth}{0.4pt}

% spazio verticale tra le celle di una tabella:
\renewcommand{\arraystretch}{1.5}

% profondità di indicizzazione:
\setcounter{tocdepth}{4}
\setcounter{secnumdepth}{4}

% numerazione innestata per elenchi numerati:
\renewcommand{\labelenumii}{\theenumii}
\renewcommand{\theenumii}{\theenumi.\arabic{enumii}.}


\version{0.0.1}
\creaz{20 dicembre 2016}
\author{\AZ}
\supervisor{\LS}
\uso{interno}
\dest{\TV, \ZU}
\title{Studio di Fattibilità}

\usepackage{hyperref}
\hypersetup{hidelinks}

\begin{document}
\maketitle
% diario delle modifiche per l'analisi dei requisiti
% da includere con % diario delle modifiche per l'analisi dei requisiti
% da includere con % diario delle modifiche per l'analisi dei requisiti
% da includere con \include{diario}

\begin{diario}
	4.0.0 & {\LB} (Responsabile) & 02/05/2017 & Approvazione del documento \\ \hline
	3.1.0 & {\PB} (Verificatore) & 02/05/2017 & Verifica del documento \\ \hline
	3.0.1 & {\MM} (Analista) & 01/05/2017 & 
	\begin{itemize}
	\item Inserimento UC5.35 e relativo requisito;
	\item Inserimento UC8 e relativo requisito;
	\item Inserimento tabella Requisiti Implementati come appendice.
\end{itemize}\\ \hline
	3.0.0 & {\AZ} (Responsabile) & 19/03/2017 & Approvazione del documento \\ \hline
	2.1.0 & {\MM} (Verificatore) & 19/03/2017 & Verifica del documento \\ \hline
	2.0.3 & {\PB} (Progettista) & 18/03/2017 &  
\begin{itemize}
	\item Modifica tabella Tracciamento Fonti-Requisiti;
	\item Modifica tabella Requisiti-Fonti;
	\item Modifica Estensione UC7.
\end{itemize}\\ \hline
	2.0.2 & {\PB} (Progettista) & 17/03/2017 &  Ristrutturato UC5 e relativi requisiti\\ \hline
	2.0.1 & {\PB} (Progettista) & 16/03/2017 &  Ristrutturato UC4 e relativi requisiti\\ \hline
	2.0.0 & {\LS} (Responsabile) & 01/02/2017 & Approvazione del documento \\ \hline
	1.1.0 & {\GG} (Verificatore) & 01/02/2017 & Verifica del documento \\ \hline
	1.0.4 & {\AZ} (Analista) & 31/01/2017 & Inserito UC5.26 con relativo requisito e tracciamento nelle tabelle e inseriti i requisiti RFO7, RFO8, RFO8.1, RFO8.2, RFO9, RFO10 e RFO11\\ \hline
	1.0.3 & {\AZ} (Analista) & 29/01/2017 & Corretta la descrizione dello UC5 e approfondita la descrizione dello UC7 \\ \hline
	1.0.2 & {\AZ} (Analista) & 28/01/2017 & Corretti UC4.1.6.3.2, UC4.2.1 e inserito perimetro sistema del UC5\\ \hline
	1.0.1 & {\AZ} (Analista) & 26/01/2017 & Inserimento scenario alternativo allo UC2, creazione UC3.1 con relativo requisito e tracciamento nelle tabelle e corrette alcune postcondizioni \\ \hline
	1.0.0 & {\LB} (Responsabile) & 09/01/2017 & Approvazione documento \\ \hline
	0.4.0 & {\LS} (Verificatore) & 06/01/2017 & Verifica introduzione, descrizione generale e requisiti \\ \hline
	0.3.0 & {\MM} (Verificatore) & 06/01/2017 & Verifica UC5.3-UC7 \\ \hline
	0.2.0 & {\LB} (Verificatore) & 06/01/2017 & Verifica UC4.2-UC5.2 \\ \hline
	0.1.0 & {\AZ} (Verificatore) & 06/01/2017 & Verifica UC1-4.1.8 \\ \hline
	0.0.11 & {\LS} (Analista) & 04/01/2017 & Stesura UC6-UC7 \\ \hline
	0.0.10 & {\GG} (Analista) & 03/01/2017 & Stesura UC5.6-UC5.18 \\ \hline
	0.0.9 & {\LS} (Analista) & 03/01/2017 & Stesura UC5.3-UC5.5.6.1 \\ \hline
	0.0.8 & {\PB} (Analista) & 02/01/2017 & Stesura UC5-UC5.2 \\ \hline
	0.0.7 & {\AZ} (Analista) & 02/01/2017 & Stesura UC4.3.3.1-UC4.11 \\ \hline
	0.0.6 & {\MM} (Analista) & 30/12/2016 & Stesura UC4.2-UC4.3.3.1 \\ \hline
	0.0.5 & {\GG} (Analista) & 29/12/2016 & Stesura UC4.1.6-UC4.1.8 \\ \hline
	0.0.4 & {\PB} (Analista) & 29/12/2016 & Stesura UC4-UC4.1.5 \\ \hline
	0.0.3 & {\LB} (Analista) & 28/12/2016 & Stesura UC1-UC2-UC3 \\ \hline
	0.0.2 & {\LS} (Analista) & 27/12/2016 & Stesura introduzione e descrizione generale \\ \hline
	0.0.1 & {\AZ} (Analista) & 27/12/2016 & Stesura scheletro \\ \hline
\end{diario}


\begin{diario}
	4.0.0 & {\LB} (Responsabile) & 02/05/2017 & Approvazione del documento \\ \hline
	3.1.0 & {\PB} (Verificatore) & 02/05/2017 & Verifica del documento \\ \hline
	3.0.1 & {\MM} (Analista) & 01/05/2017 & 
	\begin{itemize}
	\item Inserimento UC5.35 e relativo requisito;
	\item Inserimento UC8 e relativo requisito;
	\item Inserimento tabella Requisiti Implementati come appendice.
\end{itemize}\\ \hline
	3.0.0 & {\AZ} (Responsabile) & 19/03/2017 & Approvazione del documento \\ \hline
	2.1.0 & {\MM} (Verificatore) & 19/03/2017 & Verifica del documento \\ \hline
	2.0.3 & {\PB} (Progettista) & 18/03/2017 &  
\begin{itemize}
	\item Modifica tabella Tracciamento Fonti-Requisiti;
	\item Modifica tabella Requisiti-Fonti;
	\item Modifica Estensione UC7.
\end{itemize}\\ \hline
	2.0.2 & {\PB} (Progettista) & 17/03/2017 &  Ristrutturato UC5 e relativi requisiti\\ \hline
	2.0.1 & {\PB} (Progettista) & 16/03/2017 &  Ristrutturato UC4 e relativi requisiti\\ \hline
	2.0.0 & {\LS} (Responsabile) & 01/02/2017 & Approvazione del documento \\ \hline
	1.1.0 & {\GG} (Verificatore) & 01/02/2017 & Verifica del documento \\ \hline
	1.0.4 & {\AZ} (Analista) & 31/01/2017 & Inserito UC5.26 con relativo requisito e tracciamento nelle tabelle e inseriti i requisiti RFO7, RFO8, RFO8.1, RFO8.2, RFO9, RFO10 e RFO11\\ \hline
	1.0.3 & {\AZ} (Analista) & 29/01/2017 & Corretta la descrizione dello UC5 e approfondita la descrizione dello UC7 \\ \hline
	1.0.2 & {\AZ} (Analista) & 28/01/2017 & Corretti UC4.1.6.3.2, UC4.2.1 e inserito perimetro sistema del UC5\\ \hline
	1.0.1 & {\AZ} (Analista) & 26/01/2017 & Inserimento scenario alternativo allo UC2, creazione UC3.1 con relativo requisito e tracciamento nelle tabelle e corrette alcune postcondizioni \\ \hline
	1.0.0 & {\LB} (Responsabile) & 09/01/2017 & Approvazione documento \\ \hline
	0.4.0 & {\LS} (Verificatore) & 06/01/2017 & Verifica introduzione, descrizione generale e requisiti \\ \hline
	0.3.0 & {\MM} (Verificatore) & 06/01/2017 & Verifica UC5.3-UC7 \\ \hline
	0.2.0 & {\LB} (Verificatore) & 06/01/2017 & Verifica UC4.2-UC5.2 \\ \hline
	0.1.0 & {\AZ} (Verificatore) & 06/01/2017 & Verifica UC1-4.1.8 \\ \hline
	0.0.11 & {\LS} (Analista) & 04/01/2017 & Stesura UC6-UC7 \\ \hline
	0.0.10 & {\GG} (Analista) & 03/01/2017 & Stesura UC5.6-UC5.18 \\ \hline
	0.0.9 & {\LS} (Analista) & 03/01/2017 & Stesura UC5.3-UC5.5.6.1 \\ \hline
	0.0.8 & {\PB} (Analista) & 02/01/2017 & Stesura UC5-UC5.2 \\ \hline
	0.0.7 & {\AZ} (Analista) & 02/01/2017 & Stesura UC4.3.3.1-UC4.11 \\ \hline
	0.0.6 & {\MM} (Analista) & 30/12/2016 & Stesura UC4.2-UC4.3.3.1 \\ \hline
	0.0.5 & {\GG} (Analista) & 29/12/2016 & Stesura UC4.1.6-UC4.1.8 \\ \hline
	0.0.4 & {\PB} (Analista) & 29/12/2016 & Stesura UC4-UC4.1.5 \\ \hline
	0.0.3 & {\LB} (Analista) & 28/12/2016 & Stesura UC1-UC2-UC3 \\ \hline
	0.0.2 & {\LS} (Analista) & 27/12/2016 & Stesura introduzione e descrizione generale \\ \hline
	0.0.1 & {\AZ} (Analista) & 27/12/2016 & Stesura scheletro \\ \hline
\end{diario}


\begin{diario}
	4.0.0 & {\LB} (Responsabile) & 02/05/2017 & Approvazione del documento \\ \hline
	3.1.0 & {\PB} (Verificatore) & 02/05/2017 & Verifica del documento \\ \hline
	3.0.1 & {\MM} (Analista) & 01/05/2017 & 
	\begin{itemize}
	\item Inserimento UC5.35 e relativo requisito;
	\item Inserimento UC8 e relativo requisito;
	\item Inserimento tabella Requisiti Implementati come appendice.
\end{itemize}\\ \hline
	3.0.0 & {\AZ} (Responsabile) & 19/03/2017 & Approvazione del documento \\ \hline
	2.1.0 & {\MM} (Verificatore) & 19/03/2017 & Verifica del documento \\ \hline
	2.0.3 & {\PB} (Progettista) & 18/03/2017 &  
\begin{itemize}
	\item Modifica tabella Tracciamento Fonti-Requisiti;
	\item Modifica tabella Requisiti-Fonti;
	\item Modifica Estensione UC7.
\end{itemize}\\ \hline
	2.0.2 & {\PB} (Progettista) & 17/03/2017 &  Ristrutturato UC5 e relativi requisiti\\ \hline
	2.0.1 & {\PB} (Progettista) & 16/03/2017 &  Ristrutturato UC4 e relativi requisiti\\ \hline
	2.0.0 & {\LS} (Responsabile) & 01/02/2017 & Approvazione del documento \\ \hline
	1.1.0 & {\GG} (Verificatore) & 01/02/2017 & Verifica del documento \\ \hline
	1.0.4 & {\AZ} (Analista) & 31/01/2017 & Inserito UC5.26 con relativo requisito e tracciamento nelle tabelle e inseriti i requisiti RFO7, RFO8, RFO8.1, RFO8.2, RFO9, RFO10 e RFO11\\ \hline
	1.0.3 & {\AZ} (Analista) & 29/01/2017 & Corretta la descrizione dello UC5 e approfondita la descrizione dello UC7 \\ \hline
	1.0.2 & {\AZ} (Analista) & 28/01/2017 & Corretti UC4.1.6.3.2, UC4.2.1 e inserito perimetro sistema del UC5\\ \hline
	1.0.1 & {\AZ} (Analista) & 26/01/2017 & Inserimento scenario alternativo allo UC2, creazione UC3.1 con relativo requisito e tracciamento nelle tabelle e corrette alcune postcondizioni \\ \hline
	1.0.0 & {\LB} (Responsabile) & 09/01/2017 & Approvazione documento \\ \hline
	0.4.0 & {\LS} (Verificatore) & 06/01/2017 & Verifica introduzione, descrizione generale e requisiti \\ \hline
	0.3.0 & {\MM} (Verificatore) & 06/01/2017 & Verifica UC5.3-UC7 \\ \hline
	0.2.0 & {\LB} (Verificatore) & 06/01/2017 & Verifica UC4.2-UC5.2 \\ \hline
	0.1.0 & {\AZ} (Verificatore) & 06/01/2017 & Verifica UC1-4.1.8 \\ \hline
	0.0.11 & {\LS} (Analista) & 04/01/2017 & Stesura UC6-UC7 \\ \hline
	0.0.10 & {\GG} (Analista) & 03/01/2017 & Stesura UC5.6-UC5.18 \\ \hline
	0.0.9 & {\LS} (Analista) & 03/01/2017 & Stesura UC5.3-UC5.5.6.1 \\ \hline
	0.0.8 & {\PB} (Analista) & 02/01/2017 & Stesura UC5-UC5.2 \\ \hline
	0.0.7 & {\AZ} (Analista) & 02/01/2017 & Stesura UC4.3.3.1-UC4.11 \\ \hline
	0.0.6 & {\MM} (Analista) & 30/12/2016 & Stesura UC4.2-UC4.3.3.1 \\ \hline
	0.0.5 & {\GG} (Analista) & 29/12/2016 & Stesura UC4.1.6-UC4.1.8 \\ \hline
	0.0.4 & {\PB} (Analista) & 29/12/2016 & Stesura UC4-UC4.1.5 \\ \hline
	0.0.3 & {\LB} (Analista) & 28/12/2016 & Stesura UC1-UC2-UC3 \\ \hline
	0.0.2 & {\LS} (Analista) & 27/12/2016 & Stesura introduzione e descrizione generale \\ \hline
	0.0.1 & {\AZ} (Analista) & 27/12/2016 & Stesura scheletro \\ \hline
\end{diario}

\tableofcontents
\newpage

\section{Introduzione}
	\subsection{Scopo del documento}
	Lo scopo del documento è quello di presentare le motivazioni che hanno spinto il gruppo \hx{} a scegliere il capitolato d'appalto C6. 
	Inoltre sarà presente una descrizione degli altri capitolati con le ragioni che hanno portato il gruppo ad escluderli dalla scelta.
	
	\subsection{Scopo del prodotto}
	\scopo{}
	
	\subsection{Glossario}
	\presgloss{}
	
	\subsection{Riferimenti}
		\subsubsection{Normativi}
		\begin{itemize}
			\item \textbf{Norme di Progetto: } \emph{\NdP};
		\end{itemize}
		\subsubsection{Informativi}
		\begin{itemize}
			\item \textbf{Glossario: }\emph{\Glossario};
			\item \textbf{Capitolato d'Appalto C1:} \emph{APIM}: An API Market Platform
			\\ \url{http://www.math.unipd.it/~tullio/IS-1/2016/Progetto/C1.pdf};
			\item \textbf{Capitolato d'Appalto C2:} \emph{AtAVi}: Accoglienza tramite Assistente Virtuale
			\\ \url{http://www.math.unipd.it/~tullio/IS-1/2016/Progetto/C2.pdf};
			\item \textbf{Capitolato d'Appalto C3:} \emph{DeGeOP}: A Designer and Geo-localizer Web App for Organizational Plants
			\\ \url{http://www.math.unipd.it/~tullio/IS-1/2016/Progetto/C3.pdf};
			\item \textbf{Capitolato d'Appalto C4:} \emph{eBread}: applicazione di lettura per dislessici
			\\ \url{http://www.math.unipd.it/~tullio/IS-1/2016/Progetto/C4.pdf};
			\item \textbf{Capitolato d'Appalto C5:} \emph{Monolith}: an interactive bubble provider
			\\ \url{http://www.math.unipd.it/~tullio/IS-1/2016/Progetto/C5.pdf};
			\item \textbf{Capitolato d'Appalto C6:} \emph{SWEDesigner}: editor di diagrammi UML con generazione di codice
			\\ \url{http://www.math.unipd.it/~tullio/IS-1/2016/Progetto/C6.pdf};
		\end{itemize}
\newpage
	
\section{Studio di fattibilità del capitolato scelto}
	\subsection{\emph{SWEDesigner}: editor di diagrammi UML con generazione di codice}
		\subsubsection{Descrizione}
		Lo scopo del capitolato è quello di realizzare un'applicazione web (o all'occorrenza anche un programma Desktop) composta da almeno due disegnatori
		ed un coordinatore, che permetta all'utente di costruire diagrammi UML rappresentativi del prodotto da svilupparsi e generi a partire da essi codice 
		Java o Javascript.
		In particolare l'editor dovrebbe concentrarsi su uno specifico dominio applicativo (l'esempio fornito nel capitolato d'appalto è relativo ai software 
		di giochi da tavolo) per andare a costituire una sorta di framework per la produzione della relativa famiglia di prodotti.
		\subsubsection{Studio del dominio}
			\paragraph{Dominio applicativo}
			L'ambito e le problematiche che il progetto si propone di affrontare sono indubbiamente di grande interesse. Da un lato affronta la necessità di 
			creare strumenti che regimentino il processo di sviluppo per facilitare la produzione di codice di qualità e che permettano di fattorizzare gli 
			aspetti comuni ad una certa famiglia di prodotti software per rendere più agile e standardizzato il relativo sviluppo.
			Dall'altro esplora la possibilità di generare codice in uno specifico linguaggio di programmazione a partire dalla rappresentazione dell'architettura 
			software una serie di convenzioni di modellazione.
			\\In particolare lo scopo profondo del capitolato è quello di analizzare quali diagrammi UML siano i più adatti allo scopo di descrivere l'architettura 
			e le componenti funzionali di un progetto e come si possa concertare una loro integrazione per arrivare a produrre un eseguibile vero e proprio. 
			Odiernamente, infatti, UML non è altro che un grande aggregato di convenzioni rappresentative che sì coprono a diversi livelli di raffinamento diverse 
			fasi e aspetti dello sviluppo di prodotti software (architettura ad alto livello ed in dettaglio, funzionamento di metodi ...) ma al contempo risultano 
			a tal punto eterogenei l'uno rispetto all'altro da non risultare integrabili e componibili. 
			La richiesta è quindi quella di indagare come si potrebbe superare la debolezza intrinseca dell'Unified Modeling Language nella sincronizazione tra 
			codice e quanto disegnato nei diagrammi stessi e sperimentare con la possibilità di avvicinare le due fasi di progettazione e codifica, esplorando le 
			interazioni e le eventuali estensioni necessarie a legare fra loro queste due fasi.
			\paragraph{Dominio tecnologico}
			Il capitolato richiede che il sistema sia realizzato con tecnologie Web. In particolare la parte server deve essere realizzata in Java con il server 
			Tomcat o in Javascript con il server Node.Js. La parte client, invece, deve essere eseguibile in un browser HTML5 ed utilizzare fogli di stile CSS per 
			l'aspetto estetico e Javascript per la parte attiva.
		\subsubsection{Aspetti positivi e negativi}
		Molti sono stati riconosciuti dal gruppo come gli aspetti positivi di tale progetto.
		\\Prima di tutto le problematiche affrontate dal capitolato hanno suscitato in tutti i membri una forte motivazione, legata alla necessità di confrontarsi 
		con problemi - quello della generazione automatica di codice e del raffinamento dell'UML – ancora vivi nell'ambito informatico. In secondo luogo hanno giocato 
		a favore della scelta la chiarezza e precisione con cui le caratteristiche e i requisiti del progetto sono stati elencati e descritti oltre che l'indiscutibile 
		autorevolezza e prestigio dell'azienda proponente. Preziosa, inoltre, è stata considerata la possibilità di collaborare attivamente con quella che è considerata 
		la prima software house italiana e con un esponente di essa legato da così tanto tempo al mondo della programmazione.
		\\Un altro punto a favore del capitolato è la possibilità di operare con un set di tecnologie assodato, parte integrante del programma di studio e di ampia diffusione.
		\\Fra le criticità discusse la più sentita riguarda la vastità e la complessità del problema da affrontare, certamente non banale e necessitante di una riflessione 
		estremamente approfondita e ponderata. A ciò si aggiunge la necessità di conoscere al meglio i costrutti e le astrazioni del linguaggio di generazione del codice, 
		in modo tale da garantire una buona integrazione fra le due fasi di architettura e scrittura di codice.
		\subsubsection{Valutazione finale}
	
\newpage

\section{Studio di fattibilità degli altri capitolati}
	\subsection{Capitolato C1: \emph{APIM}: An API Market Platform}
		\subsubsection{Descrizione}
		Lo scopo di questo capitolato è quello di creare un'applicazione web che rappresenti una sorta di marketplace per la consultazione e 
		condivisione di microservizi. Tra le funzionalità che il prodotto software dovrebbe fornire vi sono la possibilità di registrare le API 
		di un microservizio, consultarle e ritrovare la relativa documentazione e visualizzare i relativi dati tecnici; inoltre è richiesta la 
		funzionalità di associare ad ogni API diverse chiavi d'uso, monitorare il suo utilizzo, limitare il suo uso da parte di utenti in possesso 
		di chiavi scadute o non valide.
		\subsubsection{Studio del dominio}
			\paragraph{Dominio applicativo}
			Come stanno facendo anche grandi aziende operanti nel settore della tecnologia digitale (Netflix ed Amazon, che hanno iniziato ad usare 
			in modo estensivo i microservizi), il team di ItalianaSoftware sta focalizzando i propri sforzi su una nuova visione dell'architettura 
			dei programmi basata non su grandi moduli funzionali bensì su unità minimali - microservizi - in grado di comporsi e cooperare in modo 
			tale da formare aggregati sempre più complessi fino ad arrivare ad un'unica applicazione monolitica. Un'innovazione, questa, che potrebbe 
			trasformare completamente la struttura dei sistemi informativi: da monolitici, con processi e dati difficilmente modificabili, diventerebbero 
			adattivi e flessibili in quanto composti da tanti componenti interdipendenti, ciascuno dei quali offre funzionalità ben delineate.
			\paragraph{Dominio tecnologico}
			Per quanto riguarda il set di tecnologie da utilizzarsi, ItalianaSoftware consente al gruppo (ai gruppi) impegnato sul progetto un buon 
			margine di libertà. Il front end della web application è da svilupparsi con l'ausilio della terna JavaScript, HTML e CSS3 mentre a 
			livello di database è garantita la possibilità di scegliere tra una base di dati SQL o NoSQL.\\
			Inevitabilmente il vincolo principale risiede nel forte suggerimento di utilizzare il linguaggio orientato ai microservizi sviluppato 
			dall'azienda stessa - Jolie - per la rappresentazione delle interfacce e per la creazione dell'API Gateway.
		\subsubsection{Aspetti positivi e negativi}
		Fra le note positive emerse nell'analisi del capitolato vi è innanzitutto la motivazione di confrontarsi con quella che potrebbe essere 
		a tutti gli effetti una vera e propria rivoluzione nell'intendere l'architettura dei sistemi informativi, composti da moduli unitari 
		componibili, riusabili e scalabili.\\
		Inoltre l'azienda, a cui si deve riconoscere un grande coraggio nello spingere verso l'innovazione, si è detta pronta a collaborare attivamente 
		con i team impegnati nel progetto soprattutto nel garantire formazione a proposito del linguaggio proprietario Jolie.\\
		Uno dei fattori chiave nella decisione se partecipare o meno all'appalto di tale progetto è proprio il suo uso; da un lato la necessità di 
		imparare un nuovo linguaggio di programmazione ha sempre un importante valore a livello formativo, dall'altro, però, non è ancora ben chiaro 
		quali siano le reali potenzialità di Jolie per affermarsi nel sovraffollato panorama tecnologico odierno.
		\subsubsection{Valutazione finale}
		Il gruppo \hx{} ha ritenuto opportuno scegliere un capitolato che permettesse il confronto con tecnologie che per i membri fossero nuove,
		ma al contempo già pienamente affermate in quanto a utilizzo, funzionalità, potenzialità. Tale scelta non trova riscontro in questo capitolato.
		
	\subsection{Capitolato C2: \emph{AtAVi}: Accoglienza tramite Assistente Virtuale}
		\subsubsection{Descrizione}
		Lo scopo di questo capitolato è quello di realizzare un applicativo web che permetta una prima accoglienza degli ospiti in visita all'ufficio 
		dell'azienda zero12. Il progetto dovrebbe comporsi di tre parti ben distinte: un'interfaccia web per permettere all'utente di interagire con 
		l'utilizzatore del sistema, un'interfaccia per trasmettere informazioni attraverso il canale comunicativo aziendale - Slack - e un sistema che 
		sfrutti i servizi AWS Lambda per l'interazione con le API dell'assistente virtuale scelto.
		\subsubsection{Studio del dominio}
			\paragraph{Dominio applicativo}
			Il progetto proposto si inserisce all'interno di un quadro di ampio respiro quale la trasformazione digitale dei sistemi aziendali e la 
			presenza sempre più pregnante nella vita quotidiana di applicativi basati proprio sull'interazione con assistenti virtuali, in grado di 
			organizzare e gestire i nostri impegni e adattarsi alle nostre preferenze e richieste.\\
			Il capitolato si configura pertanto sia come un'indagine su ulteriori ambiti di integrazione di tali tecnologie, sia come la dimostrazione 
			di come anche in ambito aziendale esse possano contribuire a rendere più efficiente e moderno l'ambiente lavorativo.
			\paragraph{Dominio tecnologico}
			Fra le tecnologie con cui è richiesto confrontarsi per la realizzazione di tale progetto vi sono:
			\begin{itemize}
				\item I servizi offerti dall'infrastruttura degli Amazon Web Services, ed in particolare le innovazioni fornite dalla Lambda expressions 
				di AWS che consentono di implementare un servizio di elaborazione servless;
				\item Database NoSQL come MongoDN o DynamoDB (per la memorizzazione e l'analisi delle informazioni raccolte nel corso delle interazioni con il cliente);
				\item Linguaggio di programmazione NOdeJS;
				\item Il sistema di comunicazione aziendale Slack;
				\item SDK dei principali assistenti virtuali sul mercato (Cortana, Siri, Google Home).
			\end{itemize}
		\subsubsection{Aspetti positivi e negativi}
		Fra i lati positivi del capitolato figurano l'offerta di formazione da parte dell'azienda sulle principali tecnologie e la possibilità di 
		lavorare con il supporto di un server dedicato configurato tramite i servizi AWS.\\
		Tuttavia si è sottolineato come lo scopo del capitolato consista più nel trovare un modo per concertare e coordinare fra loro diversi sistemi 
		esistenti più che nell'elaborare qualcosa di effettivamente nuovo nell'ambito applicativo.\\
		Fra le criticità individuate dai membri del team vi è inoltre la necessità di confrontarsi con un vasto assortimento di tecnologie non ben 
		conosciute e da riuscire ad indagare a fondo, alcune delle quali - librerie di sviluppo degli assistenti virtuali in primis - di indubbia 
		complessità; tale mancanza di esperienza avrebbe potuto togliere ampio spazio ad una riflessione più approfondita sulle funzionalità e sulla 
		struttura del progetto stesso.\\
		A ciò si aggiunge una mancanza di chiarezza a livello dell'architettura software da realizzare considerata particolarmente problematica da 
		parte di alcuni membri del gruppo.
		\subsubsection{Valutazione finale}
		
	\subsection{Capitolato C3: \emph{DeGeOP}: A Designer and Geo-localizer Web App for Organizational Plants}
		\subsubsection{Descrizione}
		Lo scopo del capitolato in esame è quello di creare un'interfaccia di tipo web app utilizzabile anche su mobile, per visualizzare su di una 
		mappa geografica gli elementi costitutivi del processo produttivo di una particolare azienda (macchinari, magazzini, fornitori, distributori, etc.).
		\\Tale web app deve permettere di:
		\begin{itemize}
			\item definire la struttura organizzativa di un'azienda;
			\item riportare tale struttura a livello di mappa geografica;
			\item disegnare gli scenari di danno che possono colpire l'azienda;
			\item ottenere dei risultati di analisi relativamente a tali scenari di rischio utilizzando il database per le situazioni di danno dell'azienda e il loro server di analisi che opera utilizzando un algoritmo proprietario;
		\end{itemize}
		L'applicazione deve inoltre garantire un corretto funzionamento su desktop, tablet e dispositivi mobile e utilizzare API e database proprietari dell'azienda;
		\subsubsection{Studio del dominio}
			\paragraph{Dominio applicativo}
			Esso costituisce una realtà di attuale interesse, soprattutto in un paese come il nostro dove sempre più spesso le catastrofi climatiche 
			(terremoti, inondazioni, trombe d'aria) mettono in difficoltà intere aziende. Data pertanto la mancanza di una piena consapevolezza 
			riguardo a tale problematica (il rischio relativo alle interruzioni di lavoro è poco conosciuto) e la frequenza con cui questa si presenta 
			il contributo di tale applicazione potrà essere in futuro estremamente significativo.
			\\Un altro punto a favore è costituito dalle aziende che si appoggiano al servizio RiskApp, alcune di notevole importanza nel ramo assicurazioni.
			\\Il servizio offerto dall'applicativo andrebbe inoltre a coprire un vuoto: le aziende di assicurazioni, infatti, non hanno ora strumenti 
			adeguati per valutare il rischio legato all'interruzione dei lavori per cause esterne.
			\paragraph{Dominio tecnologico}
			Nonostante la relativa libertà nell'utilizzo dello stack tecnologico le tecnologie proposte sono:
			\begin{itemize}
				\item Amazon WS, come servizio per l'organizzazione di un server entro cui far operare l'applicativo;
				\item Django, come framework in Python per la realizzazione del front-end dell'applicativo;
				\item PostgreSQL, per la realizzazione del database di appoggio del software;
				\item Librerie e framework quali Bootstrap e React.js per la creazione di interfacce utente, hammer.js e Yeoman.
				\item Html, Css e JavaScript per la realizzazione dell'interfaccia utente.
			\end{itemize}
		\subsubsection{Aspetti positivi e negativi}
		\subsubsection{Valutazione finale}
		Nessun membro del gruppo ha dimostrato particolare interesse o motivazione per il progetto in esame; pertanto sia per la poca chiarezza e 
		nebulosità nella descrizione e nei requisiti del progetto, sia per la presenza di altri capitolati in grado di stimolare una maggiore 
		motivazione, si è deciso di non procedere troppo oltre nell'esame delle caratteristiche di tale proposta.
		
	\subsection{Capitolato C4: \emph{eBread}: applicazione di lettura per dislessici}
		\subsubsection{Descrizione}
		Il progetto prevede la realizzazione di un'applicazione per dispositivi mobili - smartphone e tablet - in ambiente Android che agevoli la 
		lettura da parte di una persona affetta da dislessia grazie all'uso di appropriate tecnologie. La proposta richiede inoltre che l'applicativo, 
		indifferentemente un lettore di ebook o un client di messaggistica, utilizzi un motore di sintesi vocale come ausilio alla lettura, evidenziando 
		le parole del testo, in modo sincronizzato con l'audio sintetizzato. Un'ulteriore funzionalità da fornirsi consiste nella la possibilità di 
		modificare la visualizzazione del contenuto da leggersi attraverso particolari sistemi di colori, spaziature fra lettere, differenti dimensioni 
		e tipo dei caratteri e layout di visualizzazione - tutte agevolazioni preziose per persone affette da dislessia.
		\subsubsection{Studio del dominio}
			\paragraph{Dominio applicativo}
			Le problematiche che il progetto proposto dal capitolato si prevede di affrontare sono sicuramente di grande attualità. Da un lato la 
			dislessia, che nel nostro paese al giorno d'oggi rappresenta il disturbo dell'apprendimento più diffuso e che incide non poco nella 
			vita delle persone che ne sono affette in quanto rende difficoltosa una delle attività fondamentali della routine quotidiana: la lettura. 
			Dall'altro si spinge il gruppo fornitore a confrontarsi con una delle tecnologie che sta avendo negli ultimi tempi una diffusione sempre 
			maggiore, quella del Text-to-Speech, alla base di applicazioni di grande importanza quali screenreader, lettori di messaggi, voci di 
			assistenti virtuali e probabilmente futura interfaccia per tutte quelle attività in cui è precluso, o comunque limitato, l'utilizzo della vista.
			\paragraph{Dominio tecnologico}
			Viene lasciato un ampio margine di libertà per quanto riguarda le tecnologie da impiegarsi, purché risultino adeguate allo scopo. 
			Gli unici due vincoli principali in questo senso sono la realizzazione di una applicazione per dispositivi mobili, preferibilmente per 
			sistema operativo Android, e l'utilizzo di un servizio di sintesi vocale, con consiglio per il motore di sintesi “Flexible and Adaptive 
			Text-To-Speech”.
			\\Nel caso si proceda allo sviluppo di un ebook reader il proponente consiglia di utilizzare una libreria di terze parti preferibilmente 
			open-source per accedere al contenuto; allo stesso modo, nel caso in cui si decida di estendere le funzionalità di un client di messaggistica, 
			si suggerisce di partire da uno già esistente, possibilmente open-source (come Telegram).
		\subsubsection{Aspetti positivi e negativi}
		\subsubsection{Valutazione finale}
		Complice una presentazione del progetto forse sottotono e nebulosa ed una tematica, anche se attuale e viva soprattutto negli ambienti 
		educativi della società odierna, non capace di spingere la motivazione dei membri del team, si è scelto di procedere all'appalto di un 
		differente capitolato.
		
	\subsection{Capitolato C5: \emph{Monolith}: an interactive bubble provider}
		\subsubsection{Descrizione}
		Lo scopo del capitolato è quello di realizzare un framework che consenta agli sviluppatori di creare in modo relativamente semplice delle 
		interactive bubbles in grado di integrarsi con l'ambiente Rocket.chat e che a scopo dimostrativo includa delle bolle interattive predefinite 
		pronte da essere utilizzate da parte degli utenti finali.
		\\Una interactive bubble viene intesa dal proponente come una sorta di mini-applicativo autonomo all'interno del canale di comunicazione 
		che consente di arricchire di funzionalità il client di messaggistica in uso e condividere informazioni complesse e strutturate.

		\subsubsection{Studio del dominio}
			\paragraph{Dominio applicativo}
			Le problematiche affrontate dal presente capitolato risultano sicuramente interessanti ed attuali; si riconosce prima di tutto che al 
			giorno d'oggi il più importante canale comunicativo è costituito dalle applicazioni di chat, utilizzate in ogni ambito della vita 
			quotidiana: per relazionarci con gli amici, per collaborare in ambito lavorativo, per rimanere connessi a distanza. Inoltre sempre 
			più stringente è la necessità di scambiarsi all'interno di tali contesti informazioni più complesse e strutturate di singoli messaggi 
			(come ad esempio dati relativi ad un volo aereo, alla prenotazione in un certo locale, etc.) che rimangano aggiornate e coerenti con 
			le fonti da cui sono state tratte. Le interactive bubbles rappresenterebbero la soluzione ideale a tale problematica, che al giorno 
			d'oggi si risolver invece uno scambio intensivo di numerosi messaggi singoli. 
			\paragraph{Dominio tecnologico}
			Le tecnologie da utilizzarsi nel progetto sono le seguenti:
			\begin{itemize}
				\item Javascript versione 6 per la versione dimostrativa da mostrare al proponente e il framework finale;
				\item Framework quali Angular.js e React.js sono consigliati per la realizzazione del front-end dell'applicazione;
				\item SCSS per l'interfaccia utente.
			\end{itemize}
			Viene inoltre richiesto che il codice sorgente di Monolith e la relativa versione dimostrativa siano pubblicate e versionate utilizzando 
			GitHub o Bitbucket.
		\subsubsection{Aspetti positivi e negativi}
		L'ambito applicativo del progetto risulta sicuramente ricco di spunti interessanti, soprattutto per l'importanza cruciale che i client di 
		messaggistica odierni hanno assunto nello scambio di informazioni.
		\\Un aspetto negativo è rappresentato dalla lontananza dei proponenti che potrebbe incidere su una comunicazione proficua con l'azienda 
		proponente.
		\subsubsection{Valutazione finale}
		La maggior parte dei membri del gruppo non ha dimostrato grande interesse in merito alla proposta, e pertanto si è scelto di procedere 
		all'appalto di un differente capitolato.

\end{document}