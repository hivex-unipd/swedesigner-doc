% Studio di Fattibilita
% da compilare con il comando pdflatex Studio_di_Fattibilita_x.x.x.tex

% Dichiarazioni di ambiente e inclusione di pacchetti
% da usare tramite il comando % Dichiarazioni di ambiente e inclusione di pacchetti
% da usare tramite il comando % Dichiarazioni di ambiente e inclusione di pacchetti
% da usare tramite il comando \input{../../util/hx-ambiente}

\documentclass[a4paper,titlepage]{article}
\usepackage[T1]{fontenc}
\usepackage[utf8]{inputenc}
\usepackage[english,italian]{babel}
\usepackage{microtype}
\usepackage{lmodern}
\usepackage{underscore}
\usepackage{graphicx}
\usepackage{eurosym}
\usepackage{float}
\usepackage{fancyhdr}
\usepackage[table,dvipsnames]{xcolor}
\usepackage{multirow}
\usepackage{longtable}
\usepackage{chngpage}
\usepackage{grffile}
\usepackage[titles]{tocloft}
\usepackage{hyperref}
\hypersetup{hidelinks}

\usepackage{../../util/hx-vers}
\usepackage{../../util/hx-macro}
\usepackage{../../util/hx-front}

% solo se si vuole una nuova pagina ad ogni \section:
\usepackage{titlesec}
\newcommand{\sectionbreak}{\clearpage}

% stile di pagina:
\pagestyle{fancy}

% solo se si vuole eliminare l'indentazione ad ogni paragrafo:
\setlength{\parindent}{0pt}

% intestazione:
\lhead{\Large{\proj}}
\rhead{\includegraphics[keepaspectratio=true,width=50px]{../../util/hivex_logo2.png}}
\renewcommand{\headrulewidth}{0.4pt}

% pie' di pagina:
\lfoot{\email}
\rfoot{\thepage}
\cfoot{}
\renewcommand{\footrulewidth}{0.4pt}

% spazio verticale tra le celle di una tabella:
\renewcommand{\arraystretch}{1.5}

% profondità di indicizzazione:
\setcounter{tocdepth}{4}
\setcounter{secnumdepth}{4}

% numerazione innestata per elenchi numerati:
\renewcommand{\labelenumii}{\theenumii}
\renewcommand{\theenumii}{\theenumi.\arabic{enumii}.}


\documentclass[a4paper,titlepage]{article}
\usepackage[T1]{fontenc}
\usepackage[utf8]{inputenc}
\usepackage[english,italian]{babel}
\usepackage{microtype}
\usepackage{lmodern}
\usepackage{underscore}
\usepackage{graphicx}
\usepackage{eurosym}
\usepackage{float}
\usepackage{fancyhdr}
\usepackage[table,dvipsnames]{xcolor}
\usepackage{multirow}
\usepackage{longtable}
\usepackage{chngpage}
\usepackage{grffile}
\usepackage[titles]{tocloft}
\usepackage{hyperref}
\hypersetup{hidelinks}

\usepackage{../../util/hx-vers}
\usepackage{../../util/hx-macro}
\usepackage{../../util/hx-front}

% solo se si vuole una nuova pagina ad ogni \section:
\usepackage{titlesec}
\newcommand{\sectionbreak}{\clearpage}

% stile di pagina:
\pagestyle{fancy}

% solo se si vuole eliminare l'indentazione ad ogni paragrafo:
\setlength{\parindent}{0pt}

% intestazione:
\lhead{\Large{\proj}}
\rhead{\includegraphics[keepaspectratio=true,width=50px]{../../util/hivex_logo2.png}}
\renewcommand{\headrulewidth}{0.4pt}

% pie' di pagina:
\lfoot{\email}
\rfoot{\thepage}
\cfoot{}
\renewcommand{\footrulewidth}{0.4pt}

% spazio verticale tra le celle di una tabella:
\renewcommand{\arraystretch}{1.5}

% profondità di indicizzazione:
\setcounter{tocdepth}{4}
\setcounter{secnumdepth}{4}

% numerazione innestata per elenchi numerati:
\renewcommand{\labelenumii}{\theenumii}
\renewcommand{\theenumii}{\theenumi.\arabic{enumii}.}


\documentclass[a4paper,titlepage]{article}
\usepackage[T1]{fontenc}
\usepackage[utf8]{inputenc}
\usepackage[english,italian]{babel}
\usepackage{microtype}
\usepackage{lmodern}
\usepackage{underscore}
\usepackage{graphicx}
\usepackage{eurosym}
\usepackage{float}
\usepackage{fancyhdr}
\usepackage[table,dvipsnames]{xcolor}
\usepackage{multirow}
\usepackage{longtable}
\usepackage{chngpage}
\usepackage{grffile}
\usepackage[titles]{tocloft}
\usepackage{hyperref}
\hypersetup{hidelinks}

\usepackage{../../util/hx-vers}
\usepackage{../../util/hx-macro}
\usepackage{../../util/hx-front}

% solo se si vuole una nuova pagina ad ogni \section:
\usepackage{titlesec}
\newcommand{\sectionbreak}{\clearpage}

% stile di pagina:
\pagestyle{fancy}

% solo se si vuole eliminare l'indentazione ad ogni paragrafo:
\setlength{\parindent}{0pt}

% intestazione:
\lhead{\Large{\proj}}
\rhead{\includegraphics[keepaspectratio=true,width=50px]{../../util/hivex_logo2.png}}
\renewcommand{\headrulewidth}{0.4pt}

% pie' di pagina:
\lfoot{\email}
\rfoot{\thepage}
\cfoot{}
\renewcommand{\footrulewidth}{0.4pt}

% spazio verticale tra le celle di una tabella:
\renewcommand{\arraystretch}{1.5}

% profondità di indicizzazione:
\setcounter{tocdepth}{4}
\setcounter{secnumdepth}{4}

% numerazione innestata per elenchi numerati:
\renewcommand{\labelenumii}{\theenumii}
\renewcommand{\theenumii}{\theenumi.\arabic{enumii}.}


\version{0.0.1}
\creaz{20 dicembre 2016}
\author{\AZ}
\supervisor{\LS}
\uso{interno}
\dest{\TV, \ZU}
\title{Studio di Fattibilità}

\usepackage{hyperref}
\hypersetup{hidelinks}

\begin{document}
\maketitle
% diario delle modifiche per l'analisi dei requisiti
% da includere con % diario delle modifiche per l'analisi dei requisiti
% da includere con % diario delle modifiche per l'analisi dei requisiti
% da includere con \include{diario}

\begin{diario}
	4.0.0 & {\LB} (Responsabile) & 02/05/2017 & Approvazione del documento \\ \hline
	3.1.0 & {\PB} (Verificatore) & 02/05/2017 & Verifica del documento \\ \hline
	3.0.1 & {\MM} (Analista) & 01/05/2017 & 
	\begin{itemize}
	\item Inserimento UC5.35 e relativo requisito;
	\item Inserimento UC8 e relativo requisito;
	\item Inserimento tabella Requisiti Implementati come appendice.
\end{itemize}\\ \hline
	3.0.0 & {\AZ} (Responsabile) & 19/03/2017 & Approvazione del documento \\ \hline
	2.1.0 & {\MM} (Verificatore) & 19/03/2017 & Verifica del documento \\ \hline
	2.0.3 & {\PB} (Progettista) & 18/03/2017 &  
\begin{itemize}
	\item Modifica tabella Tracciamento Fonti-Requisiti;
	\item Modifica tabella Requisiti-Fonti;
	\item Modifica Estensione UC7.
\end{itemize}\\ \hline
	2.0.2 & {\PB} (Progettista) & 17/03/2017 &  Ristrutturato UC5 e relativi requisiti\\ \hline
	2.0.1 & {\PB} (Progettista) & 16/03/2017 &  Ristrutturato UC4 e relativi requisiti\\ \hline
	2.0.0 & {\LS} (Responsabile) & 01/02/2017 & Approvazione del documento \\ \hline
	1.1.0 & {\GG} (Verificatore) & 01/02/2017 & Verifica del documento \\ \hline
	1.0.4 & {\AZ} (Analista) & 31/01/2017 & Inserito UC5.26 con relativo requisito e tracciamento nelle tabelle e inseriti i requisiti RFO7, RFO8, RFO8.1, RFO8.2, RFO9, RFO10 e RFO11\\ \hline
	1.0.3 & {\AZ} (Analista) & 29/01/2017 & Corretta la descrizione dello UC5 e approfondita la descrizione dello UC7 \\ \hline
	1.0.2 & {\AZ} (Analista) & 28/01/2017 & Corretti UC4.1.6.3.2, UC4.2.1 e inserito perimetro sistema del UC5\\ \hline
	1.0.1 & {\AZ} (Analista) & 26/01/2017 & Inserimento scenario alternativo allo UC2, creazione UC3.1 con relativo requisito e tracciamento nelle tabelle e corrette alcune postcondizioni \\ \hline
	1.0.0 & {\LB} (Responsabile) & 09/01/2017 & Approvazione documento \\ \hline
	0.4.0 & {\LS} (Verificatore) & 06/01/2017 & Verifica introduzione, descrizione generale e requisiti \\ \hline
	0.3.0 & {\MM} (Verificatore) & 06/01/2017 & Verifica UC5.3-UC7 \\ \hline
	0.2.0 & {\LB} (Verificatore) & 06/01/2017 & Verifica UC4.2-UC5.2 \\ \hline
	0.1.0 & {\AZ} (Verificatore) & 06/01/2017 & Verifica UC1-4.1.8 \\ \hline
	0.0.11 & {\LS} (Analista) & 04/01/2017 & Stesura UC6-UC7 \\ \hline
	0.0.10 & {\GG} (Analista) & 03/01/2017 & Stesura UC5.6-UC5.18 \\ \hline
	0.0.9 & {\LS} (Analista) & 03/01/2017 & Stesura UC5.3-UC5.5.6.1 \\ \hline
	0.0.8 & {\PB} (Analista) & 02/01/2017 & Stesura UC5-UC5.2 \\ \hline
	0.0.7 & {\AZ} (Analista) & 02/01/2017 & Stesura UC4.3.3.1-UC4.11 \\ \hline
	0.0.6 & {\MM} (Analista) & 30/12/2016 & Stesura UC4.2-UC4.3.3.1 \\ \hline
	0.0.5 & {\GG} (Analista) & 29/12/2016 & Stesura UC4.1.6-UC4.1.8 \\ \hline
	0.0.4 & {\PB} (Analista) & 29/12/2016 & Stesura UC4-UC4.1.5 \\ \hline
	0.0.3 & {\LB} (Analista) & 28/12/2016 & Stesura UC1-UC2-UC3 \\ \hline
	0.0.2 & {\LS} (Analista) & 27/12/2016 & Stesura introduzione e descrizione generale \\ \hline
	0.0.1 & {\AZ} (Analista) & 27/12/2016 & Stesura scheletro \\ \hline
\end{diario}


\begin{diario}
	4.0.0 & {\LB} (Responsabile) & 02/05/2017 & Approvazione del documento \\ \hline
	3.1.0 & {\PB} (Verificatore) & 02/05/2017 & Verifica del documento \\ \hline
	3.0.1 & {\MM} (Analista) & 01/05/2017 & 
	\begin{itemize}
	\item Inserimento UC5.35 e relativo requisito;
	\item Inserimento UC8 e relativo requisito;
	\item Inserimento tabella Requisiti Implementati come appendice.
\end{itemize}\\ \hline
	3.0.0 & {\AZ} (Responsabile) & 19/03/2017 & Approvazione del documento \\ \hline
	2.1.0 & {\MM} (Verificatore) & 19/03/2017 & Verifica del documento \\ \hline
	2.0.3 & {\PB} (Progettista) & 18/03/2017 &  
\begin{itemize}
	\item Modifica tabella Tracciamento Fonti-Requisiti;
	\item Modifica tabella Requisiti-Fonti;
	\item Modifica Estensione UC7.
\end{itemize}\\ \hline
	2.0.2 & {\PB} (Progettista) & 17/03/2017 &  Ristrutturato UC5 e relativi requisiti\\ \hline
	2.0.1 & {\PB} (Progettista) & 16/03/2017 &  Ristrutturato UC4 e relativi requisiti\\ \hline
	2.0.0 & {\LS} (Responsabile) & 01/02/2017 & Approvazione del documento \\ \hline
	1.1.0 & {\GG} (Verificatore) & 01/02/2017 & Verifica del documento \\ \hline
	1.0.4 & {\AZ} (Analista) & 31/01/2017 & Inserito UC5.26 con relativo requisito e tracciamento nelle tabelle e inseriti i requisiti RFO7, RFO8, RFO8.1, RFO8.2, RFO9, RFO10 e RFO11\\ \hline
	1.0.3 & {\AZ} (Analista) & 29/01/2017 & Corretta la descrizione dello UC5 e approfondita la descrizione dello UC7 \\ \hline
	1.0.2 & {\AZ} (Analista) & 28/01/2017 & Corretti UC4.1.6.3.2, UC4.2.1 e inserito perimetro sistema del UC5\\ \hline
	1.0.1 & {\AZ} (Analista) & 26/01/2017 & Inserimento scenario alternativo allo UC2, creazione UC3.1 con relativo requisito e tracciamento nelle tabelle e corrette alcune postcondizioni \\ \hline
	1.0.0 & {\LB} (Responsabile) & 09/01/2017 & Approvazione documento \\ \hline
	0.4.0 & {\LS} (Verificatore) & 06/01/2017 & Verifica introduzione, descrizione generale e requisiti \\ \hline
	0.3.0 & {\MM} (Verificatore) & 06/01/2017 & Verifica UC5.3-UC7 \\ \hline
	0.2.0 & {\LB} (Verificatore) & 06/01/2017 & Verifica UC4.2-UC5.2 \\ \hline
	0.1.0 & {\AZ} (Verificatore) & 06/01/2017 & Verifica UC1-4.1.8 \\ \hline
	0.0.11 & {\LS} (Analista) & 04/01/2017 & Stesura UC6-UC7 \\ \hline
	0.0.10 & {\GG} (Analista) & 03/01/2017 & Stesura UC5.6-UC5.18 \\ \hline
	0.0.9 & {\LS} (Analista) & 03/01/2017 & Stesura UC5.3-UC5.5.6.1 \\ \hline
	0.0.8 & {\PB} (Analista) & 02/01/2017 & Stesura UC5-UC5.2 \\ \hline
	0.0.7 & {\AZ} (Analista) & 02/01/2017 & Stesura UC4.3.3.1-UC4.11 \\ \hline
	0.0.6 & {\MM} (Analista) & 30/12/2016 & Stesura UC4.2-UC4.3.3.1 \\ \hline
	0.0.5 & {\GG} (Analista) & 29/12/2016 & Stesura UC4.1.6-UC4.1.8 \\ \hline
	0.0.4 & {\PB} (Analista) & 29/12/2016 & Stesura UC4-UC4.1.5 \\ \hline
	0.0.3 & {\LB} (Analista) & 28/12/2016 & Stesura UC1-UC2-UC3 \\ \hline
	0.0.2 & {\LS} (Analista) & 27/12/2016 & Stesura introduzione e descrizione generale \\ \hline
	0.0.1 & {\AZ} (Analista) & 27/12/2016 & Stesura scheletro \\ \hline
\end{diario}


\begin{diario}
	4.0.0 & {\LB} (Responsabile) & 02/05/2017 & Approvazione del documento \\ \hline
	3.1.0 & {\PB} (Verificatore) & 02/05/2017 & Verifica del documento \\ \hline
	3.0.1 & {\MM} (Analista) & 01/05/2017 & 
	\begin{itemize}
	\item Inserimento UC5.35 e relativo requisito;
	\item Inserimento UC8 e relativo requisito;
	\item Inserimento tabella Requisiti Implementati come appendice.
\end{itemize}\\ \hline
	3.0.0 & {\AZ} (Responsabile) & 19/03/2017 & Approvazione del documento \\ \hline
	2.1.0 & {\MM} (Verificatore) & 19/03/2017 & Verifica del documento \\ \hline
	2.0.3 & {\PB} (Progettista) & 18/03/2017 &  
\begin{itemize}
	\item Modifica tabella Tracciamento Fonti-Requisiti;
	\item Modifica tabella Requisiti-Fonti;
	\item Modifica Estensione UC7.
\end{itemize}\\ \hline
	2.0.2 & {\PB} (Progettista) & 17/03/2017 &  Ristrutturato UC5 e relativi requisiti\\ \hline
	2.0.1 & {\PB} (Progettista) & 16/03/2017 &  Ristrutturato UC4 e relativi requisiti\\ \hline
	2.0.0 & {\LS} (Responsabile) & 01/02/2017 & Approvazione del documento \\ \hline
	1.1.0 & {\GG} (Verificatore) & 01/02/2017 & Verifica del documento \\ \hline
	1.0.4 & {\AZ} (Analista) & 31/01/2017 & Inserito UC5.26 con relativo requisito e tracciamento nelle tabelle e inseriti i requisiti RFO7, RFO8, RFO8.1, RFO8.2, RFO9, RFO10 e RFO11\\ \hline
	1.0.3 & {\AZ} (Analista) & 29/01/2017 & Corretta la descrizione dello UC5 e approfondita la descrizione dello UC7 \\ \hline
	1.0.2 & {\AZ} (Analista) & 28/01/2017 & Corretti UC4.1.6.3.2, UC4.2.1 e inserito perimetro sistema del UC5\\ \hline
	1.0.1 & {\AZ} (Analista) & 26/01/2017 & Inserimento scenario alternativo allo UC2, creazione UC3.1 con relativo requisito e tracciamento nelle tabelle e corrette alcune postcondizioni \\ \hline
	1.0.0 & {\LB} (Responsabile) & 09/01/2017 & Approvazione documento \\ \hline
	0.4.0 & {\LS} (Verificatore) & 06/01/2017 & Verifica introduzione, descrizione generale e requisiti \\ \hline
	0.3.0 & {\MM} (Verificatore) & 06/01/2017 & Verifica UC5.3-UC7 \\ \hline
	0.2.0 & {\LB} (Verificatore) & 06/01/2017 & Verifica UC4.2-UC5.2 \\ \hline
	0.1.0 & {\AZ} (Verificatore) & 06/01/2017 & Verifica UC1-4.1.8 \\ \hline
	0.0.11 & {\LS} (Analista) & 04/01/2017 & Stesura UC6-UC7 \\ \hline
	0.0.10 & {\GG} (Analista) & 03/01/2017 & Stesura UC5.6-UC5.18 \\ \hline
	0.0.9 & {\LS} (Analista) & 03/01/2017 & Stesura UC5.3-UC5.5.6.1 \\ \hline
	0.0.8 & {\PB} (Analista) & 02/01/2017 & Stesura UC5-UC5.2 \\ \hline
	0.0.7 & {\AZ} (Analista) & 02/01/2017 & Stesura UC4.3.3.1-UC4.11 \\ \hline
	0.0.6 & {\MM} (Analista) & 30/12/2016 & Stesura UC4.2-UC4.3.3.1 \\ \hline
	0.0.5 & {\GG} (Analista) & 29/12/2016 & Stesura UC4.1.6-UC4.1.8 \\ \hline
	0.0.4 & {\PB} (Analista) & 29/12/2016 & Stesura UC4-UC4.1.5 \\ \hline
	0.0.3 & {\LB} (Analista) & 28/12/2016 & Stesura UC1-UC2-UC3 \\ \hline
	0.0.2 & {\LS} (Analista) & 27/12/2016 & Stesura introduzione e descrizione generale \\ \hline
	0.0.1 & {\AZ} (Analista) & 27/12/2016 & Stesura scheletro \\ \hline
\end{diario}

\tableofcontents
\newpage

\section{Introduzione}
	\subsection{Scopo del documento}
	%da fare
	
	\subsection{Scopo del prodotto}
	\scopo{}
	
	\subsection{Glossario}
	\presgloss{}
	
	\subsection{Riferimenti}
		\subsubsection{Normativi}
		\begin{itemize}
			\item \textbf{Norme di Progetto: } \emph{\NdP};
		\end{itemize}
		\subsubsection{Informativi}
		\begin{itemize}
			\item \textbf{Glossario: }\emph{\Glossario};
			\item \textbf{Capitolato d'Appalto C1: } \emph{APIM}: An API Market Platform
			\\ \url{http://www.math.unipd.it/~tullio/IS-1/2016/Progetto/C1.pdf};
			\item \textbf{Capitolato d'Appalto C2: } \emph{AtAVi}: Accoglienza tramite Assistente Virtuale
			\\ \url{http://www.math.unipd.it/~tullio/IS-1/2016/Progetto/C2.pdf};
			\item \textbf{Capitolato d'Appalto C3: } \emph{DeGeOP}: A Designer and Geo-localizer Web App for Organizational Plants
			\\ \url{http://www.math.unipd.it/~tullio/IS-1/2016/Progetto/C3.pdf};
			\item \textbf{Capitolato d'Appalto C4: } \emph{eBread}: applicazione di lettura per dislessici
			\\ \url{http://www.math.unipd.it/~tullio/IS-1/2016/Progetto/C4.pdf};
			\item \textbf{Capitolato d'Appalto C5: } \emph{Monolith}: an interactive bubble provider
			\\ \url{http://www.math.unipd.it/~tullio/IS-1/2016/Progetto/C5.pdf};
			\item \textbf{Capitolato d'Appalto C6: } \emph{SWEDesigner}: editor di diagrammi UML con generazione di codice
			\\ \url{http://www.math.unipd.it/~tullio/IS-1/2016/Progetto/C6.pdf};
		\end{itemize}
\newpage
	
\section{Studio di fattibilità del capitolato scelto\\ \emph{SWEDesigner}: editor di diagrammi UML con generazione di codice}
	\subsection{Descrizione}
	\subsection{Studio del dominio}
		\subsubsection{Dominio applicativo}
		\subsubsection{Dominio tecnologico}
	\subsection{Aspetti positivi e negativi}
	\subsection{Valutazione finale}
\newpage

\section{Studio di fattibilità degli altri capitolati}
	\subsection{Capitolato C1: \emph{APIM}: An API Market Platform}
		\subsubsection{Descrizione}
		Lo scopo di questo capitolato è quello di creare un'applicazione web che rappresenti una sorta di marketplace per la consultazione e 
		condivisione di microservizi. Tra le funzionalità che il prodotto software dovrebbe fornire vi sono la possibilità di registrare le API 
		di un microservizio, consultarle e ritrovare la relativa documentazione e visualizzare i relativi dati tecnici; inoltre è richiesta la 
		funzionalità di associare ad ogni API diverse chiavi d'uso, monitorare il suo utilizzo, limitare il suo uso da parte di utenti in possesso 
		di chiavi scadute o non valide.
		\subsubsection{Studio del dominio}
			\paragraph{Dominio applicativo}
			Come stanno facendo anche grandi aziende operanti nel settore della tecnologia digitale (Netflix ed Amazon, che hanno iniziato ad usare 
			in modo estensivo i microservizi), il team di ItalianaSoftware sta focalizzando i propri sforzi su una nuova visione dell'architettura 
			dei programmi basata non su grandi moduli funzionali bensì su unità minimali - microservizi - in grado di comporsi e cooperare in modo 
			tale da formare aggregati sempre più complessi fino ad arrivare ad un'unica applicazione monolitica. Un'innovazione, questa, che potrebbe 
			trasformare completamente la struttura dei sistemi informativi: da monolitici, con processi e dati difficilmente modificabili, diventerebbero 
			adattivi e flessibili in quanto composti da tanti componenti interdipendenti, ciascuno dei quali offre funzionalità ben delineate.
			\paragraph{Dominio tecnologico}
			Per quanto riguarda il set di tecnologie da utilizzarsi, ItalianaSoftware consente al gruppo (ai gruppi) impegnato sul progetto un buon 
			margine di libertà. Il front end della web application è da svilupparsi con l'ausilio della terna JavaScript, HTML e CSS3 mentre a 
			livello di database è garantita la possibilità di scegliere tra una base di dati SQL o NoSQL.\\
			Inevitabilmente il vincolo principale risiede nel forte suggerimento di utilizzare il linguaggio orientato ai microservizi sviluppato 
			dall'azienda stessa - Jolie - per la rappresentazione delle interfacce e per la creazione dell'API Gateway.
		\subsubsection{Aspetti positivi e negativi}
		Fra le note positive emerse nell'analisi del capitolato vi è innanzitutto la motivazione di confrontarsi con quella che potrebbe essere 
		a tutti gli effetti una vera e propria rivoluzione nell'intendere l'architettura dei sistemi informativi, composti da moduli unitari 
		componibili, riusabili e scalabili.\\
		Inoltre l'azienda, a cui si deve riconoscere un grande coraggio nello spingere verso l'innovazione, si è detta pronta a collaborare attivamente 
		con i team impegnati nel progetto soprattutto nel garantire formazione a proposito del linguaggio proprietario Jolie.\\
		Uno dei fattori chiave nella decisione se partecipare o meno all'appalto di tale progetto è proprio il suo uso; da un lato la necessità di 
		imparare un nuovo linguaggio di programmazione ha sempre un importante valore a livello formativo, dall'altro, però, non è ancora ben chiaro 
		quali siano le reali potenzialità di Jolie per affermarsi nel sovraffollato panorama tecnologico odierno.
		
		\subsubsection{Valutazione finale}
		Il gruppo \hx{} ha ritenuto opportuno scegliere un capitolato che permettesse il confronto con tecnologie che per i membri fossero nuove,
		ma al contempo già pienamente affermate in quanto a utilizzo, funzionalità, potenzialità. Tale scelta non trova riscontro in questo capitolato.
		
	\subsection{Capitolato C2: \emph{AtAVi}: Accoglienza tramite Assistente Virtuale}
		\subsubsection{Descrizione}
		Lo scopo di questo capitolato è quello di realizzare un applicativo web che permetta una prima accoglienza degli ospiti in visita all'ufficio 
		dell'azienda zero12. Il progetto dovrebbe comporsi di tre parti ben distinte: un'interfaccia web per permettere all'utente di interagire con 
		l'utilizzatore del sistema, un'interfaccia per trasmettere informazioni attraverso il canale comunicativo aziendale - Slack - e un sistema che 
		sfrutti i servizi AWS Lambda per l'interazione con le API dell'assistente virtuale scelto.
		\subsubsection{Studio del dominio}
			\paragraph{Dominio applicativo}
			Il progetto proposto si inserisce all'interno di un quadro di ampio respiro quale la trasformazione digitale dei sistemi aziendali e la 
			presenza sempre più pregnante nella vita quotidiana di applicativi basati proprio sull'interazione con assistenti virtuali, in grado di 
			organizzare e gestire i nostri impegni e adattarsi alle nostre preferenze e richieste.\\
			Il capitolato si configura pertanto sia come un'indagine su ulteriori ambiti di integrazione di tali tecnologie, sia come la dimostrazione 
			di come anche in ambito aziendale esse possano contribuire a rendere più efficiente e moderno l'ambiente lavorativo.
			\paragraph{Dominio tecnologico}
			Fra le tecnologie con cui è richiesto confrontarsi per la realizzazione di tale progetto vi sono:
			\begin{itemize}
				\item I servizi offerti dall'infrastruttura degli Amazon Web Services, ed in particolare le innovazioni fornite dalla Lambda expressions 
				di AWS che consentono di implementare un servizio di elaborazione servless;
				\item Database NoSQL come MongoDN o DynamoDB (per la memorizzazione e l'analisi delle informazioni raccolte nel corso delle interazioni con il cliente);
				\item Linguaggio di programmazione NOdeJS;
				\item Il sistema di comunicazione aziendale Slack;
				\item SDK dei principali assistenti virtuali sul mercato (Cortana, Siri, Google Home).
			\end{itemize}
		\subsubsection{Aspetti positivi e negativi}
		Fra i lati positivi del capitolato figurano l'offerta di formazione da parte dell'azienda sulle principali tecnologie e la possibilità di 
		lavorare con il supporto di un server dedicato configurato tramite i servizi AWS.\\
		Tuttavia si è sottolineato come lo scopo del capitolato consista più nel trovare un modo per concertare e coordinare fra loro diversi sistemi 
		esistenti più che nell'elaborare qualcosa di effettivamente nuovo nell'ambito applicativo.\\
		Fra le criticità individuate dai membri del team vi è inoltre la necessità di confrontarsi con un vasto assortimento di tecnologie non ben 
		conosciute e da riuscire ad indagare a fondo, alcune delle quali - librerie di sviluppo degli assistenti virtuali in primis - di indubbia 
		complessità; tale mancanza di esperienza avrebbe potuto togliere ampio spazio ad una riflessione più approfondita sulle funzionalità e sulla 
		struttura del progetto stesso.\\
		A ciò si aggiunge una mancanza di chiarezza a livello dell'architettura software da realizzare considerata particolarmente problematica da 
		parte di alcuni membri del gruppo.
		\subsubsection{Valutazione finale}
		%da fare
		
	\subsection{Capitolato C3: \emph{DeGeOP}: A Designer and Geo-localizer Web App for Organizational Plants}
		\subsubsection{Descrizione}
		\subsubsection{Studio del dominio}
			\paragraph{Dominio applicativo}
			\paragraph{Dominio tecnologico}
		\subsubsection{Aspetti positivi e negativi}
		\subsubsection{Valutazione finale}
		
	\subsection{Capitolato C4: \emph{eBread}: applicazione di lettura per dislessici}
		\subsubsection{Descrizione}
		\subsubsection{Studio del dominio}
			\paragraph{Dominio applicativo}
			\paragraph{Dominio tecnologico}
		\subsubsection{Aspetti positivi e negativi}
		\subsubsection{Valutazione finale}
		
	\subsection{Capitolato C5: \emph{Monolith}: an interactive bubble provider}
		\subsubsection{Descrizione}
		\subsubsection{Studio del dominio}
			\paragraph{Dominio applicativo}
			\paragraph{Dominio tecnologico}
		\subsubsection{Aspetti positivi e negativi}
		\subsubsection{Valutazione finale}
		
\end{document}