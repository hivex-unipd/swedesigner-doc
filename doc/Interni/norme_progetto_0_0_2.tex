% Norme di progetto
% da compilare con il comando pdflatex [nome del file].tex

% Dichiarazioni di ambiente e inclusione di pacchetti
% da usare tramite il comando % Dichiarazioni di ambiente e inclusione di pacchetti
% da usare tramite il comando % Dichiarazioni di ambiente e inclusione di pacchetti
% da usare tramite il comando \input{../../util/hx-ambiente}

\documentclass[a4paper,titlepage]{article}
\usepackage[T1]{fontenc}
\usepackage[utf8]{inputenc}
\usepackage[english,italian]{babel}
\usepackage{microtype}
\usepackage{lmodern}
\usepackage{underscore}
\usepackage{graphicx}
\usepackage{eurosym}
\usepackage{float}
\usepackage{fancyhdr}
\usepackage[table,dvipsnames]{xcolor}
\usepackage{multirow}
\usepackage{longtable}
\usepackage{chngpage}
\usepackage{grffile}
\usepackage[titles]{tocloft}
\usepackage{hyperref}
\hypersetup{hidelinks}

\usepackage{../../util/hx-vers}
\usepackage{../../util/hx-macro}
\usepackage{../../util/hx-front}

% solo se si vuole una nuova pagina ad ogni \section:
\usepackage{titlesec}
\newcommand{\sectionbreak}{\clearpage}

% stile di pagina:
\pagestyle{fancy}

% solo se si vuole eliminare l'indentazione ad ogni paragrafo:
\setlength{\parindent}{0pt}

% intestazione:
\lhead{\Large{\proj}}
\rhead{\includegraphics[keepaspectratio=true,width=50px]{../../util/hivex_logo2.png}}
\renewcommand{\headrulewidth}{0.4pt}

% pie' di pagina:
\lfoot{\email}
\rfoot{\thepage}
\cfoot{}
\renewcommand{\footrulewidth}{0.4pt}

% spazio verticale tra le celle di una tabella:
\renewcommand{\arraystretch}{1.5}

% profondità di indicizzazione:
\setcounter{tocdepth}{4}
\setcounter{secnumdepth}{4}

% numerazione innestata per elenchi numerati:
\renewcommand{\labelenumii}{\theenumii}
\renewcommand{\theenumii}{\theenumi.\arabic{enumii}.}


\documentclass[a4paper,titlepage]{article}
\usepackage[T1]{fontenc}
\usepackage[utf8]{inputenc}
\usepackage[english,italian]{babel}
\usepackage{microtype}
\usepackage{lmodern}
\usepackage{underscore}
\usepackage{graphicx}
\usepackage{eurosym}
\usepackage{float}
\usepackage{fancyhdr}
\usepackage[table,dvipsnames]{xcolor}
\usepackage{multirow}
\usepackage{longtable}
\usepackage{chngpage}
\usepackage{grffile}
\usepackage[titles]{tocloft}
\usepackage{hyperref}
\hypersetup{hidelinks}

\usepackage{../../util/hx-vers}
\usepackage{../../util/hx-macro}
\usepackage{../../util/hx-front}

% solo se si vuole una nuova pagina ad ogni \section:
\usepackage{titlesec}
\newcommand{\sectionbreak}{\clearpage}

% stile di pagina:
\pagestyle{fancy}

% solo se si vuole eliminare l'indentazione ad ogni paragrafo:
\setlength{\parindent}{0pt}

% intestazione:
\lhead{\Large{\proj}}
\rhead{\includegraphics[keepaspectratio=true,width=50px]{../../util/hivex_logo2.png}}
\renewcommand{\headrulewidth}{0.4pt}

% pie' di pagina:
\lfoot{\email}
\rfoot{\thepage}
\cfoot{}
\renewcommand{\footrulewidth}{0.4pt}

% spazio verticale tra le celle di una tabella:
\renewcommand{\arraystretch}{1.5}

% profondità di indicizzazione:
\setcounter{tocdepth}{4}
\setcounter{secnumdepth}{4}

% numerazione innestata per elenchi numerati:
\renewcommand{\labelenumii}{\theenumii}
\renewcommand{\theenumii}{\theenumi.\arabic{enumii}.}


\documentclass[a4paper,titlepage]{article}
\usepackage[T1]{fontenc}
\usepackage[utf8]{inputenc}
\usepackage[english,italian]{babel}
\usepackage{microtype}
\usepackage{lmodern}
\usepackage{underscore}
\usepackage{graphicx}
\usepackage{eurosym}
\usepackage{float}
\usepackage{fancyhdr}
\usepackage[table,dvipsnames]{xcolor}
\usepackage{multirow}
\usepackage{longtable}
\usepackage{chngpage}
\usepackage{grffile}
\usepackage[titles]{tocloft}
\usepackage{hyperref}
\hypersetup{hidelinks}

\usepackage{../../util/hx-vers}
\usepackage{../../util/hx-macro}
\usepackage{../../util/hx-front}

% solo se si vuole una nuova pagina ad ogni \section:
\usepackage{titlesec}
\newcommand{\sectionbreak}{\clearpage}

% stile di pagina:
\pagestyle{fancy}

% solo se si vuole eliminare l'indentazione ad ogni paragrafo:
\setlength{\parindent}{0pt}

% intestazione:
\lhead{\Large{\proj}}
\rhead{\includegraphics[keepaspectratio=true,width=50px]{../../util/hivex_logo2.png}}
\renewcommand{\headrulewidth}{0.4pt}

% pie' di pagina:
\lfoot{\email}
\rfoot{\thepage}
\cfoot{}
\renewcommand{\footrulewidth}{0.4pt}

% spazio verticale tra le celle di una tabella:
\renewcommand{\arraystretch}{1.5}

% profondità di indicizzazione:
\setcounter{tocdepth}{4}
\setcounter{secnumdepth}{4}

% numerazione innestata per elenchi numerati:
\renewcommand{\labelenumii}{\theenumii}
\renewcommand{\theenumii}{\theenumi.\arabic{enumii}.}


\title{Norme di progetto}
\vers{0.0.2}
\creaz{16 dicembre 2016}
\author{\GG, ?}
\supervisor{?, ?}
\uso{interno}
\dest{Membri del gruppo}

\begin{document}

\maketitle
\tableofcontents



\section{Diario delle modifiche}

\begin{tabular}{c | c | c | p{0.5\textwidth}}
	\textbf{Versione} & \textbf{Data} & \textbf{Autore} & \textbf{Modifiche} \\ \hline
	0.0.2 & 24/12/2016 & \GG & Trascritto il documento da formato .txt a formato .tex \\ \hline
	0.0.1 & 16/12/2016 & \GG & Creato il documento, per ora contenente solo alcune norme sulla documentazione \\ \hline
\end{tabular}



\section{Introduzione}

\dots + altre sezioni \dots



\section{Bozza - Norme sulla documentazione}

\begin{itemize}
	\item il nome di un file non deve contenere spazi 
	\item ogni documento approvato dev'essere in formato .tex (esportabile in .pdf A4)
	\item ogni documento, anche se breve, deve avere una pagina di copertina con: titolo del documento, identificatore univoco (che specifichi anche la versione), nome del nostro gruppo, nome del progetto, versione, data, autori, revisori, lista di distribuzione... (magari con il nostro logo)
	\item corrispondenza biunivoca tra l'identificatore del documento e il titolo del file .tex (?)
	\item ogni pagina deve essere numerata e deve riportare il titolo del documento e del capitolo (e anche, magari, il nome/logo del gruppo e il nome del progetto)
	\item dev'esserci un'unica struttura di sezioni, sottosezioni e paragrafi, comune a tutti i documenti [dello stesso tipo (?)]
	\item i documenti lunghi o troppo tecnici devono riportare, all'inizio, un indice dei contenuti
	\item nel malaugurato caso che la documentazione debba essere stampata in bianco e nero, bisogna trovare un modo per tradurre in simboli i colori dei diagrammi
	\item all'interno di un documento, la prima occorrenza di un termine che si trovi nel glossario dev'essere segnata in modo da far capire che si trova nel glossario
	\item mantenere i documenti in una struttura a directory che sia ben organizzata ma possibilmente non troppo profonda
	\item ogni documento potrebbe includere un "diario delle modifiche" che spiega i cambiamenti rispetto alla versione precedente (?)
	\item il numero di versionamento si compone di tre numeri [...] (approvazione, revisione, modifica)
	\item 
	\item 
	\item 
	\item 
\end{itemize}

\end{document}
