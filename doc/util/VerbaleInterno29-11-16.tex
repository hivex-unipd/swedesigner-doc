% Dichiarazioni di ambiente e inclusione di pacchetti
% da usare tramite il comando % Dichiarazioni di ambiente e inclusione di pacchetti
% da usare tramite il comando % Dichiarazioni di ambiente e inclusione di pacchetti
% da usare tramite il comando \input{../../util/hx-ambiente}

\documentclass[a4paper,titlepage]{article}
\usepackage[T1]{fontenc}
\usepackage[utf8]{inputenc}
\usepackage[english,italian]{babel}
\usepackage{microtype}
\usepackage{lmodern}
\usepackage{underscore}
\usepackage{graphicx}
\usepackage{eurosym}
\usepackage{float}
\usepackage{fancyhdr}
\usepackage[table,dvipsnames]{xcolor}
\usepackage{multirow}
\usepackage{longtable}
\usepackage{chngpage}
\usepackage{grffile}
\usepackage[titles]{tocloft}
\usepackage{hyperref}
\hypersetup{hidelinks}

\usepackage{../../util/hx-vers}
\usepackage{../../util/hx-macro}
\usepackage{../../util/hx-front}

% solo se si vuole una nuova pagina ad ogni \section:
\usepackage{titlesec}
\newcommand{\sectionbreak}{\clearpage}

% stile di pagina:
\pagestyle{fancy}

% solo se si vuole eliminare l'indentazione ad ogni paragrafo:
\setlength{\parindent}{0pt}

% intestazione:
\lhead{\Large{\proj}}
\rhead{\includegraphics[keepaspectratio=true,width=50px]{../../util/hivex_logo2.png}}
\renewcommand{\headrulewidth}{0.4pt}

% pie' di pagina:
\lfoot{\email}
\rfoot{\thepage}
\cfoot{}
\renewcommand{\footrulewidth}{0.4pt}

% spazio verticale tra le celle di una tabella:
\renewcommand{\arraystretch}{1.5}

% profondità di indicizzazione:
\setcounter{tocdepth}{4}
\setcounter{secnumdepth}{4}

% numerazione innestata per elenchi numerati:
\renewcommand{\labelenumii}{\theenumii}
\renewcommand{\theenumii}{\theenumi.\arabic{enumii}.}


\documentclass[a4paper,titlepage]{article}
\usepackage[T1]{fontenc}
\usepackage[utf8]{inputenc}
\usepackage[english,italian]{babel}
\usepackage{microtype}
\usepackage{lmodern}
\usepackage{underscore}
\usepackage{graphicx}
\usepackage{eurosym}
\usepackage{float}
\usepackage{fancyhdr}
\usepackage[table,dvipsnames]{xcolor}
\usepackage{multirow}
\usepackage{longtable}
\usepackage{chngpage}
\usepackage{grffile}
\usepackage[titles]{tocloft}
\usepackage{hyperref}
\hypersetup{hidelinks}

\usepackage{../../util/hx-vers}
\usepackage{../../util/hx-macro}
\usepackage{../../util/hx-front}

% solo se si vuole una nuova pagina ad ogni \section:
\usepackage{titlesec}
\newcommand{\sectionbreak}{\clearpage}

% stile di pagina:
\pagestyle{fancy}

% solo se si vuole eliminare l'indentazione ad ogni paragrafo:
\setlength{\parindent}{0pt}

% intestazione:
\lhead{\Large{\proj}}
\rhead{\includegraphics[keepaspectratio=true,width=50px]{../../util/hivex_logo2.png}}
\renewcommand{\headrulewidth}{0.4pt}

% pie' di pagina:
\lfoot{\email}
\rfoot{\thepage}
\cfoot{}
\renewcommand{\footrulewidth}{0.4pt}

% spazio verticale tra le celle di una tabella:
\renewcommand{\arraystretch}{1.5}

% profondità di indicizzazione:
\setcounter{tocdepth}{4}
\setcounter{secnumdepth}{4}

% numerazione innestata per elenchi numerati:
\renewcommand{\labelenumii}{\theenumii}
\renewcommand{\theenumii}{\theenumi.\arabic{enumii}.}


\documentclass[a4paper,titlepage]{article}
% \usepackage{accanthis} % scegliere il font da http://www.tug.dk/FontCatalogue/
\usepackage[T1]{fontenc}
\usepackage[utf8]{inputenc}
\usepackage[english,italian]{babel}
\usepackage{microtype}
\usepackage{lmodern}
\usepackage{underscore}
\usepackage{graphicx}
\usepackage{hyperref}

\usepackage[table]{xcolor}

\usepackage{../../util/hx-vers}
\usepackage{../../util/hx-macro}
\usepackage{../../util/hx-front}

\renewcommand{\arraystretch}{1.5}
\setcounter{tocdepth}{4}
\setcounter{secnumdepth}{4}


\begin{document}
\section{Sommario}

  Documento contenente il primo verbale interno del 29 Novembre 2016 per il progetto SWEDesigner del gruppo Hivex.

\section{Informazioni generali}
\begin{itemize}
\item {Data incontro:} 2016-11-29;
\item {Ora inizio incontro:} 13:30;
\item {Ora fine incontro:} 15:30;
\item {Luogo incontro:} 1C150, Torre Archimede; 
\item {Durata:} 2 Ore;
\item {Oggetto:} scelta del nome e del logo del gruppo, del capitolato d’appalto, degli strumenti per l’organizzazione e il versionamento che si andranno a utilizzare ""e la definizione dei ruoli"";
\item {Segretario:} 	\LB; 
\item {Partecipanti:} \GR.
\end{itemize}

\section{Riassunto della riunione}
\subsection{Descrizione} 
Durante la riunione sono stati discussi i seguenti argomenti: il nome e il logo del gruppo, il capitolato da scegliere, ""le modalità con le quali verranno distribuiti i ruoli ai vari componenti del gruppo"" e gli strumenti che verranno utilizzati per l’organizzazione ed il versionamento.
\subsection{Decisioni prese} 
\begin{itemize}
\item Scelta del nome del gruppo: dopo varie proposte è stato scelto all’unaminità il seguente nome: Hivex; 
\item Si è scelto il capitolato C6, SWEDesigner proposto dall’azienda \ZU; 
\item ""Si è discusso su come distribuire i ruoli ai vari componenti del team\ped{G}. Questa decisione è stata riportata ed è consultabile nel file Ruoli.txt in Google Drive\ped{G}"";
\item Sono stati definiti gli strumenti e i servizi che si andranno ad utilizzare nel corso del progetto. Come strumento di organizzazione abbiamo scelto di utilizzare Asana\ped{G} in quanto offre per esempio l' integrazione con Slack\ped{G} e si è reso necessario integrare questo strumento ad Instagantt\ped{G}, allo scopo di visualizzare in maniera immediata tramite diagrammi di Gantt\ped{G} tutti i task inseriti in Asana\ped{G}. Prima di scegliere Asana avevamo rivolto la nostra attenzione a Trello\ped{G} ma durante l'analisi delle features offerte sono sorte varie criticità. Per la comunicazione abbiamo scelto di utilizzare Slack\ped{G}: esso permette la suddivisione in diversi canali e l'integrazione di diversi servizi, come l'interfacciamento ai servizi di Asana\ped{G}. Per quanto riguarda il versionamento, abbiamo scelto di utilizzare  GitHub\ped{G} perchè offre un ottimo servizio gratuitamente.

\end{itemize}


\end{document}