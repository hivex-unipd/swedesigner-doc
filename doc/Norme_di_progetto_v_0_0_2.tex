% norme di progetto
% da compilare con il comando pdflatex Norme_di_progetto_v_x.x.x.tex

% Dichiarazioni di ambiente e inclusione di pacchetti
% da usare tramite il comando % Dichiarazioni di ambiente e inclusione di pacchetti
% da usare tramite il comando % Dichiarazioni di ambiente e inclusione di pacchetti
% da usare tramite il comando \input{../../util/hx-ambiente}

\documentclass[a4paper,titlepage]{article}
\usepackage[T1]{fontenc}
\usepackage[utf8]{inputenc}
\usepackage[english,italian]{babel}
\usepackage{microtype}
\usepackage{lmodern}
\usepackage{underscore}
\usepackage{graphicx}
\usepackage{eurosym}
\usepackage{float}
\usepackage{fancyhdr}
\usepackage[table,dvipsnames]{xcolor}
\usepackage{multirow}
\usepackage{longtable}
\usepackage{chngpage}
\usepackage{grffile}
\usepackage[titles]{tocloft}
\usepackage{hyperref}
\hypersetup{hidelinks}

\usepackage{../../util/hx-vers}
\usepackage{../../util/hx-macro}
\usepackage{../../util/hx-front}

% solo se si vuole una nuova pagina ad ogni \section:
\usepackage{titlesec}
\newcommand{\sectionbreak}{\clearpage}

% stile di pagina:
\pagestyle{fancy}

% solo se si vuole eliminare l'indentazione ad ogni paragrafo:
\setlength{\parindent}{0pt}

% intestazione:
\lhead{\Large{\proj}}
\rhead{\includegraphics[keepaspectratio=true,width=50px]{../../util/hivex_logo2.png}}
\renewcommand{\headrulewidth}{0.4pt}

% pie' di pagina:
\lfoot{\email}
\rfoot{\thepage}
\cfoot{}
\renewcommand{\footrulewidth}{0.4pt}

% spazio verticale tra le celle di una tabella:
\renewcommand{\arraystretch}{1.5}

% profondità di indicizzazione:
\setcounter{tocdepth}{4}
\setcounter{secnumdepth}{4}

% numerazione innestata per elenchi numerati:
\renewcommand{\labelenumii}{\theenumii}
\renewcommand{\theenumii}{\theenumi.\arabic{enumii}.}


\documentclass[a4paper,titlepage]{article}
\usepackage[T1]{fontenc}
\usepackage[utf8]{inputenc}
\usepackage[english,italian]{babel}
\usepackage{microtype}
\usepackage{lmodern}
\usepackage{underscore}
\usepackage{graphicx}
\usepackage{eurosym}
\usepackage{float}
\usepackage{fancyhdr}
\usepackage[table,dvipsnames]{xcolor}
\usepackage{multirow}
\usepackage{longtable}
\usepackage{chngpage}
\usepackage{grffile}
\usepackage[titles]{tocloft}
\usepackage{hyperref}
\hypersetup{hidelinks}

\usepackage{../../util/hx-vers}
\usepackage{../../util/hx-macro}
\usepackage{../../util/hx-front}

% solo se si vuole una nuova pagina ad ogni \section:
\usepackage{titlesec}
\newcommand{\sectionbreak}{\clearpage}

% stile di pagina:
\pagestyle{fancy}

% solo se si vuole eliminare l'indentazione ad ogni paragrafo:
\setlength{\parindent}{0pt}

% intestazione:
\lhead{\Large{\proj}}
\rhead{\includegraphics[keepaspectratio=true,width=50px]{../../util/hivex_logo2.png}}
\renewcommand{\headrulewidth}{0.4pt}

% pie' di pagina:
\lfoot{\email}
\rfoot{\thepage}
\cfoot{}
\renewcommand{\footrulewidth}{0.4pt}

% spazio verticale tra le celle di una tabella:
\renewcommand{\arraystretch}{1.5}

% profondità di indicizzazione:
\setcounter{tocdepth}{4}
\setcounter{secnumdepth}{4}

% numerazione innestata per elenchi numerati:
\renewcommand{\labelenumii}{\theenumii}
\renewcommand{\theenumii}{\theenumi.\arabic{enumii}.}


\documentclass[a4paper,titlepage]{article}
\usepackage[T1]{fontenc}
\usepackage[utf8]{inputenc}
\usepackage[english,italian]{babel}
\usepackage{microtype}
\usepackage{lmodern}
\usepackage{underscore}
\usepackage{graphicx}
\usepackage{eurosym}
\usepackage{float}
\usepackage{fancyhdr}
\usepackage[table,dvipsnames]{xcolor}
\usepackage{multirow}
\usepackage{longtable}
\usepackage{chngpage}
\usepackage{grffile}
\usepackage[titles]{tocloft}
\usepackage{hyperref}
\hypersetup{hidelinks}

\usepackage{../../util/hx-vers}
\usepackage{../../util/hx-macro}
\usepackage{../../util/hx-front}

% solo se si vuole una nuova pagina ad ogni \section:
\usepackage{titlesec}
\newcommand{\sectionbreak}{\clearpage}

% stile di pagina:
\pagestyle{fancy}

% solo se si vuole eliminare l'indentazione ad ogni paragrafo:
\setlength{\parindent}{0pt}

% intestazione:
\lhead{\Large{\proj}}
\rhead{\includegraphics[keepaspectratio=true,width=50px]{../../util/hivex_logo2.png}}
\renewcommand{\headrulewidth}{0.4pt}

% pie' di pagina:
\lfoot{\email}
\rfoot{\thepage}
\cfoot{}
\renewcommand{\footrulewidth}{0.4pt}

% spazio verticale tra le celle di una tabella:
\renewcommand{\arraystretch}{1.5}

% profondità di indicizzazione:
\setcounter{tocdepth}{4}
\setcounter{secnumdepth}{4}

% numerazione innestata per elenchi numerati:
\renewcommand{\labelenumii}{\theenumii}
\renewcommand{\theenumii}{\theenumi.\arabic{enumii}.}

\version{0.0.1}
\creaz{24 dicembre 2016}
\author{\GG, \MM}
\supervisor{\LB, \AZ}
\uso{interno}
\dest{Tutti i membri del gruppo}
\title{Norme di progetto}

\renewcommand{\arraystretch}{1.5}
\setcounter{tocdepth}{4}
\setcounter{secnumdepth}{4}

\begin{document}
\maketitle
\section*{Registro delle modifiche}
\begin{figure}[htb]
	\centering
	\begin{tabular}{cp{3cm}cp{3cm}}
	Versione & Autore e Ruolo       & Data       & Descrizione \\ \hline
	0.0.2    & {\GG} Amministratore & 25/12/2016 & Iniziata la stesura dello scopo del documento e abbozzate alcune norme sulla documentazione \\ \hline
	0.0.1    & {\MM} Amministratore & 24/12/2016 & Stesura scheletro \\ \hline
	\end{tabular}
\end{figure}
\tableofcontents

\section{Introduzione}

\subsection{Scopo del documento}
Questo documento (interno al gruppo) regolamenta i processi del progetto didattico e va quindi letto da ciascun membro del gruppo. Le convenzioni qui prescritte servono a:
\begin{itemize}
	\item garantire ordine all'interno dei documenti e delle varie parti della configurazione del prodotto;
	\item mantenere coerenza nelle notazioni e nelle procedure;
	\item minimizzare i conflitti tra i vari ruoli;
	\item garantire che l'infrastruttura di lavoro sia il più possibile semplice e gestibile --- quindi fruibile;
\end{itemize}

\subsection{Scopo del prodotto}
% [macro da utilizzare in tutti documenti, come proposto da Lucab]

	\subsection{Glossario}
	\subsection{Riferimenti}
		\subsubsection{Normativi}
\section{Processi di sviluppo}
	\subsection{Studio di fattibilità}
	\subsection{Analisi dei requisiti}
		\subsubsection{Requisiti}
		\subsubsection{Casi d'uso}
	\subsection{Progettazione}
	\subsection{Codifica}
		\subsubsection{Formattazione del codice}
	\subsection{Strumenti}
		\subsubsection{PragmaDB}
		\subsubsection{Astah}
		\subsubsection{Ide ancora da scegliere}
	\subsection{Tecnologie utilizzate}
\section{Processi di supporto}

\subsection{Processi di documentazione}
Bozza.
\begin{itemize}
	\item il nome di un file non deve contenere spazi 
	\item ogni documento approvato dev'essere in formato .tex (esportabile in .pdf A4)
	\item ogni documento, anche se breve, deve avere una pagina di copertina con: titolo del documento, identificatore univoco (che specifichi anche la versione), nome del nostro gruppo, nome del progetto, versione, data, autori, revisori, lista di distribuzione... (magari con il nostro logo)
	\item corrispondenza biunivoca tra l'identificatore del documento e il titolo del file .tex (?)
	\item ogni pagina deve essere numerata e deve riportare il titolo del documento e del capitolo (e anche, magari, il nome/logo del gruppo e il nome del progetto)
	\item dev'esserci un'unica struttura di sezioni, sottosezioni e paragrafi, comune a tutti i documenti [dello stesso tipo (?)]
	\item i documenti lunghi o troppo tecnici devono riportare, all'inizio, un indice dei contenuti
	\item nel malaugurato caso che la documentazione debba essere stampata in bianco e nero, bisogna trovare un modo per tradurre in simboli i colori dei diagrammi
	\item all'interno di un documento, la prima occorrenza di un termine che si trovi nel glossario dev'essere segnata in modo da far capire che si trova nel glossario
	\item mantenere i documenti in una struttura a directory che sia ben organizzata ma possibilmente non troppo profonda
	\item ogni documento potrebbe includere un "diario delle modifiche" che spiega i cambiamenti rispetto alla versione precedente (?)
	\item il numero di versionamento si compone di tre numeri [...] (approvazione, revisione, modifica)
\end{itemize}

		\subsubsection{Strumenti}
		\subsubsection{Ciclo di vita di un documento}
		\subsubsection{Documenti finali}
			\paragraph{Studio di fattibilità}
			\paragraph{Norme di progetto}
			\paragraph{Piano di progetto}
			\paragraph{Piano di qualifica}
			\paragraph{Analisi dei requisiti}
			\paragraph{Specifica tecnica}
			\paragraph{Definizione di prodotto}
			\paragraph{Glossario}
			\paragraph{Manuale utente}
			\paragraph{Verbali}
	\subsection{Processi di verifica}
	\subsection{Strumenti}
\section{Processi organizzativi}
	\subsection{Strumenti}
\end{document}
