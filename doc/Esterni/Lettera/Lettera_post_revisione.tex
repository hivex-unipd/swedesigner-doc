\documentclass[a4paper]{letter} % non possiamo usare hx-ambiente
\usepackage{accanthis} % scegliere da http://www.tug.dk/FontCatalogue/
\usepackage[T1]{fontenc}
\usepackage[utf8]{inputenc}
\usepackage[italian]{babel}
\usepackage{microtype}
\usepackage{eurosym}
\usepackage{siunitx}
\usepackage{graphicx}
\usepackage{hyperref}
\usepackage{../../util/hx-macro}
\usepackage{../../util/hx-vers}
\date{Padova, \today}
\signature{\LB\\Responsabile \hx{}\\
\includegraphics[scale=1]{../Piano_di_progetto/img/firmalb.jpg}}

\begin{document}

\begin{letter}{Egr. Prof. Tullio Vardanega\\
Università degli Studi di Padova\\
Dipartimento di Matematica “Tullio Levi-Civita”\\
via Trieste, 63\\
35121 Padova}

\includegraphics[width=80px]{../../util/hivex_logo3.png}

\opening{Egregio Professor Vardanega,}

Con la presente, il gruppo \hx{} Le comunica che ha rimediato alla non-conformità presente nel piano di progetto inviatoLe in data 11 gennaio, impegnandosi a rispettare il vincolo di ore previsto.

In particolare, abbiamo deciso di rivalutare gli sforzi necessari per la progettazione, trovando la relativa preventivazione troppo pessimistica: il ruolo di progettista, infatti, è quello che richiede più del doppio di ore rispetto a qualsiasi altro ruolo all'interno del progetto. Abbiamo dunque rivisto al ribasso le ore di lavoro pianificate per alcuni componenti del gruppo, le quali comunque ammontano in totale a 16 ore, in modo da rispettare il vincolo totale di 105 ore/persona rendicontate massime. Non è stato dunque necessario diminuire il numero di requisiti del nostro prodottoM confidiamo nel fatto che questa sia stata una buona scelta.

Ci impegniamo, dunque, a rispettare gli aggiustamenti apportati al piano di progetto, che Lei potrà leggere assieme agli altri documenti entro la data della prossima revisione.

\closing{Distinti Saluti,}

\end{letter}

\end{document}
