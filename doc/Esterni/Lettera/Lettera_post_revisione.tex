\documentclass[a4paper]{letter} % non possiamo usare hx-ambiente
\usepackage{accanthis} % scegliere da http://www.tug.dk/FontCatalogue/
\usepackage[T1]{fontenc}
\usepackage[utf8]{inputenc}
\usepackage[italian]{babel}
\usepackage{microtype}
\usepackage{eurosym}
\usepackage{siunitx}
\usepackage{graphicx}
\usepackage{hyperref}
\usepackage{../../util/hx-macro}
\usepackage{../../util/hx-vers}
\date{Padova, \today}
\signature{\LB\\Responsabile \hx{}\\
\includegraphics[scale=1]{../Piano_di_progetto/img/firmalb.jpg}}

\begin{document}

\begin{letter}{Egr. Prof. Tullio Vardanega\\
Università degli Studi di Padova\\
Dipartimento di Matematica “Tullio Levi-Civita”\\
via Trieste, 63\\
35121 Padova}

\includegraphics[width=80px]{../../util/hivex_logo3.png}

\opening{Egregio Professor Vardanega,}

Con la presente, il gruppo \hx{} Le comunica che ha rimediato alle non-conformità presenti nel Piano di Progetto inviatoLe in data 11 gennaio (documento \PdP).

In particolare, abbiamo rivalutato gli sforzi necessari per la progettazione. Infatti, in questo periodo era previsto che ogni ruolo richiedesse almeno il doppio di ore rispetto agli stessi ruoli negli altri periodi del progetto; abbiamo quindi rivisto al ribasso le ore di lavoro pianificate per alcuni componenti del gruppo, al fine di rientrare nel massimo numero di ore spendibili nel progetto didattico (105 ore a persona).

\closing{Distinti Saluti,}

\end{letter}

\end{document}
