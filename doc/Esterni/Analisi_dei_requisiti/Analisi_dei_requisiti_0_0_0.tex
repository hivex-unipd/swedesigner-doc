% Analisi dei Requisiti
% da compilare con il comando pdflatex Analisi_dei_Requisiti_x.x.x.tex

% Dichiarazioni di ambiente e inclusione di pacchetti
% da usare tramite il comando % Dichiarazioni di ambiente e inclusione di pacchetti
% da usare tramite il comando % Dichiarazioni di ambiente e inclusione di pacchetti
% da usare tramite il comando \input{../../util/hx-ambiente}

\documentclass[a4paper,titlepage]{article}
\usepackage[T1]{fontenc}
\usepackage[utf8]{inputenc}
\usepackage[english,italian]{babel}
\usepackage{microtype}
\usepackage{lmodern}
\usepackage{underscore}
\usepackage{graphicx}
\usepackage{eurosym}
\usepackage{float}
\usepackage{fancyhdr}
\usepackage[table,dvipsnames]{xcolor}
\usepackage{multirow}
\usepackage{longtable}
\usepackage{chngpage}
\usepackage{grffile}
\usepackage[titles]{tocloft}
\usepackage{hyperref}
\hypersetup{hidelinks}

\usepackage{../../util/hx-vers}
\usepackage{../../util/hx-macro}
\usepackage{../../util/hx-front}

% solo se si vuole una nuova pagina ad ogni \section:
\usepackage{titlesec}
\newcommand{\sectionbreak}{\clearpage}

% stile di pagina:
\pagestyle{fancy}

% solo se si vuole eliminare l'indentazione ad ogni paragrafo:
\setlength{\parindent}{0pt}

% intestazione:
\lhead{\Large{\proj}}
\rhead{\includegraphics[keepaspectratio=true,width=50px]{../../util/hivex_logo2.png}}
\renewcommand{\headrulewidth}{0.4pt}

% pie' di pagina:
\lfoot{\email}
\rfoot{\thepage}
\cfoot{}
\renewcommand{\footrulewidth}{0.4pt}

% spazio verticale tra le celle di una tabella:
\renewcommand{\arraystretch}{1.5}

% profondità di indicizzazione:
\setcounter{tocdepth}{4}
\setcounter{secnumdepth}{4}

% numerazione innestata per elenchi numerati:
\renewcommand{\labelenumii}{\theenumii}
\renewcommand{\theenumii}{\theenumi.\arabic{enumii}.}


\documentclass[a4paper,titlepage]{article}
\usepackage[T1]{fontenc}
\usepackage[utf8]{inputenc}
\usepackage[english,italian]{babel}
\usepackage{microtype}
\usepackage{lmodern}
\usepackage{underscore}
\usepackage{graphicx}
\usepackage{eurosym}
\usepackage{float}
\usepackage{fancyhdr}
\usepackage[table,dvipsnames]{xcolor}
\usepackage{multirow}
\usepackage{longtable}
\usepackage{chngpage}
\usepackage{grffile}
\usepackage[titles]{tocloft}
\usepackage{hyperref}
\hypersetup{hidelinks}

\usepackage{../../util/hx-vers}
\usepackage{../../util/hx-macro}
\usepackage{../../util/hx-front}

% solo se si vuole una nuova pagina ad ogni \section:
\usepackage{titlesec}
\newcommand{\sectionbreak}{\clearpage}

% stile di pagina:
\pagestyle{fancy}

% solo se si vuole eliminare l'indentazione ad ogni paragrafo:
\setlength{\parindent}{0pt}

% intestazione:
\lhead{\Large{\proj}}
\rhead{\includegraphics[keepaspectratio=true,width=50px]{../../util/hivex_logo2.png}}
\renewcommand{\headrulewidth}{0.4pt}

% pie' di pagina:
\lfoot{\email}
\rfoot{\thepage}
\cfoot{}
\renewcommand{\footrulewidth}{0.4pt}

% spazio verticale tra le celle di una tabella:
\renewcommand{\arraystretch}{1.5}

% profondità di indicizzazione:
\setcounter{tocdepth}{4}
\setcounter{secnumdepth}{4}

% numerazione innestata per elenchi numerati:
\renewcommand{\labelenumii}{\theenumii}
\renewcommand{\theenumii}{\theenumi.\arabic{enumii}.}


\documentclass[a4paper,titlepage]{article}
\usepackage[T1]{fontenc}
\usepackage[utf8]{inputenc}
\usepackage[english,italian]{babel}
\usepackage{microtype}
\usepackage{lmodern}
\usepackage{underscore}
\usepackage{graphicx}
\usepackage{eurosym}
\usepackage{float}
\usepackage{fancyhdr}
\usepackage[table,dvipsnames]{xcolor}
\usepackage{multirow}
\usepackage{longtable}
\usepackage{chngpage}
\usepackage{grffile}
\usepackage[titles]{tocloft}
\usepackage{hyperref}
\hypersetup{hidelinks}

\usepackage{../../util/hx-vers}
\usepackage{../../util/hx-macro}
\usepackage{../../util/hx-front}

% solo se si vuole una nuova pagina ad ogni \section:
\usepackage{titlesec}
\newcommand{\sectionbreak}{\clearpage}

% stile di pagina:
\pagestyle{fancy}

% solo se si vuole eliminare l'indentazione ad ogni paragrafo:
\setlength{\parindent}{0pt}

% intestazione:
\lhead{\Large{\proj}}
\rhead{\includegraphics[keepaspectratio=true,width=50px]{../../util/hivex_logo2.png}}
\renewcommand{\headrulewidth}{0.4pt}

% pie' di pagina:
\lfoot{\email}
\rfoot{\thepage}
\cfoot{}
\renewcommand{\footrulewidth}{0.4pt}

% spazio verticale tra le celle di una tabella:
\renewcommand{\arraystretch}{1.5}

% profondità di indicizzazione:
\setcounter{tocdepth}{4}
\setcounter{secnumdepth}{4}

% numerazione innestata per elenchi numerati:
\renewcommand{\labelenumii}{\theenumii}
\renewcommand{\theenumii}{\theenumi.\arabic{enumii}.}

\usepackage{hyperref}
\usepackage{longtable}
\hypersetup{hidelinks}

\version{0.0.0}
\creaz{19 dicembre 2016}
\author{\ALL}
\supervisor{\ALL}
\uso{esterno}
\dest{Prof. Tullio Vardanega, Zucchetti S.P.A.}
\title{Analisi dei Requisiti}

\renewcommand{\arraystretch}{1.5}
\setcounter{tocdepth}{4}
\setcounter{secnumdepth}{4}

\begin{document}
\maketitle
% diario delle modifiche per l'analisi dei requisiti
% da includere con % diario delle modifiche per l'analisi dei requisiti
% da includere con % diario delle modifiche per l'analisi dei requisiti
% da includere con \include{diario}

\begin{diario}
	4.0.0 & {\LB} (Responsabile) & 02/05/2017 & Approvazione del documento \\ \hline
	3.1.0 & {\PB} (Verificatore) & 02/05/2017 & Verifica del documento \\ \hline
	3.0.1 & {\MM} (Analista) & 01/05/2017 & 
	\begin{itemize}
	\item Inserimento UC5.35 e relativo requisito;
	\item Inserimento UC8 e relativo requisito;
	\item Inserimento tabella Requisiti Implementati come appendice.
\end{itemize}\\ \hline
	3.0.0 & {\AZ} (Responsabile) & 19/03/2017 & Approvazione del documento \\ \hline
	2.1.0 & {\MM} (Verificatore) & 19/03/2017 & Verifica del documento \\ \hline
	2.0.3 & {\PB} (Progettista) & 18/03/2017 &  
\begin{itemize}
	\item Modifica tabella Tracciamento Fonti-Requisiti;
	\item Modifica tabella Requisiti-Fonti;
	\item Modifica Estensione UC7.
\end{itemize}\\ \hline
	2.0.2 & {\PB} (Progettista) & 17/03/2017 &  Ristrutturato UC5 e relativi requisiti\\ \hline
	2.0.1 & {\PB} (Progettista) & 16/03/2017 &  Ristrutturato UC4 e relativi requisiti\\ \hline
	2.0.0 & {\LS} (Responsabile) & 01/02/2017 & Approvazione del documento \\ \hline
	1.1.0 & {\GG} (Verificatore) & 01/02/2017 & Verifica del documento \\ \hline
	1.0.4 & {\AZ} (Analista) & 31/01/2017 & Inserito UC5.26 con relativo requisito e tracciamento nelle tabelle e inseriti i requisiti RFO7, RFO8, RFO8.1, RFO8.2, RFO9, RFO10 e RFO11\\ \hline
	1.0.3 & {\AZ} (Analista) & 29/01/2017 & Corretta la descrizione dello UC5 e approfondita la descrizione dello UC7 \\ \hline
	1.0.2 & {\AZ} (Analista) & 28/01/2017 & Corretti UC4.1.6.3.2, UC4.2.1 e inserito perimetro sistema del UC5\\ \hline
	1.0.1 & {\AZ} (Analista) & 26/01/2017 & Inserimento scenario alternativo allo UC2, creazione UC3.1 con relativo requisito e tracciamento nelle tabelle e corrette alcune postcondizioni \\ \hline
	1.0.0 & {\LB} (Responsabile) & 09/01/2017 & Approvazione documento \\ \hline
	0.4.0 & {\LS} (Verificatore) & 06/01/2017 & Verifica introduzione, descrizione generale e requisiti \\ \hline
	0.3.0 & {\MM} (Verificatore) & 06/01/2017 & Verifica UC5.3-UC7 \\ \hline
	0.2.0 & {\LB} (Verificatore) & 06/01/2017 & Verifica UC4.2-UC5.2 \\ \hline
	0.1.0 & {\AZ} (Verificatore) & 06/01/2017 & Verifica UC1-4.1.8 \\ \hline
	0.0.11 & {\LS} (Analista) & 04/01/2017 & Stesura UC6-UC7 \\ \hline
	0.0.10 & {\GG} (Analista) & 03/01/2017 & Stesura UC5.6-UC5.18 \\ \hline
	0.0.9 & {\LS} (Analista) & 03/01/2017 & Stesura UC5.3-UC5.5.6.1 \\ \hline
	0.0.8 & {\PB} (Analista) & 02/01/2017 & Stesura UC5-UC5.2 \\ \hline
	0.0.7 & {\AZ} (Analista) & 02/01/2017 & Stesura UC4.3.3.1-UC4.11 \\ \hline
	0.0.6 & {\MM} (Analista) & 30/12/2016 & Stesura UC4.2-UC4.3.3.1 \\ \hline
	0.0.5 & {\GG} (Analista) & 29/12/2016 & Stesura UC4.1.6-UC4.1.8 \\ \hline
	0.0.4 & {\PB} (Analista) & 29/12/2016 & Stesura UC4-UC4.1.5 \\ \hline
	0.0.3 & {\LB} (Analista) & 28/12/2016 & Stesura UC1-UC2-UC3 \\ \hline
	0.0.2 & {\LS} (Analista) & 27/12/2016 & Stesura introduzione e descrizione generale \\ \hline
	0.0.1 & {\AZ} (Analista) & 27/12/2016 & Stesura scheletro \\ \hline
\end{diario}


\begin{diario}
	4.0.0 & {\LB} (Responsabile) & 02/05/2017 & Approvazione del documento \\ \hline
	3.1.0 & {\PB} (Verificatore) & 02/05/2017 & Verifica del documento \\ \hline
	3.0.1 & {\MM} (Analista) & 01/05/2017 & 
	\begin{itemize}
	\item Inserimento UC5.35 e relativo requisito;
	\item Inserimento UC8 e relativo requisito;
	\item Inserimento tabella Requisiti Implementati come appendice.
\end{itemize}\\ \hline
	3.0.0 & {\AZ} (Responsabile) & 19/03/2017 & Approvazione del documento \\ \hline
	2.1.0 & {\MM} (Verificatore) & 19/03/2017 & Verifica del documento \\ \hline
	2.0.3 & {\PB} (Progettista) & 18/03/2017 &  
\begin{itemize}
	\item Modifica tabella Tracciamento Fonti-Requisiti;
	\item Modifica tabella Requisiti-Fonti;
	\item Modifica Estensione UC7.
\end{itemize}\\ \hline
	2.0.2 & {\PB} (Progettista) & 17/03/2017 &  Ristrutturato UC5 e relativi requisiti\\ \hline
	2.0.1 & {\PB} (Progettista) & 16/03/2017 &  Ristrutturato UC4 e relativi requisiti\\ \hline
	2.0.0 & {\LS} (Responsabile) & 01/02/2017 & Approvazione del documento \\ \hline
	1.1.0 & {\GG} (Verificatore) & 01/02/2017 & Verifica del documento \\ \hline
	1.0.4 & {\AZ} (Analista) & 31/01/2017 & Inserito UC5.26 con relativo requisito e tracciamento nelle tabelle e inseriti i requisiti RFO7, RFO8, RFO8.1, RFO8.2, RFO9, RFO10 e RFO11\\ \hline
	1.0.3 & {\AZ} (Analista) & 29/01/2017 & Corretta la descrizione dello UC5 e approfondita la descrizione dello UC7 \\ \hline
	1.0.2 & {\AZ} (Analista) & 28/01/2017 & Corretti UC4.1.6.3.2, UC4.2.1 e inserito perimetro sistema del UC5\\ \hline
	1.0.1 & {\AZ} (Analista) & 26/01/2017 & Inserimento scenario alternativo allo UC2, creazione UC3.1 con relativo requisito e tracciamento nelle tabelle e corrette alcune postcondizioni \\ \hline
	1.0.0 & {\LB} (Responsabile) & 09/01/2017 & Approvazione documento \\ \hline
	0.4.0 & {\LS} (Verificatore) & 06/01/2017 & Verifica introduzione, descrizione generale e requisiti \\ \hline
	0.3.0 & {\MM} (Verificatore) & 06/01/2017 & Verifica UC5.3-UC7 \\ \hline
	0.2.0 & {\LB} (Verificatore) & 06/01/2017 & Verifica UC4.2-UC5.2 \\ \hline
	0.1.0 & {\AZ} (Verificatore) & 06/01/2017 & Verifica UC1-4.1.8 \\ \hline
	0.0.11 & {\LS} (Analista) & 04/01/2017 & Stesura UC6-UC7 \\ \hline
	0.0.10 & {\GG} (Analista) & 03/01/2017 & Stesura UC5.6-UC5.18 \\ \hline
	0.0.9 & {\LS} (Analista) & 03/01/2017 & Stesura UC5.3-UC5.5.6.1 \\ \hline
	0.0.8 & {\PB} (Analista) & 02/01/2017 & Stesura UC5-UC5.2 \\ \hline
	0.0.7 & {\AZ} (Analista) & 02/01/2017 & Stesura UC4.3.3.1-UC4.11 \\ \hline
	0.0.6 & {\MM} (Analista) & 30/12/2016 & Stesura UC4.2-UC4.3.3.1 \\ \hline
	0.0.5 & {\GG} (Analista) & 29/12/2016 & Stesura UC4.1.6-UC4.1.8 \\ \hline
	0.0.4 & {\PB} (Analista) & 29/12/2016 & Stesura UC4-UC4.1.5 \\ \hline
	0.0.3 & {\LB} (Analista) & 28/12/2016 & Stesura UC1-UC2-UC3 \\ \hline
	0.0.2 & {\LS} (Analista) & 27/12/2016 & Stesura introduzione e descrizione generale \\ \hline
	0.0.1 & {\AZ} (Analista) & 27/12/2016 & Stesura scheletro \\ \hline
\end{diario}


\begin{diario}
	4.0.0 & {\LB} (Responsabile) & 02/05/2017 & Approvazione del documento \\ \hline
	3.1.0 & {\PB} (Verificatore) & 02/05/2017 & Verifica del documento \\ \hline
	3.0.1 & {\MM} (Analista) & 01/05/2017 & 
	\begin{itemize}
	\item Inserimento UC5.35 e relativo requisito;
	\item Inserimento UC8 e relativo requisito;
	\item Inserimento tabella Requisiti Implementati come appendice.
\end{itemize}\\ \hline
	3.0.0 & {\AZ} (Responsabile) & 19/03/2017 & Approvazione del documento \\ \hline
	2.1.0 & {\MM} (Verificatore) & 19/03/2017 & Verifica del documento \\ \hline
	2.0.3 & {\PB} (Progettista) & 18/03/2017 &  
\begin{itemize}
	\item Modifica tabella Tracciamento Fonti-Requisiti;
	\item Modifica tabella Requisiti-Fonti;
	\item Modifica Estensione UC7.
\end{itemize}\\ \hline
	2.0.2 & {\PB} (Progettista) & 17/03/2017 &  Ristrutturato UC5 e relativi requisiti\\ \hline
	2.0.1 & {\PB} (Progettista) & 16/03/2017 &  Ristrutturato UC4 e relativi requisiti\\ \hline
	2.0.0 & {\LS} (Responsabile) & 01/02/2017 & Approvazione del documento \\ \hline
	1.1.0 & {\GG} (Verificatore) & 01/02/2017 & Verifica del documento \\ \hline
	1.0.4 & {\AZ} (Analista) & 31/01/2017 & Inserito UC5.26 con relativo requisito e tracciamento nelle tabelle e inseriti i requisiti RFO7, RFO8, RFO8.1, RFO8.2, RFO9, RFO10 e RFO11\\ \hline
	1.0.3 & {\AZ} (Analista) & 29/01/2017 & Corretta la descrizione dello UC5 e approfondita la descrizione dello UC7 \\ \hline
	1.0.2 & {\AZ} (Analista) & 28/01/2017 & Corretti UC4.1.6.3.2, UC4.2.1 e inserito perimetro sistema del UC5\\ \hline
	1.0.1 & {\AZ} (Analista) & 26/01/2017 & Inserimento scenario alternativo allo UC2, creazione UC3.1 con relativo requisito e tracciamento nelle tabelle e corrette alcune postcondizioni \\ \hline
	1.0.0 & {\LB} (Responsabile) & 09/01/2017 & Approvazione documento \\ \hline
	0.4.0 & {\LS} (Verificatore) & 06/01/2017 & Verifica introduzione, descrizione generale e requisiti \\ \hline
	0.3.0 & {\MM} (Verificatore) & 06/01/2017 & Verifica UC5.3-UC7 \\ \hline
	0.2.0 & {\LB} (Verificatore) & 06/01/2017 & Verifica UC4.2-UC5.2 \\ \hline
	0.1.0 & {\AZ} (Verificatore) & 06/01/2017 & Verifica UC1-4.1.8 \\ \hline
	0.0.11 & {\LS} (Analista) & 04/01/2017 & Stesura UC6-UC7 \\ \hline
	0.0.10 & {\GG} (Analista) & 03/01/2017 & Stesura UC5.6-UC5.18 \\ \hline
	0.0.9 & {\LS} (Analista) & 03/01/2017 & Stesura UC5.3-UC5.5.6.1 \\ \hline
	0.0.8 & {\PB} (Analista) & 02/01/2017 & Stesura UC5-UC5.2 \\ \hline
	0.0.7 & {\AZ} (Analista) & 02/01/2017 & Stesura UC4.3.3.1-UC4.11 \\ \hline
	0.0.6 & {\MM} (Analista) & 30/12/2016 & Stesura UC4.2-UC4.3.3.1 \\ \hline
	0.0.5 & {\GG} (Analista) & 29/12/2016 & Stesura UC4.1.6-UC4.1.8 \\ \hline
	0.0.4 & {\PB} (Analista) & 29/12/2016 & Stesura UC4-UC4.1.5 \\ \hline
	0.0.3 & {\LB} (Analista) & 28/12/2016 & Stesura UC1-UC2-UC3 \\ \hline
	0.0.2 & {\LS} (Analista) & 27/12/2016 & Stesura introduzione e descrizione generale \\ \hline
	0.0.1 & {\AZ} (Analista) & 27/12/2016 & Stesura scheletro \\ \hline
\end{diario}

\tableofcontents
\newpage
\section{Introduzione}
	\subsection{Scopo del documento}
	Questo documento ha lo scopo di elencare, classificare e tracciare i requisiti del prodotto \proj{} come identificati dal gruppo \hx{}. 
	\\L'individuazione di tali requisiti si è svolta dapprima tramite l'analisi del relativo capitolato d'appalto e successivamente attraverso una serie di incontri interni al team ed esterni alla presenza del proponente.
	\\Con tale documento il gruppo \hx{} si impegna a sviluppare un prodotto con implementate le caratteristiche di seguito elencate.

	\subsection{Scopo del prodotto}
	\scopo{}
	
	\subsection{Glossario}
	\presgloss{}
	
	\subsection{Riferimenti}
		\subsubsection{Normativi}
		\begin{itemize}
			\item \emph{\NdP};
			\item \textbf{Capitolato d'Appalto C6: \proj}:
			\\ \url{http://www.math.unipd.it/~tullio/IS-1/2016/Progetto/C6p.pdf};
		\end{itemize}
		
		\subsubsection{Informativi}
		\begin{itemize}
			\item \textbf{Studio di Fattibilità: }\emph{\SdF};
			\item \textbf{Diagrammi dei casi d'uso:}
			\\ \url{http://www.math.unipd.it/~tullio/IS-1/2016/Dispense/E01b.pdf};
			\item \textbf{Analisi dei requisiti:}
			\\ \url{http://www.math.unipd.it/~tullio/IS-1/2016/Dispense/L06.pdf};
		\end{itemize}
\newpage

\section{Descrizione generale}
	\subsection{Obiettivo del prodotto}
	Il prodotto sviluppato  ha come scopo primario quello di fornire un editor in grado di regimentare e guidare l'utente nella progettazione e nella codifica dell'applicativo desiderato e generare codice eseguibile sulla base delle informazioni contenute nei diagrammi realizzati. 
	\\In particolare, il software risulterà ottimizzato per la realizzazione di programmi appartenenti ad uno specifico dominio applicativo, quello dei giochi da tavolo.
	\subsection{Funzioni del prodotto}
	In dettaglio, le funzionalità e le caratteristiche del prodotto sviluppato garantite dal team \hx{} sono le seguenti:
	\subsection{Caratteristiche degli utenti}
	Il prodotto \proj{} si rivolge principalmente ad una categoria di utenti ben specifica, quella dei programmatori o, in generale, dei lavoratori impiegati nell'ambito informatico con una conoscenza approfondita e strutturata della programmazione.
	\\In particolare è richiesto che l'utente dell'editor abbia un certo grado di familiarità con:
		\begin{itemize}
			\item Le principali convenzioni della modellazione \gloss{UML}, e più in specifico con i diagrammi delle classi e delle attività;
			\item Le astrazioni e i paradigmi della programmazione orientata agli oggetti; infatti il linguaggio target di generazione del codice, i diagrammi e le funzionalità offerte suggeriscono un'architettura dell'applicativo da realizzarsi ben strutturata in moduli funzionali specifici e distinti.
		\end{itemize} 
	Per garantire inoltre una comprensione accurata e completa dell'insieme di funzionalità implementate il prodotto sarà corredato di un manuale utente completo delle indicazioni necessarie per poterne fruire in modo efficacie.
	\subsection{Piattaforma di esecuzione}
	Il prodotto sviluppato risulterà fruibile da qualsiasi piattaforma desktop con operante un browser web compatibile con le tecnologie \gloss{HTML5}, \gloss{CSS3} e \gloss{Javascript}. In particolare l'elenco dei browser web il cui supporto è garantito è contenuto nella sezione relativa ai requisiti di vincolo.
	\subsection{Vincoli generali}
\newpage

\section{Casi d'uso}
All'interno di questa sezione vengono elencati i casi d'uso rilevanti per il prodotto \proj{} individuati e definiti attraverso l'analisi del capitolato d'appalto, gli incontri con il proponente e le riunioni interne al team \hx{}. 
\\In particolare ogni
\subsection{UC1: Pagina Iniziale}
\label{UC1}
\begin{itemize}
\item \textbf{Attori}: Utente;
\item \textbf{Descrizione}: nella pagina iniziale l'attore crea un nuovo progetto oppure ne apre uno già esistente.
Successivamente l'attore può:
	\begin{itemize}
	\item realizzare e gestire un diagramma delle classi;
	\item realizzare e gestire un diagramma delle attività per ogni metodo presente nel diagramma delle classi;
	\item salvare le modifiche apportate al progetto (se questo è stato caricato);
	\item generare l'applicativo realizzato.
	\end{itemize}
\item \textbf{Precondizione}: il sistema risulta avviato e mostra la pagina iniziale dell'applicazione;
\item \textbf{Postcondizione}: il sistema ha ricevuto tutti gli input dall'attore sulle operazioni che vuole effettuare;
\item \textbf{Scenario principale}:
	\begin{enumerate}
	\item l'attore può aprire un progetto esistente (UC2);
	\item l'attore può creare un nuovo progetto (UC3);
	\item l'attore può realizzare e gestire un diagramma delle classi (UC4);
	\item l'attore può realizzare e gestire un diagramma delle attività per ogni metodo presente nel diagramma delle classi (UC5);
	\item l'attore può salvare il progetto (UC6);
	\item l'attore può generare l'applicativo realizzato (UC7).
	\end{enumerate}

\end{itemize}

\subsection{UC2: Apertura Progetto}
\label{UC2}
	\begin{itemize}
		\item \textbf{Attori}: Utente;
		\item \textbf{Descrizione}: l'attore può caricare un progetto precedentemente realizzato con il programma ora in uso allo scopo di apportarvi 		ulteriori modifiche e/o ampliamenti;
		\item \textbf{Precondizione}: il sistema avviato mostra la pagina iniziale dell'applicazione;
		\item \textbf{Postcondizione}: il sistema ha caricato il progetto selezionato mostrandone il contenuto come precedentemente salvato;
		\item \textbf{Scenario principale}:
l'attore carica il progetto selezionato.
		\item \textbf{Estensioni}:
l'attore visualizza un messaggio d'errore riguardo al file inserito(UC2.1).
	\end{itemize}

	\subsubsection{UC2.1: Messaggio d'errore caricamento file}
	\label{UC2.1}
		\begin{itemize}
			\item \textbf{Attori}: Utente;
			\item \textbf{Descrizione}: l'attore visualizza un opportuno messaggio di errore nel caso in cui tenti di inserire un file incompatibile rispetto all'editor in uso;
			\item \textbf{Precondizione}: l'attore ha selezionato per il caricamento un file con estensione incompatibile;
			\item \textbf{Postcondizione}: il sistema visualizza un messaggio di errore per informare l'utente;
			\item \textbf{Scenario principale}: l'attore visualizza un messaggio di errore che indica l'incompatibilità del file richiesto per il caricamento.
		\end{itemize}

\subsection{UC3: Nuovo Progetto}
\label{UC3}
\begin{itemize}
\item \textbf{Attori}: Utente;
\item \textbf{Descrizione}: l'attore può creare un nuovo progetto; 
\item \textbf{Precondizione}: il sistema avviato mostra la pagina iniziale dell'applicazione;
\item \textbf{Postcondizione}: il sistema crea un progetto vuoto;
\item \textbf{Scenario principale}:
l'attore crea un nuovo progetto interagendo con l'editor in uso tramite il comando appropriato.
\end{itemize}

\subsection{UC4: Realizzazione Diagramma Delle Classi}
\label{UC4}
\begin{itemize}
\item \textbf{Attori}: Utente;
\item \textbf{Descrizione}: l'attore realizza e gestisce un diagramma delle classi con la possibilità di aggiungere nuovi elementi e rimuovere elementi precedentemente inseriti che non corrispondono più alle sue intenzioni di progettazione;
\item \textbf{Precondizione}: il sistema è avviato e mostra il contenuto di un progetto;
\item \textbf{Postcondizione}: il sistema visualizza il diagramma delle classi risultante dalla gestione operata dall'utente;
\item \textbf{Scenario principale}:
	\begin{enumerate}
	 \item l'attore può inserire una classe (UC4.1);
	 \item l'attore può inserire una relazione (UC4.2);
	 \item l'attore può inserire un commento (UC4.3);
	 \item l'attore può rimuovere una classe (UC4.4);
	 \item l'attore può rimuovere una relazione (UC4.5);
	 \item l'attore può rimuovere un commento (UC4.6);
	 \item l'attore può ridurre una classe (UC4.7);
	 \item l'attore può espandere una classe (UC4.8);
	 \item l'attore può ridurre un commento (UC4.9);
	 \item l'attore può espandere un commento (UC4.10);
	  \item l'attore può modificare una classe (UC4.11);
	 \item l'attore può modificare una relazione (UC4.12);
	 \item l'attore può modificare un commento (UC4.13).
	\end{enumerate}
\end{itemize}

\subsubsection{UC4.1: Inserimento Classe}
\label{UC4.1}
\begin{itemize}
\item \textbf{Attori}: Utente;
\item \textbf{Descrizione}: l'attore inserisce una classe fornendo i dati richiesti;
\item \textbf{Precondizione}: il sistema è avviato, mostra il contenuto di un particolare progetto ed è nell'area per la realizzazione e gestione del diagramma delle classi;
\item \textbf{Postcondizione}: il sistema ha ricevuto tutti gli input dall'attore sulle classi e sulle relazioni che ha definito e per ogni metodo crea un riferimento ad un diagramma delle attività inizialmente vuoto;
\item \textbf{Scenario principale}:
	\begin{enumerate}
		\item l'attore può impostare la visibilità della classe (UC4.1.1);
		\item l'attore deve inserire il nome della classe (univoco) (UC4.1.2);
		\item l'attore può scegliere uno stereotipo tra quelli forniti e previsti (UC4.1.3);
		\item l'attore può aggiungere un attributo (UC4.1.4);
		\item l'attore può rimuovere un attributo (UC4.1.5);
		\item l'attore può aggiungere un metodo (UC4.1.6);
		\item l'attore può rimuovere un metodo (UC4.1.7);
		\item l'attore può scegliere attributi opzionali (UC4.1.8);
		\item l'attore può confermare i dati inseriti (UC4.1.9).
	\end{enumerate}
\end{itemize}

\paragraph{UC4.1.1: Inserimento Visibilità Classe}
\label{UC4.1.1}
\begin{itemize}
\item \textbf{Attori}: Utente;
\item \textbf{Descrizione}: l'attore inserisce la visibilità desiderata per la classe corrente;
\item \textbf{Precondizione}: il sistema visualizza il form per l'inserimento dei dati;
\item \textbf{Postcondizione}: il sistema ha ricevuto il dato richiesto;
\item \textbf{Scenario principale}:l'attore inserisce la visibilità della classe fra l'insieme dei valori per essa possibili;
\item \textbf{Scenario alternativo}:l'attore non inserisce alcun valore, viene pertanto applicato il valore di default per la visibilità.
\end{itemize}

\paragraph{UC4.1.2: Inserimento Nome Classe}
\label{UC4.1.2}
\begin{itemize}
\item \textbf{Attori}: Utente;
\item \textbf{Descrizione}: l'attore inserisce un nome univoco per la classe corrente;
\item \textbf{Precondizione}: il sistema visualizza il form per l'inserimento dei dati;
\item \textbf{Postcondizione}: il sistema ha ricevuto il dato richiesto;
\item \textbf{Scenario principale}:
l'attore inserisce un nome per identificare la classe corrente univoco all'interno del diagramma delle classi che sta gestendo.
\end{itemize}

\paragraph{UC4.1.3: Inserimento Stereotipo Classe}
\label{UC4.1.3}
\begin{itemize}
\item \textbf{Attori}: Utente;
\item \textbf{Descrizione}: l'attore sceglie uno stereotipo per la classe corrente fra quelli previsti dall'editor;
\item \textbf{Precondizione}: il sistema visualizza il form per l'inserimento dei dati;
\item \textbf{Postcondizione}: il sistema ha ricevuto il dato richiesto;
\item \textbf{Scenario principale}:
l'attore sceglie uno stereotipo per la classe corrente fra quelli prestabiliti.
\end{itemize}

\paragraph{UC4.1.4: Inserimento Attributo}
\label{UC4.1.4}
\begin{itemize}
\item \textbf{Attori}: Utente;
\item \textbf{Descrizione}: l'attore inserisce un attributo per la classe corrente;
\item \textbf{Precondizione}: il sistema visualizza il form per l'inserimento dei dati;
\item \textbf{Postcondizione}: il sistema visualizza l'attributo fornito dall'attore nella classe correntemente gestita dallo stesso;
\item \textbf{Scenario principale}:
\begin{enumerate}
	\item l'attore può impostare la visibilità dell'attributo (UC4.1.4.1);
	\item l'attore deve inserire il nome dell'attributo (UC4.1.4.2);
	\item l'attore deve inserire il tipo dell'attributo (UC4.1.4.3);
	\item l'attore può scegliere la molteplicità dell'attributo (UC4.1.4.4);
	\item l'attore può inserire un valore di default per l'attributo (UC4.1.4.5);
	\item l'attore può confermare i dati inseriti (UC4.1.4.6).
\end{enumerate}
\end{itemize}

\subparagraph{UC4.1.4.1: Inserimento Visibilità Attributo}
\label{UC4.1.4.1}
\begin{itemize}
\item \textbf{Attori}: Utente;
\item \textbf{Descrizione}: l'attore può scegliere la visibilità per l'attributo;
\item \textbf{Precondizione}: il sistema visualizza il form per l'inserimento dei dati;
\item \textbf{Postcondizione}: il sistema ha ricevuto il dato richiesto;
\item \textbf{Scenario principale}:
l'attore sceglie la visibilità per l'attributo fra l'insieme di valori per essa disponibili.
\item \textbf{Scenario alternativo}:l'attore non inserisce alcun valore, viene pertanto applicato il valore di default per la visibilità.
\end{itemize}

\subparagraph{UC4.1.4.2: Inserimento Nome Attributo}
\label{UC4.1.4.2}
\begin{itemize}
\item \textbf{Attori}: Utente;
\item \textbf{Descrizione}: l'attore deve inserire il nome dell'attributo, che deve essere univoco nel contesto della classe correntemente gestita;
\item \textbf{Precondizione}: il sistema visualizza il form per l'inserimento dei dati;
\item \textbf{Postcondizione}: il sistema ha ricevuto il dato richiesto;
\item \textbf{Scenario principale}:
l'attore inserisce il nome dell'attributo.
\end{itemize}

\subparagraph{UC4.1.4.3: Inserimento Tipo Attributo}
\label{UC4.1.4.3}
\begin{itemize}
\item \textbf{Attori}: Utente;
\item \textbf{Descrizione}: l'attore deve inserire il tipo dell'attributo;
\item \textbf{Precondizione}: il sistema visualizza il form per l'inserimento dei dati;
\item \textbf{Postcondizione}: il sistema ha ricevuto il dato richiesto;
\item \textbf{Scenario principale}:
l'attore inserisce il tipo dell'attributo corrente.
\end{itemize}

\subparagraph{UC4.1.4.4: Inserimento Molteplicità Attributo}
\label{UC4.1.4.4}
\begin{itemize}
\item \textbf{Attori}: Utente;
\item \textbf{Descrizione}: l'attore può scegliere la molteplicità dell'attributo;
\item \textbf{Precondizione}: il sistema visualizza il form per l'inserimento dei dati;
\item \textbf{Postcondizione}: il sistema ha ricevuto il dato richiesto;
\item \textbf{Scenario principale}:
l'attore sceglie la molteplicità dell'attributo;
\item \textbf{Scenario alternativo}:l'attore non inserisce alcun valore, viene pertanto applicato il valore di default per la molteplicità. 
\end{itemize}

\subparagraph{UC4.1.4.5: Inserimento Valore di Default}
\label{UC4.1.4.5}
\begin{itemize}
\item \textbf{Attori}: Utente;
\item \textbf{Descrizione}: l'attore può inserire il valore di default dell'attributo;
\item \textbf{Precondizione}: il sistema visualizza il form per l'inserimento dei dati;
\item \textbf{Postcondizione}: il sistema ha ricevuto il dato richiesto;
\item \textbf{Scenario principale}:
l'attore inserisce un valore di default per l'attributo.
\end{itemize}

\subparagraph{UC4.1.4.6: Conferma Inserimento Attributo}
\label{UC4.1.4.6}
\begin{itemize}
\item \textbf{Attori}: Utente;
\item \textbf{Descrizione}: l'attore può confermare l'inserimento dell'attributo;
\item \textbf{Precondizione}: il sistema visualizza il form per l'inserimento dei dati;	
\item \textbf{Postcondizione}: il sistema visualizza l'attributo inserito fornito dall'attore nella classe correntemente gestita dallo stesso;	
\item \textbf{Scenario principale}:
l'attore conferma l'inserimento dell'attributo;	
\item \textbf{Estensioni}:
l'attore visualizza un messaggio d'errore sull'inserimento dei dati (UC4.1.4.6.1);	
\item \textbf{Scenari alternativi}:
\begin{itemize}
	\item l'attore non ha inserito il nome dell'attributo;
	\item l'attore non ha inserito il tipo dell'attributo;
	\item l'attore ha inserito per l'attributo un nome già esistente all'interno della classe stessa.
\end{itemize}
\end{itemize}

\subparagraph{UC4.1.4.6.1: Visualizzazione Errore Inserimento Attributo}
\label{UC4.1.4.6.1}
\begin{itemize}
\item \textbf{Attori}: Utente;
\item \textbf{Descrizione}: l'attore può visualizzare un messaggio d'errore al momento della conferma dell'inserimento nel caso si fossero verificati uno o più scenari alternativi durante la fase di inserimento di un attributo;	
\item \textbf{Precondizione}: il sistema non ha ricevuto sufficienti informazioni per l'inserimento di un attributo;	
\item \textbf{Postcondizione}: il sistema avvisa l'attore dell'errore verificatosi tramite un opportuno messaggio di errore;	
\item \textbf{Scenario principale}:
l'attore visualizza un messaggio d'errore.	
\end{itemize}

\paragraph{UC4.1.5: Rimozione Attributo}
\label{UC4.1.5}
\begin{itemize}
\item \textbf{Attori}: Utente;
\item \textbf{Descrizione}: l'attore può rimuovere un attributo precedentemente inserito;
\item \textbf{Precondizione}: il sistema visualizza la classe correntemente gestita dall'attore;
\item \textbf{Postcondizione}: il sistema rimuove l'attributo selezionato;
\item \textbf{Scenario principale}:
l'attore rimuove l'attributo selezionato usando l'apposito comando previsto.
\end{itemize}

\paragraph{UC4.1.6: Inserimento Metodo}
\label{UC4.1.6}
\begin{itemize}
\item \textbf{Attori}: Utente;
\item \textbf{Descrizione}: l'attore inserisce un metodo per la classe;
\item \textbf{Precondizione}: il sistema visualizza il form per l'inserimento dei dati;
\item \textbf{Postcondizione}: il sistema visualizza il metodo fornito dall'attore nella classe correntemente gestita dallo stesso;
\item \textbf{Scenario principale}:
\begin{enumerate}
	\item l'attore può scegliere la visibilità del metodo (UC4.1.6.1);
	\item l'attore deve inserire il nome del metodo (UC4.1.6.2);
	\item l'attore può inserire un parametro del metodo (UC4.1.6.3);
	\item l'attore deve inserire il tipo di ritorno del metodo (UC4.1.6.4);
	\item l'attore può confermare l'inserimento del metodo (UC4.1.6.5);
	\item  l'attore può rimuovere un parametro del metodo (UC4.1.6.6).
\end{enumerate}
\end{itemize}

\subparagraph{UC4.1.6.1: Inserimento Visibilità Metodo}
\label{UC4.1.6.1}
\begin{itemize}
\item \textbf{Attori}: Utente;
\item \textbf{Descrizione}: l'attore può scegliere la visibilità per il metodo;
\item \textbf{Precondizione}: il sistema visualizza il form per l'inserimento dei dati;
\item \textbf{Postcondizione}: il sistema ha ricevuto il dato richiesto;
\item \textbf{Scenario principale}:
l'attore sceglie la visibilità del metodo altrimenti viene applicato il valore di default.
\end{itemize}

\subparagraph{UC4.1.6.2: Inserimento Nome Metodo}
\label{UC4.1.6.2}
\begin{itemize}
\item \textbf{Attori}: Utente;
\item \textbf{Descrizione}: l'attore deve inserire il nome del metodo, che deve essere univoco nella classe correntemente gestita rispetto ad altri metodi già inseriti con la stessa segnatura;
\item \textbf{Precondizione}: il sistema visualizza il form per l'inserimento dei dati;
\item \textbf{Postcondizione}: il sistema ha ricevuto il dato richiesto;
\item \textbf{Scenario principale}:
l'attore inserisce il nome del metodo.
\end{itemize}

\subparagraph{UC4.1.6.3: Inserimento Parametro Metodo}
\label{UC4.1.6.3}
\begin{itemize}
\item \textbf{Attori}: Utente;
\item \textbf{Descrizione}: l'attore inserisce un parametro per il metodo corrente;
\item \textbf{Precondizione}: il sistema visualizza il form per l'inserimento dei dati;
\item \textbf{Postcondizione}: il sistema visualizza il parametro fornito fornito dall'attore nella classe correntemente gestita dallo stesso;
\item \textbf{Scenario principale}:
\begin{enumerate}
	\item l'attore deve inserire il nome del parametro (UC4.1.6.3.1);
	\item l'attore deve inserire il tipo del parametro (UC4.1.6.3.2);
	\item l'attore conferma l'inserimento del parametro. (UC4.1.6.3.3).
\end{enumerate}
\end{itemize}

\subparagraph{UC4.1.6.3.1: Inserimento Nome Parametro}
\label{UC4.1.6.3.1}
\begin{itemize}
\item \textbf{Attori}: Utente;
\item \textbf{Descrizione}: l'attore deve inserire il nome del parametro del metodo, che deve essere univoco nel metodo correntemente gestito rispetto ad altri parametri già inseriti con lo stesso nome;
\item \textbf{Precondizione}: il sistema visualizza il form per l'inserimento dei dati;
\item \textbf{Postcondizione}: il sistema ha ricevuto il dato richiesto;
\item \textbf{Scenario principale}:
l'attore inserisce il nome del parametro.
\end{itemize}

\subparagraph{UC4.1.6.3.2: Inserimento Tipo Parametro}
\label{UC4.1.6.3.2}
\begin{itemize}
\item \textbf{Attori}: Utente;
\item \textbf{Descrizione}: l'attore deve inserire il tipo del metodo;
\item \textbf{Precondizione}: il sistema visualizza il form per l'inserimento dei dati;	
\item \textbf{Postcondizione}: il sistema ha ricevuto il dato richiesto;	
\item \textbf{Scenario principale}:
l'attore inserisce il tipo del parametro.
\end{itemize}

\subparagraph{UC4.1.6.3.3: Conferma Inserimento Parametro}
\label{UC4.1.6.3.3}
\begin{itemize}
\item \textbf{Attori}: Utente;
\item \textbf{Descrizione}: l'attore può confermare l'inserimento del parametro;
\item \textbf{Precondizione}: il sistema visualizza il form per l'inserimento dei dati;	
\item \textbf{Postcondizione}: il sistema visualizza il parametro inserito dall'attore nel metodo della classe correntemente gestita dallo stesso;	
\item \textbf{Scenario principale}:
l'attore visualizza il parametro inserito;
\item \textbf{Estensioni}:
l'attore visualizza un messaggio d'errore sull'inserimento dei dati (UC4.1.6.3.3.1);
\item \textbf{Scenari alternativi}:
\begin{itemize}
	\item l'attore non ha inserito il nome del parametro;
	\item l'attore non ha inserito il tipo del parametro.
\end{itemize}
\end{itemize}

\subparagraph{UC4.1.6.3.3.1: Visualizzazione Errore Inserimento Parametro }
\label{UC4.1.6.3.3.1}
\begin{itemize}
\item \textbf{Attori}: Utente;
\item \textbf{Descrizione}: l'attore può visualizzare un messaggio d'errore nel caso si fossero verificati uno o più scenari alternativi durante la fase di inserimento di un parametro;	
\item \textbf{Precondizione}: il sistema non ha ricevuto sufficienti informazioni per l'inserimento di un parametro;	
\item \textbf{Postcondizione}: il sistema avvisa l'attore dell'errore verificatosi tramite un opportuno messaggio;	
\item \textbf{Scenario principale}:
l'attore visualizza un messaggio d'errore.	
\end{itemize}

\subparagraph{UC4.1.6.4: Inserimento Tipo Ritorno Metodo}
\label{UC4.1.6.4}
\begin{itemize}
\item \textbf{Attori}: Utente;
\item \textbf{Descrizione}: l'attore deve inserire il tipo di ritorno del metodo;
\item \textbf{Precondizione}: il sistema visualizza il form per l'inserimento dei dati;
\item \textbf{Postcondizione}: il sistema ha ricevuto il dato richiesto;
\item \textbf{Scenario principale}:
l'attore inserisce il tipo di ritorno del metodo.
\end{itemize}

\subparagraph{UC4.1.6.5: Conferma Inserimento Metodo}
\label{UC4.1.6.5}
\begin{itemize}
\item \textbf{Attori}: Utente;
\item \textbf{Descrizione}: l'attore può confermare l'inserimento del metodo;	
\item \textbf{Precondizione}: il sistema visualizza il form per l'inserimento dei dati;	
\item \textbf{Postcondizione}: il sistema visualizza il metodo inserito dall'attore nella classe correntemente gestita dallo stesso;	
\item \textbf{Scenario principale}:
l'attore conferma l'inserimento del metodo;	
\item \textbf{Estensioni}:
l'attore visualizza un messaggio d'errore sull'inserimento dei dati (UC4.1.6.5.1);	
\item \textbf{Scenari alternativi}:
\begin{itemize}
	\item l'attore non ha inserito il nome del metodo;
	\item l'attore non ha inserito il tipo di ritorno del metodo;
	\item l'attore ha inserito un metodo con una segnatura già esistente.
\end{itemize}
\end{itemize}

\subparagraph{UC4.1.6.6: Rimozione Parametro Metodo}
\label{UC4.1.6.6}
\begin{itemize}
\item \textbf{Attori}: Utente;
\item \textbf{Descrizione}: l'attore rimuove un parametro del metodo;
\item \textbf{Precondizione}: il sistema visualizza il parametro del metodo selezionato;	
\item \textbf{Postcondizione}: il sistema rimuove il parametro selezionato per la rimozione;
\item \textbf{Scenario principale}:
l'attore rimuove il parametro selezionato.
\end{itemize}

\subparagraph{UC4.1.6.5.1: Visualizzazione Errore Inserimento Metodo}
\label{UC4.1.6.5.1}
\begin{itemize}
\item \textbf{Attori}: Utente;
\item \textbf{Descrizione}: l'attore può visualizzare un messaggio d'errore al momento della conferma dell'inserimento nel caso si fossero verificati uno o più scenari alternativi durante la fase di inserimento di un metodo;	
\item \textbf{Precondizione}: il sistema non ha ricevuto tutti i dati necessari per l'inserimento di un metodo;	
\item \textbf{Postcondizione}: il sistema avvisa l'attore dell'errore verificatosi tramite un opportuno messaggio;	
\item \textbf{Scenario principale}:
l'attore visualizza un messaggio d'errore.	
\end{itemize}

\paragraph{UC4.1.7: Rimozione Metodo}
\label{UC4.1.7}
\begin{itemize}
\item \textbf{Attori}: Utente;
\item \textbf{Descrizione}: l'attore può rimuovere un metodo precedentemente inserito;
\item \textbf{Precondizione}: il sistema visualizza la classe fornita dall'attore;
\item \textbf{Postcondizione}: il sistema rimuove il metodo selezionato;
\item \textbf{Scenario principale}:
l'attore rimuove il metodo selezionato utilizzando l'opportuno comando previsto.
\end{itemize}

\paragraph{UC4.1.8: Conferma Inserimento Classe}
\label{UC4.1.8}
\begin{itemize}
\item \textbf{Attori}: Utente;
\item \textbf{Descrizione}: l'attore può confermare l'inserimento della classe correntemente gestita;
\item \textbf{Precondizione}: il sistema visualizza il form per l'inserimento dei dati;	
\item \textbf{Postcondizione}: il sistema visualizza la classe inserita nella porzione dell'editor dedicata alla gestione e realizzazione del diagramma delle classi; 
\item \textbf{Scenario principale}:
l'attore conferma l'inserimento della classe attraverso l'opportuno comando previsto;
\item \textbf{Estensioni}:
l'attore visualizza un messaggio d'errore sull'inserimento dei dati (UC4.1.9.1);
\item \textbf{Scenari alternativi}:
\begin{itemize}
	\item l'attore non ha inserito il nome della classe;
	\item l'attore ha inserito un nome per la classe già esistente nel diagramma delle classi correntemente gestito.
\end{itemize}
\end{itemize}

\subparagraph{UC4.1.8.1: Visualizzazione Errore Inserimento Classe}
\label{UC4.1.8.1}
\begin{itemize}
\item \textbf{Attori}: Utente;
\item \textbf{Descrizione}: l'attore può visualizzare un messaggio d'errore alla conferma dell'inserimento nel caso si fossero verificati uno o più scenari alternativi durante la fase di inserimento di una classe;	
\item \textbf{Precondizione}: il sistema ha ricevuto dei dati errati e/o mancanti per l'inserimento di una classe;	
\item \textbf{Postcondizione}: il sistema avvisa l'attore dell'errore verificatosi tramite un opportuno messaggio;	
\item \textbf{Scenario principale}:
l'attore visualizza un messaggio d'errore.	
\end{itemize}

\subsubsection{UC4.2: Inserimento Relazione}
\label{UC4.2}
\begin{itemize}
\item \textbf{Attori}: Utente;
\item \textbf{Descrizione}: l'attore può inserire una relazione tra due classi;
\item \textbf{Precondizione}: il sistema visualizza le classi fornite dall'attore nella sezione dell'editor dedicata alla gestione e realizzazione del diagramma delle classi;
\item \textbf{Postcondizione}:il sistema visualizza graficamente la relazione creando un collegamento grafico tra le classi indicate dall'attore conforme alla tipologia di relazione specificata;
\item \textbf{Scenario principale}:
\begin{enumerate}
	\item l'attore sceglie il tipo della relazione (UC4.2.1); 
	\item l'attore imposta la classe1 della relazione - ovvero la classe di partenza della relazione(UC4.2.2);
	\item l'attore imposta la classe2 della relazione - ovvero la classe di destinazione della relazione(UC4.2.3);
	\item l'attore inserisce la cardinalità della relazione (UC4.2.4);
	\item l'attore sceglie per la relazione uno stereotipo tra quelli forniti (UC4.2.5);
	\item l'attore può confermare l'inserimento della relazione (UC4.2.6);
	\item l'attore inserisce il nome da assegnare alla relazione (UC4.2.7).
\end{enumerate}
\end{itemize}

\paragraph{UC4.2.1: Inserimento Tipo Relazione}
\label{UC4.2.1}
\begin{itemize}
\item \textbf{Attori}: Utente;
\item \textbf{Descrizione}: l'attore deve scegliere il tipo di una relazione fra quelli previsti dall'editor;
\item \textbf{Precondizione}: il sistema visualizza il form per l'inserimento dei dati;
\item \textbf{Postcondizione}: il sistema ha ricevuto il dato richiesto;
\item \textbf{Scenario principale}:
l'attore sceglie il tipo della relazione tra quelli proposti.
\end{itemize}

\paragraph{UC4.2.2: Inserimento Classe1 Relazione}
\label{UC4.2.2}
\begin{itemize}
\item \textbf{Attori}: Utente;
\item \textbf{Descrizione}: l'attore deve scegliere la classe1 della relazione scelta come classe di partenza della relazione stessa;
\item \textbf{Precondizione}: il sistema visualizza il form per l'inserimento dei dati
\item \textbf{Postcondizione}: il sistema ha ricevuto il dato richiesto;
\item \textbf{Scenario principale}:
l'attore sceglie la classe di partenza della relazione;
\end{itemize}

\paragraph{UC4.2.3: Inserimento Classe2 Relazione}
\label{UC4.2.3}
\begin{itemize}
\item \textbf{Attori}: Utente;
\item \textbf{Descrizione}: l'attore deve scegliere la classe2 della relazione scelta come classe di destinazione della stessa se opta per una relazione unidirezionale, altrimenti può essere anche scelta come classe1;
\item \textbf{Precondizione}: il sistema visualizza il form per l'inserimento dei dati	
\item \textbf{Postcondizione}: il sistema ha ricevuto il dato richiesto;	
\item \textbf{Scenario principale}:
l'attore sceglie la classe di destinazione della relazione;	
\end{itemize}

\paragraph{UC4.2.4: Inserimento Cardinalità Relazione}
\label{UC4.2.4}
\begin{itemize}
\item \textbf{Attori}: Utente;
\item \textbf{Descrizione}: l'attore può scegliere la cardinalità di una relazione;
\item \textbf{Precondizione}: il sistema visualizza il form per l'inserimento dei dati	
\item \textbf{Postcondizione}: il sistema ha ricevuto il dato richiesto;	
\item \textbf{Scenario principale}:
l'attore sceglie la cardinalità della relazione tra quelle proposte;
\item \textbf{Scenario alternativo}: l'attore non inserisce alcun valore, viene pertanto applicato il valore di default per la cardinalità.
\end{itemize}

\paragraph{UC4.2.5: Inserimento Stereotipo Relazione}
\label{UC4.2.5}
\begin{itemize}
\item \textbf{Attori}: Utente;
\item \textbf{Descrizione}: l'attore può scegliere uno stereotipo per una relazione fra quelli disponibili;
\item \textbf{Precondizione}: il sistema visualizza il form per l'inserimento dei dati	
\item \textbf{Postcondizione}: il sistema ha ricevuto il dato richiesto;	
\item \textbf{Scenario principale}:
l'attore sceglie uno stereotipo tra quelli proposti;
\end{itemize}

\paragraph{UC4.2.6: Conferma Inserimento Relazione}
\label{UC4.2.6}
\begin{itemize}
\item \textbf{Attori}: Utente;
\item \textbf{Descrizione}: l'attore può confermare l'inserimento di una relazione;	
\item \textbf{Precondizione}: il sistema visualizza il form per l'inserimento dei dati;	
\item \textbf{Postcondizione}:  graficamente la relazione creando un collegamento grafico tra le classi indicate dall'attore conforme alla tipologia di relazione specificata;
\item \textbf{Scenario principale}:
l'attore conferma l'inserimento della relazione;	
\item \textbf{Estensioni}:
l'attore visualizza un messaggio d'errore sull'inserimento dei dati (UC4.2.6.1);
\item \textbf{Scenari alternativi}:
\begin{itemize}
	\item l'attore non ha scelto il tipo della relazione;
	\item l'attore non ha inserito il nome della relazione;
	\item l'attore non ha scelto la classe di partenza della relazione;
	\item l'attore non ha scelto la classe di destinazione della relazione.
\end{itemize}
\end{itemize}

\subparagraph{UC4.2.6.1: Visualizzazione Errore Inserimento Relazione}
\label{UC4.2.6.1}
\begin{itemize}
\item \textbf{Attori}: Utente;
\item \textbf{Descrizione}: l'attore può visualizzare un messaggio d'errore nel caso si fossero verificati uno o più scenari alternativi durante la fase di inserimento di una relazione;	
\item \textbf{Precondizione}: il sistema non ha ricevuto tutti i dati necessari per l'inserimento di una relazione;	
\item \textbf{Postcondizione}: il sistema avvisa l'attore dell'errore verificatosi tramite un opportuno messaggio;	
\item \textbf{Scenario principale}:
l'attore visualizza un messaggio d'errore.	
\end{itemize}


\paragraph{UC4.2.7: Inserimento Attributo Relazione}
\label{UC4.2.7}
\begin{itemize}
\item \textbf{Attori}: Utente;
\item \textbf{Descrizione}: l'attore deve scegliere l'attributo di una relazione per chiarire il ruolo della classe di destinazione, nel caso si tratti di una associazione, composizione o aggregazione;
\item \textbf{Precondizione}: il sistema visualizza il form per l'inserimento dei dati;
\item \textbf{Postcondizione}: il sistema ha ricevuto il dato richiesto;
\item \textbf{Scenario principale}:
l'attore sceglie l'attributo della relazione.
\end{itemize}

\subsubsection{UC4.3: Inserimento Commento}
\label{UC4.3}
\begin{itemize}
\item \textbf{Attori}: Utente;
\item \textbf{Descrizione}: l'attore può inserire un commento;
\item \textbf{Precondizione}: il sistema visualizza la struttura di classi realizzata dall'attore nella sezione dell'editor dedicata alla gestione del diagramma delle classi;
\item \textbf{Postcondizione}: il sistema visualizza il commento fornito dall'attore collegato graficamente all'elemento del grafico a cui esso si riferisce;	
\item \textbf{Scenario principale}:
\begin{enumerate}
\item l'attore può inserire il testo del commento (UC4.3.1);
\item l'attore deve scegliere il "parent" del commento (UC4.3.2);
\item l'attore può confermare l'inserimento del commento (UC4.3.3).
\end{enumerate}
\end{itemize}

\paragraph{UC4.3.1: Inserimento Testo Commento}
\label{UC4.3.1}
\begin{itemize}
\item \textbf{Attori}: Utente;
\item \textbf{Descrizione}: l'attore può inserire il testo di un commento;
\item \textbf{Precondizione}: il sistema visualizza il form per l'inserimento dei dati	
\item \textbf{Postcondizione}: il sistema ha ricevuto il dato richiesto;	
\item \textbf{Scenario principale}:
l'attore inserisce il testo del commento;
\end{itemize}

\paragraph{UC4.3.2: Inserimento Parent Commento}
\label{UC4.3.2}
\begin{itemize}
\item \textbf{Attori}: Utente;
\item \textbf{Descrizione}: l'attore deve scegliere il "parent" di un commento, ovvero l'elemento del diagramma delle classi da lui realizzato, sia esso una classe o una relazione, a cui il commento si riferisce;
\item \textbf{Precondizione}: il sistema visualizza il form per l'inserimento dei dati;	
\item \textbf{Postcondizione}: il sistema ha ricevuto il dato richiesto;	
\item \textbf{Scenario principale}:
l'attore sceglie il parent del commento tra quelli possibili.
\end{itemize}

\paragraph{UC4.3.3: Conferma Inserimento Commento}
\label{UC4.3.3}
\begin{itemize}
\item \textbf{Attori}: Utente;
\item \textbf{Descrizione}: l'attore può confermare l'inserimento di un commento;	
\item \textbf{Precondizione}: il sistema visualizza il form per l'inserimento dei dati;	
\item \textbf{Postcondizione}: il sistema visualizza il commento inserito collegandolo in modo grafico con l'elemento del diagramma delle classi a cui si riferisce;	
\item \textbf{Scenario principale}:
l'attore conferma l'inserimento del commento;	
\item \textbf{Estensioni}:
l'attore visualizza un messaggio d'errore sull'inserimento dei dati (UC4.3.3.1);	
\item \textbf{Scenari alternativi}:
\begin{itemize}
\item l'attore non ha scelto l'elemento a cui il commento si riferisce.
\end{itemize}
\end{itemize}

\subparagraph{UC4.3.3.1: Visualizzazione Errore Inserimento Commento}
\label{UC4.3.3.1}
\begin{itemize}
\item \textbf{Attori}: Utente;
\item \textbf{Descrizione}: l'attore può visualizzare un messaggio d'errore al momento della conferma di inserimento nel caso si fossero verificati uno o più scenari alternativi durante la fase di inserimento di un commento;	
\item \textbf{Precondizione}: il sistema non ha ricevuto i dati necessari per il corretto inserimento di un commento;	
\item \textbf{Postcondizione}: il sistema avvisa l'attore dell'errore verificatosi tramite un opportuno messaggio;	
\item \textbf{Scenario principale}:
l'attore visualizza un messaggio d'errore.	
\end{itemize}

\subsubsection{UC4.4: Rimozione Classe}
\label{UC4.4}
\begin{itemize}
\item \textbf{Attori}: Utente;
\item \textbf{Descrizione}: l'attore può rimuovere una classe;
\item \textbf{Precondizione}: il sistema visualizza il diagramma delle classi realizzato dall'utente, ed in particolare la classe che sarà selezionata per la cancellazione non ha dipendenze;
\item \textbf{Postcondizione}: il sistema rimuove la classe indicata dall'utente cancellandola graficamente dal diagramma;
\item \textbf{Scenario principale}:
\begin{itemize}
	\item l'attore seleziona la classe da eliminare;
	\item l'attore richiede l'eliminazione della classe selezionata.
\end{itemize}
\end{itemize}

\subsubsection{UC4.5: Rimozione Relazione}
\label{UC4.5}
\begin{itemize}
\item \textbf{Attori}: Utente;
\item \textbf{Descrizione}: l'attore può rimuovere una relazione;	
\item \textbf{Precondizione}: il sistema visualizza il diagramma delle classi realizzato dall'utente, ed in particolare la relazione che sarà selezionata per la cancellazione;
\item \textbf{Postcondizione}: il sistema rimuove la relazione indicata dall'utente cancellandola graficamente dal diagramma delle classi precedentemente visualizzato;
\item \textbf{Scenario principale}:
\begin{itemize}
	\item l'attore seleziona la relazione da eliminare;
	\item l'attore richiede l'eliminazione della relazione selezionata.
\end{itemize}	
\end{itemize}

\subsubsection{UC4.6: Rimozione Commento	}
\label{UC4.6}
\begin{itemize}
\item \textbf{Attori}: Utente;
\item \textbf{Descrizione}: l'attore può rimuovere un commento;
\item \textbf{Precondizione}: il sistema visualizza il diagramma delle classi realizzato dall'utente, ed in particolare il commento che sarà selezionato per la cancellazione;
\item \textbf{Postcondizione}: il sistema rimuove il commento indicato dall'utente cancellandolo graficamente dal diagramma delle classi precedentemente visualizzato;
\item \textbf{Scenario principale}:
\begin{itemize}
	\item l'attore seleziona il commento da eliminare;
	\item l'attore richiede l'eliminazione del commento selezionato.
\end{itemize}		
\end{itemize}

\subsubsection{UC4.7: Riduzione Classe}
\label{UC4.7}
\begin{itemize}
\item \textbf{Attori}: Utente;
\item \textbf{Descrizione}: l'attore può ridurre la classe;
\item \textbf{Precondizione}: il sistema visualizza il diagramma delle classi realizzato dall'utente, ed in particolare la classe che sarà selezionata per la riduzione in formato espanso;	
\item \textbf{Postcondizione}: il sistema visualizza la classe selezionata dall'utente in formato ridotto, ovvero con la sola specifica del nome senza i metodi o gli attributi inseriti;	
\item \textbf{Scenario principale}:
l'attore riduce la classe nascondendo i vari metodi ed attributi.
\end{itemize}

\subsubsection{UC4.8: Espansione Classe	}
\label{UC4.8}
\begin{itemize}
\item \textbf{Attori}: Utente;
\item \textbf{Descrizione}: l'attore può espandere una classe;	
\item \textbf{Precondizione}: il sistema visualizza il diagramma delle classi realizzato dall'utente, ed in particolare la classe che sarà selezionata per l'espansione in formato ridotto, ovvero con la sola segnatura del nome e senza la visibilità dei metodi e degli attributi inseriti;	
\item \textbf{Postcondizione}:  il sistema visualizza la classe selezionata dall'utente in formato espanso, ovvero con la specifica del nome, degli attributi e dei metodi inseriti;
\item \textbf{Scenario principale}:
l'attore espande la classe mostrando i vari metodi ed attributi.	
\end{itemize}

\subsubsection{UC4.9: Riduzione Commento}
\label{UC4.9}
\begin{itemize}
\item \textbf{Attori}: Utente;
\item \textbf{Descrizione}: l'attore può ridurre un commento;	
\item \textbf{Precondizione}: il sistema visualizza il commento precedentemente realizzato ed espanso;	
\item \textbf{Postcondizione}: il sistema visualizza il commento ridotto;	
\item \textbf{Scenario principale}:
l'attore riduce il commento nascondendone il contenuto.	
\end{itemize}

\subsubsection{UC4.10: Espansione Commento	}
\label{UC4.10}
\begin{itemize}
\item \textbf{Attori}: Utente;
\item \textbf{Descrizione}: l'attore può espandere un commento;	
\item \textbf{Precondizione}: il sistema visualizza il commento precedentemente realizzato e ridotto;	
\item \textbf{Postcondizione}: il sistema visualizza il commento espanso;	
\item \textbf{Scenario principale}:
l'attore espande il commento mostrandone il contenuto.	
\end{itemize}

\subsubsection{UC4.11:Modifica Classe}
\label{UC4.11}
\begin{itemize}
\item \textbf{Attori}: Utente;
\item \textbf{Descrizione}: l'attore modifica una classe precedentemente creata; 
\item \textbf{Precondizione}: l'attore ha selezionato la classe da modificare;	
\item \textbf{Postcondizione}: il sistema ha ricevuto tutti gli input dall'attore sulle operazioni che vuole effettuare e visualizza la classe modificata;
\item \textbf{Scenario principale}:
\begin{enumerate}
	\item l'attore può modificare la visibilità della classe (UC4.11.1);
	\item l'attore può modificare il nome della classe (univoco) (UC4.11.2);
	\item l'attore può modificare un attributo (UC4.11.3);
	\item l'attore può modificare un metodo (UC4.11.4);
	\item l'attore può confermare le modifiche apportate alla classe (UC4.11.5).
\end{enumerate}
\end{itemize}

\paragraph{UC4.11.1:Modifica Visibilità Classe}
\label{UC}
\begin{itemize}
\item \textbf{Attori}: Utente;
\item \textbf{Descrizione}: l'attore può modificare la visibilità della classe;	
\item \textbf{Precondizione}: il sistema visualizza il form per l'inserimento dei dati;	
\item \textbf{Postcondizione}: il sistema ha ricevuto il dato richiesto;
\item \textbf{Scenario principale}: l'attore modifica la visibilità della classe.
\end{itemize}

\paragraph{UC4.11.2: Modifica Nome Classe}
\label{UC4.11.2}
\begin{itemize}
\item \textbf{Attori}: Utente;
\item \textbf{Descrizione}: l'attore può modificare il nome della classe;
\item \textbf{Precondizione}: il sistema visualizza il form per l'inserimento dei dati;	
\item \textbf{Postcondizione}: il sistema ha ricevuto il dato richiesto;
\item \textbf{Scenario principale}:	l'attore inserisce il nome modificato della classe mantenendone l'univocità.
\end{itemize}

\paragraph{UC4.11.3: Modifica Attributo Classe}
\label{UC4.11.3}
\begin{itemize}
\item \textbf{Attori}: Utente;
\item \textbf{Descrizione}: l'attore modifica un attributo della classe;	
\item \textbf{Precondizione}: il sistema visualizza il form per l'inserimento dei dati;	
\item \textbf{Postcondizione}: il sistema visualizza l'attributo modificato dall'attore;	
\item \textbf{Scenario principale}:
\begin{enumerate}
\item l'attore può modificare la visibilità dell'attributo altrimenti verrà applicato il valore di default (UC4.11.3.1);
\item l'attore può modificare il nome dell'attributo (UC4.11.3.2);
\item l'attore può modificare il tipo dell'attributo (UC4.11.3.3);
\item l'attore può modificare la molteplicità dell'attributo altrimenti verrà applicato il valore di default (UC4.11.3.4);
\item l'attore può modificare il valore di default per l'attributo (UC4.11.3.5);
\item l'attore può confermare le modifiche apportate all'attributo (UC4.11.3.6).
\end{enumerate}
\end{itemize}

\subparagraph{UC4.11.3.1: Modifica Visibilità Attributo}
\label{UC4.11.3.1}
\begin{itemize}
\item \textbf{Attori}: Utente;
\item \textbf{Descrizione}: l'attore può modificare la visibilità dell'attributo;	
\item \textbf{Precondizione}: il sistema visualizza il form per l'inserimento dei dati;	
\item \textbf{Postcondizione}: il sistema ha ricevuto il dato richiesto;	
\item \textbf{Scenario principale}:
l'attore modifica la visibilità dell'attributo.
\end{itemize}

\subparagraph{UC4.11.3.2: Modifica Nome Attributo}
\label{UC4.11.3.2}
\begin{itemize}
\item \textbf{Attori}: Utente;
\item \textbf{Descrizione}: l'attore può modificare il nome dell'attributo;	
\item \textbf{Precondizione}: il sistema visualizza il form per l'inserimento dei dati;	
\item \textbf{Postcondizione}: il sistema ha ricevuto il dato richiesto;	
\item \textbf{Scenario principale}:
l'attore inserisce il nome modificato dell'attributo.	
\end{itemize}

\subparagraph{UC4.11.3.3: Modifica Tipo Attributo}
\label{UC4.11.3.3}
\begin{itemize}
\item \textbf{Attori}: Utente;
\item \textbf{Descrizione}: l'attore può modificare il tipo dell'attributo;	
\item \textbf{Precondizione}: il sistema visualizza il form per l'inserimento dei dati;	
\item \textbf{Postcondizione}: il sistema ha ricevuto il dato richiesto;	
\item \textbf{Scenario principale}:
l'attore inserisce il tipo modificato dell'attributo.	
\end{itemize}

\subparagraph{UC4.11.3.4: Modifica Molteplicità Attributo}
\label{UC4.11.3.4}
\begin{itemize}
\item \textbf{Attori}: Utente;
\item \textbf{Descrizione}: l'attore può modificare la molteplicità dell'attributo;	
\item \textbf{Precondizione}: il sistema visualizza il form per l'inserimento dei dati;	
\item \textbf{Postcondizione}: il sistema ha ricevuto il dato richiesto;	
\item \textbf{Scenario principale}:
l'attore sceglie la molteplicità dell'attributo.
\end{itemize}

\subparagraph{UC4.11.3.5: Modifica Valore Default Attributo}
\label{UC4.11.3.5}
\begin{itemize}
\item \textbf{Attori}: Utente;
\item \textbf{Descrizione}: l'attore può modificare il valore di default dell'attributo;	
\item \textbf{Precondizione}: il sistema visualizza il form per l'inserimento dei dati;	
\item \textbf{Postcondizione}: il sistema ha ricevuto il dato richiesto;	
\item \textbf{Scenario principale}:
l'attore inserisce un valore di default modificato.	
\end{itemize}

\subparagraph{UC4.11.3.6: Conferma Modifica Attributo}
\label{UC4.11.3.6}
\begin{itemize}
\item \textbf{Attori}: Utente;
\item \textbf{Descrizione}: l’attore può confermare le modifiche apportate all'attributo;	
\item \textbf{Precondizione}: il sistema visualizza il form per l'inserimento dei dati;	
\item \textbf{Postcondizione}: il sistema visualizza l'attributo modificato;	
\item \textbf{Scenario principale}:
l’attore conferma le modifiche dell'attributo;	
\item \textbf{Inclusioni}:
Visualizzazione Errore Inserimento Attributo (UC4.1.4.6.1);
\item \textbf{Scenari alternativi}:
\begin{itemize}
\item l'attore non ha inserito il nome dell'attributo;
\item l'attore non ha inserito il tipo dell'attributo;
\item l'attore ha inserito un nome già esistente.
\end{itemize}
\end{itemize}

\paragraph{UC4.11.4: Modifica Metodo Classe}
\label{UC4.11.4}
\begin{itemize}
\item \textbf{Attori}: Utente;
\item \textbf{Descrizione}: l'attore modifica un metodo della classe;	
\item \textbf{Precondizione}: il sistema visualizza il form per l'inserimento dei dati;	
\item \textbf{Postcondizione}: il sistema visualizza il metodo modificato dall'attore;	
\item \textbf{Scenario principale}:
\begin{enumerate}
\item l'attore può modificare la visibilità del metodo (UC4.11.4.1);
\item l'attore può modificare il nome del metodo (UC4.11.4.2);
\item l'attore può modificare un parametro del metodo (UC4.11.4.3);
\item l'attore può modificare il tipo di ritorno del metodo (UC4.11.4.4);
\item l'attore può confermare le modifiche apportate al metodo (UC4.11.4.5).
\end{enumerate}
\end{itemize}

\subparagraph{UC4.11.4.1: Modifica Visibilità Metodo}
\label{UC4.11.4.1}
\begin{itemize}
\item \textbf{Attori}: Utente;
\item \textbf{Descrizione}: l'attore può modificare la visibilità del metodo;	
\item \textbf{Precondizione}: il sistema visualizza il form per l'inserimento dei dati;	
\item \textbf{Postcondizione}: il sistema ha ricevuto il dato richiesto;	
\item \textbf{Scenario principale}:
l'attore sceglie la visibilità del metodo.
\end{itemize}

\subparagraph{UC4.11.4.2: Modifica Nome Metodo}
\label{UC4.11.4.2}
\begin{itemize}
\item \textbf{Attori}: Utente;
\item \textbf{Descrizione}: l'attore può modificare il nome del metodo;	
\item \textbf{Precondizione}: il sistema visualizza il form per l'inserimento dei dati;	
\item \textbf{Postcondizione}: il sistema ha ricevuto il dato richiesto;	
\item \textbf{Scenario principale}:
l'attore modifica il nome del metodo.	
\end{itemize}

\subparagraph{UC4.11.4.3: Modifica Parametro Metodo}
\label{UC4.11.4.3}
\begin{itemize}
\item \textbf{Attori}: Utente;
\item \textbf{Descrizione}: l'attore modifica un parametro del metodo;	
\item \textbf{Precondizione}: il sistema visualizza il form per l'inserimento dei dati;	
\item \textbf{Postcondizione}: il sistema visualizza il parametro modificato dall'attore;	
\item \textbf{Scenario principale}:
\begin{enumerate}
\item l'attore può modificare il nome del parametro (UC4.11.4.3.1);
\item l'attore può modificare il tipo del parametro (UC4.11.4.3.2);
\item l'attore conferma le modifiche apportate al parametro (UC4.11.4.3.3).
\end{enumerate}	
\end{itemize}

\subparagraph{UC4.11.4.3.1: Modifica Nome Parametro}
\label{UC4.11.4.3.1}
\begin{itemize}
\item \textbf{Attori}: Utente;
\item \textbf{Descrizione}: l'attore può modificare il nome del parametro del metodo;	
\item \textbf{Precondizione}: il sistema visualizza il form per l'inserimento dei dati;	
\item \textbf{Postcondizione}: il sistema ha ricevuto il dato richiesto;	
\item \textbf{Scenario principale}:
l'attore inserisce il nome modificato del parametro.	
\end{itemize}

\subparagraph{UC4.11.4.3.2: Modifica Tipo Parametro}
\label{UC4.11.4.3.2}
\begin{itemize}
\item \textbf{Attori}: Utente;
\item \textbf{Descrizione}: l'attore può modificare il tipo del metodo;	
\item \textbf{Precondizione}: il sistema visualizza il form per l'inserimento dei dati;	
\item \textbf{Postcondizione}: il sistema ha ricevuto il dato richiesto;	
\item \textbf{Scenario principale}:
l'attore inserisce il tipo modificato del parametro.	
\end{itemize}

\subparagraph{UC4.11.4.3.3: Conferma Modifica Parametro}
\label{UC4.11.4.3.3}
\begin{itemize}
\item \textbf{Attori}: Utente;
\item \textbf{Descrizione}: l’attore può confermare le modifiche apportate al parametro;	
\item \textbf{Precondizione}: il sistema visualizza il form per l'inserimento dei dati;	
\item \textbf{Postcondizione}: il sistema visualizza il parametro modificato;	
\item \textbf{Scenario principale}:
l’attore visualizza il parametro modificato;	
\item \textbf{Inclusioni}:
Visualizzazione Errore Inserimento Parametro (UC4.1.6.3.3.1);
\item \textbf{Scenari alternativi}:
\begin{itemize}
\item l'attore non ha inserito il nome del parametro;
\item l'attore non ha inserito il tipo del parametro.
\end{itemize}
\end{itemize}

\subparagraph{UC4.11.4.4: Modifica Tipo Ritorno Metodo}
\label{UC4.11.4.4}
\begin{itemize}
\item \textbf{Attori}: Utente;
\item \textbf{Descrizione}: l'attore può modificare il tipo di ritorno del metodo;	
\item \textbf{Precondizione}: il sistema visualizza il form per l'inserimento dei dati;	
\item \textbf{Postcondizione}: il sistema ha ricevuto il dato richiesto;	
\item \textbf{Scenario principale}:
l'attore inserisce il tipo di ritorno modificato del metodo.	
\end{itemize}

\subparagraph{UC4.11.4.5: Conferma Modifica Metodo}
\label{UC4.11.4.5}
\begin{itemize}
\item \textbf{Attori}: Utente;
\item \textbf{Descrizione}: l’attore può confermare le modifiche apportate al metodo;	
\item \textbf{Precondizione}: il sistema visualizza il form per l'inserimento dei dati;	
\item \textbf{Postcondizione}: il sistema visualizza il metodo modificato;	
\item \textbf{Scenario principale}:
l’attore conferma le modifiche apportate al metodo;	
\item \textbf{Inclusioni}:
Visualizzazione Errore Inserimento Metodo (UC4.1.6.5.1);
\item \textbf{Scenari alternativi}:
\begin{itemize}
\item l'attore non ha inserito il nome del metodo;
\item l'attore non ha inserito il tipo di ritorno del metodo;
\item l'attore ha inserito un metodo con una segnatura già esistente.
\end{itemize}
\end{itemize}

\paragraph{UC4.11.5: Conferma Modifica Classe}
\label{UC4.11.5}
\begin{itemize}
\item \textbf{Attori}: Utente;
\item \textbf{Descrizione}: l'attore può confermare le modifiche apportate alla classe;
\item \textbf{Precondizione}: il sistema visualizza il form per l'inserimento dei dati;	
\item \textbf{Postcondizione}: il sistema visualizza la classe modificata;	
\item \textbf{Scenario principale}:
l’attore conferma le modifiche apportate alla classe;	
\item \textbf{Inclusioni}:
Visualizzazione Errore Inserimento Classe (UC4.1.8.1);
\item \textbf{Scenari alternativi}:
\begin{itemize}
\item l'attore non ha inserito il nome della classe;
\item l'attore ha inserito un nome per la classe già esistente.
\end{itemize}
\end{itemize}

\subsubsection{UC4.12: Modifica Relazione}
\label{UC4.12}
\begin{itemize}
\item \textbf{Attori}: ;
\item \textbf{Descrizione}: l'attore può modificare una relazione tra due classi;	
\item \textbf{Precondizione}: l'attore seleziona la relazione da modificare;	
\item \textbf{Postcondizione}: il sistema visualizza la relazione modificata dall'attore;	
\item \textbf{Scenario principale}:
\begin{enumerate}
\item l'attore può modificare il tipo della relazione (UC4.12.1);
\item l'attore può modificare l'attributo della relazione (UC4.12.2);
\item l'attore può modificare la classe1 della relazione (UC4.12.3);
\item l'attore può modificare la classe2 della relazione (UC4.12.4);
\item l'attore può modificare la cardinalità della relazione (UC4.12.5); 
\item l'attore può confermare le modifiche apportate alla relazione (UC4.12.6);

\end{enumerate}	
\end{itemize}

\paragraph{UC4.12.1: Modifica Tipo Relazione}
\label{UC4.12.1}
\begin{itemize}
\item \textbf{Attori}: Utente;
\item \textbf{Descrizione}: l'attore può modificare il tipo di una relazione già inserita;
\item \textbf{Precondizione}: l'attore seleziona la relazione da modificare e il sistema visualizza il form per l'inserimento dei dati;
\item \textbf{Postcondizione}: il sistema ha ricevuto il dato richiesto;
\item \textbf{Scenario principale}:
l'attore modifica il tipo della relazione selezionata.
\end{itemize}

\paragraph{UC4.12.2: Modifica Attributo Relazione}
\label{UC4.12.2}
\begin{itemize}
\item \textbf{Attori}: Utente;
\item \textbf{Descrizione}: l'attore può modificare l'attributo di una relazione già inserita;
\item \textbf{Precondizione}: l'attore seleziona la relazione da modificare e il sistema visualizza il form per l'inserimento dei dati;
\item \textbf{Postcondizione}: il sistema ha ricevuto il dato richiesto;
\item \textbf{Scenario principale}:
l'attore modifica l'attributo della relazione selezionata.
\end{itemize}

\paragraph{UC4.12.3: Modifica Classe1 Relazione}
\label{UC4.12.3}
\begin{itemize}
\item \textbf{Attori}: Utente;
\item \textbf{Descrizione}: l'attore può modificare la classe1 della relazione selezionata;
\item \textbf{Precondizione}: l'attore seleziona la relazione da modificare il sistema visualizza il form per l'inserimento dei dati
\item \textbf{Postcondizione}: il sistema ha ricevuto il dato richiesto;
\item \textbf{Scenario principale}:
l'attore modifica la classe di partenza della relazione;
\end{itemize}

\paragraph{UC4.12.4: Modifica Classe2 Relazione}
\label{UC4.12.4}
\begin{itemize}
\item \textbf{Attori}: Utente;
\item \textbf{Descrizione}: l'attore può modificare la classe2 della relazione selezionata per la modifica;
\item \textbf{Precondizione}: l'attore seleziona la relazione da modificare e il sistema visualizza il form per l'inserimento dei dati	
\item \textbf{Postcondizione}: il sistema ha ricevuto il dato richiesto;	
\item \textbf{Scenario principale}:
l'attore modifica la classe di destinazione della relazione;	
\end{itemize}

\paragraph{UC4.12.5: Modifica Cardinalità Relazione}
\label{UC4.12.5}
\begin{itemize}
\item \textbf{Attori}: Utente;
\item \textbf{Descrizione}: l'attore può modificare la cardinalità di una relazione;
\item \textbf{Precondizione}: l'attore seleziona la relazione da modificare e il sistema visualizza il form per l'inserimento dei dati	
\item \textbf{Postcondizione}: il sistema ha ricevuto il dato richiesto;	
\item \textbf{Scenario principale}:
l'attore modifica la cardinalità della relazione.
\end{itemize}

\paragraph{UC4.12.6: Conferma Modifica Relazione}
\label{UC4.12.6}
\begin{itemize}
\item \textbf{Attori}: Utente;
\item \textbf{Descrizione}: l'attore può confermare le modifiche apportate alla relazione;	
\item \textbf{Precondizione}: l'attore seleziona la relazione da modificare e il sistema visualizza il form per l'inserimento dei dati;	
\item \textbf{Postcondizione}:  il sistema visualizza graficamente la relazione modificata creando un collegamento grafico tra le classi indicate dall'attore conforme alla tipologia di relazione specificata;
\item \textbf{Scenario principale}:
l'attore conferma le modifiche apportate alla relazione;	
\item \textbf{Estensioni}:
l'attore visualizza un messaggio d'errore sull'inserimento dei dati (UC4.2.6.1);
\item \textbf{Scenari alternativi}:
\begin{itemize}
	\item l'attore non ha scelto il tipo della relazione;
	\item l'attore non ha inserito il nome della relazione;
	\item l'attore non ha scelto la classe di partenza della relazione;
	\item l'attore non ha scelto la classe di destinazione della relazione.
\end{itemize}
\end{itemize}

\subsubsection{UC4.13: Modifica Commento}
\label{UC4.13}
\begin{itemize}
\item \textbf{Attori}: ;
\item \textbf{Descrizione}: l'attore può modificare un commento;	
\item \textbf{Precondizione}: l'attore seleziona il commento da modificare;	
\item \textbf{Postcondizione}: il sistema visualizza il commento modificato dall'attore;	
\item \textbf{Scenario principale}:
\begin{enumerate}
\item l'attore può modificare il testo del commento (UC4.13.1);
\item l'attore può modificare il "parent" del commento (UC4.13.2);
\item l'attore può confermare le modifiche apportate al commento (UC4.13.3).
\end{enumerate}
\end{itemize}

\paragraph{UC4.13.1: Modifica Testo Commento}
\label{UC4.13.1}
\begin{itemize}
\item \textbf{Attori}: ;
\item \textbf{Descrizione}: l'attore può modificare il testo del commento;	
\item \textbf{Precondizione}: il sistema visualizza il form per l'inserimento dei dati	
\item \textbf{Postcondizione}: il sistema ha ricevuto il dato richiesto;	
\item \textbf{Scenario principale}:
l'attore inserisce il testo modificato del commento;	
\end{itemize}

\paragraph{UC4.13.2: Modifica Parent Commento}
\label{UC4.13.2}
\begin{itemize}
\item \textbf{Attori}: ;
\item \textbf{Descrizione}: l'attore può modificare il parent del commento;	
\item \textbf{Precondizione}: il sistema visualizza il form per l'inserimento dei dati	
\item \textbf{Postcondizione}: il sistema ha ricevuto il dato richiesto;	
\item \textbf{Scenario principale}:
l'attore sceglie il parent del commento.
\end{itemize}

\paragraph{UC4.13.3: Conferma Modifica Commento}
\label{UC4.13.3}
\begin{itemize}
\item \textbf{Attori}: ;
\item \textbf{Descrizione}: l’attore può confermare le modifiche apportate al commento;	
\item \textbf{Precondizione}: il sistema visualizza il form per l'inserimento dei dati;	
\item \textbf{Postcondizione}: il sistema visualizza il commento modificato;	
\item \textbf{Scenario principale}:
l’attore conferma le modifiche apportate al commento;	
\item \textbf{Inclusioni}:
Visualizzazione Errore Inserimento Commento (UC4.3.3.1);
\item \textbf{Scenari alternativi}:
\begin{itemize}
\item l'attore non ha scelto il parent del commento. 
\end{itemize}
\end{itemize}

\subsection{UC5: Realizzazione Diagramma delle Attività}
\label{UC5}
\begin{itemize}
\item \textbf{Attori}: Utente;
\item \textbf{Descrizione}: l'attore realizza e gestisce un diagramma delle attività per ogni metodo presente nel diagramma delle classi in modo tale da darne una implementazione più in dettaglio;	
\item \textbf{Precondizione}: il sistema è passato dalla sezione dell'editor dedicata alla gestione del diagramma delle classi alla sezione dell'editor dedicata alla gestione del diagramma delle attività del metodo selezionato;	
\item \textbf{Postcondizione}: il sistema visualizza il diagramma delle attività del metodo selezionato come realizzato dall'utente del programma;
\item \textbf{Scenario principale}:
\begin{enumerate}
\item l'attore può inserire un blocco variabile (UC5.1);
\item l'attore può inserire un blocco metodo (UC5.2);
\item l'attore può inserire un blocco if/else (UC5.3);
\item l'attore può inserire un blocco while (UC5.4);
\item l'attore può inserire un blocco for (UC5.5);
\item l'attore può inserire un blocco custom (UC5.6);
\item l'attore può rimuovere un blocco variabile (UC5.7);
\item l'attore può rimuovere un blocco metodo (UC5.8);
\item l'attore può rimuovere un blocco if/else (UC5.9);
\item l'attore può rimuovere  un blocco while (UC5.10);
\item l'attore può rimuovere un blocco for (UC5.11);
\item l'attore può rimuovere un blocco custom (UC5.12);
\item l'attore può ridurre un blocco if/else (UC5.13);
\item l'attore può espandere un blocco if/else (UC5.14);
\item l'attore può ridurre un blocco while (UC5.15);
\item l'attore può espandere un blocco while (UC5.16);
\item l'attore può ridurre un blocco for (UC5.17);
\item l'attore può espandere un blocco for (UC5.18);
\item l'attore può spostare un qualsiasi blocco già inserito (UC5.19).
\end{enumerate}
\end{itemize}

\subsubsection{UC5.1: Inserimento Blocco Variabile}
\label{UC5.1}
\begin{itemize}
\item \textbf{Attori}: Utente;
\item \textbf{Descrizione}: l'attore può inserire un blocco variabile scegliendo tra un blocco variabile di inizializzazione e un blocco variabile di assegnazione;
\item \textbf{Precondizione}: il sistema, nella sezione dell'editor dedicata alla gestione del diagramma delle attività, visualizza un diagramma delle attività (eventualmente vuoto) associato ad un particolare metodo;
\item \textbf{Postcondizione}: il sistema visualizza il blocco inserito mostrando eventualmente il commento specificato dall'attore e posizionandolo successivamente all'ultimo blocco generico precedentemente inserito nel diagramma delle attività corrente;
\item \textbf{Scenario principale}:
\begin{enumerate}
\item l'attore può scegliere di creare e inizializzare una variabile (UC5.1.1);
\item l'attore può scegliere di assegnare un valore ad una variabile esistente da scegliere (UC5.1.2);
\item l'attore può scegliere di inserire un commento (UC5.1.3);
\item l'attore può confermare l'inserimento del blocco variabile (UC5.1.4).
\end{enumerate}
\end{itemize}

\paragraph{UC5.1.1: Inizializzazione Variabile}
\label{UC5.1.1}
\begin{itemize}
\item \textbf{Attori}: Utente;
\item \textbf{Descrizione}: l'attore può creare ed inizializzare una variabile assegnandole un valore;
\item \textbf{Precondizione}: l'attore ha selezionato un blocco di inizializzazione variabile da inserire e il sistema visualizza il relativo form per l'inserimento dei dati;
\item \textbf{Postcondizione}: il sistema ha ricevuto il dato richiesto;
\item \textbf{Scenario principale}:
l'attore deve indicare il nome la variabile da creare ed eventualmente un valore da assegnarle: tale valore può essere anche ritornato da un metodo che può essere scelto tra quelli proposti.
\end{itemize}

\paragraph{UC5.1.2: Assegnazione Variabile}
\label{UC5.1.2}
\begin{itemize}
\item \textbf{Attori}: Utente;
\item \textbf{Descrizione}: l'attore può assegnare un valore ad una variabile esistente e visibile;
\item \textbf{Precondizione}: l'attore ha selezionato un blocco di assegnazione variabile da inserire e il sistema visualizza il relativo form per l'inserimento dei dati;	
\item \textbf{Postcondizione}: il sistema ha ricevuto il dato richiesto;	
\item \textbf{Scenario principale}:
l'attore deve fornire un valore da assegnare ad una variabile esistente e visibile: tale valore può essere anche ritornato da un metodo che può essere scelto tra quelli proposti.
\end{itemize}

\paragraph{UC5.1.3: Inserimento Commento Blocco Variabile}
\label{UC5.1.3}
\begin{itemize}
\item \textbf{Attori}: Utente;
\item \textbf{Descrizione}: l'attore deve inserire un commento per il blocco variabile selezionato;	
\item \textbf{Precondizione}: l'attore ha selezionato un blocco di assegnazione variabile da inserire e il sistema visualizza il relativo form per l'inserimento dei dati;
\item \textbf{Postcondizione}: il sistema ha ricevuto il dato richiesto;	
\item \textbf{Scenario principale}:
l'attore inserisce un commento per il blocco variabile selezionato;	
\end{itemize}

\paragraph{UC5.1.4: Conferma Inserimento Blocco Variabile}
\label{UC5.1.4}
\begin{itemize}
\item \textbf{Attori}: Utente;
\item \textbf{Descrizione}: l'attore può confermare l'inserimento del blocco variabile selezionato;	
\item \textbf{Precondizione}: l'attore ha selezionato un blocco di assegnazione variabile da inserire e il sistema visualizza il relativo form per l'inserimento dei dati;
\item \textbf{Postcondizione}: il sistema visualizza il blocco variabile inserito posizionandolo successivamente all'ultimo blocco generico precedentemente inserito nel diagramma delle attività corrente;	
\item \textbf{Scenario principale}:
l'attore conferma l'inserimento del blocco variabile selezionato;	
\item \textbf{Estensioni}:
l'attore visualizza un messaggio d'errore sull'inserimento dei dati (UC5.1.4.1);	
\item \textbf{Scenari alternativi}:
\begin{itemize}
\item l'attore non ha indicato il nome della variabile da inizializzare;
\item l'attore non ha fornito un valore da assegnare alla variabile già esistente e visibile;
\item l'attore non ha inserito un commento.	
\end{itemize}
\end{itemize}

\subparagraph{UC5.1.4.1: Visualizzazione Errore Inserimento B.Variabile	}
\label{UC5.1.4.1}
\begin{itemize}
\item \textbf{Attori}: Utente;
\item \textbf{Descrizione}: l'attore può visualizzare un messaggio d'errore in corrispondenza della conferma di inserimento nel caso si siano verificati uno o più scenari alternativi durante la fase di inserimento di un blocco variabile;	
\item \textbf{Precondizione}: il sistema non ha ricevuto sufficienti informazioni per l'inserimento di un blocco variabile;	
\item \textbf{Postcondizione}: il sistema avvisa l'attore dell'errore verificatosi tramite un opportuno messaggio;	
\item \textbf{Scenario principale}:
l'attore visualizza un messaggio d'errore.	
\end{itemize}

\subsubsection{UC5.2: Inserimento Blocco Metodo}
\label{UC5.2}
\begin{itemize}
\item \textbf{Attori}: Utente;
\item \textbf{Descrizione}: l'attore può inserire un blocco metodo allo scopo di inserire all'interno del diagramma la chiamata ad un particolare metodo definito altrove (eventualmente nel corrispondente diagramma delle classi);	
\item \textbf{Precondizione}: il sistema, nella sezione dell'editor dedicata alla gestione del diagramma delle attività, visualizza un diagramma delle attività (eventualmente vuoto) associato ad un particolare metodo;	
\item \textbf{Postcondizione}:  il sistema visualizza il blocco inserito mostrando eventualmente il commento specificato dall'attore e posizionandolo successivamente all'ultimo blocco generico precedentemente inserito nel diagramma delle attività corrente;
\item \textbf{Scenario principale}:
\begin{enumerate}
\item l'attore può scegliere di inserire un metodo (UC5.2.1);
\item l'attore può scegliere di inserire un commento (UC5.2.2);
\item l'attore può confermare l'inserimento del blocco metodo (UC5.2.3).
\end{enumerate}
\end{itemize}

\paragraph{UC5.2.1: Chiamata di Metodo}
\label{UC5.2.1}
\begin{itemize}
\item \textbf{Attori}: Utente;
\item \textbf{Descrizione}: l'attore deve invocare un metodo da inserire all'interno del blocco tra quelli disponibili;	
\item \textbf{Precondizione}:  l'attore ha selezionato un blocco metodo da inserire e il sistema visualizza il form per l'inserimento dei dati;	
\item \textbf{Postcondizione}: il sistema ha ricevuto il dato richiesto;	
\item \textbf{Scenario principale}:
l'attore deve scegliere un metodo da invocare tra quelli proposti.	
\end{itemize}

\paragraph{UC5.2.2: Inserimento Commento B.Metodo}
\label{UC5.2.2}
\begin{itemize}
\item \textbf{Attori}: Utente;
\item \textbf{Descrizione}: l'attore deve inserire un commento per il blocco metodo da inserirsi;	
\item \textbf{Precondizione}: l'attore ha selezionato un blocco metodo da inserire e il sistema visualizza il form per l'inserimento dei dati;	
\item \textbf{Postcondizione}: il sistema ha ricevuto il dato richiesto;	
\item \textbf{Scenario principale}:
l'attore inserisce un commento per il blocco metodo selezionato.
\end{itemize}

\paragraph{UC5.2.3: Conferma Inserimento Blocco Metodo}
\label{UC5.2.3}
\begin{itemize}
\item \textbf{Attori}: Utente;
\item \textbf{Descrizione}: l'attore può confermare l'inserimento del blocco metodo;	
\item \textbf{Precondizione}:l'attore ha selezionato un blocco metodo da inserire e il sistema visualizza il form per l'inserimento dei dati;	
\item \textbf{Postcondizione}: il sistema visualizza il blocco metodo inserito posizionandolo successivamente all'ultimo blocco generico precedentemente inserito nel diagramma delle attività corrente;		
\item \textbf{Scenario principale}:
l'attore conferma l'inserimento del blocco metodo;	
\item \textbf{Estensioni}:
l'attore visualizza un messaggio d'errore sull'inserimento dei dati (UC5.2.3.1);	
\item \textbf{Scenari alternativi}:
\begin{itemize}
\item l'attore non ha indicato un metodo da inserire;
\item l'attore non ha inserito un commento a corredo del blocco metodo.
\end{itemize}
\end{itemize}

\subparagraph{UC5.2.3.1: Visualizzazione Errore Inserimento B.Metodo}
\label{UC5.2.3.1}
\begin{itemize}
\item \textbf{Attori}: Utente;
\item \textbf{Descrizione}: l'attore può visualizzare un messaggio d'errore in corrispondenza della conferma di inserimento nel caso si siano verificati uno o più scenari alternativi durante la fase di inserimento di un blocco metodo;	
\item \textbf{Precondizione}: il sistema non ha ricevuto sufficienti informazioni per l'inserimento di un blocco metodo;	
\item \textbf{Postcondizione}: il sistema avvisa l'attore dell'errore verificatosi tramite un opportuno messaggio;	
\item \textbf{Scenario principale}:
l'attore visualizza un messaggio d'errore.	
\end{itemize}

\subsubsection{UC5.3: Inserimento Blocco If/Else}
\label{UC5.3}
\begin{itemize}
\item \textbf{Attori}: Utente;
\item \textbf{Descrizione}: l'attore può inserire un blocco if/else e in caso voglia indicare semplicemente la condizione di if senza andare a specificare quella antitetica, sarà per lui sufficiente lasciare vuoto il corpo del blocco else;	
\item \textbf{Precondizione}: il sistema, nella sezione dell'editor dedicata alla gestione del diagramma delle attività, visualizza un diagramma delle attività (eventualmente vuoto) associato ad un particolare metodo;	
\item \textbf{Postcondizione}: il sistema visualizza il blocco inserito mostrando il commento inserito dall'attore a corredo e posizionandolo successivamente all'ultimo blocco generico precedentemente inserito nel diagramma delle attività corrente;	
\item \textbf{Scenario principale}:
\begin{enumerate}
\item l'attore può scegliere di inserire la condizione da verificare (UC5.3.1);
\item l'attore può scegliere di inserire il contenuto del blocco if (UC5.3.2);
\item l'attore può scegliere di inserire il contenuto del blocco else (UC5.3.3);
\item l'attore può scegliere di inserire un commento (UC5.3.4);
\item l'attore può confermare l'inserimento del blocco if/else (UC5.3.5).
\end{enumerate}
\end{itemize}

\paragraph{UC5.3.1: Condizione Blocco If}
\label{UC5.3.1}
\begin{itemize}
\item \textbf{Attori}: Utente;
\item \textbf{Descrizione}: l'attore può inserire la condizione da verificare del blocco if;	da inserire e il sistema visualizza il relativo form per l'inserimento dei dati; 
\item \textbf{Postcondizione}: il sistema ha ricevuto il dato richiesto;	
\item \textbf{Scenario principale}:
\item \textbf{Precondizione}: l'attore ha selezionato un blocco if 
l'attore deve inserire la condizione da verificare del blocco if come espressione booleana espressa con il linguaggio target di generazione del codice dell'applicazione.
\end{itemize}

\paragraph{UC5.3.2: Corpo Blocco If}
\label{UC5.3.2}
\begin{itemize}
\item \textbf{Attori}: Utente;
\item \textbf{Descrizione}: l'attore può inserire una serie di blocchi tra quelli resi disponibili dall'editor per descrivere il comportamento del corpo del blocco if stesso;	
\item \textbf{Precondizione}: l'attore ha selezionato un blocco if da inserire e il sistema visualizza il relativo form per l'inserimento dei dati;	
\item \textbf{Postcondizione}: il sistema ha ricevuto il dato richiesto;	
\item \textbf{Scenario principale}:
l'attore deve inserire uno o più blocchi tra quelli resi disponibili dall'editor.	
\end{itemize}

\paragraph{UC5.3.3: Corpo Blocco Else	}
\label{UC5.3.3}
\begin{itemize}
\item \textbf{Attori}: Utente;
\item \textbf{Descrizione}: l'attore può inserire una serie di blocchi tra quelli resi disponibili dall'editor per descrivere il comportamento del corpo del blocco else stesso;	
\item \textbf{Precondizione}: l'attore ha selezionato un blocco if da inserire e il sistema visualizza il relativo form per l'inserimento dei dati;	
\item \textbf{Postcondizione}: il sistema ha ricevuto il dato richiesto;	
\item \textbf{Scenario principale}:
l'attore deve inserire uno o più blocchi tra quelli resi disponibili dall'editor.	
\end{itemize}

\paragraph{UC5.3.4: Inserimento Commento B.If/Else	}
\label{UC5.3.4}
\begin{itemize}
\item \textbf{Attori}: Utente;
\item \textbf{Descrizione}: l'attore deve inserire un commento per il blocco if/else;	
\item \textbf{Precondizione}: l'attore ha selezionato un blocco if da inserire e il sistema visualizza il relativo form per l'inserimento dei dati;	
\item \textbf{Postcondizione}: il sistema ha ricevuto il dato richiesto;	
\item \textbf{Scenario principale}:
l'attore inserisce un commento per il blocco if/else.
\end{itemize}

\paragraph{UC5.3.5: Conferma Inserimento Blocco If/Else}
\label{UC5.3.5}
\begin{itemize}
\item \textbf{Attori}: Utente;
\item \textbf{Descrizione}: l'attore può confermare l'inserimento del blocco if/else;	
\item \textbf{Precondizione}: l'attore ha selezionato un blocco if da inserire e il sistema visualizza il relativo form per l'inserimento dei dati;	
\item \textbf{Postcondizione}: il sistema visualizza il blocco if/else inserito posizionandolo successivamente all'ultimo blocco generico precedentemente inserito nel diagramma delle attività corrente;
\item \textbf{Scenario principale}:
l'attore conferma l'inserimento del blocco if/else selezionato;	
\item \textbf{Estensioni}:
l'attore visualizza un messaggio d'errore sull'inserimento dei dati (UC5.3.5.1);	
\item \textbf{Scenari alternativi}:
\begin{itemize}
\item l'attore non ha fornito la condizione da verificare;
\item l'attore non ha inserito un commento.
\end{itemize}
\end{itemize}

\subparagraph{UC5.3.5.1: Visualizzazione Errore Inserimento B.If/Else}
\label{UC5.3.5.1}
\begin{itemize}
\item \textbf{Attori}: Utente;
\item \textbf{Descrizione}: l'attore può visualizzare un messaggio d'errore in corrispondenza della conferma di inserimento nel caso si siano verificati uno o più scenari alternativi durante la fase di inserimento di un blocco if/else;	
\item \textbf{Precondizione}: il sistema non ha ricevuto sufficienti informazioni per l'inserimento di un blocco if/else;	
\item \textbf{Postcondizione}: il sistema avvisa l'attore dell'errore verificatosi tramite un opportuno messaggio;	
\item \textbf{Scenario principale}:
l'attore visualizza un messaggio d'errore.	
\end{itemize}

\subsubsection{UC5.4: Inserimento Blocco While}
\label{UC5.4}
\begin{itemize}
\item \textbf{Attori}: Utente;
\item \textbf{Descrizione}: l'attore può inserire un blocco while specificandone condizione e istruzioni del corrispondente corpo di codice;	
\item \textbf{Precondizione}: il sistema, nella sezione dell'editor dedicata alla gestione del diagramma delle attività, visualizza un diagramma delle attività (eventualmente vuoto) associato ad un particolare metodo;	
\item \textbf{Postcondizione}: il sistema visualizza il blocco inserito mostrando il commento inserito dall'attore e posizionandolo successivamente all'ultimo blocco generico precedentemente inserito nel diagramma delle attività corrente;	
\item \textbf{Scenario principale}:
\begin{enumerate}
\item l'attore può scegliere di inserire la condizione da verificare (UC5.4.1);
\item l'attore può scegliere di inserire il corpo del blocco while (UC5.4.2);
\item l'attore può scegliere di inserire un commento (UC5.4.3);
\item l'attore può confermare l'inserimento del blocco while (UC5.4.4).
\end{enumerate}
\end{itemize}

\paragraph{UC5.4.1: Condizione Blocco While	}
\label{UC5.4.1}
\begin{itemize}
\item \textbf{Attori}: Utente;
\item \textbf{Descrizione}: l'attore può inserire la condizione da verificare del blocco while;	
\item \textbf{Precondizione}: l'attore ha selezionato un blocco while da inserire e il sistema visualizza il relativo form per l'inserimento dei dati;
\item \textbf{Postcondizione}: il sistema ha ricevuto il dato richiesto;	
\item \textbf{Scenario principale}:
l'attore deve inserire la condizione da verificare del blocco while come espressione booleana espressa con il linguaggio target di generazione del codice dell'applicazione.
\end{itemize}

\paragraph{UC5.4.2: Corpo Blocco While}
\label{UC5.4.2}
\begin{itemize}
\item \textbf{Attori}: Utente;
\item \textbf{Descrizione}: l'attore può inserire uno o più blocchi tra quelli resi disponibili dall'editor per descrivere il comportamento del corpo del blocco while stesso;	
\item \textbf{Precondizione}: l'attore ha selezionato un blocco while da inserire e il sistema visualizza il relativo form per l'inserimento dei dati;	
\item \textbf{Postcondizione}: il sistema ha ricevuto il dato richiesto;	
\item \textbf{Scenario principale}:
l'attore deve inserire uno o più blocchi tra quelli resi disponibili dall'editor.	
\end{itemize}

\paragraph{UC5.4.3: Inserimento Commento B.While	}
\label{UC5.4.3}
\begin{itemize}
\item \textbf{Attori}: Utente;
\item \textbf{Descrizione}: l'attore deve inserire un commento per il blocco while;	
\item \textbf{Precondizione}: l'attore ha selezionato un blocco while da inserire e il sistema visualizza il relativo form per l'inserimento dei dati;	
\item \textbf{Postcondizione}: il sistema ha ricevuto il dato richiesto;	
\item \textbf{Scenario principale}:
l'attore inserisce un commento per il blocco while.
\item \textbf{Scenari alternativi}:
	

\end{itemize}

\paragraph{UC5.4.4: Conferma Inserimento Blocco While}
\label{UC5.4.4}
\begin{itemize}
\item \textbf{Attori}: Utente;
\item \textbf{Descrizione}: l'attore può confermare l'inserimento del blocco while;	
\item \textbf{Precondizione}: l'attore ha selezionato un blocco while da inserire e il sistema visualizza il relativo form per l'inserimento dei dati;	
\item \textbf{Postcondizione}: il sistema visualizza il blocco while inserito posizionandolo successivamente all'ultimo blocco generico precedentemente inserito nel diagramma delle attività corrente;	
\item \textbf{Scenario principale}:
l'attore conferma l'inserimento del blocco while;	
\item \textbf{Estensioni}:
l'attore visualizza un messaggio d'errore sull'inserimento dei dati (UC5.4.4.1);	
\item \textbf{Scenari alternativi}:
\begin{itemize}
\item l'attore non ha fornito la condizione da verificare;
\item l'attore non ha inserito un commento.
\end{itemize}
\end{itemize}

\subparagraph{UC5.4.4.1: Visualizzazione Errore Inserimento B.While}
\label{UC5.4.4.1}
\begin{itemize}
\item \textbf{Attori}: Utente;
\item \textbf{Descrizione}: l'attore può visualizzare un messaggio d'errore in corrispondenza della conferma di inserimento nel caso si fossero verificati uno o più scenari alternativi durante la fase di inserimento di un blocco while;	
\item \textbf{Precondizione}: il sistema non ha ricevuto sufficienti informazioni per l'inserimento di un blocco while;	
\item \textbf{Postcondizione}: il sistema avvisa l'attore dell'errore verificatosi tramite un opportuno messaggio;	
\item \textbf{Scenario principale}:
l'attore visualizza un messaggio d'errore.	
\end{itemize}

\subsection{UC5.5: Inserimento Blocco For}
\label{UC5.5}
\begin{itemize}
\item \textbf{Attori}: Utente;
\item \textbf{Descrizione}: l'attore può inserire un blocco for specificandone condizione, passo e le istruzioni che andranno a comporre il relativo corpo di codice;	
\item \textbf{Precondizione}: il sistema, nella sezione dell'editor dedicata alla gestione del diagramma delle attività, visualizza un diagramma delle attività (eventualmente vuoto) associato ad un particolare metodo;	
\item \textbf{Postcondizione}: il sistema visualizza il blocco inserito mostrando il commento inserito dall'attore e posizionandolo successivamente all'ultimo blocco generico precedentemente 
inserito nel diagramma delle attività corrente;	
\item \textbf{Scenario principale}:
\begin{enumerate}
\item l'attore può scegliere di inserire il codice di inizializzazione (UC5.5.1);
\item l'attore può scegliere di inserire la condizione da verificare (UC5.5.2);
\item l'attore può scegliere di inserire il passo di incremento/decremento (UC5.5.3);
\item l'attore può scegliere di inserire il corpo del for (UC5.5.4);
\item l'attore può scegliere di inserire un commento (UC5.5.5);
\item l'attore può confermare l'inserimento del blocco for (UC5.5.6).
\end{enumerate}
\end{itemize}

\paragraph{UC5.5.1: Inizializzazione Blocco For}
\label{UC5.5.1}
\begin{itemize}
\item \textbf{Attori}: Utente;
\item \textbf{Descrizione}: l'attore può inserire l'inizializzazione del blocco for;
\item \textbf{Precondizione}:  l'attore ha selezionato un blocco for da inserire e il sistema visualizza il relativo form per l'inserimento dei dati;	
\item \textbf{Postcondizione}: il sistema ha ricevuto il dato richiesto;	
\item \textbf{Scenario principale}:
l'attore deve inserire l'inizializzazione del blocco for creando ed inizializzando una variabile o scegliendone una tra quelle disponibili.
\end{itemize}

\paragraph{UC5.5.2: Condizione Blocco For}
\label{UC5.5.2}
\begin{itemize}
\item \textbf{Attori}: Utente;
\item \textbf{Descrizione}: l'attore può inserire la condizione da verificare all'interno blocco for;	
\item \textbf{Precondizione}:  l'attore ha selezionato un blocco for da inserire e il sistema visualizza il relativo form per l'inserimento dei dati;	
\item \textbf{Postcondizione}: il sistema ha ricevuto il dato richiesto;	
\item \textbf{Scenario principale}:
l'attore deve inserire la condizione da verificare del blocco for come espressione booleana espressa con il linguaggio target di generazione del codice dell'applicazione.	
\end{itemize}

\paragraph{UC5.5.3: Incremento/Decremento Blocco For}
\label{UC5.5.3}
\begin{itemize}
\item \textbf{Attori}: Utente;
\item \textbf{Descrizione}: l'attore può inserire il passo di incremento/decremento del blocco for;	
\item \textbf{Precondizione}: l'attore ha selezionato un blocco for da inserire e il sistema visualizza il relativo form per l'inserimento dei dati;	
\item \textbf{Postcondizione}: il sistema ha ricevuto il dato richiesto;	
\item \textbf{Scenario principale}:
l'attore deve inserire il passo di incremento/decremento del blocco for scegliendo una variabile tra quelle disponibili e visibili.	
\end{itemize}

\paragraph{UC5.5.4: Corpo Blocco For}
\label{UC5.5.4}
\begin{itemize}
\item \textbf{Attori}: Utente;
\item \textbf{Descrizione}: l'attore può inserire uno o più blocchi tra quelli resi disponibili dall'editor per descrivere il comportamento del corpo del blocco for stesso;	
\item \textbf{Precondizione}: l'attore ha selezionato un blocco for da inserire e il sistema visualizza il relativo form per l'inserimento dei dati;	
\item \textbf{Postcondizione}: il sistema ha ricevuto il dato richiesto;	
\item \textbf{Scenario principale}:
l'attore deve inserire uno o più blocchi tra quelli resi disponibili dall'editor.	
\end{itemize}

\paragraph{UC5.5.5: Inserimento Commento B.For	}
\label{UC5.5.5}
\begin{itemize}
\item \textbf{Attori}: Utente;
\item \textbf{Descrizione}: l'attore deve inserire un commento per il blocco for;	
\item \textbf{Precondizione}: l'attore ha selezionato un blocco for da inserire e il sistema visualizza il relativo form per l'inserimento dei dati;	
\item \textbf{Postcondizione}: il sistema ha ricevuto il dato richiesto;	
\item \textbf{Scenario principale}:
l'attore inserisce un commento per il blocco for.	
\item \textbf{Scenari alternativi}:
\end{itemize}

\paragraph{UC5.5.6: Conferma Inserimento Blocco For}
\label{UC5.5.6}
\begin{itemize}
\item \textbf{Attori}: Utente;
\item \textbf{Descrizione}: l'attore può confermare l'inserimento del blocco for;	
\item \textbf{Precondizione}: l'attore ha selezionato un blocco for da inserire e il sistema visualizza il relativo form per l'inserimento dei dati;	
\item \textbf{Postcondizione}: il sistema visualizza il blocco for inserito posizionandolo successivamente all'ultimo blocco generico precedentemente inserito nel diagramma delle attività corrente;
\item \textbf{Scenario principale}:
l'attore conferma l'inserimento del blocco for;	
\item \textbf{Estensioni}:
l'attore visualizza un messaggio d'errore sull'inserimento dei dati (UC5.5.6.1);	
\item \textbf{Scenari alternativi}:
\begin{itemize}
\item l'attore non ha fornito la condizione da verificare;
\item l'attore non ha fornito l'incremento/decremento;
\item l'attore non ha inserito un commento.
\end{itemize}
\end{itemize}

\subparagraph{UC5.5.6.1: Visualizzazione Errore Inserimento B.For}
\label{UC5.5.6.1}
\begin{itemize}
\item \textbf{Attori}: Utente;
\item \textbf{Descrizione}: l'attore può visualizzare un messaggio d'errore in corrispondenza della conferma di inserimento nel caso si fossero verificati uno o più scenari alternativi durante la fase di inserimento di un blocco for;	
\item \textbf{Precondizione}: il sistema non ha ricevuto sufficienti informazioni per l'inserimento di un blocco for;	
\item \textbf{Postcondizione}: il sistema avvisa l'attore dell'errore verificatosi tramite un opportuno messaggio;	
\item \textbf{Scenario principale}:
l'attore visualizza un messaggio d'errore.	
\end{itemize}

\subsubsection{UC5.6: Inserimento Blocco Custom	}
\label{UC5.6}
\begin{itemize}
\item \textbf{Attori}: Utente;
\item \textbf{Descrizione}: l'attore può inserire un blocco custom di codice all'interno del quale inserire liberamente uno o più statement nel linguaggio target di generazione di codice dell'editor;	
\item \textbf{Precondizione}: il sistema, nella sezione dell'editor dedicata alla gestione del diagramma delle attività, visualizza un diagramma delle attività (eventualmente vuoto) associato ad un particolare metodo;	
\item \textbf{Postcondizione}: il sistema visualizza il blocco inserito mostrando il commento inserito dall'attore e posizionandolo successivamente all'ultimo blocco generico precedentemente 
inserito nel diagramma delle attività corrente;	
\item \textbf{Scenario principale}:
\begin{enumerate}
\item l'attore può inserisce il contenuto in codice del blocco custom (UC5.6.1);
\item l'attore può inserisce un commento a corredo del blocco custom(UC5.6.2);
\item l'attore può confermare l'inserimento del blocco custom (UC5.6.3).	
\end{enumerate}
\end{itemize}

\paragraph{UC5.6.1: Contenuto Blocco Custom}
\label{UC5.6.1}
\begin{itemize}
\item \textbf{Attori}: Utente;
\item \textbf{Descrizione}: l'attore inserisce il contenuto in codice del blocco custom;	
\item \textbf{Precondizione}: l'attore ha selezionato un blocco custom da inserire e il sistema visualizza il form per l'inserimento dei dati;	
\item \textbf{Postcondizione}: il sistema ha ricevuto il dato richiesto;	
\item \textbf{Scenario principale}:
l'attore deve inserire il contenuto in codice del blocco custom in Java, il linguaggio target di generazione del codice dell'applicativo.	
\end{itemize}

\paragraph{UC5.6.2: Inserimento Commento B.Custom}
\label{UC5.6.2}
\begin{itemize}
\item \textbf{Attori}: Utente;
\item \textbf{Descrizione}: l'attore deve inserire un commento per il blocco custom;	
\item \textbf{Precondizione}: l'attore ha selezionato un blocco custom da inserire e il sistema visualizza il form per l'inserimento dei dati;	
\item \textbf{Postcondizione}: il sistema ha ricevuto il dato richiesto;	
\item \textbf{Scenario principale}:
l'attore inserisce un commento per il blocco custom che ne illustri il principale comportamento.
\end{itemize}

\paragraph{UC5.6.3: Conferma Inserimento Blocco Custom}
\label{UC5.6.3}
\begin{itemize}
\item \textbf{Attori}: Utente;
\item \textbf{Descrizione}: l'attore può confermare l'inserimento del blocco custom;	
\item \textbf{Precondizione}: il sistema visualizza il form per l'inserimento dei dati;	
\item \textbf{Postcondizione}: il sistema visualizza il blocco custom inserito;	
\item \textbf{Scenario principale}:
l'attore conferma l'inserimento del blocco custom.
\end{itemize}

\subsubsection{UC5.7: Rimozione Blocco Variabile}
\label{UC5.7}
\begin{itemize}
\item \textbf{Attori}: Utente;
\item \textbf{Descrizione}: l'attore può rimuovere un blocco variabile;	
\item \textbf{Precondizione}: il sistema visualizza il blocco variabile precedentemente realizzato;	
\item \textbf{Postcondizione}: il sistema rimuove il blocco variabile;	
\item \textbf{Scenario principale}:
l'attore rimuove il blocco variabile selezionato.	
\end{itemize}

\subsubsection{UC5.8: Rimozione Blocco Metodo}
\label{UC5.8}
\begin{itemize}
\item \textbf{Attori}: Utente;
\item \textbf{Descrizione}: l'attore può rimuovere un blocco metodo;	
\item \textbf{Precondizione}: il sistema visualizza il blocco metodo precedentemente realizzato;	
\item \textbf{Postcondizione}: il sistema rimuove il blocco metodo;	
\item \textbf{Scenario principale}:
l'attore rimuove il blocco metodo selezionato.	
\end{itemize}

\subsubsection{UC5.9: Rimozione Blocco If/Else}
\label{UC5.9}
\begin{itemize}
\item \textbf{Attori}: Utente;
\item \textbf{Descrizione}: l'attore può rimuovere un blocco if/else;	
\item \textbf{Precondizione}: il sistema visualizza il blocco if/else precedentemente realizzato;	
\item \textbf{Postcondizione}: il sistema rimuove il blocco if/else;	
\item \textbf{Scenario principale}:
l'attore rimuove il blocco if/else selezionato.	
\end{itemize}

\subsubsection{UC5.10: Rimozione Blocco While}
\label{UC5.10}
\begin{itemize}
\item \textbf{Attori}: Utente;
\item \textbf{Descrizione}: l'attore può rimuovere un blocco while;	
\item \textbf{Precondizione}: il sistema visualizza il blocco while precedentemente realizzato;	
\item \textbf{Postcondizione}: il sistema rimuove il blocco while;	
\item \textbf{Scenario principale}:
l'attore rimuove il blocco while selezionato.	
\end{itemize}

\subsubsection{UC5.11: Rimozione Blocco For}
\label{UC5.11}
\begin{itemize}
\item \textbf{Attori}: Utente;
\item \textbf{Descrizione}: l'attore può rimuovere un blocco for;	
\item \textbf{Precondizione}: il sistema visualizza il blocco for precedentemente realizzato;	
\item \textbf{Postcondizione}: il sistema rimuove il blocco for;	
\item \textbf{Scenario principale}:
l'attore rimuove il blocco for selezionato.	
\end{itemize}

\subsubsection{UC5.12: Rimozione Blocco Custom}
\label{UC5.12}
\begin{itemize}
\item \textbf{Attori}: Utente;
\item \textbf{Descrizione}: l'attore può rimuovere un blocco custom;	
\item \textbf{Precondizione}: il sistema visualizza il blocco custom precedentemente realizzato;	
\item \textbf{Postcondizione}: il sistema rimuove il blocco custom;	
\item \textbf{Scenario principale}:
l'attore rimuove il blocco custom selezionato.	
\end{itemize}

\subsubsection{UC5.13: Riduzione Blocco If/Else}
\label{UC5.13}
\begin{itemize}
\item \textbf{Attori}: Utente;
\item \textbf{Descrizione}: l'attore può ridurre il blocco if/else;	
\item \textbf{Precondizione}: il sistema visualizza il blocco if/else precedentemente realizzato ed espanso;	
\item \textbf{Postcondizione}: il sistema visualizza il blocco if/else ridotto;	
\item \textbf{Scenario principale}:
l'attore riduce il blocco if/else nascondendone il corpo.	
\end{itemize}

\subsubsection{UC5.14: Espansione Blocco If/Else	}
\label{UC5.14}
\begin{itemize}
\item \textbf{Attori}: Utente;
\item \textbf{Descrizione}: l'attore può espandere il blocco if/else;	
\item \textbf{Precondizione}: il sistema visualizza il blocco if/else precedentemente realizzato e ridotto;	
\item \textbf{Postcondizione}: il sistema visualizza il blocco if/else espanso;	
\item \textbf{Scenario principale}:
l'attore espande il blocco if/else mostrandone il corpo.	
\end{itemize}

\subsubsection{UC5.15: Riduzione Blocco While}
\label{UC5.15}
\begin{itemize}
\item \textbf{Attori}: Utente;
\item \textbf{Descrizione}: l'attore può ridurre il blocco while;	
\item \textbf{Precondizione}: il sistema visualizza il blocco while precedentemente realizzato ed espanso;	
\item \textbf{Postcondizione}: il sistema visualizza il blocco while ridotto;	
\item \textbf{Scenario principale}:
l'attore riduce il blocco while nascondendone il corpo.	
\end{itemize}

\subsubsection{UC5.16: Espansione Blocco While	}
\label{UC5.16}
\begin{itemize}
\item \textbf{Attori}: Utente;
\item \textbf{Descrizione}: l'attore può espandere il blocco while;	
\item \textbf{Precondizione}: il sistema visualizza il blocco while precedentemente realizzato e ridotto;	
\item \textbf{Postcondizione}: il sistema visualizza il blocco while espanso;	
\item \textbf{Scenario principale}:
l'attore espande il blocco while mostrandone il corpo.	
\end{itemize}

\subsubsection{UC5.17: Riduzione Blocco For	}
\label{UC5.17}
\begin{itemize}
\item \textbf{Attori}: Utente;
\item \textbf{Descrizione}: l'attore può ridurre il blocco for;	
\item \textbf{Precondizione}: il sistema visualizza il blocco for precedentemente realizzato ed espanso;	
\item \textbf{Postcondizione}: il sistema visualizza il blocco for ridotto;	
\item \textbf{Scenario principale}:
l'attore riduce il blocco for nascondendone il corpo.	
\end{itemize}

\subsubsection{UC5.18: Espansione Blocco For}
\label{UC5.18}
\begin{itemize}
\item \textbf{Attori}: Utente;
\item \textbf{Descrizione}: l'attore può espandere il blocco for;	
\item \textbf{Precondizione}: il sistema visualizza il blocco for precedentemente realizzato e ridotto;	
\item \textbf{Postcondizione}: il sistema visualizza il blocco for espanso;	
\item \textbf{Scenario principale}:
l'attore espande il blocco for mostrandone il corpo.	
\end{itemize}

\subsubsection{UC5.19: Spostamento Blocco}
\label{UC5.19}
\begin{itemize}
\item \textbf{Attori}: Utente;
\item \textbf{Descrizione}: l'attore può spostare uno dei qualsiasi blocchi già inseriti modificandone la posizione all'interno del diagramma di attività correntemente gestito;	
\item \textbf{Precondizione}: il sistema visualizza il diagramma delle attività correntemente gestito dall'attore ed in particolare il blocco che sarà selezionato per lo spostamento;	
\item \textbf{Postcondizione}: il sistema visualizza il blocco for nella nuova posizione specificata dall'utente;	
\item \textbf{Scenario principale}:
l'attore muove il blocco selezionato in una nuova posizione del diagramma delle attività correntemente gestito.	
\end{itemize}

\subsection{UC6: Salvataggio}
\label{UC6}
\begin{itemize}
\item \textbf{Attori}: Utente;
\item \textbf{Descrizione}: l'attore può salvare il progetto correntemente aperto;
\item \textbf{Precondizione}: il sistema avviato mostra il progetto realizzato fino ad ora dall'utente;
\item \textbf{Postcondizione}: il sistema salva il progetto in una particolare destinazione specificata dall'utente sovrascrivendo eventualmente lo stesso in una sua precedente versione;
\item \textbf{Scenario principale}:
\begin{enumerate}
\item l'attore può specificare il nome del file da salvare (UC6.1);
\item l'attore può specificare la destinazione del file da salvare (UC6.2);
\item l'attore può confermare il salvataggio del file (UC6.3).
\end{enumerate}
\end{itemize}

\subsubsection{UC6.1 Specifica Titolo File}
\label{UC6.1}
\begin{itemize}
\item \textbf{Attori}: Utente;
\item \textbf{Descrizione}: l'attore indica il nome del file attraverso cui realizzare il salvataggio;
\item \textbf{Precondizione}: l'utente ha selezionato il comando di salvataggio e il sistema visualizza il relativo form;
\item \textbf{Postcondizione}: il form ha ricevuto l'informazione di cui necessita;
\item \textbf{Scenario principale}:
l'attore indica il nome per il file che verrà salvato.
\end{itemize}

\subsubsection{UC6.2 Specifica Percorso File}
\label{UC6.2}
\begin{itemize}
\item \textbf{Attori}: Utente;
\item \textbf{Descrizione}: l'attore indica il percorso di directory in cui salvare il progetto corrente;
\item \textbf{Precondizione}: l'utente ha selezionato il comando di salvataggio e il sistema visualizza il relativo form;
\item \textbf{Postcondizione}: il form ha ricevuto l'informazione di cui necessita;
\item \textbf{Scenario principale}:
l'attore specifica la destinazione per il file che sarà salvato.
\end{itemize}

\subsubsection{UC6.3 Conferma Salvataggio File}
\label{UC6.3}
\begin{itemize}
\item \textbf{Attori}: Utente;
\item \textbf{Descrizione}: l'attore conferma il salvataggio del progetto corrente;
\item \textbf{Precondizione}: l'utente ha selezionato il comando di salvataggio e il sistema visualizza il relativo form;
\item \textbf{Postcondizione}: il file viene salvato secondo le specifiche indicate dall'utente;
\item \textbf{Scenario principale}:
l'attore conferma il salvataggio del progetto corrente.
\item \textbf{Estensioni}:l'attore visualizza un messaggio d'errore sull'inserimento dei dati;
\item \textbf{Scenari alternativi}:
\begin{itemize}
\item l'attore non inserisce il nome del file;
\item l'attore non specifica un percorso di directory valido.
\end{itemize}
\end{itemize}

\paragraph{UC6.3.1 Visualizzazione Errore Salvataggio File}
\begin{itemize}
\item \textbf{Attori}: Utente;
\item \textbf{Descrizione}: l'attore può visualizzare un messaggio d'errore in corrispondenza della conferma di salvataggio nel caso si siano verificati uno o più scenari alternativi durante la fase di inserimento dei dati di salvataggio;	
\item \textbf{Precondizione}: il sistema non ha ricevuto sufficienti informazioni per il salvataggio del progetto;	
\item \textbf{Postcondizione}: il sistema avvisa l'attore dell'errore verificatosi tramite un opportuno messaggio;	
\item \textbf{Scenario principale}:
l'attore visualizza un messaggio d'errore.	
\end{itemize}

\subsection{UC7: Generazione Codice}
\label{UC7}
\begin{itemize}
\item \textbf{Attori}: Utente;
\item \textbf{Descrizione}: l'attore può generare l'applicativo realizzato mediante l'editor;
\item \textbf{Precondizione}: il diagramma delle classi è completo e per ogni metodo è disponibile un diagramma delle attività corretto;
\item \textbf{Postcondizione}: il sistema mostra l'applicativo generato;
\item \textbf{Scenario principale}:
l'attore genera l'applicativo;
\item \textbf{Estensioni}:l'attore visualizza un messaggio d'errore relativo ai diagrammi realizzati (UC7.1);
\item \textbf{Scenari alternativi}:
\begin{itemize}
\item l'attore non ha rispettato i vincoli imposti dagli stereotipi scelti.
\end{itemize}
\end{itemize}

\subsubsection{UC7.1: Visualizzazione Errore Generazione Codice}
\label{UC7.1}
\begin{itemize}
\item \textbf{Attori}: Utente;
\item \textbf{Descrizione}: l'attore può visualizzare un messaggio d'errore in corrispondenza della richiesta di generazione codice nel caso si siano verificati uno o più scenari alternativi durante la fase di creazione dei diagrammi delle classi e delle attività;	
\item \textbf{Precondizione}: i diagrammi realizzati dall'utente non sono coerenti con i vincoli imposti dal sistema di stereotipi già presente nell'applicazione;	
\item \textbf{Postcondizione}: il sistema avvisa l'attore dell'errore verificatosi tramite un opportuno messaggio;	
\item \textbf{Scenario principale}:
l'attore visualizza un messaggio d'errore.	
\end{itemize}


\subsection{Requisiti Funzionali}
\normalsize
\begin{longtable}{|c|>{\centering}m{7cm}|c|}
\hline 
\textbf{Id Requisito} & \textbf{Descrizione} & \textbf{Fonti}\\
\hline
\endhead
\hypertarget{RFO1}{RFO1} & l'attore può caricare un progetto &  \hyperlink{Interno}{Interno}\\
& & \hyperref[UC2]{UC2}\\ \hline

\hypertarget{RFO2}{RFO2} & l'attore può creare un nuovo progetto &  \hyperlink{Interno}{Interno}\\
& & \hyperref[UC3]{UC3}\\ \hline

\hypertarget{RFO3}{RFO3} & l'attore può realizzare e gestire un diagramma delle classi & \hyperlink{Capitolato}{Capitolato}\\
& & \hyperref[UC4]{UC4}\\ \hline

\hypertarget{RFO3.1}{RFO3.1} & l'attore può inserire una classe &  \hyperlink{Interno}{Interno}\\
& &\hyperref[UC4]{UC4}\\
& &\hyperref[UC4.1]{UC4.1}\\ \hline

\hypertarget{RFO3.1.1}{RFO3.1.1} & l'attore può inserire la visibilità della classe & \hyperlink{Interno}{Interno}\\
& &\hyperref[UC4.1]{UC4.1}\\
& &\hyperref[UC4.1.1]{UC4.1.1}\\ \hline

\hypertarget{RFO3.1.2}{RFO3.1.2} & l'attore può inserire il nome della classe & \hyperlink{Interno}{Interno}\\
& &\hyperref[UC4.1]{UC4.1}\\
& &\hyperref[UC4.1.2]{UC4.1.2}\\ \hline

\hypertarget{RFO3.1.3}{RFO3.1.3} & l'attore può scegliere uno stereotipo per la classe & \hyperlink{Riunione Esterna}{Riunione Esterna}\\
& &\hyperref[UC4.1]{UC4.1}\\
& &\hyperref[UC4.1.3]{UC4.1.3}\\ \hline

\hypertarget{RFO3.1.4}{RFO3.1.4} & l'attore può inserire un attributo per la classe & \hyperlink{Interno}{Interno}\\
& &\hyperref[UC4.1]{UC4.1}\\
& &\hyperref[UC4.1.4]{UC4.1.4}\\ \hline

\hypertarget{RFO3.1.4.1}{RFO3.1.4.1} & l'attore può scegliere la visibilità per l'attributo &  \hyperlink{Interno}{Interno}\\
& &\hyperref[UC4.1.4]{UC4.1.4}\\
& &\hyperref[UC4.1.4.1]{UC4.1.4.1}\\ \hline

\hypertarget{RFO3.1.4.2}{RFO3.1.4.2} & l'attore può inserire il nome dell'attributo & \hyperlink{Interno}{Interno}\\
& &\hyperref[UC4.1.4]{UC4.1.4}\\
& &\hyperref[UC4.1.4.2]{UC4.1.4.2}\\ \hline

\hypertarget{RFO3.1.4.3}{RFO3.1.4.3} & l'attore può inserire il tipo dell'attributo & \hyperlink{Interno}{Interno}\\
& &\hyperref[UC4.1.4]{UC4.1.4}\\
& &\hyperref[UC4.1.4.2]{UC4.1.4.2}\\ \hline

\hypertarget{RFO3.1.4.4}{RFO3.1.4.4} & l'attore può scegliere la molteplicità dell'attributo & \hyperlink{Interno}{Interno}\\
& &\hyperref[UC4.1.4]{UC4.1.4}\\
& &\hyperref[UC4.1.4.4]{UC4.1.4.4}\\ \hline

\hypertarget{RFO3.1.4.5}{RFO3.1.4.5} & l'attore può inserire il valore di default dell'attributo & \hyperlink{Interno}{Interno}\\
& &\hyperref[UC4.1.4]{UC4.1.4}\\
& &\hyperref[UC4.1.4.5]{UC4.1.4.5}\\ \hline

\hypertarget{RFO3.1.4.6}{RFO3.1.4.6} & l’attore può confermare l'inserimento dell'attributo & \hyperlink{Interno}{Interno}\\
& &\hyperref[UC4.1.4]{UC4.1.4}\\
& &\hyperref[UC4.1.4.6]{UC4.1.4.6}\\ \hline

\hypertarget{RFO3.1.4.6.1}{RFO3.1.4.6.1} & il sistema deve visualizzare un messaggio d’errore se l'attore non ha inserito il nome dell'attributo, non ha inserito il tipo dell'attributo o ha inserito un nome per l'attributo già esistente & \hyperlink{Interno}{Interno}\\
& &\hyperref[UC4.1.4.6]{UC4.1.4.6}\\
& &\hyperref[UC4.1.4.6.1]{UC4.1.4.6.1}\\ \hline

\hypertarget{RFO3.1.5}{RFO3.1.5} & l'attore può rimuovere un attributo precedentemente inserito & \hyperlink{Interno}{Interno}\\
& &\hyperref[UC4.1]{UC4.1}\\
& &\hyperref[UC4.1.5]{UC4.1.5}\\ \hline

\hypertarget{RFO3.1.6}{RFO3.1.6} & l'attore può inserire un metodo per la classe &\hyperlink{Interno}{Interno}\\
& &\hyperref[UC4.1]{UC4.1}\\
& &\hyperref[UC4.1.6]{UC4.1.6}\\ \hline

\hypertarget{RFO3.1.6.1}{RFO3.1.6.1} & l'attore può scegliere la visibilità per il metodo &\hyperlink{Interno}{Interno}\\
& &\hyperref[UC4.1.6]{UC4.1.6}\\
& &\hyperref[UC4.1.6.1]{UC4.1.6.1}\\ \hline

\hypertarget{RFO3.1.6.2}{RFO3.1.6.2} & l'attore può inserire il nome del metodo & \hyperlink{Interno}{Interno}\\
& &\hyperref[UC4.1.6]{UC4.1.6}\\
& &\hyperref[UC4.1.6.2]{UC4.1.6.2}\\ \hline

\hypertarget{RFO3.1.6.3}{RFO3.1.6.3} & l'attore può inserire un parametro per il metodo & \hyperlink{Interno}{Interno}\\
& &\hyperref[UC4.1.6]{UC4.1.6}\\
& &\hyperref[UC4.1.6.3]{UC4.1.6.3}\\ \hline

\hypertarget{RFO3.1.6.3.1}{RFO3.1.6.3.1} & l'attore può inserire il parametro del metodo &\hyperlink{Interno}{Interno}\\
& & \hyperref[UC4.1.6.3]{UC4.1.6.3}\\
& & \hyperref[UC4.1.6.3.1]{UC4.1.6.3.1}\\ \hline

\hypertarget{RFO3.1.6.3.2}{RFO3.1.6.3.2} & l'attore può inserire il tipo del metodo &\hyperlink{Interno}{Interno}\\
& &\hyperref[UC4.1.6.3]{UC4.1.6.3}\\
& &\hyperref[UC4.1.6.3.2]{UC4.1.6.3.2}\\ \hline

\hypertarget{RFO3.1.6.3.3}{RFO3.1.6.3.3} & l’attore può confermare l'inserimento del parametro & \hyperlink{Interno}{Interno}\\
& &\hyperref[UC4.1.6.3]{UC4.1.6.3}\\
& &\hyperref[UC4.1.6.3.3]{UC4.1.6.3.3}\\ \hline

\hypertarget{RFO3.1.6.3.3.1}{RFO3.1.6.3.3.1} & il sistema deve visualizzare un messaggio d’errore se l'attore non ha inserito il nome del parametro o non ha inserito il tipo del parametro &  \hyperlink{Interno}{Interno}\\
& &\hyperref[UC4.1.6.3.3]{UC4.1.6.3.3}\\
& &\hyperref[UC4.1.6.3.3.1]{UC4.1.6.3.3.1}\\ \hline

\hypertarget{RFO3.1.6.4}{RFO3.1.6.4} & l'attore può inserire il tipo di ritorno del metodo &  \hyperlink{Interno}{Interno}\\
& &\hyperref[UC4.1.6]{UC4.1.6}\\
& &\hyperref[UC4.1.6.4]{UC4.1.6.4}\\ \hline

\hypertarget{RFO3.1.6.5}{RFO3.1.6.5} & l’attore può confermare l'inserimento del metodo & \hyperlink{Interno}{Interno}\\
& &\hyperref[UC4.1.6]{UC4.1.6}\\
& &\hyperref[UC4.1.6.5]{UC4.1.6.5}\\ \hline

\hypertarget{RFO3.1.6.5.1}{RFO3.1.6.5.1} & il sistema deve visualizzare un messaggio d'errore se l'attore non ha inserito il nome del metodo, non ha inserito il tipo di ritorno del metodo o ha inserito un metodo con una segnatura già esistente & \hyperlink{Interno}{Interno}\\
& &\hyperref[UC4.1.6.5]{UC4.1.6.5}\\
& &\hyperref[UC4.1.6.5.1]{UC4.1.6.5.1}\\ \hline

\hypertarget{RFO3.1.6.6}{RFO3.1.6.6} & l’attore può rimuovere il parametro del metodo & \hyperlink{Interno}{Interno}\\
& &\hyperref[UC4.1.6]{UC4.1.6}\\
& &\hyperref[UC4.1.6.6]{UC4.1.6.6}\\ \hline

\hypertarget{RFO3.1.7}{RFO3.1.7} & l'attore può rimuovere un metodo precedentemente inserito & \hyperlink{Interno}{Interno}\\
& &\hyperref[UC4.1]{UC4.1}\\
& &\hyperref[UC4.1.7]{UC4.1.7}\\ \hline

\hypertarget{RFO3.1.8}{RFO3.1.8} & l'attore può confermare l'inserimento della classe & \hyperlink{Interno}{Interno}\\
& &\hyperref[UC4.1]{UC4.1}\\
& &\hyperref[UC4.1.8]{UC4.1.8}\\ \hline

\hypertarget{RFO3.1.8.1}{RFO3.1.8.1} & il sistema deve visualizzare un messaggio d'errore se l'attore non ha inserito il nome della classe o ha inserito un nome per la classe già esistente &  \hyperlink{Interno}{Interno}\\
& &\hyperref[UC4.1.8]{UC4.1.8}\\
& &\hyperref[UC4.1.8.1]{UC4.1.8.1}\\ \hline

\hypertarget{RFO3.2}{RFO3.2} & l'attore può inserire una relazione tra due classi & \hyperlink{Interno}{Interno}\\
& &\hyperref[UC4]{UC4}\\
& &\hyperref[UC4.2]{UC4.2}\\ \hline

\hypertarget{RFO3.2.1}{RFO3.2.1} & l'attore può scegliere il nome della relazione & \hyperlink{Interno}{Interno}\\
& &\hyperref[UC4.2]{UC4.2}\\
& &\hyperref[UC4.2.1]{UC4.2.1}\\ \hline

\hypertarget{RFO3.2.2}{RFO3.2.2} & l'attore può scegliere la classe di partenza per la relazione & \hyperlink{Interno}{Interno}\\
& &\hyperref[UC4.2]{UC4.2}\\
& &\hyperref[UC4.2.2]{UC4.2.2}\\ \hline

\hypertarget{RFO3.2.2.1}{RFO3.2.2.1} & l'attore può rimuovere un commento &  \hyperlink{Interno}{Interno}\\
& &\hyperref[UC4]{UC4}\\
& &\hyperref[UC4.6]{UC4.6}\\ \hline

\hypertarget{RFO3.2.3}{RFO3.2.3} & l'attore può scegliere la classe di destinazione per la relazione & \hyperlink{Interno}{Interno}\\
& &\hyperref[UC4.2]{UC4.2}\\
& &\hyperref[UC4.2.3]{UC4.2.3}\\ \hline

\hypertarget{RFO3.2.4}{RFO3.2.4} & l'attore può scegliere la cardinalità della relazione & \hyperlink{Interno}{Interno}\\
& &\hyperref[UC4.2]{UC4.2}\\
& &\hyperref[UC4.2.4]{UC4.2.4}\\ \hline

\hypertarget{RFO3.2.5}{RFO3.2.5} & l'attore può scegliere uno stereotipo per la relazione & \hyperlink{Interno}{Interno}\\
& &\hyperref[UC4.2]{UC4.2}\\
& &\hyperref[UC4.2.5]{UC4.2.5}\\ \hline

\hypertarget{RFO3.2.6}{RFO3.2.6} & l'attore può confermare l'inserimento della relazione & \hyperlink{Interno}{Interno}\\
& &\hyperref[UC4.2]{UC4.2}\\
& &\hyperref[UC4.2.6]{UC4.2.6}\\ \hline

\hypertarget{RFO3.2.6.1}{RFO3.2.6.1} & il sistema deve visualizzare un messaggio d'errore se l'attore non ha scelto la relazione, on ha scelto la di partenza o non ha scelto la classe di destinazione &  \hyperlink{Interno}{Interno}\\
& &\hyperref[UC4.2.6]{UC4.2.6}\\
& &\hyperref[UC4.2.6.1]{UC4.2.6.1}\\ \hline

\hypertarget{RFO3.3}{RFO3.3} & l'attore può inserire un commento &  \hyperlink{Interno}{Interno}\\
& &\hyperref[UC4]{UC4}\\
& &\hyperref[UC4.3]{UC4.3}\\ \hline

\hypertarget{RFO3.3.1}{RFO3.3.1} & l'attore può inserire il testo del commento & \hyperlink{Interno}{Interno}\\
& &\hyperref[UC4.3]{UC4.3}\\
& &\hyperref[UC4.3.1]{UC4.3.1}\\ \hline

\hypertarget{RFO3.3.2}{RFO3.3.2} & l'attore può scegliere il "parent" del commento & \hyperlink{Interno}{Interno}\\
& &\hyperref[UC4.3]{UC4.3}\\
& &\hyperref[UC4.3.2]{UC4.3.2}\\ \hline

\hypertarget{RFO3.3.3}{RFO3.3.3} & l’attore può confermare l'inserimento del commento & \hyperlink{Interno}{Interno}\\
& &\hyperref[UC4.3]{UC4.3}\\
& &\hyperref[UC4.3.3]{UC4.3.3}\\ \hline

\hypertarget{RFO3.3.3.1}{RFO3.3.3.1} & il sistema deve visualizzare un messaggio d’errore se l'attore non ha scelto il "parent" del commento. & \hyperlink{Interno}{Interno}\\
& &\hyperref[UC4.3.3]{UC4.3.3}\\
& &\hyperref[UC4.3.3.1]{UC4.3.3.1}\\ \hline

\hypertarget{RFO3.4}{RFO3.4} & l'attore può rimuovere una classe &  \hyperlink{Interno}{Interno}\\
& &\hyperref[UC4]{UC4}\\
& &\hyperref[UC4.4]{UC4.4}\\ \hline

\hypertarget{RFO3.5}{RFO3.5} & l'attore può rimuovere una relazione &  \hyperlink{Interno}{Interno}\\
& &\hyperref[UC4]{UC4}\\
& &\hyperref[UC4.5]{UC4.5}\\ \hline

\hypertarget{RFO3.6}{RFO3.6} & l'attore può rimuovere un commento &  \hyperlink{Interno}{Interno}\\
& &\hyperref[UC4]{UC4}\\
& &\hyperref[UC4.6]{UC4.6}\\ \hline

\hypertarget{RFO3.7}{RFO3.7} & l'attore può ridurre una classe &  \hyperlink{Interno}{Interno}\\
& &\hyperref[UC4]{UC4}\\
& &\hyperref[UC4.7]{UC4.7}\\ \hline

\hypertarget{RFO3.8}{RFO3.8} & l'attore può espandere una classe &  \hyperlink{Interno}{Interno}\\
& &\hyperref[UC4]{UC4}\\
& &\hyperref[UC4.8]{UC4.8}\\ \hline

\hypertarget{RFO3.9}{RFO3.9} & l'attore può ridurre un commento &  \hyperlink{Interno}{Interno}\\
& &\hyperref[UC4]{UC4}\\
& &\hyperref[UC4.9]{UC4.9}\\ \hline

\hypertarget{RFO3.10}{RFO3.10} & l'attore può espandere un commento &  \hyperlink{Interno}{Interno}\\
& &\hyperref[UC4]{UC4}\\
& &\hyperref[UC4.10]{UC4.10}\\ \hline

\hypertarget{RFO3.11}{RFO3.11} & l'attore può modificare una classe &  \hyperlink{Interno}{Interno}\\
& &\hyperref[UC4]{UC4}\\
& &\hyperref[UC4.11]{UC4.11}\\ \hline

\hypertarget{RFO3.11.1}{RFO3.11.1} & l'attore può modificare la visibilità della classe &  \hyperlink{Interno}{Interno}\\
& &\hyperref[UC4.11]{UC4.11}\\
& &\hyperref[UC4.11.1]{UC4.11.1}\\ \hline

\hypertarget{RFO3.11.2}{RFO3.11.2} & l'attore può modificare il nome della classe &  \hyperlink{Interno}{Interno}\\
& &\hyperref[UC4.11]{UC4.11}\\
& &\hyperref[UC4.11.2]{UC4.11.2}\\ \hline

\hypertarget{RFO3.11.3}{RFO3.11.3} & l'attore può modificare un attributo della classe &  \hyperlink{Interno}{Interno}\\
& &\hyperref[UC4.11]{UC4.11}\\
& &\hyperref[UC4.11.3]{UC4.11.3}\\ \hline

\hypertarget{RFO3.11.3.1}{RFO3.11.3.1} & l'attore può modificare la visibilità dell'attributo &  \hyperlink{Interno}{Interno}\\
& &\hyperref[UC4.11.3]{UC4.11.3}\\
& &\hyperref[UC4.11.3.1]{UC4.11.3.1}\\ \hline

\hypertarget{RFO3.11.3.2}{RFO3.11.3.2} & l'attore può modificare il nome dell'attributo &  \hyperlink{Interno}{Interno}\\
& &\hyperref[UC4.11.3]{UC4.11.3}\\
& &\hyperref[UC4.11.3.2]{UC4.11.3.2}\\ \hline

\hypertarget{RFO3.11.3.3}{RFO3.11.3.3} & l'attore può modificare il tipo dell'attributo &  \hyperlink{Interno}{Interno}\\
& &\hyperref[UC4.11.3]{UC4.11.3}\\
& &\hyperref[UC4.11.3.3]{UC4.11.3.3}\\ \hline

\hypertarget{RFO3.11.3.4}{RFO3.11.3.4} & l'attore può modificare la molteplicità dell'attributo &  \hyperlink{Interno}{Interno}\\
& &\hyperref[UC4.11.3]{UC4.11.3}\\
& &\hyperref[UC4.11.3.4]{UC4.11.3.4}\\ \hline

\hypertarget{RFO3.11.3.5}{RFO3.11.3.5} & l'attore può modificare il valore di default dell'attributo &  \hyperlink{Interno}{Interno}\\
& &\hyperref[UC4.11.3]{UC4.11.3}\\
& &\hyperref[UC4.11.3.5]{UC4.11.3.5}\\ \hline

\hypertarget{RFO3.11.3.6}{RFO3.11.3.6} & l'attore può confermare le modifiche apportate all'attributo &  \hyperlink{Interno}{Interno}\\
& &\hyperref[UC4.11.3]{UC4.11.3}\\
& &\hyperref[UC4.11.3.6]{UC4.11.3.6}\\ 
& &\hyperref[UC4.1.4.6.1]{UC4.1.4.6.1}\\ \hline

\hypertarget{RFO3.11.4}{RFO3.11.4} & l'attore può modificare un metodo della classe &  \hyperlink{Interno}{Interno}\\
& &\hyperref[UC4.11]{UC4.11}\\
& &\hyperref[UC4.11.4]{UC4.11.4}\\ \hline

\hypertarget{RFO3.11.4.1}{RFO3.11.4.1} & l'attore può modificare la visibilità del metodo &  \hyperlink{Interno}{Interno}\\
& &\hyperref[UC4.11.4]{UC4.11.4}\\
& &\hyperref[UC4.11.4.1]{UC4.11.4.1}\\ \hline

\hypertarget{RFO3.11.4.2}{RFO3.11.4.2} & l'attore può modificare il nome del metodo &  \hyperlink{Interno}{Interno}\\
& &\hyperref[UC4.11.4]{UC4.11.4}\\
& &\hyperref[UC4.11.4.2]{UC4.11.4.2}\\ \hline

\hypertarget{RFO3.11.4.3}{RFO3.11.4.3} & l'attore può modificare un parametro del metodo &  \hyperlink{Interno}{Interno}\\
& &\hyperref[UC4.11.4]{UC4.11.4}\\
& &\hyperref[UC4.11.4.3]{UC4.11.4.3}\\ \hline

\hypertarget{RFO3.11.4.3.1}{RFO3.11.4.3.1} & l'attore può modificare il nome del parametro del metodo &  \hyperlink{Interno}{Interno}\\
& &\hyperref[UC4.11.4.3]{UC4.11.4.3}\\
& &\hyperref[UC4.11.4.3.1]{UC4.11.4.3.1}\\ \hline

\hypertarget{RFO3.11.4.3.2}{RFO3.11.4.3.2} & l'attore può modificare il tipo del parametro del metodo &  \hyperlink{Interno}{Interno}\\
& &\hyperref[UC4.11.4.3]{UC4.11.4.3}\\
& &\hyperref[UC4.11.4.3.2]{UC4.11.4.3.2}\\ \hline

\hypertarget{RFO3.11.4.3.3}{RFO3.11.4.3.3} & l'attore può confermare le modifiche apportate al parametro del metodo &  \hyperlink{Interno}{Interno}\\
& &\hyperref[UC4.11.4.3]{UC4.11.4.3}\\
& &\hyperref[UC4.11.4.3.3]{UC4.11.4.3.3}\\
& &\hyperref[UC4.1.6.3.3.1]{UC4.1.6.3.3.1}\\ \hline

\hypertarget{RFO3.11.4.4}{RFO3.11.4.4} & l'attore può modificare il tipo di ritorno del metodo &  \hyperlink{Interno}{Interno}\\
& &\hyperref[UC4.11.4]{UC4.11.4}\\
& &\hyperref[UC4.11.4.4]{UC4.11.4.4}\\ \hline

\hypertarget{RFO3.11.4.5}{RFO3.11.4.5} & l'attore può confermare le modifiche apportate al metodo &  \hyperlink{Interno}{Interno}\\
& &\hyperref[UC4.11.4]{UC4.11.4}\\
& &\hyperref[UC4.11.4.5]{UC4.11.4.5}\\ 
& &\hyperref[UC4.1.6.5.1]{UC4.1.6.5.1}\\ \hline

\hypertarget{RFO3.11.5}{RFO3.11.5} & l'attore può confermare le modifiche apportate alla classe &  \hyperlink{Interno}{Interno}\\
& &\hyperref[UC4.11]{UC4.11}\\
& &\hyperref[UC4.11.5]{UC4.11.5}\\ 
& &\hyperref[UC4.1.8.1]{UC4.1.8.1}\\ \hline

\hypertarget{RFO3.12}{RFO3.12} & l'attore può modificare una relazione &  \hyperlink{Interno}{Interno}\\
& &\hyperref[UC4]{UC4}\\
& &\hyperref[UC4.12]{UC4.12}\\ \hline

\hypertarget{RFO3.12.1}{RFO3.12.1} & l'attore può modificare il tipo di una relazione &  \hyperlink{Interno}{Interno}\\
& &\hyperref[UC4.12]{UC4.12}\\
& &\hyperref[UC4.12.1]{UC4.12.1}\\ \hline

\hypertarget{RFO3.12.2}{RFO3.12.2} & l'attore può modificare l'attributo di una relazione &  \hyperlink{Interno}{Interno}\\
& &\hyperref[UC4.12]{UC4.12}\\
& &\hyperref[UC4.12.2]{UC4.12.2}\\ \hline

\hypertarget{RFO3.12.3}{RFO3.12.3} & l'attore può modificare la classe1 di una relazione &  \hyperlink{Interno}{Interno}\\
& &\hyperref[UC4.12]{UC4.12}\\
& &\hyperref[UC4.12.3]{UC4.12.3}\\ \hline

\hypertarget{RFO3.12.4}{RFO3.12.4} & l'attore può modificare la classe2 di una relazione &  \hyperlink{Interno}{Interno}\\
& &\hyperref[UC4.12]{UC4.12}\\
& &\hyperref[UC4.12.4]{UC4.12.4}\\ \hline

\hypertarget{RFO3.12.5}{RFO3.12.5} & l'attore può modificare la cardinalità di una relazione &  \hyperlink{Interno}{Interno}\\
& &\hyperref[UC4.12]{UC4.12}\\
& &\hyperref[UC4.12.5]{UC4.12.5}\\ \hline

\hypertarget{RFO3.12.6}{RFO3.12.6} & l'attore può confermare le modifiche apportate alla relazione &  \hyperlink{Interno}{Interno}\\
& &\hyperref[UC4.12]{UC4.12}\\
& &\hyperref[UC4.12.6]{UC4.12.6}\\ 
& &\hyperref[UC4.2.6.1]{UC4.2.6.1}\\ \hline

\hypertarget{RFO3.13}{RFO3.13} & l'attore può modificare un commento &  \hyperlink{Interno}{Interno}\\
& &\hyperref[UC4]{UC4}\\
& &\hyperref[UC4.13]{UC4.13}\\ \hline

\hypertarget{RFO3.13.1}{RFO3.13.1} & l'attore può modificare il testo di un commento &  \hyperlink{Interno}{Interno}\\
& &\hyperref[UC4.13]{UC4.13}\\
& &\hyperref[UC4.13.1]{UC4.13.1}\\ \hline

\hypertarget{RFO3.13.2}{RFO3.13.2} & l'attore può modificare il parent di un commento &  \hyperlink{Interno}{Interno}\\
& &\hyperref[UC4.13]{UC4.13}\\
& &\hyperref[UC4.13.2]{UC4.13.2}\\ \hline

\hypertarget{RFO3.13.3}{RFO3.13.3} & l'attore può confermare le modifiche apportate al commento &  \hyperlink{Interno}{Interno}\\
& &\hyperref[UC4.13]{UC4.13}\\
& &\hyperref[UC4.13.3]{UC4.13.3}\\ \hline

\hypertarget{RFO4}{RFO4} & l'attore può realizzare e gestire un diagramma delle attività per ogni metodo presente nel diagramma delle classi & \hyperlink{Capitolato}{Capitolato}\\
& & \hyperref[UC5]{UC5}\\ \hline

\hypertarget{RFO4.1}{RFO4.1} & l'attore può inserire un blocco variabile scegliendo tra inizializzazione e assegnazione &  \hyperlink{Riunione Esterna}{Riunione Esterna}\\
& &\hyperref[UC5]{UC5}\\
& &\hyperref[UC5.1]{UC5.1}\\ \hline

\hypertarget{RFO4.1.1}{RFO4.1.1} & l'attore può creare ed inizializzare una variabile assegnandole un valore & \hyperlink{Interno}{Interno}\\
& \hyperref[UC5.1]{UC5.1}\\
& \hyperref[UC5.1.1]{UC5.1.1}\\ \hline

\hypertarget{RFO4.1.2}{RFO4.1.2} & l'attore può assegnare un valore ad una variabile esistente e visibile & \hyperlink{Interno}{Interno}\\
& \hyperref[UC5.1]{UC5.1}\\
& \hyperref[UC5.1.2]{UC5.1.2}\\ \hline

\hypertarget{RFO4.1.3}{RFO4.1.3} & l’attore può inserire un commento per il blocco variabile & \hyperlink{Interno}{Interno}\\
& \hyperref[UC5.1]{UC5.1}\\
& \hyperref[UC5.1.3]{UC5.1.3}\\ \hline

\hypertarget{RFO4.1.4}{RFO4.1.4} & l’attore può confermare l'inserimento del blocco variabile &\hyperlink{Interno}{Interno}\\
& \hyperref[UC5.1]{UC5.1}\\
& \hyperref[UC5.1.4]{UC5.1.4}\\ \hline

\hypertarget{RFO4.1.4.1}{RFO4.1.4.1} & il sistema deve visualizzare un messaggio d’errore se l'attore non ha fornito la variabile da inizializzare, non ha fornito un valore da assegnare alla variabile o non ha inserito un commento. & \hyperlink{Interno}{Interno}\\
& \hyperref[UC5.1.4]{UC5.1.4}\\
& \hyperref[UC5.1.4.1]{UC5.1.4.1}\\ \hline

\hypertarget{RFO4.2}{RFO4.2} & l'attore può inserire un blocco metodo & \hyperlink{Riunione Esterna}{Riunione Esterna}\\
& \hyperref[UC5]{UC5}\\
& \hyperref[UC5.2]{UC5.2}\\ \hline

\hypertarget{RFO4.2.1}{RFO4.2.1} & l'attore può invocare un metodo tra quelli disponibili & \hyperlink{Interno}{Interno}\\
& \hyperref[UC5.2]{UC5.2}\\
& \hyperref[UC5.2.1]{UC5.2.1}\\ \hline

\hypertarget{RFO4.2.2}{RFO4.2.2} & l’attore può inserire un commento per il blocco metodo & \hyperlink{Interno}{Interno}\\
& \hyperref[UC5.2]{UC5.2}\\
& \hyperref[UC5.2.2]{UC5.2.2}\\ \hline

\hypertarget{RFO4.2.3}{RFO4.2.3} & l’attore può confermare l'inserimento del blocco metodo & \hyperlink{Interno}{Interno}\\
& \hyperref[UC5.2]{UC5.2}\\
& \hyperref[UC5.2.3]{UC5.2.3}\\ \hline

\hypertarget{RFO4.2.3.1}{RFO4.2.3.1} & il sistema deve visualizzare un messaggio d’errore se l'attore non ha fornito un metodo da inserire o non ha inserito un commento & \hyperlink{Interno}{Interno}\\
& \hyperref[UC5.2.3]{UC5.2.3}\\
& \hyperref[UC5.2.3.1]{UC5.2.3.1}\\ \hline

\hypertarget{RFO4.3}{RFO4.3} & l'attore può inserire un blocco if/else & \hyperlink{Riunione Esterna}{Riunione Esterna}\\
& \hyperref[UC5]{UC5}\\
& \hyperref[UC5.3]{UC5.3}\\ \hline

\hypertarget{RFO4.3.1}{RFO4.3.1} & l'attore può inserire la condizione da verificare del blocco if & \hyperlink{Interno}{Interno}\\
& \hyperref[UC5.3]{UC5.3}\\
& \hyperref[UC5.3.1]{UC5.3.1}\\ \hline

\hypertarget{RFO4.3.2}{RFO4.3.2} & l'attore può completare il corpo del blocco if, inserendo una serie di blocchi tra quelli resi disponibili dall'editor  & \hyperlink{Interno}{Interno}\\
& \hyperref[UC5.3]{UC5.3}\\
& \hyperref[UC5.3.2]{UC5.3.2}\\ \hline

\hypertarget{RFO4.3.3}{RFO4.3.3} & l'attore può completare il corpo del blocco else, inserendo una serie di blocchi tra quelli resi disponibili dall'editor  & \hyperlink{Interno}{Interno}\\
& \hyperref[UC5.3]{UC5.3}\\
& \hyperref[UC5.3.3]{UC5.3.3}\\ \hline

\hypertarget{RFO4.3.4}{RFO4.3.4} & l’attore deve inserire un commento per il blocco if/else & \hyperlink{Interno}{Interno}\\
& \hyperref[UC5.3]{UC5.3}\\
& \hyperref[UC5.3.4]{UC5.3.4}\\ \hline

\hypertarget{RFO4.3.5}{RFO4.3.5} & l’attore può confermare l'inserimento del blocco if/else & \hyperlink{Interno}{Interno}\\
& \hyperref[UC5.3]{UC5.3}\\
& \hyperref[UC5.3.5]{UC5.3.5}\\ \hline

\hypertarget{RFO4.3.5.1}{RFO4.3.5.1} & il sistema visualizza un messaggio d’errore se l'attore non ha fornito la condizione da verificare o non ha inserito un commento & \hyperlink{Interno}{Interno}\\
& \hyperref[UC5.3.5]{UC5.3.5}\\
& \hyperref[UC5.3.5.1]{UC5.3.5.1}\\ \hline

\hypertarget{RFO4.4}{RFO4.4} & l'attore può inserire un blocco while & \hyperlink{Riunione Esterna}{Riunione Esterna}\\
& \hyperref[UC5]{UC5}\\
& \hyperref[UC5.4]{UC5.4}\\ \hline

\hypertarget{RFO4.4.1}{RFO4.4.1} & l'attore può inserire la condizione da verificare del blocco while & \hyperlink{Interno}{Interno}\\
& \hyperref[UC5.4]{UC5.4}\\
& \hyperref[UC5.4.1]{UC5.4.1}\\ \hline

\hypertarget{RFO4.4.2}{RFO4.4.2} & l'attore può completare il corpo del blocco while, inserendo una serie di blocchi tra quelli resi disponibili dall'editor & \hyperlink{Interno}{Interno}\\
& \hyperref[UC5.4]{UC5.4}\\
& \hyperref[UC5.4.2]{UC5.4.2}\\ \hline

\hypertarget{RFO4.4.3}{RFO4.4.3} & l’attore può inserire un commento per il blocco while & \hyperlink{Interno}{Interno}\\
& \hyperref[UC5.4]{UC5.4}\\
& \hyperref[UC5.4.3]{UC5.4.3}\\ \hline

\hypertarget{RFO4.4.4}{RFO4.4.4} & l’attore può confermare l'inserimento del blocco while & \hyperlink{Interno}{Interno}\\
& \hyperref[UC5.4]{UC5.4}\\
& \hyperref[UC5.4.4]{UC5.4.4}\\ \hline

\hypertarget{RFO4.4.4.1}{RFO4.4.4.1} & il sistema deve visualizzare un messaggio d’errore se l'attore non ha fornito la condizione da verificare o non ha inserito un commento. &\hyperlink{Interno}{Interno}\\
& \hyperref[UC5.4.4]{UC5.4.4}\\
& \hyperref[UC5.4.4.1]{UC5.4.4.1}\\ \hline

\hypertarget{RFO4.5}{RFO4.5} & l'attore può inserire un blocco for &  \hyperlink{Riunione Esterna}{Riunione Esterna}\\
& \hyperref[UC5]{UC5}\\
& \hyperref[UC5.5]{UC5.5}\\ \hline

\hypertarget{RFO4.5.1}{RFO4.5.1} & l'attore può inserire l'inizializzazione del blocco for & \hyperlink{Interno}{Interno}\\
& \hyperref[UC5.5]{UC5.5}\\
& \hyperref[UC5.5.1]{UC5.5.1}\\ \hline

\hypertarget{RFO4.5.2}{RFO4.5.2} & l'attore può inserire la condizione da verificare del blocco for & \textcolor{red}{\textit{Non Soddisfatto}}\\ \hline \hyperlink{Interno}{Interno}\\
& \hyperref[UC5.5]{UC5.5}\\
& \hyperref[UC5.5.2]{UC5.5.2}\\ \hline

\hypertarget{RFO4.5.3}{RFO4.5.3} & l'attore può inserire l'incremento/decremento del blocco for & \hyperlink{Interno}{Interno}\\
& \hyperref[UC5.5]{UC5.5}\\
& \hyperref[UC5.5.3]{UC5.5.3}\\ \hline

\hypertarget{RFO4.5.4}{RFO4.5.4} & l'attore può completare il corpo del blocco for, inserendo una serie di blocchi tra quelli resi disponibili dall'editor & \hyperlink{Interno}{Interno}\\
& \hyperref[UC5.5]{UC5.5}\\
& \hyperref[UC5.5.4]{UC5.5.4}\\ \hline

\hypertarget{RFO4.5.5}{RFO4.5.5} & l’attore può inserire un commento per il blocco for & \hyperlink{Interno}{Interno}\\
& \hyperref[UC5.5]{UC5.5}\\
& \hyperref[UC5.5.5]{UC5.5.5}\\ \hline

\hypertarget{RFO4.5.6}{RFO4.5.6} & l’attore può confermare l'inserimento del blocco for &  \hyperlink{Interno}{Interno}\\
& \hyperref[UC5.5]{UC5.5}\\
& \hyperref[UC5.5.6]{UC5.5.6}\\ \hline

\hypertarget{RFO4.5.6.1}{RFO4.5.6.1} & il sistema deve visualizzare un messaggio d'errore se l'attore non ha fornito la condizione da verificare, non ha fornito l'incremento/decremento o non ha inserito un commento & \hyperlink{Interno}{Interno}\\
& \hyperref[UC5.5.6]{UC5.5.6}\\
& \hyperref[UC5.5.6.1]{UC5.5.6.1}\\ \hline

\hypertarget{RFO4.6}{RFO4.6} & l'attore può inserire un blocco custom di codice & \hyperlink{Riunione Esterna}{Riunione Esterna}\\
& \hyperref[UC5]{UC5}\\
& \hyperref[UC5.6]{UC5.6}\\ \hline

\hypertarget{RFO4.6.1}{RFO4.6.1} & l'attore può inserire il contenuto del blocco custom & \hyperlink{Interno}{Interno}\\
& \hyperref[UC5.6]{UC5.6}\\
& \hyperref[UC5.6.1]{UC5.6.1}\\ \hline

\hypertarget{RFO4.6.2}{RFO4.6.2} & l’attore può inserire un commento per il blocco custom & \hyperlink{Interno}{Interno}\\
& \hyperref[UC5.6]{UC5.6}\\
& \hyperref[UC5.6.2]{UC5.6.2}\\ \hline

\hypertarget{RFO4.6.3}{RFO4.6.3} & l’attore può confermare l'inserimento del blocco custom &\hyperlink{Interno}{Interno}\\
& \hyperref[UC5.6]{UC5.6}\\
& \hyperref[UC5.6.3]{UC5.6.3}\\ \hline

\hypertarget{RFO4.7}{RFO4.7} & \hyperlink{Interno}{Interno}\\
& \hyperref[UC5]{UC5}\\
& & \hyperref[UC5.7]{UC5.7}\\ \hline

\hypertarget{RFO4.8}{RFO4.8} & l'attore può rimuovere un blocco metodo & \hyperlink{Interno}{Interno}\\
& \hyperref[UC5]{UC5}\\
& \hyperref[UC5.8]{UC5.8}\\ \hline

\hypertarget{RFO4.9}{RFO4.9} & l'attore può rimuovere un blocco if/else & \hyperlink{Interno}{Interno}\\
& \hyperref[UC5]{UC5}\\
& \hyperref[UC5.9]{UC5.9}\\ \hline

\hypertarget{RFO4.10}{RFO4.10} & \hyperlink{Interno}{Interno}\\
& \hyperref[UC5]{UC5}\\
& & \hyperref[UC5.10]{UC5.10}\\ \hline

\hypertarget{RFO4.11}{RFO4.11} & l'attore può rimuovere un blocco for & \hyperlink{Interno}{Interno}\\
& \hyperref[UC5]{UC5}\\
& \hyperref[UC5.11]{UC5.11}\\ \hline

\hypertarget{RFO4.12}{RFO4.12} & l'attore può rimuovere un blocco custom & \hyperlink{Interno}{Interno}\\
& \hyperref[UC5]{UC5}\\
& \hyperref[UC5.12]{UC5.12}\\ \hline

\hypertarget{RFD4.13}{RFD4.13} & l'attore può ridurre il blocco if/else & \hyperlink{Interno}{Interno}\\
& \hyperref[UC5]{UC5}\\
& \hyperref[UC5.12]{UC5.12}\\ \hline

\hypertarget{RFD4.14}{RFD4.14} & l'attore può espandere il blocco if/else & \hyperlink{Riunione Esterna}{Riunione Esterna}\\
& \hyperref[UC5]{UC5}\\
& \hyperref[UC5.14]{UC5.14}\\ \hline

\hypertarget{RFD4.15}{RFD4.15} & l'attore può ridurre il blocco while & \hyperlink{Riunione Esterna}{Riunione Esterna}\\
& \hyperref[UC5]{UC5}\\
& \hyperref[UC5.15]{UC5.15}\\ \hline

\hypertarget{RFD4.16}{RFD4.16} & l'attore può espandere il blocco while & \hyperlink{Riunione Esterna}{Riunione Esterna}\\
& \hyperref[UC5]{UC5}\\
& \hyperref[UC5.16]{UC5.16}\\ \hline

\hypertarget{RFD4.17}{RFD4.17} & \hyperlink{Riunione Esterna}{Riunione Esterna}\\
& \hyperref[UC5]{UC5}\\
& & \hyperref[UC5.17]{UC5.17}\\ \hline

\hypertarget{RFD4.18}{RFD4.18} & l'attore può espandere il blocco for &  \hyperlink{Riunione Esterna}{Riunione Esterna}\\
& \hyperref[UC5]{UC5}\\
& \hyperref[UC5.18]{UC5.18}\\ \hline

\hypertarget{RFO5}{RFO5} & l'attore può salvare il progetto &  \hyperlink{Interno}{Interno}\\
& & \hyperref[UC6]{UC6}\\ \hline

\hypertarget{RFO6}{RFO6} & l'attore può generare l'applicativo dai diagrammi da lui realizzati & \hyperlink{Riunione Esterna}{Riunione Esterna}\\
& & \hyperref[UC7]{UC7}\\ \hline

\caption[Requisiti Funzionali]{Requisiti Funzionali}
\label{tabella:req0}
\end{longtable}
\clearpage

\subsection{Requisiti di Qualità}
\normalsize
\begin{longtable}{|c|>{\centering}m{7cm}|c|}
\hline 
\textbf{Id Requisito} & \textbf{Descrizione} & \textbf{Stato}\\
\hline
\endhead
\hypertarget{RQO1}{RQO1} & è fornito un manuale utente & \textcolor{red}{\textit{Non Soddisfatto}}\\ \hline

\hypertarget{RQO1.1}{RQO1.1} & il manuale utente contiene una sezione in cui è descritta ogni interazione discussa come Use Case. & \textcolor{red}{\textit{Non Soddisfatto}}\\ \hline

\hypertarget{RQO1.1.1}{RQO1.1.1} & il manuale utente contiene una sezione con le possibili operazioni di troubleshooting nell'uso del software. & \textcolor{red}{\textit{Non Soddisfatto}}\\ \hline

\hypertarget{RQO1.2}{RQO1.2} & il manuale utente spiega come procedere all'installazione del server & \textcolor{red}{\textit{Non Soddisfatto}}\\ \hline

\hypertarget{RQO1.2.1}{RQO1.2.1} & il manuale utente contiene una sezione con le possibili operazioni di troubleshooting per l'installazione del server & \textcolor{red}{\textit{Non Soddisfatto}}\\ \hline

\hypertarget{RQO1.3}{RQO1.3} & il manuale utente è fornito in lingua italiana & \textcolor{red}{\textit{Non Soddisfatto}}\\ \hline

\hypertarget{RQF1.4}{RQF1.4} & il manuale utente è fornito in lingua inglese & \textcolor{red}{\textit{Non Soddisfatto}}\\ \hline

\hypertarget{RQO2}{RQO2} & è fornito un manuale utente sviluppatore, dedicato a chiunque voglia estendere l'applicazione & \textcolor{red}{\textit{Non Soddisfatto}}\\ \hline

\hypertarget{RQF2.1}{RQF2.1} & il manuale utente sviluppatore è fornito in lingua inglese & \textcolor{red}{\textit{Non Soddisfatto}}\\ \hline

\hypertarget{RQO2.2}{RQO2.2} & il manuale utente sviluppatore è fornito in lingua italiana & \textcolor{red}{\textit{Non Soddisfatto}}\\ \hline

\caption[Requisiti di Qualità]{Requisiti di Qualità}
\label{tabella:req2}
\end{longtable}
\clearpage
\subsection{Requisiti di Vincolo}
\normalsize
\begin{longtable}{|c|>{\centering}m{7cm}|c|}
\hline 
\textbf{Id Requisito} & \textbf{Descrizione} & \textbf{Stato}\\
\hline
\endhead
\hypertarget{RVO1}{RVO1} & l'applicativo deve essere sviluppato tramite tecnologie web. & \textcolor{red}{\textit{Non Soddisfatto}}\\ \hline

\hypertarget{RVO1.1}{RVO1.1} & l'applicativo lato client è realizzato in HTML5, CSS e Javascript. & \textcolor{red}{\textit{Non Soddisfatto}}\\ \hline

\hypertarget{RVO1.2}{RVO1.2} & l'applicativo lato server è realizzato in Java con server Tomcat. & \textcolor{red}{\textit{Non Soddisfatto}}\\ \hline

\hypertarget{RVO1.3}{RVO1.3} & l'applicativo funziona su Mozilla Firefox versione 43 o superiore & \textcolor{red}{\textit{Non Soddisfatto}}\\ \hline

\hypertarget{RVO1.4}{RVO1.4} & l'applicativo funziona su Google Chrome versione 47 o superiore	 & \textcolor{red}{\textit{Non Soddisfatto}}\\ \hline

\hypertarget{RVD1.5}{RVD1.5} & l'applicativo funziona su Internet Explorer versione 11 o superiore	 & \textcolor{red}{\textit{Non Soddisfatto}}\\ \hline

\hypertarget{RVO1.6}{RVO1.6} & l'applicativo funziona su Safari versione 9 o superiore & \textcolor{red}{\textit{Non Soddisfatto}}\\ \hline

\hypertarget{RVD1.7}{RVD1.7} & l'applicativo funziona su Microsoft Edge versione 25 o superiore & \textcolor{red}{\textit{Non Soddisfatto}}\\ \hline

\hypertarget{RVO2}{RVO2} & il progetto deve essere sviluppato sulla piattaforma GitHub in modalità pubblica & \textcolor{red}{\textit{Non Soddisfatto}}\\ \hline

\caption[Requisiti di Vincolo]{Requisiti di Vincolo}
\label{tabella:req3}
\end{longtable}
\clearpage

\subsection{Tracciamento Requisiti-Fonti}
\normalsize
\begin{longtable}{|>{\centering}m{5cm}|m{5cm}<{\centering}|}
\hline 
\textbf{Id Requisito} & \textbf{Fonti}\\
\hline
\endhead
\hyperlink{RFO1}{RFO1} & \hyperlink{Interno}{Interno}\\
& \hyperref[UC2]{UC2}\\ \hline

\hyperlink{RFO2}{RFO2} & \hyperlink{Interno}{Interno}\\
& \hyperref[UC3]{UC3}\\ \hline

\hyperlink{RFO3}{RFO3} & \hyperlink{Capitolato}{Capitolato}\\
& \hyperref[UC4]{UC4}\\ \hline

\hyperlink{RFO3.1}{RFO3.1} & \hyperlink{Interno}{Interno}\\
& \hyperref[UC4]{UC4}\\
& \hyperref[UC4.1]{UC4.1}\\ \hline

\hyperlink{RFO3.1.1}{RFO3.1.1} & \hyperlink{Interno}{Interno}\\
& \hyperref[UC4.1]{UC4.1}\\
& \hyperref[UC4.1.1]{UC4.1.1}\\ \hline

\hyperlink{RFO3.1.2}{RFO3.1.2} & \hyperlink{Interno}{Interno}\\
& \hyperref[UC4.1]{UC4.1}\\
& \hyperref[UC4.1.2]{UC4.1.2}\\ \hline

\hyperlink{RFO3.1.3}{RFO3.1.3} & \hyperlink{Riunione Esterna}{Riunione Esterna}\\
& \hyperref[UC4.1]{UC4.1}\\
& \hyperref[UC4.1.3]{UC4.1.3}\\ \hline

\hyperlink{RFO3.1.4}{RFO3.1.4} & \hyperlink{Interno}{Interno}\\
& \hyperref[UC4.1]{UC4.1}\\
& \hyperref[UC4.1.4]{UC4.1.4}\\ \hline

\hyperlink{RFO3.1.4.1}{RFO3.1.4.1} & \hyperlink{Interno}{Interno}\\
& \hyperref[UC4.1.4]{UC4.1.4}\\
& \hyperref[UC4.1.4.1]{UC4.1.4.1}\\ \hline

\hyperlink{RFO3.1.4.2}{RFO3.1.4.2} & \hyperlink{Interno}{Interno}\\
& \hyperref[UC4.1.4]{UC4.1.4}\\
& \hyperref[UC4.1.4.2]{UC4.1.4.2}\\ \hline

\hyperlink{RFO3.1.4.3}{RFO3.1.4.3} & \hyperlink{Interno}{Interno}\\
& \hyperref[UC4.1.4]{UC4.1.4}\\
& \hyperref[UC4.1.4.3]{UC4.1.4.3}\\ \hline

\hyperlink{RFO3.1.4.4}{RFO3.1.4.4} & \hyperlink{Interno}{Interno}\\
& \hyperref[UC4.1.4]{UC4.1.4}\\
& \hyperref[UC4.1.4.4]{UC4.1.4.4}\\ \hline

\hyperlink{RFO3.1.4.5}{RFO3.1.4.5} & \hyperlink{Interno}{Interno}\\
& \hyperref[UC4.1.4]{UC4.1.4}\\
& \hyperref[UC4.1.4.5]{UC4.1.4.5}\\ \hline

\hyperlink{RFO3.1.4.6}{RFO3.1.4.6} & \hyperlink{Interno}{Interno}\\
& \hyperref[UC4.1.4]{UC4.1.4}\\
& \hyperref[UC4.1.4.6]{UC4.1.4.6}\\ \hline

\hyperlink{RFO3.1.4.6.1}{RFO3.1.4.6.1} & \hyperlink{Interno}{Interno}\\
& \hyperref[UC4.1.4.6]{UC4.1.4.6}\\
& \hyperref[UC4.1.4.6.1]{UC4.1.4.6.1}\\ \hline

\hyperlink{RFO3.1.5}{RFO3.1.5} & \hyperlink{Interno}{Interno}\\
& \hyperref[UC4.1]{UC4.1}\\
& \hyperref[UC4.1.5]{UC4.1.5}\\ \hline

\hyperlink{RFO3.1.6}{RFO3.1.6} & \hyperlink{Interno}{Interno}\\
& \hyperref[UC4.1]{UC4.1}\\
& \hyperref[UC4.1.6]{UC4.1.6}\\ \hline

\hyperlink{RFO3.1.6.1}{RFO3.1.6.1} & \hyperlink{Interno}{Interno}\\
& \hyperref[UC4.1.6]{UC4.1.6}\\
& \hyperref[UC4.1.6.1]{UC4.1.6.1}\\ \hline

\hyperlink{RFO3.1.6.2}{RFO3.1.6.2} & \hyperlink{Interno}{Interno}\\
& \hyperref[UC4.1.6]{UC4.1.6}\\
& \hyperref[UC4.1.6.2]{UC4.1.6.2}\\ \hline

\hyperlink{RFO3.1.6.3}{RFO3.1.6.3} & \hyperlink{Interno}{Interno}\\
& \hyperref[UC4.1.6]{UC4.1.6}\\
& \hyperref[UC4.1.6.3]{UC4.1.6.3}\\ \hline

\hyperlink{RFO3.1.6.3.1}{RFO3.1.6.3.1} & \hyperlink{Interno}{Interno}\\
& \hyperref[UC4.1.6.3]{UC4.1.6.3}\\
& \hyperref[UC4.1.6.3.1]{UC4.1.6.3.1}\\ \hline

\hyperlink{RFO3.1.6.3.2}{RFO3.1.6.3.2} & \hyperlink{Interno}{Interno}\\
& \hyperref[UC4.1.6.3]{UC4.1.6.3}\\
& \hyperref[UC4.1.6.3.2]{UC4.1.6.3.2}\\ \hline

\hyperlink{RFO3.1.6.3.3}{RFO3.1.6.3.3} & \hyperlink{Interno}{Interno}\\
& \hyperref[UC4.1.6.3]{UC4.1.6.3}\\
& \hyperref[UC4.1.6.3.3]{UC4.1.6.3.3}\\ \hline

\hyperlink{RFO3.1.6.3.3.1}{RFO3.1.6.3.3.1} & \hyperlink{Interno}{Interno}\\
& \hyperref[UC4.1.6.3.3]{UC4.1.6.3.3}\\
& \hyperref[UC4.1.6.3.3.1]{UC4.1.6.3.3.1}\\ \hline

\hyperlink{RFO3.1.6.4}{RFO3.1.6.4} & \hyperlink{Interno}{Interno}\\
& \hyperref[UC4.1.6]{UC4.1.6}\\
& \hyperref[UC4.1.6.4]{UC4.1.6.4}\\ \hline

\hyperlink{RFO3.1.6.5}{RFO3.1.6.5} & \hyperlink{Interno}{Interno}\\
& \hyperref[UC4.1.6]{UC4.1.6}\\
& \hyperref[UC4.1.6.5]{UC4.1.6.5}\\ \hline

\hyperlink{RFO3.1.6.5.1}{RFO3.1.6.5.1} & \hyperlink{Interno}{Interno}\\
& \hyperref[UC4.1.6.5]{UC4.1.6.5}\\
& \hyperref[UC4.1.6.5.1]{UC4.1.6.5.1}\\ \hline

\hyperlink{RFO3.1.7}{RFO3.1.7} & \hyperlink{Interno}{Interno}\\
& \hyperref[UC4.1]{UC4.1}\\
& \hyperref[UC4.1.7]{UC4.1.7}\\ \hline

\hyperlink{RFO3.1.8}{RFO3.1.8} & \hyperlink{Interno}{Interno}\\
& \hyperref[UC4.1]{UC4.1}\\
& \hyperref[UC4.1.8]{UC4.1.8}\\ \hline

\hyperlink{RFO3.1.8.1}{RFO3.1.8.1} & \hyperlink{Interno}{Interno}\\
& \hyperref[UC4.1.8]{UC4.1.8}\\ \hline

\hyperlink{RFO3.2}{RFO3.2} & \hyperlink{Interno}{Interno}\\
& \hyperref[UC4]{UC4}\\
& \hyperref[UC4.2]{UC4.2}\\ \hline

\hyperlink{RFO3.2.1}{RFO3.2.1} & \hyperlink{Interno}{Interno}\\
& \hyperref[UC4.2]{UC4.2}\\
& \hyperref[UC4.2.1]{UC4.2.1}\\ \hline

\hyperlink{RFO3.2.2}{RFO3.2.2} & \hyperlink{Interno}{Interno}\\
& \hyperref[UC4.2]{UC4.2}\\
& \hyperref[UC4.2.2]{UC4.2.2}\\ \hline

\hyperlink{RFO3.2.2.1}{RFO3.2.2.1} & \hyperlink{Interno}{Interno}\\
& \hyperref[UC4]{UC4}\\
& \hyperref[UC4.6]{UC4.6}\\ \hline

\hyperlink{RFO3.2.3}{RFO3.2.3} & \hyperlink{Interno}{Interno}\\
& \hyperref[UC4.2]{UC4.2}\\
& \hyperref[UC4.2.3]{UC4.2.3}\\ \hline

\hyperlink{RFO3.2.4}{RFO3.2.4} & \hyperlink{Interno}{Interno}\\
& \hyperref[UC4.2]{UC4.2}\\
& \hyperref[UC4.2.4]{UC4.2.4}\\ \hline

\hyperlink{RFO3.2.5}{RFO3.2.5} & \hyperlink{Interno}{Interno}\\
& \hyperref[UC4.2]{UC4.2}\\
& \hyperref[UC4.2.5]{UC4.2.5}\\ \hline

\hyperlink{RFO3.2.6}{RFO3.2.6} & \hyperlink{Interno}{Interno}\\
& \hyperref[UC4.2]{UC4.2}\\
& \hyperref[UC4.2.6]{UC4.2.6}\\ \hline

\hyperlink{RFO3.2.6.1}{RFO3.2.6.1} & \hyperlink{Interno}{Interno}\\
& \hyperref[UC4.2.6]{UC4.2.6}\\
& \hyperref[UC4.2.6.1]{UC4.2.6.1}\\ \hline

\hyperlink{RFO3.3}{RFO3.3} & \hyperlink{Interno}{Interno}\\
& \hyperref[UC4]{UC4}\\
& \hyperref[UC4.3]{UC4.3}\\ \hline

\hyperlink{RFO3.3.1}{RFO3.3.1} & \hyperlink{Interno}{Interno}\\
& \hyperref[UC4.3]{UC4.3}\\
& \hyperref[UC4.3.1]{UC4.3.1}\\ \hline

\hyperlink{RFO3.3.2}{RFO3.3.2} & \hyperlink{Interno}{Interno}\\
& \hyperref[UC4.3]{UC4.3}\\
& \hyperref[UC4.3.2]{UC4.3.2}\\ \hline

\hyperlink{RFO3.3.3}{RFO3.3.3} & \hyperlink{Interno}{Interno}\\
& \hyperref[UC4.3]{UC4.3}\\
& \hyperref[UC4.3.3]{UC4.3.3}\\ \hline

\hyperlink{RFO3.3.3.1}{RFO3.3.3.1} & \hyperlink{Interno}{Interno}\\
& \hyperref[UC4.3.3]{UC4.3.3}\\
& \hyperref[UC4.3.3.1]{UC4.3.3.1}\\ \hline

\hyperlink{RFO3.4}{RFO3.4} & \hyperlink{Interno}{Interno}\\
& \hyperref[UC4]{UC4}\\
& \hyperref[UC4.4]{UC4.4}\\ \hline

\hyperlink{RFO3.5}{RFO3.5} & \hyperlink{Interno}{Interno}\\
& \hyperref[UC4]{UC4}\\
& \hyperref[UC4.5]{UC4.5}\\ \hline

\hyperlink{RFO4}{RFO4} & \hyperlink{Capitolato}{Capitolato}\\
& \hyperref[UC5]{UC5}\\ \hline

\hyperlink{RFO4.1}{RFO4.1} & \hyperlink{Riunione Esterna}{Riunione Esterna}\\
& \hyperref[UC5]{UC5}\\
& \hyperref[UC5.1]{UC5.1}\\ \hline

\hyperlink{RFO4.1.1}{RFO4.1.1} & \hyperlink{Interno}{Interno}\\
& \hyperref[UC5.1]{UC5.1}\\
& \hyperref[UC5.1.1]{UC5.1.1}\\ \hline

\hyperlink{RFO4.1.2}{RFO4.1.2} & \hyperlink{Interno}{Interno}\\
& \hyperref[UC5.1]{UC5.1}\\
& \hyperref[UC5.1.2]{UC5.1.2}\\ \hline

\hyperlink{RFO4.1.3}{RFO4.1.3} & \hyperlink{Interno}{Interno}\\
& \hyperref[UC5.1]{UC5.1}\\
& \hyperref[UC5.1.3]{UC5.1.3}\\ \hline

\hyperlink{RFO4.1.4}{RFO4.1.4} & \hyperlink{Interno}{Interno}\\
& \hyperref[UC5.1]{UC5.1}\\
& \hyperref[UC5.1.4]{UC5.1.4}\\ \hline

\hyperlink{RFO4.1.4.1}{RFO4.1.4.1} & \hyperlink{Interno}{Interno}\\
& \hyperref[UC5.1.4]{UC5.1.4}\\
& \hyperref[UC5.1.4.1]{UC5.1.4.1}\\ \hline

\hyperlink{RFO4.2}{RFO4.2} & \hyperlink{Riunione Esterna}{Riunione Esterna}\\
& \hyperref[UC5]{UC5}\\
& \hyperref[UC5.2]{UC5.2}\\ \hline

\hyperlink{RFO4.2.1}{RFO4.2.1} & \hyperlink{Interno}{Interno}\\
& \hyperref[UC5.2]{UC5.2}\\
& \hyperref[UC5.2.1]{UC5.2.1}\\ \hline

\hyperlink{RFO4.2.2}{RFO4.2.2} & \hyperlink{Interno}{Interno}\\
& \hyperref[UC5.2]{UC5.2}\\
& \hyperref[UC5.2.2]{UC5.2.2}\\ \hline

\hyperlink{RFO4.2.3}{RFO4.2.3} & \hyperlink{Interno}{Interno}\\
& \hyperref[UC5.2]{UC5.2}\\
& \hyperref[UC5.2.3]{UC5.2.3}\\ \hline

\hyperlink{RFO4.2.3.1}{RFO4.2.3.1} & \hyperlink{Interno}{Interno}\\
& \hyperref[UC5.2.3]{UC5.2.3}\\
& \hyperref[UC5.2.3.1]{UC5.2.3.1}\\ \hline

\hyperlink{RFO4.3}{RFO4.3} & \hyperlink{Riunione Esterna}{Riunione Esterna}\\
& \hyperref[UC5]{UC5}\\
& \hyperref[UC5.3]{UC5.3}\\ \hline

\hyperlink{RFO4.3.1}{RFO4.3.1} & \hyperlink{Interno}{Interno}\\
& \hyperref[UC5.3]{UC5.3}\\
& \hyperref[UC5.3.1]{UC5.3.1}\\ \hline

\hyperlink{RFO4.3.2}{RFO4.3.2} & \hyperlink{Interno}{Interno}\\
& \hyperref[UC5.3]{UC5.3}\\
& \hyperref[UC5.3.2]{UC5.3.2}\\ \hline

\hyperlink{RFO4.3.3}{RFO4.3.3} & \hyperlink{Interno}{Interno}\\
& \hyperref[UC5.3]{UC5.3}\\
& \hyperref[UC5.3.3]{UC5.3.3}\\ \hline

\hyperlink{RFO4.3.4}{RFO4.3.4} & \hyperlink{Interno}{Interno}\\
& \hyperref[UC5.3]{UC5.3}\\
& \hyperref[UC5.3.4]{UC5.3.4}\\ \hline

\hyperlink{RFO4.3.5}{RFO4.3.5} & \hyperlink{Interno}{Interno}\\
& \hyperref[UC5.3]{UC5.3}\\
& \hyperref[UC5.3.5]{UC5.3.5}\\ \hline

\hyperlink{RFO4.3.5.1}{RFO4.3.5.1} & \hyperlink{Interno}{Interno}\\
& \hyperref[UC5.3.5]{UC5.3.5}\\
& \hyperref[UC5.3.5.1]{UC5.3.5.1}\\ \hline

\hyperlink{RFO4.4}{RFO4.4} & \hyperlink{Riunione Esterna}{Riunione Esterna}\\
& \hyperref[UC5]{UC5}\\
& \hyperref[UC5.4]{UC5.4}\\ \hline

\hyperlink{RFO4.4.1}{RFO4.4.1} & \hyperlink{Interno}{Interno}\\
& \hyperref[UC5.4]{UC5.4}\\
& \hyperref[UC5.4.1]{UC5.4.1}\\ \hline

\hyperlink{RFO4.4.2}{RFO4.4.2} & \hyperlink{Interno}{Interno}\\
& \hyperref[UC5.4]{UC5.4}\\
& \hyperref[UC5.4.2]{UC5.4.2}\\ \hline

\hyperlink{RFO4.4.3}{RFO4.4.3} & \hyperlink{Interno}{Interno}\\
& \hyperref[UC5.4]{UC5.4}\\
& \hyperref[UC5.4.3]{UC5.4.3}\\ \hline

\hyperlink{RFO4.4.4}{RFO4.4.4} & \hyperlink{Interno}{Interno}\\
& \hyperref[UC5.4]{UC5.4}\\
& \hyperref[UC5.4.4]{UC5.4.4}\\ \hline

\hyperlink{RFO4.4.4.1}{RFO4.4.4.1} & \hyperlink{Interno}{Interno}\\
& \hyperref[UC5.4.4]{UC5.4.4}\\
& \hyperref[UC5.4.4.1]{UC5.4.4.1}\\ \hline

\hyperlink{RFO4.5}{RFO4.5} & \hyperlink{Riunione Esterna}{Riunione Esterna}\\
& \hyperref[UC5]{UC5}\\
& \hyperref[UC5.5]{UC5.5}\\ \hline

\hyperlink{RFO4.5.1}{RFO4.5.1} & \hyperlink{Interno}{Interno}\\
& \hyperref[UC5.5]{UC5.5}\\
& \hyperref[UC5.5.1]{UC5.5.1}\\ \hline

\hyperlink{RFO4.5.2}{RFO4.5.2} & \hyperlink{Interno}{Interno}\\
& \hyperref[UC5.5]{UC5.5}\\
& \hyperref[UC5.5.2]{UC5.5.2}\\ \hline

\hyperlink{RFO4.5.3}{RFO4.5.3} & \hyperlink{Interno}{Interno}\\
& \hyperref[UC5.5]{UC5.5}\\
& \hyperref[UC5.5.3]{UC5.5.3}\\ \hline

\hyperlink{RFO4.5.4}{RFO4.5.4} & \hyperlink{Interno}{Interno}\\
& \hyperref[UC5.5]{UC5.5}\\
& \hyperref[UC5.5.4]{UC5.5.4}\\ \hline

\hyperlink{RFO4.5.5}{RFO4.5.5} & \hyperlink{Interno}{Interno}\\
& \hyperref[UC5.5]{UC5.5}\\
& \hyperref[UC5.5.5]{UC5.5.5}\\ \hline

\hyperlink{RFO4.5.6}{RFO4.5.6} & \hyperlink{Interno}{Interno}\\
& \hyperref[UC5.5]{UC5.5}\\
& \hyperref[UC5.5.6]{UC5.5.6}\\ \hline

\hyperlink{RFO4.5.6.1}{RFO4.5.6.1} & \hyperlink{Interno}{Interno}\\
& \hyperref[UC5.5.6]{UC5.5.6}\\
& \hyperref[UC5.5.6.1]{UC5.5.6.1}\\ \hline

\hyperlink{RFO4.6}{RFO4.6} & \hyperlink{Riunione Esterna}{Riunione Esterna}\\
& \hyperref[UC5]{UC5}\\
& \hyperref[UC5.6]{UC5.6}\\ \hline

\hyperlink{RFO4.6.1}{RFO4.6.1} & \hyperlink{Interno}{Interno}\\
& \hyperref[UC5.6]{UC5.6}\\
& \hyperref[UC5.6.1]{UC5.6.1}\\ \hline

\hyperlink{RFO4.6.2}{RFO4.6.2} & \hyperlink{Interno}{Interno}\\
& \hyperref[UC5.6]{UC5.6}\\
& \hyperref[UC5.6.2]{UC5.6.2}\\ \hline

\hyperlink{RFO4.6.3}{RFO4.6.3} & \hyperlink{Interno}{Interno}\\
& \hyperref[UC5.6]{UC5.6}\\
& \hyperref[UC5.6.3]{UC5.6.3}\\ \hline

\hyperlink{RFO4.7}{RFO4.7} & \hyperlink{Interno}{Interno}\\
& \hyperref[UC5]{UC5}\\
& \hyperref[UC5.7]{UC5.7}\\ \hline

\hyperlink{RFO4.8}{RFO4.8} & \hyperlink{Interno}{Interno}\\
& \hyperref[UC5]{UC5}\\
& \hyperref[UC5.8]{UC5.8}\\ \hline

\hyperlink{RFO4.9}{RFO4.9} & \hyperlink{Interno}{Interno}\\
& \hyperref[UC5]{UC5}\\
& \hyperref[UC5.9]{UC5.9}\\ \hline

\hyperlink{RFO4.10}{RFO4.10} & \hyperlink{Interno}{Interno}\\
& \hyperref[UC5]{UC5}\\
& \hyperref[UC5.10]{UC5.10}\\ \hline

\hyperlink{RFO4.11}{RFO4.11} & \hyperlink{Interno}{Interno}\\
& \hyperref[UC5]{UC5}\\
& \hyperref[UC5.11]{UC5.11}\\ \hline

\hyperlink{RFO4.12}{RFO4.12} & \hyperlink{Interno}{Interno}\\
& \hyperref[UC5]{UC5}\\
& \hyperref[UC5.12]{UC5.12}\\ \hline

\hyperlink{RFD4.13}{RFD4.13} & \hyperlink{Riunione Esterna}{Riunione Esterna}\\
& \hyperref[UC5]{UC5}\\
& \hyperref[UC5.13]{UC5.13}\\ \hline

\hyperlink{RFD4.14}{RFD4.14} & \hyperlink{Riunione Esterna}{Riunione Esterna}\\
& \hyperref[UC5]{UC5}\\
& \hyperref[UC5.14]{UC5.14}\\ \hline

\hyperlink{RFD4.15}{RFD4.15} & \hyperlink{Riunione Esterna}{Riunione Esterna}\\
& \hyperref[UC5]{UC5}\\
& \hyperref[UC5.15]{UC5.15}\\ \hline

\hyperlink{RFD4.16}{RFD4.16} & \hyperlink{Riunione Esterna}{Riunione Esterna}\\
& \hyperref[UC5]{UC5}\\
& \hyperref[UC5.16]{UC5.16}\\ \hline

\hyperlink{RFD4.17}{RFD4.17} & \hyperlink{Riunione Esterna}{Riunione Esterna}\\
& \hyperref[UC5]{UC5}\\
& \hyperref[UC5.17]{UC5.17}\\ \hline

\hyperlink{RFD4.18}{RFD4.18} & \hyperlink{Riunione Esterna}{Riunione Esterna}\\
& \hyperref[UC5]{UC5}\\
& \hyperref[UC5.18]{UC5.18}\\ \hline

\hyperlink{RFO5}{RFO5} & \hyperlink{Interno}{Interno}\\
& \hyperref[UC6]{UC6}\\ \hline

\hyperlink{RFO6}{RFO6} & \hyperlink{Riunione Esterna}{Riunione Esterna}\\
& \hyperref[UC7]{UC7}\\ \hline

\hyperlink{RQO1}{RQO1} & \hyperlink{Capitolato}{Capitolato}\\ \hline

\hyperlink{RQO1.1}{RQO1.1} & \hyperlink{Interno}{Interno}\\ \hline

\hyperlink{RQO1.1.1}{RQO1.1.1} & \hyperlink{Interno}{Interno}\\ \hline

\hyperlink{RQO1.2}{RQO1.2} & \hyperlink{Interno}{Interno}\\ \hline

\hyperlink{RQO1.2.1}{RQO1.2.1} & \hyperlink{Interno}{Interno}\\ \hline

\hyperlink{RQO1.3}{RQO1.3} & \hyperlink{Interno}{Interno}\\ \hline

\hyperlink{RQF1.4}{RQF1.4} & \hyperlink{Interno}{Interno}\\ \hline

\hyperlink{RQO2}{RQO2} & \hyperlink{Capitolato}{Capitolato}\\ \hline

\hyperlink{RQF2.1}{RQF2.1} & \hyperlink{Interno}{Interno}\\ \hline

\hyperlink{RQO2.2}{RQO2.2} & \hyperlink{Interno}{Interno}\\ \hline

\hyperlink{RVO1}{RVO1} & \hyperlink{Capitolato}{Capitolato}\\ \hline

\hyperlink{RVO1.1}{RVO1.1} & \hyperlink{Capitolato}{Capitolato}\\ \hline

\hyperlink{RVO1.2}{RVO1.2} & \hyperlink{Capitolato}{Capitolato}\\ \hline

\hyperlink{RVO1.3}{RVO1.3} & \hyperlink{Interno}{Interno}\\ \hline

\hyperlink{RVO1.4}{RVO1.4} & \hyperlink{Interno}{Interno}\\ \hline

\hyperlink{RVD1.5}{RVD1.5} & \hyperlink{Interno}{Interno}\\ \hline

\hyperlink{RVO1.6}{RVO1.6} & \hyperlink{Interno}{Interno}\\ \hline

\hyperlink{RVD1.7}{RVD1.7} & \hyperlink{Interno}{Interno}\\ \hline

\hyperlink{RVO2}{RVO2} & \hyperlink{Capitolato}{Capitolato}\\ \hline

\caption[Tracciamento Requisiti-Fonti]{Tracciamento Requisiti-Fonti}
\label{tabella:requi-fonti}
\end{longtable}

\clearpage

\subsection{Tracciamento Fonti-Requisiti}
\normalsize
\begin{longtable}{|>{\centering}m{5cm}|m{5cm}<{\centering}|}
\hline 
\textbf{Fonte} & \textbf{Id Requisiti}\\
\hline
\endhead
\hyperlink{Capitolato}{Capitolato} & \hyperlink{RFO3}{RFO3}\\
& \hyperlink{RFO4}{RFO4}\\
& \hyperlink{RQO1}{RQO1}\\
& \hyperlink{RQO2}{RQO2}\\
& \hyperlink{RVO1}{RVO1}\\
& \hyperlink{RVO1.1}{RVO1.1}\\
& \hyperlink{RVO1.2}{RVO1.2}\\
& \hyperlink{RVO2}{RVO2}\\ \hline
\hyperlink{Interno}{Interno} & \hyperlink{RFO1}{RFO1}\\
& \hyperlink{RFO2}{RFO2}\\
& \hyperlink{RFO3.1}{RFO3.1}\\
& \hyperlink{RFO3.1.1}{RFO3.1.1}\\
& \hyperlink{RFO3.1.2}{RFO3.1.2}\\
& \hyperlink{RFO3.1.4}{RFO3.1.4}\\
& \hyperlink{RFO3.1.4.1}{RFO3.1.4.1}\\
& \hyperlink{RFO3.1.4.2}{RFO3.1.4.2}\\
& \hyperlink{RFO3.1.4.3}{RFO3.1.4.3}\\
& \hyperlink{RFO3.1.4.4}{RFO3.1.4.4}\\
& \hyperlink{RFO3.1.4.5}{RFO3.1.4.5}\\
& \hyperlink{RFO3.1.4.6}{RFO3.1.4.6}\\
& \hyperlink{RFO3.1.4.6.1}{RFO3.1.4.6.1}\\
& \hyperlink{RFO3.1.5}{RFO3.1.5}\\
& \hyperlink{RFO3.1.6}{RFO3.1.6}\\
& \hyperlink{RFO3.1.6.1}{RFO3.1.6.1}\\
& \hyperlink{RFO3.1.6.2}{RFO3.1.6.2}\\
& \hyperlink{RFO3.1.6.3}{RFO3.1.6.3}\\
& \hyperlink{RFO3.1.6.3.1}{RFO3.1.6.3.1}\\
& \hyperlink{RFO3.1.6.3.2}{RFO3.1.6.3.2}\\
& \hyperlink{RFO3.1.6.3.3}{RFO3.1.6.3.3}\\
& \hyperlink{RFO3.1.6.3.3.1}{RFO3.1.6.3.3.1}\\
& \hyperlink{RFO3.1.6.4}{RFO3.1.6.4}\\
& \hyperlink{RFO3.1.6.5}{RFO3.1.6.5}\\
& \hyperlink{RFO3.1.6.5.1}{RFO3.1.6.5.1}\\
& \hyperlink{RFO3.1.7}{RFO3.1.7}\\
& \hyperlink{RFO3.1.8}{RFO3.1.8}\\
& \hyperlink{RFO3.1.8.1}{RFO3.1.8.1}\\
& \hyperlink{RFO3.2}{RFO3.2}\\
& \hyperlink{RFO3.2.1}{RFO3.2.1}\\
& \hyperlink{RFO3.2.2}{RFO3.2.2}\\
& \hyperlink{RFO3.2.2.1}{RFO3.2.2.1}\\
& \hyperlink{RFO3.2.3}{RFO3.2.3}\\
& \hyperlink{RFO3.2.4}{RFO3.2.4}\\
& \hyperlink{RFO3.2.5}{RFO3.2.5}\\
& \hyperlink{RFO3.2.6}{RFO3.2.6}\\
& \hyperlink{RFO3.2.6.1}{RFO3.2.6.1}\\
& \hyperlink{RFO3.3}{RFO3.3}\\
& \hyperlink{RFO3.3.1}{RFO3.3.1}\\
& \hyperlink{RFO3.3.2}{RFO3.3.2}\\
& \hyperlink{RFO3.3.3}{RFO3.3.3}\\
& \hyperlink{RFO3.3.3.1}{RFO3.3.3.1}\\
& \hyperlink{RFO3.4}{RFO3.4}\\
& \hyperlink{RFO3.5}{RFO3.5}\\
& \hyperlink{RFO4.1.1}{RFO4.1.1}\\
& \hyperlink{RFO4.1.2}{RFO4.1.2}\\
& \hyperlink{RFO4.1.3}{RFO4.1.3}\\
& \hyperlink{RFO4.1.4}{RFO4.1.4}\\
& \hyperlink{RFO4.1.4.1}{RFO4.1.4.1}\\
& \hyperlink{RFO4.2.1}{RFO4.2.1}\\
& \hyperlink{RFO4.2.2}{RFO4.2.2}\\
& \hyperlink{RFO4.2.3}{RFO4.2.3}\\
& \hyperlink{RFO4.2.3.1}{RFO4.2.3.1}\\
& \hyperlink{RFO4.3.1}{RFO4.3.1}\\
& \hyperlink{RFO4.3.2}{RFO4.3.2}\\
& \hyperlink{RFO4.3.3}{RFO4.3.3}\\
& \hyperlink{RFO4.3.4}{RFO4.3.4}\\
& \hyperlink{RFO4.3.5}{RFO4.3.5}\\
& \hyperlink{RFO4.3.5.1}{RFO4.3.5.1}\\
& \hyperlink{RFO4.4.1}{RFO4.4.1}\\
& \hyperlink{RFO4.4.2}{RFO4.4.2}\\
& \hyperlink{RFO4.4.3}{RFO4.4.3}\\
& \hyperlink{RFO4.4.4}{RFO4.4.4}\\
& \hyperlink{RFO4.4.4.1}{RFO4.4.4.1}\\
& \hyperlink{RFO4.5.1}{RFO4.5.1}\\
& \hyperlink{RFO4.5.2}{RFO4.5.2}\\
& \hyperlink{RFO4.5.3}{RFO4.5.3}\\
& \hyperlink{RFO4.5.4}{RFO4.5.4}\\
& \hyperlink{RFO4.5.5}{RFO4.5.5}\\
& \hyperlink{RFO4.5.6}{RFO4.5.6}\\
& \hyperlink{RFO4.5.6.1}{RFO4.5.6.1}\\
& \hyperlink{RFO4.6.1}{RFO4.6.1}\\
& \hyperlink{RFO4.6.2}{RFO4.6.2}\\
& \hyperlink{RFO4.6.3}{RFO4.6.3}\\
& \hyperlink{RFO4.7}{RFO4.7}\\
& \hyperlink{RFO4.8}{RFO4.8}\\
& \hyperlink{RFO4.9}{RFO4.9}\\
& \hyperlink{RFO4.10}{RFO4.10}\\
& \hyperlink{RFO4.11}{RFO4.11}\\
& \hyperlink{RFO4.12}{RFO4.12}\\
& \hyperlink{RFO5}{RFO5}\\
& \hyperlink{RQO1.1}{RQO1.1}\\
& \hyperlink{RQO1.1.1}{RQO1.1.1}\\
& \hyperlink{RQO1.2}{RQO1.2}\\
& \hyperlink{RQO1.2.1}{RQO1.2.1}\\
& \hyperlink{RQO1.3}{RQO1.3}\\
& \hyperlink{RQF1.4}{RQF1.4}\\
& \hyperlink{RQF2.1}{RQF2.1}\\
& \hyperlink{RQO2.2}{RQO2.2}\\
& \hyperlink{RVO1.3}{RVO1.3}\\
& \hyperlink{RVO1.4}{RVO1.4}\\
& \hyperlink{RVD1.5}{RVD1.5}\\
& \hyperlink{RVO1.6}{RVO1.6}\\
& \hyperlink{RVD1.7}{RVD1.7}\\ \hline
\hyperlink{Riunione Esterna}{Riunione Esterna} & \hyperlink{RFO3.1.3}{RFO3.1.3}\\
& \hyperlink{RFO4.1}{RFO4.1}\\
& \hyperlink{RFO4.2}{RFO4.2}\\
& \hyperlink{RFO4.3}{RFO4.3}\\
& \hyperlink{RFO4.4}{RFO4.4}\\
& \hyperlink{RFO4.5}{RFO4.5}\\
& \hyperlink{RFO4.6}{RFO4.6}\\
& \hyperlink{RFD4.13}{RFD4.13}\\
& \hyperlink{RFD4.14}{RFD4.14}\\
& \hyperlink{RFD4.15}{RFD4.15}\\
& \hyperlink{RFD4.16}{RFD4.16}\\
& \hyperlink{RFD4.17}{RFD4.17}\\
& \hyperlink{RFD4.18}{RFD4.18}\\
& \hyperlink{RFO6}{RFO6}\\ \hline
\hyperref[UC2]{UC2} & \hyperlink{RFO1}{RFO1}\\ \hline
\hyperref[UC3]{UC3} & \hyperlink{RFO2}{RFO2}\\ \hline
\hyperref[UC4]{UC4} & \hyperlink{RFO3}{RFO3}\\
& \hyperlink{RFO3.1}{RFO3.1}\\
& \hyperlink{RFO3.2}{RFO3.2}\\
& \hyperlink{RFO3.2.2.1}{RFO3.2.2.1}\\
& \hyperlink{RFO3.3}{RFO3.3}\\
& \hyperlink{RFO3.4}{RFO3.4}\\
& \hyperlink{RFO3.5}{RFO3.5}\\ \hline
\hyperref[UC4.1]{UC4.1} & \hyperlink{RFO3.1}{RFO3.1}\\
& \hyperlink{RFO3.1.1}{RFO3.1.1}\\
& \hyperlink{RFO3.1.2}{RFO3.1.2}\\
& \hyperlink{RFO3.1.3}{RFO3.1.3}\\
& \hyperlink{RFO3.1.4}{RFO3.1.4}\\
& \hyperlink{RFO3.1.5}{RFO3.1.5}\\
& \hyperlink{RFO3.1.6}{RFO3.1.6}\\
& \hyperlink{RFO3.1.7}{RFO3.1.7}\\
& \hyperlink{RFO3.1.8}{RFO3.1.8}\\ \hline
\hyperref[UC4.1.1]{UC4.1.1} & \hyperlink{RFO3.1.1}{RFO3.1.1}\\ \hline
\hyperref[UC4.1.2]{UC4.1.2} & \hyperlink{RFO3.1.2}{RFO3.1.2}\\ \hline
\hyperref[UC4.1.3]{UC4.1.3} & \hyperlink{RFO3.1.3}{RFO3.1.3}\\ \hline
\hyperref[UC4.1.4]{UC4.1.4} & \hyperlink{RFO3.1.4}{RFO3.1.4}\\
& \hyperlink{RFO3.1.4.1}{RFO3.1.4.1}\\
& \hyperlink{RFO3.1.4.2}{RFO3.1.4.2}\\
& \hyperlink{RFO3.1.4.3}{RFO3.1.4.3}\\
& \hyperlink{RFO3.1.4.4}{RFO3.1.4.4}\\
& \hyperlink{RFO3.1.4.5}{RFO3.1.4.5}\\
& \hyperlink{RFO3.1.4.6}{RFO3.1.4.6}\\ \hline
\hyperref[UC4.1.4.1]{UC4.1.4.1} & \hyperlink{RFO3.1.4.1}{RFO3.1.4.1}\\ \hline
\hyperref[UC4.1.4.2]{UC4.1.4.2} & \hyperlink{RFO3.1.4.2}{RFO3.1.4.2}\\ \hline
\hyperref[UC4.1.4.3]{UC4.1.4.3} & \hyperlink{RFO3.1.4.3}{RFO3.1.4.3}\\ \hline
\hyperref[UC4.1.4.4]{UC4.1.4.4} & \hyperlink{RFO3.1.4.4}{RFO3.1.4.4}\\ \hline
\hyperref[UC4.1.4.5]{UC4.1.4.5} & \hyperlink{RFO3.1.4.5}{RFO3.1.4.5}\\ \hline
\hyperref[UC4.1.4.6]{UC4.1.4.6} & \hyperlink{RFO3.1.4.6}{RFO3.1.4.6}\\
& \hyperlink{RFO3.1.4.6.1}{RFO3.1.4.6.1}\\ \hline
\hyperref[UC4.1.4.6.1]{UC4.1.4.6.1} & \hyperlink{RFO3.1.4.6.1}{RFO3.1.4.6.1}\\ \hline
\hyperref[UC4.1.5]{UC4.1.5} & \hyperlink{RFO3.1.5}{RFO3.1.5}\\ \hline
\hyperref[UC4.1.6]{UC4.1.6} & \hyperlink{RFO3.1.6}{RFO3.1.6}\\
& \hyperlink{RFO3.1.6.1}{RFO3.1.6.1}\\
& \hyperlink{RFO3.1.6.2}{RFO3.1.6.2}\\
& \hyperlink{RFO3.1.6.3}{RFO3.1.6.3}\\
& \hyperlink{RFO3.1.6.4}{RFO3.1.6.4}\\
& \hyperlink{RFO3.1.6.5}{RFO3.1.6.5}\\ \hline
\hyperref[UC4.1.6.1]{UC4.1.6.1} & \hyperlink{RFO3.1.6.1}{RFO3.1.6.1}\\ \hline
\hyperref[UC4.1.6.2]{UC4.1.6.2} & \hyperlink{RFO3.1.6.2}{RFO3.1.6.2}\\ \hline
\hyperref[UC4.1.6.3]{UC4.1.6.3} & \hyperlink{RFO3.1.6.3}{RFO3.1.6.3}\\
& \hyperlink{RFO3.1.6.3.1}{RFO3.1.6.3.1}\\
& \hyperlink{RFO3.1.6.3.2}{RFO3.1.6.3.2}\\
& \hyperlink{RFO3.1.6.3.3}{RFO3.1.6.3.3}\\ \hline
\hyperref[UC4.1.6.3.1]{UC4.1.6.3.1} & \hyperlink{RFO3.1.6.3.1}{RFO3.1.6.3.1}\\ \hline
\hyperref[UC4.1.6.3.2]{UC4.1.6.3.2} & \hyperlink{RFO3.1.6.3.2}{RFO3.1.6.3.2}\\ \hline
\hyperref[UC4.1.6.3.3]{UC4.1.6.3.3} & \hyperlink{RFO3.1.6.3.3}{RFO3.1.6.3.3}\\
& \hyperlink{RFO3.1.6.3.3.1}{RFO3.1.6.3.3.1}\\ \hline
\hyperref[UC4.1.6.3.3.1]{UC4.1.6.3.3.1} & \hyperlink{RFO3.1.6.3.3.1}{RFO3.1.6.3.3.1}\\ \hline
\hyperref[UC4.1.6.4]{UC4.1.6.4} & \hyperlink{RFO3.1.6.4}{RFO3.1.6.4}\\ \hline
\hyperref[UC4.1.6.5]{UC4.1.6.5} & \hyperlink{RFO3.1.6.5}{RFO3.1.6.5}\\
& \hyperlink{RFO3.1.6.5.1}{RFO3.1.6.5.1}\\ \hline
\hyperref[UC4.1.6.5.1]{UC4.1.6.5.1} & \hyperlink{RFO3.1.6.5.1}{RFO3.1.6.5.1}\\ \hline
\hyperref[UC4.1.7]{UC4.1.7} & \hyperlink{RFO3.1.7}{RFO3.1.7}\\ \hline
\hyperref[UC4.1.8]{UC4.1.8} & \hyperlink{RFO3.1.8}{RFO3.1.8}\\
& \hyperlink{RFO3.1.8.1}{RFO3.1.8.1}\\ \hline
\hyperref[UC4.2]{UC4.2} & \hyperlink{RFO3.2}{RFO3.2}\\
& \hyperlink{RFO3.2.1}{RFO3.2.1}\\
& \hyperlink{RFO3.2.2}{RFO3.2.2}\\
& \hyperlink{RFO3.2.3}{RFO3.2.3}\\
& \hyperlink{RFO3.2.4}{RFO3.2.4}\\
& \hyperlink{RFO3.2.5}{RFO3.2.5}\\
& \hyperlink{RFO3.2.6}{RFO3.2.6}\\ \hline
\hyperref[UC4.2.1]{UC4.2.1} & \hyperlink{RFO3.2.1}{RFO3.2.1}\\ \hline
\hyperref[UC4.2.2]{UC4.2.2} & \hyperlink{RFO3.2.2}{RFO3.2.2}\\ \hline
\hyperref[UC4.2.3]{UC4.2.3} & \hyperlink{RFO3.2.3}{RFO3.2.3}\\ \hline
\hyperref[UC4.2.4]{UC4.2.4} & \hyperlink{RFO3.2.4}{RFO3.2.4}\\ \hline
\hyperref[UC4.2.5]{UC4.2.5} & \hyperlink{RFO3.2.5}{RFO3.2.5}\\ \hline
\hyperref[UC4.2.6]{UC4.2.6} & \hyperlink{RFO3.2.6}{RFO3.2.6}\\
& \hyperlink{RFO3.2.6.1}{RFO3.2.6.1}\\ \hline
\hyperref[UC4.2.6.1]{UC4.2.6.1} & \hyperlink{RFO3.2.6.1}{RFO3.2.6.1}\\ \hline
\hyperref[UC4.3]{UC4.3} & \hyperlink{RFO3.3}{RFO3.3}\\
& \hyperlink{RFO3.3.1}{RFO3.3.1}\\
& \hyperlink{RFO3.3.2}{RFO3.3.2}\\
& \hyperlink{RFO3.3.3}{RFO3.3.3}\\ \hline
\hyperref[UC4.3.1]{UC4.3.1} & \hyperlink{RFO3.3.1}{RFO3.3.1}\\ \hline
\hyperref[UC4.3.2]{UC4.3.2} & \hyperlink{RFO3.3.2}{RFO3.3.2}\\ \hline
\hyperref[UC4.3.3]{UC4.3.3} & \hyperlink{RFO3.3.3}{RFO3.3.3}\\
& \hyperlink{RFO3.3.3.1}{RFO3.3.3.1}\\ \hline
\hyperref[UC4.3.3.1]{UC4.3.3.1} & \hyperlink{RFO3.3.3.1}{RFO3.3.3.1}\\ \hline
\hyperref[UC4.4]{UC4.4} & \hyperlink{RFO3.4}{RFO3.4}\\ \hline
\hyperref[UC4.5]{UC4.5} & \hyperlink{RFO3.5}{RFO3.5}\\ \hline
\hyperref[UC4.6]{UC4.6} & \hyperlink{RFO3.2.2.1}{RFO3.2.2.1}\\ \hline
\hyperref[UC5]{UC5} & \hyperlink{RFO4}{RFO4}\\
& \hyperlink{RFO4.1}{RFO4.1}\\
& \hyperlink{RFO4.2}{RFO4.2}\\
& \hyperlink{RFO4.3}{RFO4.3}\\
& \hyperlink{RFO4.4}{RFO4.4}\\
& \hyperlink{RFO4.5}{RFO4.5}\\
& \hyperlink{RFO4.6}{RFO4.6}\\
& \hyperlink{RFO4.7}{RFO4.7}\\
& \hyperlink{RFO4.8}{RFO4.8}\\
& \hyperlink{RFO4.9}{RFO4.9}\\
& \hyperlink{RFO4.10}{RFO4.10}\\
& \hyperlink{RFO4.11}{RFO4.11}\\
& \hyperlink{RFO4.12}{RFO4.12}\\
& \hyperlink{RFD4.13}{RFD4.13}\\
& \hyperlink{RFD4.14}{RFD4.14}\\
& \hyperlink{RFD4.15}{RFD4.15}\\
& \hyperlink{RFD4.16}{RFD4.16}\\
& \hyperlink{RFD4.17}{RFD4.17}\\
& \hyperlink{RFD4.18}{RFD4.18}\\ \hline
\hyperref[UC5.1]{UC5.1} & \hyperlink{RFO4.1}{RFO4.1}\\
& \hyperlink{RFO4.1.1}{RFO4.1.1}\\
& \hyperlink{RFO4.1.2}{RFO4.1.2}\\
& \hyperlink{RFO4.1.3}{RFO4.1.3}\\
& \hyperlink{RFO4.1.4}{RFO4.1.4}\\ \hline
\hyperref[UC5.1.1]{UC5.1.1} & \hyperlink{RFO4.1.1}{RFO4.1.1}\\ \hline
\hyperref[UC5.1.2]{UC5.1.2} & \hyperlink{RFO4.1.2}{RFO4.1.2}\\ \hline
\hyperref[UC5.1.3]{UC5.1.3} & \hyperlink{RFO4.1.3}{RFO4.1.3}\\ \hline
\hyperref[UC5.1.4]{UC5.1.4} & \hyperlink{RFO4.1.4}{RFO4.1.4}\\
& \hyperlink{RFO4.1.4.1}{RFO4.1.4.1}\\ \hline
\hyperref[UC5.1.4.1]{UC5.1.4.1} & \hyperlink{RFO4.1.4.1}{RFO4.1.4.1}\\ \hline
\hyperref[UC5.2]{UC5.2} & \hyperlink{RFO4.2}{RFO4.2}\\
& \hyperlink{RFO4.2.1}{RFO4.2.1}\\
& \hyperlink{RFO4.2.2}{RFO4.2.2}\\
& \hyperlink{RFO4.2.3}{RFO4.2.3}\\ \hline
\hyperref[UC5.2.1]{UC5.2.1} & \hyperlink{RFO4.2.1}{RFO4.2.1}\\ \hline
\hyperref[UC5.2.2]{UC5.2.2} & \hyperlink{RFO4.2.2}{RFO4.2.2}\\ \hline
\hyperref[UC5.2.3]{UC5.2.3} & \hyperlink{RFO4.2.3}{RFO4.2.3}\\
& \hyperlink{RFO4.2.3.1}{RFO4.2.3.1}\\ \hline
\hyperref[UC5.2.3.1]{UC5.2.3.1} & \hyperlink{RFO4.2.3.1}{RFO4.2.3.1}\\ \hline
\hyperref[UC5.3]{UC5.3} & \hyperlink{RFO4.3}{RFO4.3}\\
& \hyperlink{RFO4.3.1}{RFO4.3.1}\\
& \hyperlink{RFO4.3.2}{RFO4.3.2}\\
& \hyperlink{RFO4.3.3}{RFO4.3.3}\\
& \hyperlink{RFO4.3.4}{RFO4.3.4}\\
& \hyperlink{RFO4.3.5}{RFO4.3.5}\\ \hline
\hyperref[UC5.3.1]{UC5.3.1} & \hyperlink{RFO4.3.1}{RFO4.3.1}\\ \hline
\hyperref[UC5.3.2]{UC5.3.2} & \hyperlink{RFO4.3.2}{RFO4.3.2}\\ \hline
\hyperref[UC5.3.3]{UC5.3.3} & \hyperlink{RFO4.3.3}{RFO4.3.3}\\ \hline
\hyperref[UC5.3.4]{UC5.3.4} & \hyperlink{RFO4.3.4}{RFO4.3.4}\\ \hline
\hyperref[UC5.3.5]{UC5.3.5} & \hyperlink{RFO4.3.5}{RFO4.3.5}\\
& \hyperlink{RFO4.3.5.1}{RFO4.3.5.1}\\ \hline
\hyperref[UC5.3.5.1]{UC5.3.5.1} & \hyperlink{RFO4.3.5.1}{RFO4.3.5.1}\\ \hline
\hyperref[UC5.4]{UC5.4} & \hyperlink{RFO4.4}{RFO4.4}\\
& \hyperlink{RFO4.4.1}{RFO4.4.1}\\
& \hyperlink{RFO4.4.2}{RFO4.4.2}\\
& \hyperlink{RFO4.4.3}{RFO4.4.3}\\
& \hyperlink{RFO4.4.4}{RFO4.4.4}\\ \hline
\hyperref[UC5.4.1]{UC5.4.1} & \hyperlink{RFO4.4.1}{RFO4.4.1}\\ \hline
\hyperref[UC5.4.2]{UC5.4.2} & \hyperlink{RFO4.4.2}{RFO4.4.2}\\ \hline
\hyperref[UC5.4.3]{UC5.4.3} & \hyperlink{RFO4.4.3}{RFO4.4.3}\\ \hline
\hyperref[UC5.4.4]{UC5.4.4} & \hyperlink{RFO4.4.4}{RFO4.4.4}\\
& \hyperlink{RFO4.4.4.1}{RFO4.4.4.1}\\ \hline
\hyperref[UC5.4.4.1]{UC5.4.4.1} & \hyperlink{RFO4.4.4.1}{RFO4.4.4.1}\\ \hline
\hyperref[UC5.5]{UC5.5} & \hyperlink{RFO4.5}{RFO4.5}\\
& \hyperlink{RFO4.5.1}{RFO4.5.1}\\
& \hyperlink{RFO4.5.2}{RFO4.5.2}\\
& \hyperlink{RFO4.5.3}{RFO4.5.3}\\
& \hyperlink{RFO4.5.4}{RFO4.5.4}\\
& \hyperlink{RFO4.5.5}{RFO4.5.5}\\
& \hyperlink{RFO4.5.6}{RFO4.5.6}\\ \hline
\hyperref[UC5.5.1]{UC5.5.1} & \hyperlink{RFO4.5.1}{RFO4.5.1}\\ \hline
\hyperref[UC5.5.2]{UC5.5.2} & \hyperlink{RFO4.5.2}{RFO4.5.2}\\ \hline
\hyperref[UC5.5.3]{UC5.5.3} & \hyperlink{RFO4.5.3}{RFO4.5.3}\\ \hline
\hyperref[UC5.5.4]{UC5.5.4} & \hyperlink{RFO4.5.4}{RFO4.5.4}\\ \hline
\hyperref[UC5.5.5]{UC5.5.5} & \hyperlink{RFO4.5.5}{RFO4.5.5}\\ \hline
\hyperref[UC5.5.6]{UC5.5.6} & \hyperlink{RFO4.5.6}{RFO4.5.6}\\
& \hyperlink{RFO4.5.6.1}{RFO4.5.6.1}\\ \hline
\hyperref[UC5.5.6.1]{UC5.5.6.1} & \hyperlink{RFO4.5.6.1}{RFO4.5.6.1}\\ \hline
\hyperref[UC5.6]{UC5.6} & \hyperlink{RFO4.6}{RFO4.6}\\
& \hyperlink{RFO4.6.1}{RFO4.6.1}\\
& \hyperlink{RFO4.6.2}{RFO4.6.2}\\
& \hyperlink{RFO4.6.3}{RFO4.6.3}\\ \hline
\hyperref[UC5.6.1]{UC5.6.1} & \hyperlink{RFO4.6.1}{RFO4.6.1}\\ \hline
\hyperref[UC5.6.2]{UC5.6.2} & \hyperlink{RFO4.6.2}{RFO4.6.2}\\ \hline
\hyperref[UC5.6.3]{UC5.6.3} & \hyperlink{RFO4.6.3}{RFO4.6.3}\\ \hline
\hyperref[UC5.7]{UC5.7} & \hyperlink{RFO4.7}{RFO4.7}\\ \hline
\hyperref[UC5.8]{UC5.8} & \hyperlink{RFO4.8}{RFO4.8}\\ \hline
\hyperref[UC5.9]{UC5.9} & \hyperlink{RFO4.9}{RFO4.9}\\ \hline
\hyperref[UC5.10]{UC5.10} & \hyperlink{RFO4.10}{RFO4.10}\\ \hline
\hyperref[UC5.11]{UC5.11} & \hyperlink{RFO4.11}{RFO4.11}\\ \hline
\hyperref[UC5.12]{UC5.12} & \hyperlink{RFO4.12}{RFO4.12}\\ \hline
\hyperref[UC5.13]{UC5.13} & \hyperlink{RFD4.13}{RFD4.13}\\ \hline
\hyperref[UC5.14]{UC5.14} & \hyperlink{RFD4.14}{RFD4.14}\\ \hline
\hyperref[UC5.15]{UC5.15} & \hyperlink{RFD4.15}{RFD4.15}\\ \hline
\hyperref[UC5.16]{UC5.16} & \hyperlink{RFD4.16}{RFD4.16}\\ \hline
\hyperref[UC5.17]{UC5.17} & \hyperlink{RFD4.17}{RFD4.17}\\ \hline
\hyperref[UC5.18]{UC5.18} & \hyperlink{RFD4.18}{RFD4.18}\\ \hline
\hyperref[UC6]{UC6} & \hyperlink{RFO5}{RFO5}\\ \hline
\hyperref[UC7]{UC7} & \hyperlink{RFO6}{RFO6}\\ \hline
\caption[Tracciamento Fonti-Requisiti]{Tracciamento Fonti-Requisiti}
\label{tabella:fonti-requi}
\end{longtable}

\clearpage

\end{document}


