\newglossaryentry{amministratoreDiProgetto}
{
name={amministratore di progetto},
description={chi controlla che ad ogni istante della vita del progetto le risorse (umane, materiali, economiche e strutturali) siano presenti e operanti; inoltre, gestisce la documentazione e controlla il versionamento e la configurazione}
}

\newglossaryentry{analista}
{
name={analista},
description={chi ha il compito di individuare, a partire dai bisogni del cliente, il problema da fornire ai progettisti.}
}

\newglossaryentry{asana}
{
name={Asana},
description={un servizio utilizzato per coordinare le attività di project management.
}
}

\newglossaryentry{backup}
{
name={Backup},
description={indica la replicazione di materiale informativo archiviato nella memoria di massa dei computer, al fine di prevenire la perdita definitiva dei dati in caso di eventi malevoli accidentali o intenzionali.}
}

\newglossaryentry{baseline}
{
name={baseline},
description={nel ciclo di vita di un progetto, punto d'arrivo tecnico dal quale non si retrocede.}
}

\newglossaryentry{bassoAccoppiamento}
{
name={basso accoppiamento},
description={minimizzazione delle dipendenze tra i vari componenti di un sistema.}
}

\newglossaryentry{branch}
{
name={branch},
description={insieme di versioni di file sorgente in evoluzione.}
}

\newglossaryentry{casoDiUso}
{
name={caso d'uso},
description={insieme di scenari che hanno in comune un obiettivo per un utente.}
}

\newglossaryentry{cicloDiVita}
{
name={ciclo di vita (di un prodotto)},
description={insieme degli stati che il prodotto assume, dal concepimento al ritiro.}
}

\newglossaryentry{configurazione}
{
name={configurazione},
description={di quali parti si compone un prodotto e il modo in cui esse stanno assieme.}
}

\newglossaryentry{controlloreDellaQualita}
{
name={controllore della qualità},
description={funzione aziendale (e non ruolo di progetto) che accerta la qualità dei prodotti.}
}

\newglossaryentry{csstre}
{
name={CSS3},
description={un linguaggio web utilizzato per descrivere l'aspetto e la formattazione di un sito web al browser lato client.}
}

\newglossaryentry{diagrammaCasiDiUso}
{
name={diagramma dei casi d'uso},
description={grafo orientato in cui ogni nodo rappresenta un attore o un caso d'uso; ogni arco può essere una comunicazione tra un attore e un caso d'uso oppure una relazione (di estensione, inclusione o generalizzazione) tra due casi d'uso o tra due attori.}
}

\newglossaryentry{diagrammaGantt}
{
name={Diagramma di Gantt},
description={usato principalmente nelle attività di amministrazione di progetto, permette la rappresentazione grafica di un calendario di attività, utile al fine di pianificare, coordinare e tracciare specifiche attività in un progetto dando una chiara illustrazione dello stato d’avanzamento del progetto rappresentato.}
}

\newglossaryentry{diagrammaNassiShneiderman}
{
name={Diagramma Nassi–Shneiderman},
description={un disegno grafico per la rappresentazione della programmazione strutturata a blocchi. }
}

\newglossaryentry{diagrammaWBS}
{
name={diagramma WBS},
description={diagramma che decompone in modo gerarchico le attività di un progetto in sotto-attività (coese ma non necessariamente sequenziali): Work Breakdown Structure.}
}

\newglossaryentry{documentazione}
{
name={documentazione},
description={tutto ciò che documenta le attività del progetto.}
}

\newglossaryentry{efficacia}
{
name={efficacia},
description={conformità alle attese.}
}

\newglossaryentry{efficienza}
{
name={efficienza},
description={contenimento dei costi per raggiungere un obiettivo.}
}

\newglossaryentry{github}
{
name={GitHub},
description={un servizio web di hosting per lo sviluppo di progetti software, che usa il sistema di controllo di versione Git. Può essere utilizzato anche per la condivisione e la modifica di file di testo e documenti revisionabili.}
}

\newglossaryentry{glossario}
{
name={glossario},
description={elenco dei significati dei termini più rilevanti di un documento o di un insieme di documenti.}
}

\newglossaryentry{htmlcinque}
{
name={HTML5},
description={un linguaggio di markup di pubblico dominio per la strutturazione delle pagine web.}
}

\newglossaryentry{indiceAnalitico}
{
name={indice analitico},
description={elenco ordinato delle corrispondenze tra particolari termini importanti di un documento e la loro ubicazione in esso.}
}

\newglossaryentry{indiceGenerale}
{
name={indice generale},
description={elenco (gerarchico) delle parti in cui è strutturato un documento.}
}

\newglossaryentry{infrastruttura}
{
name={infrastruttura},
description={tutte le risorse hardware e software (di un progetto).}
}

\newglossaryentry{instagantt}
{
name={Instagantt},
description={applicazione web per la creazione dei diagrammi di Gantt.}
}

\newglossaryentry{java}
{
name={Java},
description={Java è un linguaggio di programmazione orientato agli oggetti a tipizzazione statica e una piattaforma di elaborazione sviluppati da Sun Microsystems nel 1995.}
}

\newglossaryentry{javadoc}
{
name={Javadoc},
description={un programma di utilità incluso nel SDK di Java con il quale è possibile generare in modo automatico la documentazione in formato HTML dei sorgenti di un programma Java, a patto che vengano utilizzate delle regole ben precise nella scrittura dei commenti stessi.}
}

\newglossaryentry{javascript}
{
name={JavaScript},
description={ è un linguaggio di scripting orientato agli oggetti e agli eventi, comunemente utilizzato nella programmazione Web lato client per la creazione, in siti web e applicazioni web, di effetti dinamici interattivi tramite funzioni di script invocate da eventi innescati a loro volta in vari modi dall’utente sulla pagina web in uso. }
}

\newglossaryentry{latex}
{
name={LaTEX},
description={un linguaggio di markup utilizzato per la produzione di documentazione tecnica e scientifica; è lo standard de facto per la comunicazione e la pubblicazione di documenti scientifici ed è disponibile come software libero.}
}

\newglossaryentry{latoclient}
{
name={Lato client},
description={ termine che, nell’ambito delle reti di calcolatori, indica le operazioni di elaborazione effettuate da un client in un’architettura client-server. }
}

\newglossaryentry{milestone}
{
name={milestone},
description={punto nel tempo associato ad un valore strategico.}
}

\newglossaryentry{mysql}
{
name={mySQL},
description={un software per la gestione di un database relazionale.}
}

\newglossaryentry{php}
{
name={PHP},
description={un linguaggio di scripting open source molto utilizzato, specialmente per lo sviluppo web.}
}

\newglossaryentry{pianificazione}
{
name={pianificazione},
description={organizzare e controllare tempo, risorse e risultati.}
}

\newglossaryentry{pragmadb}
{
name={PragmaDB},
description={un software per la gestione di documenti creati durante il ciclo di vita di un progetto software.}
}

\newglossaryentry{progettazione}
{
name={progettazione},
description={processo di definizione dell'architettura, dei componenti, delle interfacce e delle altre caratteristiche di un sistema o componente.}
}

\newglossaryentry{progettista}
{
name={progettista},
description={chi sintetizza una soluzione a partire dalle specifiche di un problema già analizzato.}
}

\newglossaryentry{progetto}
{
name={progetto},
description={insieme di compiti da svolgere in modo collaborativo a fronte di un incarico (che diventa poi un impegno)}
}

\newglossaryentry{programmatore}
{
name={programmatore},
description={chi implementa una parte della soluzione dei progettisti.}
}

\newglossaryentry{qualifica}
{
name={qualifica},
description={verifica e validazione.}
}

\newglossaryentry{requisito}
{
name={requisito},
description={bisogno da soddisfare o vincolo da rispettare.}
}

\newglossaryentry{requisitoDiProcesso}
{
name={requisito di processo},
description={vincolo sullo sviluppo del prodotto.}
}

\newglossaryentry{requisitoDiProdotto}
{
name={requisito di prodotto},
description={bisogno o vincolo sul prodotto da sviluppare.}
}

\newglossaryentry{requisitoDiSistema}
{
name={requisito di sistema},
description={definizione formale e dettagliata di una funzione del sistema.}
}

\newglossaryentry{requisitoFunzionale}
{
name={requisito funzionale},
description={servizio che un prodotto deve fornire.}
}

\newglossaryentry{requisitoNonFunzionale}
{
name={requisito non funzionale},
description={vincolo su uno o più servizi che un prodotto fornisce.}
}

\newglossaryentry{responsabileDiProgetto}
{
name={responsabile di progetto},
description={chi pianifica il progetto, assegna le persone ai ruoli giusti e rappresenta il progetto presso il fornitore e il committente}
}

\newglossaryentry{ruolo}
{
name={ruolo},
description={funzione aziendale assegnata a progetto; identifica capacità e compiti}
}

\newglossaryentry{scenario}
{
name={scenario},
description={sequenza di passi che descrive un esempio di interazione con un sistema.}
}

\newglossaryentry{schemaPDCA}
{
name={schema PDCA},
description={schema di auto-miglioramento che consiste di quattro punti: Plan (individuare obiettivi di miglioramento), Do (eseguire ciò che si è pianificato), Check (verificare se ha funzionato) e Act (agire per correggersi)}
}

\newglossaryentry{server}
{
name={Server},
description={indica una componente, hardware e/o software, che gestisce il traffico di informazioni e fornisce servizi o risorse al client che ne fa richiesta.}
}

\newglossaryentry{sistema}
{
name={sistema},
description={insieme di componenti organizzati per compiere una o più funzioni.}
}

\newglossaryentry{slack}
{
name={Slack},
description={una piattaforma per la comunicazione all'interno del team.}
}

\newglossaryentry{sommario}
{
name={sommario},
description={breve riassunto del contenuto di un documento.}
}

\newglossaryentry{sonarqube}
{
name={SonarQube},
description={una piattaforma open source per il controllo della qualità del codice.}
}

\newglossaryentry{sql}
{
name={SQL},
description={Structured Query Language: linguaggio di programmazione dichiarativo basato sull'algebra relazionale che serve a creare, manipolare e interrogare basi di dati relazionali.}
}

\newglossaryentry{team}
{
name={Team},
description={gruppo di persone che collabora nello svolgimento di un’attività.}
}

\newglossaryentry{travis}
{
name={Travis},
description={un servizio di integrazione utilizzato per costruire e testare progetti software ospitati presso GitHub.}
}

\newglossaryentry{trello}
{
name={Trello},
description={un servizio utilizzato per coordinare le attività di project management.}
}

\newglossaryentry{uml}
{
name={UML},
description={Unified Modelling Language: famiglia di notazioni grafiche che si basano su un singolo meta-modello e servono a descrivere e progettare sistemi software.}
}

\newglossaryentry{verificatore}
{
name={verificatore},
description={chi verifica il lavoro dei programmatori.}
}

\newglossaryentry{versione}
{
name={versione},
description={istanza identificata (di una parte della configurazione) nel tempo.}
}