%%%%%%%%%%%%%%%%%%%%%%%%%
%%  Tecnologie utilizzate
%%%%%%%%%%%%%%%%%%%%%%%%%



\subsection{Client}
Per implementare il client di \proj{} abbiamo scelto di utilizzare HTML5 e CSS3 assieme a JavaScript. Di seguito elenchiamo e descriviamo le librerie JavaScript di cui il nostro client necessita. Tutte le librerie sono open-source, come richiesto dal capitolato.

\subsubsection{JointJS}
JointJS (\url{https://www.jointjs.com/opensource}) è una libreria open-source per visualizzare e manipolare grafi (e quindi diagrammi). Dipende da HTML5 e da alcune altre librerie JavaScript ma offre numerosi plug-in per realizzare particolari tipi di grafi.

JointJS dipende da:
\begin{itemize}
	\item HTML5: JointJS richiede che una pagina HTML5 sia popolata con un tag \texttt{<canvas>}, che conterrà un grafo.
	\item Le seguenti librerie JavaScript:
	\begin{itemize}
		\item JQuery;
		\item Lodash;
		\item Backbone.js.
	\end{itemize}
\end{itemize}

il client di \proj{} utilizza un plug-in di JointJS che facilita la creazione di una particolare categoria di grafi: i diagrammi UML. [\dots]

\subsubsection{Backbone.js}
% ...

\subsubsection{JQuery}
% ...

%%% [altro...]



\subsection{Server}

\subsubsection{Apache Tomcat}
% ...

\subsubsection{Pivotal Spring}
% ...

%%% [altro...]



\subsection{Linguaggio target}

%%% [...]
