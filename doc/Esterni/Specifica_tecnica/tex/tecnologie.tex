%%%%%%%%%%%%%%%%%%%%%%%%%
%%  Tecnologie utilizzate
%%%%%%%%%%%%%%%%%%%%%%%%%



\subsection{\gloss{Client}}
Implementeremo il client di \proj{} con le tecnologie web richieste dal capitolato d'appalto: \gloss{HTML5}, \gloss{CSS3} e \gloss{JavaScript}. Di seguito elenchiamo e descriviamo le librerie JavaScript di cui il nostro client necessita. Tutte le librerie sono open-source, come richiesto dal capitolato.

\subsubsection{JointJS}
La libreria \jointjs{} (\url{https://www.jointjs.com/opensource}) è una libreria open-source per realizzare \gloss{editor} di diagrammi interattivi, in maniera altamente personalizzabile. %Dipende da HTML5 e da alcune altre librerie JavaScript ma offre numerosi plug-in per realizzare particolari tipi di grafi.

La libreria sfrutta le seguenti tecnologie per il suo funzionamento:
\begin{itemize}
	\item \html{}: \jointjs{} richiede che una pagina \html{} sia popolata con un tag \texttt{<canvas>}, che conterrà un diagramma.
	\item Le seguenti librerie \js{}:
	\begin{itemize}
		\item \jquery{};
		\item \lodash{};
		\item \backbonejs{}.
	\end{itemize}
\end{itemize}

\subsubsection{\backbonejs}
La libreria \backbonejs{} permette di strutturare applicazioni web single page (SPA) fornendo \textbf{modelli} con binding di chiave-valore, eventi, \textbf{collezioni} e \textbf{viste} con una gestione degli eventi dichiarativa. Essa offre inoltre una interfaccia RESTful.

A causa della struttura data alla libreria \jointjs{}, sviluppata tramite MVC (collection), al fine di ridurre il numero di librerie necessarie allo sviluppo del progetto, si costruirà \proj{} estendendo le funzionalità di base offerte da \jointjs{} usando il modello \mvc{}. % il modello \mvc{} offerto da \backbonejs{}. % ?


\subsubsection{\jquery}
La libreria \jquery{} è sfruttata da \jointjs{} ed è necesaria per semplificare varie operazioni di basso livello.


\subsubsection{\requirejs}
La libreria \requirejs{} è un loader di moduli e file \js{}. Esso si occuperà principalmente di risolvere dipendenze delle librerie \js{} utilizzate.

\subsubsection{\lodash}
La libreria \lodash{} fornisce metodi di utilità non offerti da \js{} puro. Rispetto alla simile libreria \emph{Underscore}, questa fornisce più performance, più features e miglior documentazione. Essa inoltre è usata da \jointjs{}.


\subsubsection{qunit}
assertions, test di regressione
\subsubsection{sinonjs}
stubs, mocks, (test spies??)




\subsection{Server}


\subsubsection{Apache Tomcat}
Il \emph{back end} della nostra applicazione è ospitato su un server Apache Tomcat, come richiesto dal capitolato d'appalto nel caso il \emph{back end} fosse scritto in Java.

\subsubsection{Pivotal Spring}
Le richieste HTTP inviate dal client al server vengono gestite tramite la libreria open-source Spring, utile per aderire ai princìpi dello stile REST.

\subsubsection{StringTemplate}
StringTemplate è una libreria open-source scritta da Terence Parr e disponibile all'indirizzo \url{stringtemplate.org}. Il server la utilizza per popolare dei template (scritti nel linguaggio di mark-up definito da StringTemplate) con i dati ricevuti dal client.

\subsubsection{Google Gson}
Google Gson è una libreria 

%%% [altro...]



\subsection{Linguaggio target}

%%% [...]
