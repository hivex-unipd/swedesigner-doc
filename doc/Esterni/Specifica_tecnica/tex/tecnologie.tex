%%%%%%%%%%%%%%%%%%%%%%%%%
%%  Tecnologie utilizzate
%%%%%%%%%%%%%%%%%%%%%%%%%



\subsection{Client} \label{sec:tech_client}
Implementeremo il \gloss{client} di \proj{} con le tecnologie web richieste dal capitolato d'appalto: HTML5, CSS3 e JavaScript. Di seguito elenchiamo e descriviamo le librerie JavaScript di cui il nostro client necessita. Tutte le librerie sono open-source, come richiesto dal capitolato.

\subsubsection{JointJS}
La libreria \jointjs{} (\url{www.jointjs.com/opensource}) è una libreria open-source per realizzare editor di diagrammi interattivi, in maniera altamente personalizzabile.

La libreria sfrutta le seguenti tecnologie per il suo funzionamento:
\begin{itemize}
	\item \html: \jointjs{} richiede che una pagina \html{} sia popolata con un tag \texttt{<svg>}, che conterrà un diagramma.
	\item Le seguenti librerie \js:
	\begin{itemize}
		\item \jquery;
		\item \lodash;
		\item \backbonejs.
	\end{itemize}
\end{itemize}

Di \jointjs{} sfruttiamo un \emph{plug-in} per realizzare diagrammi UML. Tale \emph{plug-in} offre delle forme geometriche e delle frecce che ci servono per il disegno del diagramma delle classi e dei diagrammi a blocchi di ogni metodo.

\subsubsection{\backbonejs}
La libreria \backbonejs{} permette di strutturare applicazioni web single page (SPA) fornendo \textbf{modelli} con binding di chiave-valore, eventi, \textbf{collezioni} e \textbf{viste} con una gestione degli eventi dichiarativa. Essa offre inoltre una interfaccia RESTful.

A causa della struttura data alla libreria \jointjs{}, sviluppata tramite MVC (collection), al fine di ridurre il numero di librerie necessarie allo sviluppo del progetto, si costruirà il lato client di \proj{} estendendo le funzionalità di base offerte da \jointjs{} usando il modello \mvc{} offerto da \backbonejs{}.	

\subsubsection{\jquery}
La libreria \jquery{} è sfruttata da \jointjs{} ed è necesaria per semplificare varie operazioni di basso livello.

\subsubsection{\requirejs}
La libreria \requirejs{} è un loader di moduli e file \js{}. Esso si occuperà principalmente di risolvere dipendenze delle librerie \js{} utilizzate.

\subsubsection{\lodash}
La libreria \lodash{} fornisce metodi di utilità non offerti da \js{} puro. Rispetto alla simile libreria \emph{Underscore}, questa fornisce più performance, più features e miglior documentazione. Essa inoltre è usata da \jointjs{}.

% \subsubsection{qunit}
% assertions, test di regressione

% \subsubsection{sinonjs}
% stubs, mocks, (test spies??)



\subsection{Server}

\subsubsection{Apache Tomcat}
Il \emph{back end} della nostra applicazione è ospitato su un \gloss{server} Apache Tomcat, come richiesto dal capitolato d'appalto nel caso il \emph{back end} fosse scritto in Java.

\subsubsection{Pivotal Spring}
Le richieste HTTP inviate dal client al server vengono gestite tramite la libreria open-source \emph{Spring} (disponibile all'indirizzo \url{spring.io}), utile per aderire ai princìpi dello stile architetturale REST.

\subsubsection{StringTemplate}
StringTemplate è una libreria open-source scritta da Terence Parr e disponibile all'indirizzo \url{stringtemplate.org}. Il server della nostra applicazione la utilizza per:
\begin{enumerate}
	\item creare dei template di codice Java, inframmezzati da speciali marcatori definiti da StringTemplate;
	\item popolare tali template con i dati ricevuti dal client.
\end{enumerate}
I marcatori definiti dalla libreria servono da segnaposto per ospitare i dati ottenuti analizzando il documento JSON generato dal client.

\subsubsection{Google Gson}
Google Gson è una libreria open-source disponibile all'indirizzo \url{github.com/google/gson}. Essa permette di convertire un oggetto JSON in un oggetto Java che abbia, come campi dati, un sovrainsieme delle chiavi presenti nell'oggetto JSON; l'oggetto JSON dev'essere contenuto in una stringa Java. I valori delle chiavi dell'oggetto JSON diventano i valori dei campi dati dell'oggetto Java. La libreria permette anche il procedimento inverso (da Java a JSON), che però non usiamo nella nostra applicazione.
