%%%%%%%%%%%%%%%%%%%%%%%%%
%%  Tecnologie utilizzate
%%%%%%%%%%%%%%%%%%%%%%%%%



\subsection{\gloss{Client}}
Per implementare il client di \proj{} abbiamo scelto di utilizzare \gloss{HTML5} e \gloss{CSS3} assieme a \gloss{JavaScript}. Di seguito elenchiamo e descriviamo le librerie JavaScript di cui il nostro client necessita. Tutte le librerie sono open-source, come richiesto dal capitolato.

\subsubsection{JointJS}
La libreria \jointjs{} (\url{https://www.jointjs.com/opensource}) è una libreria open-source per realizzare \gloss{editor} di diagrammi interattivi, in maniera altamente personalizzabile. %Dipende da HTML5 e da alcune altre librerie JavaScript ma offre numerosi plug-in per realizzare particolari tipi di grafi.

La libreria sfrutta le seguenti tecnologie per il suo funzionamento:

\begin{itemize}
	\item \\gloss{html}{}: \jointjs{} richiede che una pagina \html{} sia popolata con un tag \texttt{<svg>}, che conterrà un diagramma.
	\item Le seguenti librerie \js{}:
	\begin{itemize}
		\item \jquery{};
		\item \lodash{};
		\item \backbonejs{}.
	\end{itemize}
\end{itemize}

\subsubsection{\backbonejs}
La libreria \backbonejs{} permette di strutturare applicazioni web single page (SPA) fornendo \textbf{modelli} con binding di chiave-valore, eventi, \textbf{collezioni} e \textbf{viste} con una gestione degli eventi dichiarativa. Essa offre inoltre una interfaccia RESTful.

A causa della struttura data alla libreria \jointjs{}, sviluppata tramite MVC (collection), al fine di ridurre il numero di librerie necessarie allo sviluppo del progetto, si costruirà \proj{} estendendo le funzionalità di base offerte da \jointjs{} usando il modello \mvc{}.


\subsubsection{\jquery}
La libreria \jquery{} è sfruttata da \jointjs{} ed è necesaria per semplificare varie operazioni di basso livello. 


\subsubsection{\requirejs}
La libreria \requirejs{} è un loader di moduli e file \js{}. Esso si occuperà principalmente di risolvere dipendenze delle librerie \js{} utilizzate.

\subsubsection{\lodash}
La libreria \lodash{} fornisce metodi di utilità non offerti da \js{} puro. Rispetto alla simile libreria \emph{Underscore}, questa fornisce più performance, più features e miglior documentazione. Essa inoltre è usata da \jointjs{}.


\subsubsection{qunit}
assertions, test di regressione
\subsubsection{sinonjs}
stubs, mocks, (test spies??)

\subsection{Server}


\subsubsection{Apache \gloss{Tomcat}}
% ...

\subsubsection{Pivotal Spring}
% ...

%%% [altro...]



\subsection{Linguaggio target}

%%% [...]
