\subsection{\nogloxy{SWEDesigner::Client}}
\label{\nogloxy{SWEDesigner::Client}}
\subsubsection{Informazioni generali}
\begin{itemize}
\item \textbf{Descrizione}\\
il package che descrive le componenti del front-end, presenti nel browser dell utente.
\item \textbf{Padre}: \hyperref[\nogloxy{SWEDesigner}]{\nogloxy{\texttt{SWEDesigner}}}
\item \textbf{Package contenuti}:
\begin{itemize}
\item \hyperref[\nogloxy{SWEDesigner::Client::Collection}]{\nogloxy{\texttt{Collection}}}\\
il package descrive le classi che contengono delle collection secondo la visione di \gloss{backbonejs}. Ulteriori considerazioni sulla visione del pattern MVC sono contenute nell'appendice A. %[INSERIRE RIFERIMENTO].
\item \hyperref[\nogloxy{SWEDesigner::Client::Model}]{\nogloxy{\texttt{Model}}}\\
questo package contiene la classi dei modelli di dati usati dal client per rappresentare l'informazione manipolata dall'utente.
\item \hyperref[\nogloxy{SWEDesigner::Client::View}]{\nogloxy{\texttt{View}}}\\
il package raccoglie le classi che descrivono le classi che interagiscono con il browser, che popolano template e si sottoscrivono alla view tramite un pattern Observer. I template non sono contenuti in questo package.
\end{itemize}
\end{itemize}

\subsection{\nogloxy{SWEDesigner::Client::Collection}}
\label{\nogloxy{SWEDesigner::Client::Collection}}
\subsubsection{Informazioni generali}
\begin{itemize}
\item \textbf{Descrizione}\\
il package descrive le classi che contengono delle collection secondo la visione di \gloss{backbonejs}. Ulteriori considerazioni sulla visione del pattern MVC sono contenute nell'appendice A. %[INSERIRE RIFERIMENTO].
\item \textbf{Padre}: \hyperref[\nogloxy{SWEDesigner::Client}]{\nogloxy{\texttt{Client}}}
\end{itemize}
\subsubsection{Classi}
\paragraph{\nogloxy{SWEDesigner::Client::Collection::DiagramCollection}}
\label{\nogloxy{SWEDesigner::Client::Collection::DiagramCollection}}
\begin{itemize}
\item \textbf{Descrizione}\\
questa classe collection possiede un certo numero di \texttt{Graph}, offerti dalla libreria \jointjs{}, che rappresentano degli elementi disegnabili. Una descrizione più dettagliata è disponibile in appendice. % [RIFERIMENTO].

\item \textbf{Utilizzo}\\
\texttt{ProjectModel} accede a questa classe per recuperare il diagramma da visualizzare di volta in volta. Inoltre \texttt{ProjectLoader} e \texttt{ProjectSaver} se ne servono per salvare e caricare il progetto su e da disco, tramite \texttt{ProjectModel}. I \texttt{Graph} al suo interno permettono di rimuovere facilmente le classi, blocchi, relazioni e commenti già esistenti.
\item \textbf{Relazioni con altre classi}:
\begin{itemize}
\item \textit{IN} \hyperref[\nogloxy{SWEDesigner::Client::Model::ProjectModel}]{\nogloxy{\texttt{ProjectModel}}}\\
questa classe ci permette di aggiungere della logica alla collezione di tutti i diagrammi che possediamo.
\end{itemize}
\end{itemize}
\subsection{\nogloxy{SWEDesigner::Client::Model}}
\label{\nogloxy{SWEDesigner::Client::Model}}
\subsubsection{Informazioni generali}
\begin{itemize}
\item \textbf{Descrizione}\\
questo package contiene la classi dei modelli di dati usati dal client per rappresentare l'informazione manipolata dall'utente.
\item \textbf{Padre}: \hyperref[\nogloxy{SWEDesigner::Client}]{\nogloxy{\texttt{Client}}}
\item \textbf{Package contenuti}:
\begin{itemize}
\item \hyperref[\nogloxy{SWEDesigner::Client::Model::CellTypes}]{\nogloxy{\texttt{CellTypes}}}\\
questo package contiene la definizione del modello degli elementi grafici dell'applicazione (e.g. diagrammi delle classi, blocchi condizionali). 
\item \hyperref[\nogloxy{SWEDesigner::Client::Model::Utility}]{\nogloxy{\texttt{Utility}}}\\
questo package racchiude le classi che definiscono i principali comandi dell'applicazione. Esse fanno parte di un unico pattern Command, usato dalla AppView principale.
\end{itemize}
\end{itemize}
\subsubsection{Classi}
\paragraph{\nogloxy{SWEDesigner::Client::Model::NewCellFactory}}
\label{\nogloxy{SWEDesigner::Client::Model::NewCellFactory}}
\begin{itemize}
\item \textbf{Descrizione}\\
questa classe si occupa di fornire un'istanza di una cella del tipo richiesto da \texttt{NewCellModel}. 
\item \textbf{Utilizzo}\\
viene utilizzata da \texttt{NewCellModel} che richiede una nuova cella (blocco o relazione) per avere un'istanza del blocco richiesto. Il design pattern descritto da questa classe è un Factory Pattern, come spiegato nel libro \emph{Learning Javascript Design Patterns} (\url{addyosmani.com/resources/essentialjsdesignpatterns/book/}).
\item \textbf{Relazioni con altre classi}:
\begin{itemize}
\item \textit{IN} \hyperref[\nogloxy{SWEDesigner::Client::Model::CellTypes::ActivityDiagramElement}]{\nogloxy{\texttt{ActivityDiagramElement}}}\\
questa classe è la base di tutte le classi che rappresentano i blocchi del diagramma delle attività.
\item \textit{IN} \hyperref[\nogloxy{SWEDesigner::Client::Model::NewCellModel}]{\nogloxy{\texttt{NewCellModel}}}\\
questa classe si occupa di fornire tutti i tipi di cell (tutti i blocchi e relazioni) da poter inserire nel diagramma corrente (o classi o attività).
\item \textit{OUT} \hyperref[\nogloxy{SWEDesigner::Client::Model::CellTypes::ClassDiagramElement}]{\nogloxy{\texttt{ClassDiagramElement}}}\\
questa classe è la base di tutte le classi che rappresentano gli elementi del diagramma delle classi.
\end{itemize}
\end{itemize}

\paragraph{\nogloxy{SWEDesigner::Client::Model::NewCellModel}}
\label{\nogloxy{SWEDesigner::Client::Model::NewCellModel}}
\begin{itemize}
\item \textbf{Descrizione}\\
questa classe si occupa di fornire tutti i tipi di cell (tutti i blocchi e relazioni) da poter inserire nel diagramma corrente (o classi o attività).
\item \textbf{Utilizzo}\\
utilizza \texttt{NewCellFactory} per recuperare una cell del tipo richiesto da \texttt{NewCellView}. Quest'ultima utilizza \texttt{NewCellModel} come modello da dove recuperare tutti i blocchi/relazioni che l'utente può inserire nel diagramma corrente.
\item \textbf{Relazioni con altre classi}:
\begin{itemize}
\item \textit{IN} \hyperref[\nogloxy{SWEDesigner::Client::View::NewCellView}]{\nogloxy{\texttt{NewCellView}}}\\
questa classe si occupa di visualizzare tutti i possibili blocchi e relazioni che si possono inserire nel diagramma delle classi o delle attività.
\item \textit{OUT} \hyperref[\nogloxy{SWEDesigner::Client::Model::NewCellFactory}]{\nogloxy{\texttt{NewCellFactory}}}\\
questa classe si occupa di fornire un'istanza di una cella del tipo richiesto da \texttt{NewCellModel}. 
\end{itemize}
\end{itemize}

\paragraph{\nogloxy{SWEDesigner::Client::Model::ProjectCommand}}
\label{\nogloxy{SWEDesigner::Client::Model::ProjectCommand}}
\begin{itemize}
\item \textbf{Descrizione}\\
questa interfaccia descrive la struttura di un comando che viene chiamato da \texttt{AppView} quando l'utente decide di creare un nuovo progetto, di caricarne uno esistente, di salvarlo o di generare il codice dal diagramma. Il pattern realizzato è il pattern Command.
\item \textbf{Utilizzo}\\
viene utilizzata da \texttt{AppView}, la quale poi chiederà comandi concreti in base all'input richiesto dall'utente.
\item \textbf{Relazioni con altre classi}:
\begin{itemize}
\item \textit{IN} \hyperref[\nogloxy{SWEDesigner::Client::Model::Utility::ProjectGenerator}]{\nogloxy{\texttt{ProjectGenerator}}}\\
questa classe si occupa di fornire il comando necessario ad inviare al server il progetto e di ricevere il codice del progetto quando l'utente lo richiede.
\item \textit{IN} \hyperref[\nogloxy{SWEDesigner::Client::Model::Utility::ProjectInitializer}]{\nogloxy{\texttt{ProjectInitializer}}}\\
questa classe si occupa di fornire il comando per la creazione di nuovi progetti vuoti quando l'utente lo richiede.
\item \textit{IN} \hyperref[\nogloxy{SWEDesigner::Client::Model::Utility::ProjectLoader}]{\nogloxy{\texttt{ProjectLoader}}}\\
questa classe si occupa di fornire il comando per il caricamento di un progetto già esistente su disco dell'utente quando l'utente lo richiede. 
\item \textit{IN} \hyperref[\nogloxy{SWEDesigner::Client::Model::Utility::ProjectSaver}]{\nogloxy{\texttt{ProjectSaver}}}\\
questa classe si occupa di fornire il comando per il salvataggio di un progetto quando l'utente lo richiede.
\item \textit{IN} \hyperref[\nogloxy{SWEDesigner::Client::View::AppView}]{\nogloxy{\texttt{AppView}}}\\
questa classe si occupa di gestire l'intera view dell'applicazione, componendo l'interfaccia principale e richiamando le altre view tramite il sistema a template di \backbonejs{}.
\item \textit{OUT} \hyperref[\nogloxy{SWEDesigner::Client::Model::ProjectModel}]{\nogloxy{\texttt{ProjectModel}}}\\
questa classe ci permette di aggiungere della logica alla collezione di tutti i diagrammi che possediamo.
\end{itemize}
\end{itemize}

\paragraph{\nogloxy{SWEDesigner::Client::Model::ProjectModel}}
\label{\nogloxy{SWEDesigner::Client::Model::ProjectModel}}
\begin{itemize}
\item \textbf{Descrizione}\\
questa classe ci permette di aggiungere della logica alla collezione di tutti i diagrammi che possediamo.
\item \textbf{Utilizzo}\\
essa è istanziata da ProjectView e possiede DiagramCollection. È possibile interagire con questo modello anche tramite \texttt{ProjectCommand}, per permettere l'implementazione di funzionalità globali dell'applicazione (e.g. salvataggio e caricamento).
\item \textbf{Relazioni con altre classi}:
\begin{itemize}
\item \textit{IN} \hyperref[\nogloxy{SWEDesigner::Client::Model::ProjectCommand}]{\nogloxy{\texttt{ProjectCommand}}}\\
questa interfaccia descrive la struttura di un comando che viene chiamato da \texttt{AppView} quando l'utente decide di creare un nuovo progetto, di caricarne uno esistente, di salvarlo o di generare il codice dal diagramma. Il pattern realizzato è il pattern Command.
\item \textit{IN} \hyperref[\nogloxy{SWEDesigner::Client::View::AppView}]{\nogloxy{\texttt{AppView}}}\\
questa classe si occupa di gestire l'intera view dell'applicazione, componendo l'interfaccia principale e richiamando le altre view tramite il sistema a template di \backbonejs{}.
\item \textit{IN} \hyperref[\nogloxy{SWEDesigner::Client::View::ProjectView}]{\nogloxy{\texttt{ProjectView}}}\\
questa classe rappresenta l'area di disegno principale dell'applicazione, che necessita di essere cambiata tra diagramma delle classi e diagramma delle attività. 
\item \textit{OUT} \hyperref[\nogloxy{SWEDesigner::Client::Collection::DiagramCollection}]{\nogloxy{\texttt{DiagramCollection}}}\\
questa classe collection possiede un certo numero di \texttt{Graph}, offerti dalla libreria \jointjs{}, che rappresentano degli elementi disegnabili. Una descrizione più dettagliata è disponibile in appendice. % [RIFERIMENTO].

\end{itemize}
\end{itemize}
\subsection{\nogloxy{SWEDesigner::Client::Model::CellTypes}}
\label{\nogloxy{SWEDesigner::Client::Model::CellTypes}}
\subsubsection{Informazioni generali}
\begin{itemize}
\item \textbf{Descrizione}\\
questo package contiene la definizione del modello degli elementi grafici dell'applicazione (e.g. diagrammi delle classi, blocchi condizionali). 
\item \textbf{Padre}: \hyperref[\nogloxy{SWEDesigner::Client::Model}]{\nogloxy{\texttt{Model}}}
\end{itemize}
\subsubsection{Classi}
\paragraph{\nogloxy{SWEDesigner::Client::Model::CellTypes::ActivityDiagramElement}}
\label{\nogloxy{SWEDesigner::Client::Model::CellTypes::ActivityDiagramElement}}
\begin{itemize}
\item \textbf{Descrizione}\\
questa classe è la base di tutte le classi che rappresentano i blocchi del diagramma delle attività.
\item \textbf{Utilizzo}\\
eredita da sottotipi della classe \texttt{Element} di \jointjs{} e viene estesa da tutte le classi specifiche di ogni blocco del diagramma delle attività. Questa classe è indirettamente correlata a \texttt{NewCellFactory}.
\item \textbf{Sottoclassi}:
\begin{itemize}
\item \hyperref[\nogloxy{SWEDesigner::Client::Model::CellTypes::HxAssignment}]{\nogloxy{\texttt{HxAssignment}}}
\item \hyperref[\nogloxy{SWEDesigner::Client::Model::CellTypes::HxCustom}]{\nogloxy{\texttt{HxCustom}}}
\item \hyperref[\nogloxy{SWEDesigner::Client::Model::CellTypes::HxFor}]{\nogloxy{\texttt{HxFor}}}
\item \hyperref[\nogloxy{SWEDesigner::Client::Model::CellTypes::HxIf}]{\nogloxy{\texttt{HxIf}}}
\item \hyperref[\nogloxy{SWEDesigner::Client::Model::CellTypes::HxInitialization}]{\nogloxy{\texttt{HxInitialization}}}
\item \hyperref[\nogloxy{SWEDesigner::Client::Model::CellTypes::HxReturn}]{\nogloxy{\texttt{HxReturn}}}
\item \hyperref[\nogloxy{SWEDesigner::Client::Model::CellTypes::HxWhile}]{\nogloxy{\texttt{HxWhile}}}
\end{itemize}
\item \textbf{Relazioni con altre classi}:
\begin{itemize}
\item \textit{OUT} \hyperref[\nogloxy{SWEDesigner::Client::Model::NewCellFactory}]{\nogloxy{\texttt{NewCellFactory}}}\\
questa classe si occupa di fornire un'istanza di una cella del tipo richiesto da \texttt{NewCellModel}. 
\end{itemize}
\end{itemize}

\paragraph{\nogloxy{SWEDesigner::Client::Model::CellTypes::ClassDiagramElement}}
\label{\nogloxy{SWEDesigner::Client::Model::CellTypes::ClassDiagramElement}}
\begin{itemize}
\item \textbf{Descrizione}\\
questa classe è la base di tutte le classi che rappresentano gli elementi del diagramma delle classi.
\item \textbf{Utilizzo}\\
eredita da sottotipi della classe element di \jointjs{} e viene estesa da tutte le classi specifiche di ogni blocco del diagramma delle classi. Questa classe è indirettamente correlata con \texttt{NewCellFactory}, derivando da \texttt{Cell}.
\item \textbf{Sottoclassi}:
\begin{itemize}
\item \hyperref[\nogloxy{SWEDesigner::Client::Model::CellTypes::HxAbstractClass}]{\nogloxy{\texttt{HxAbstractClass}}}
\item \hyperref[\nogloxy{SWEDesigner::Client::Model::CellTypes::HxAnnotation}]{\nogloxy{\texttt{HxAnnotation}}}
\item \hyperref[\nogloxy{SWEDesigner::Client::Model::CellTypes::HxClass}]{\nogloxy{\texttt{HxClass}}}
\item \hyperref[\nogloxy{SWEDesigner::Client::Model::CellTypes::HxInterface}]{\nogloxy{\texttt{HxInterface}}}
\end{itemize}
\item \textbf{Relazioni con altre classi}:
\begin{itemize}
\item \textit{IN} \hyperref[\nogloxy{SWEDesigner::Client::Model::NewCellFactory}]{\nogloxy{\texttt{NewCellFactory}}}\\
questa classe si occupa di fornire un'istanza di una cella del tipo richiesto da \texttt{NewCellModel}. 
\item \textit{OUT} \hyperref[\nogloxy{SWEDesigner::Client::Model::CellTypes::HxStereotype}]{\nogloxy{\texttt{HxStereotype}}}\\
questa classe rappresenta le caratteristiche dello stereotipo che viene assegnato ad un \texttt{ClassDiagramElement} come i meta-attributi e meta-metodi, i quali sono essere ridefiniti dalla classe o interfaccia implementata tramite \proj{}. 
\end{itemize}
\end{itemize}

\paragraph{\nogloxy{SWEDesigner::Client::Model::CellTypes::ClassDiagramLink}}
\label{\nogloxy{SWEDesigner::Client::Model::CellTypes::ClassDiagramLink}}
\begin{itemize}
\item \textbf{Descrizione}\\
questa classe è la base di tutte le classi che rappresentano le relazioni tra gli elementi del diagramma delle classi.
\item \textbf{Utilizzo}\\
eredita dalla classe \texttt{Link} di \jointjs{} e viene estesa da tutte le classi specifiche di ogni relazione del diagramma delle classi.
\item \textbf{Sottoclassi}:
\begin{itemize}
\item \hyperref[\nogloxy{SWEDesigner::Client::Model::CellTypes::GeneralizationCell}]{\nogloxy{\texttt{GeneralizationCell}}}
\item \hyperref[\nogloxy{SWEDesigner::Client::Model::CellTypes::ImplementationCell}]{\nogloxy{\texttt{ImplementationCell}}}
\end{itemize}
\end{itemize}

\paragraph{\nogloxy{SWEDesigner::Client::Model::CellTypes::GeneralizationCell}}
\label{\nogloxy{SWEDesigner::Client::Model::CellTypes::GeneralizationCell}}
\begin{itemize}
\item \textbf{Descrizione}\\
questa classe rappresenta la relazione di generalizzazione tra due celle del diagramma delle classi.
\item \textbf{Utilizzo}\\
eredita dalla classe \texttt{ClassDiagramLink}.
\item \textbf{Classi ereditate}:
\begin{itemize}
\item \hyperref[\nogloxy{SWEDesigner::Client::Model::CellTypes::ClassDiagramLink}]{\nogloxy{\texttt{ClassDiagramLink}}}
\end{itemize}
\end{itemize}

\paragraph{\nogloxy{SWEDesigner::Client::Model::CellTypes::HxAbstractClass}}
\label{\nogloxy{SWEDesigner::Client::Model::CellTypes::HxAbstractClass}}
\begin{itemize}
\item \textbf{Descrizione}\\
questa classe rappresenta la cella \texttt{abstract class} del diagramma delle classi UML. 
\item \textbf{Utilizzo}\\
la classe \texttt{NewCellFactory} ritorna un'istanza di questa classe ogni volta che l'utente richiede un nuovo elemento \emph{abstract class}.
\item \textbf{Classi ereditate}:
\begin{itemize}
\item \hyperref[\nogloxy{SWEDesigner::Client::Model::CellTypes::ClassDiagramElement}]{\nogloxy{\texttt{ClassDiagramElement}}}
\end{itemize}
\end{itemize}

\paragraph{\nogloxy{SWEDesigner::Client::Model::CellTypes::HxAnnotation}}
\label{\nogloxy{SWEDesigner::Client::Model::CellTypes::HxAnnotation}}
\begin{itemize}
\item \textbf{Descrizione}\\
questa classe rappresenta la cella di commento del diagramma delle classi UML.
\item \textbf{Utilizzo}\\
la classe \texttt{NewCellFactory} ritorna un'istanza di questa classe ogni volta che l'utente richiede un nuovo commento.
\item \textbf{Classi ereditate}:
\begin{itemize}
\item \hyperref[\nogloxy{SWEDesigner::Client::Model::CellTypes::ClassDiagramElement}]{\nogloxy{\texttt{ClassDiagramElement}}}
\end{itemize}
\end{itemize}

\paragraph{\nogloxy{SWEDesigner::Client::Model::CellTypes::HxAssignment}}
\label{\nogloxy{SWEDesigner::Client::Model::CellTypes::HxAssignment}}
\begin{itemize}
\item \textbf{Descrizione}\\
questa classe rappresenta il blocco di assegnazione di una variabile del diagramma delle attività.
\item \textbf{Utilizzo}\\
la classe \texttt{NewCellFactory} ritorna un'istanza di questa classe ogni volta che l'utente richiede un nuovo blocco di assegnazione di una variabile.
\item \textbf{Classi ereditate}:
\begin{itemize}
\item \hyperref[\nogloxy{SWEDesigner::Client::Model::CellTypes::ActivityDiagramElement}]{\nogloxy{\texttt{ActivityDiagramElement}}}
\end{itemize}
\end{itemize}

\paragraph{\nogloxy{SWEDesigner::Client::Model::CellTypes::HxClass}}
\label{\nogloxy{SWEDesigner::Client::Model::CellTypes::HxClass}}
\begin{itemize}
\item \textbf{Descrizione}\\
questa classe rappresenta il blocco \texttt{class} del diagramma delle classi UML.
\item \textbf{Utilizzo}\\
la classe \texttt{NewCellFactory} ritorna un'istanza di questa classe ogni volta che l'utente richiede un nuovo elemento \emph{class}.
\item \textbf{Classi ereditate}:
\begin{itemize}
\item \hyperref[\nogloxy{SWEDesigner::Client::Model::CellTypes::ClassDiagramElement}]{\nogloxy{\texttt{ClassDiagramElement}}}
\end{itemize}
\end{itemize}

\paragraph{\nogloxy{SWEDesigner::Client::Model::CellTypes::HxCustom}}
\label{\nogloxy{SWEDesigner::Client::Model::CellTypes::HxCustom}}
\begin{itemize}
\item \textbf{Descrizione}\\
questa classe rappresenta il blocco custom del diagramma delle attività che permette all'utente di inserire liberamente codice nel linguaggio target scelto.
\item \textbf{Utilizzo}\\
la classe \texttt{NewCellFactory} ritorna un'istanza di questa classe ogni volta che l'utente richiede un nuovo blocco di codice personalizzato.
\item \textbf{Classi ereditate}:
\begin{itemize}
\item \hyperref[\nogloxy{SWEDesigner::Client::Model::CellTypes::ActivityDiagramElement}]{\nogloxy{\texttt{ActivityDiagramElement}}}
\end{itemize}
\end{itemize}

\paragraph{\nogloxy{SWEDesigner::Client::Model::CellTypes::HxFor}}
\label{\nogloxy{SWEDesigner::Client::Model::CellTypes::HxFor}}
\begin{itemize}
\item \textbf{Descrizione}\\
questa classe rappresenta il blocco \texttt{for} del diagramma delle attività.
\item \textbf{Utilizzo}\\
la classe \texttt{NewCellFactory} ritorna un'istanza di questa classe ogni volta che l'utente richiede un nuovo blocco \texttt{for}.
\item \textbf{Classi ereditate}:
\begin{itemize}
\item \hyperref[\nogloxy{SWEDesigner::Client::Model::CellTypes::ActivityDiagramElement}]{\nogloxy{\texttt{ActivityDiagramElement}}}
\end{itemize}
\end{itemize}

\paragraph{\nogloxy{SWEDesigner::Client::Model::CellTypes::HxIf}}
\label{\nogloxy{SWEDesigner::Client::Model::CellTypes::HxIf}}
\begin{itemize}
\item \textbf{Descrizione}\\
questa classe rappresenta il blocco \texttt{if} del diagramma delle attività.
\item \textbf{Utilizzo}\\
la classe \texttt{NewCellFactory} ritorna un'istanza di questa classe ogni volta che l'utente richiede un nuovo blocco \texttt{if}.
\item \textbf{Classi ereditate}:
\begin{itemize}
\item \hyperref[\nogloxy{SWEDesigner::Client::Model::CellTypes::ActivityDiagramElement}]{\nogloxy{\texttt{ActivityDiagramElement}}}
\end{itemize}
\end{itemize}

\paragraph{\nogloxy{SWEDesigner::Client::Model::CellTypes::HxInitialization}}
\label{\nogloxy{SWEDesigner::Client::Model::CellTypes::HxInitialization}}
\begin{itemize}
\item \textbf{Descrizione}\\
questa classe rappresenta il blocco di inizializzazione di una variabile del diagramma delle attività.
\item \textbf{Utilizzo}\\
la classe \texttt{NewCellFactory} ritorna un'istanza di questa classe ogni volta che l'utente richiede un nuovo blocco di inizializzazione di una variabile.
\item \textbf{Classi ereditate}:
\begin{itemize}
\item \hyperref[\nogloxy{SWEDesigner::Client::Model::CellTypes::ActivityDiagramElement}]{\nogloxy{\texttt{ActivityDiagramElement}}}
\end{itemize}
\end{itemize}

\paragraph{\nogloxy{SWEDesigner::Client::Model::CellTypes::HxInterface}}
\label{\nogloxy{SWEDesigner::Client::Model::CellTypes::HxInterface}}
\begin{itemize}
\item \textbf{Descrizione}\\
questa classe rappresenta il costrutto \emph{interface} del diagramma delle classi UML.
\item \textbf{Utilizzo}\\
la classe \texttt{NewCellFactory} ritorna un'istanza di questa classe ogni volta che l'utente richiede un nuovo elemento \emph{interface}.
\item \textbf{Classi ereditate}:
\begin{itemize}
\item \hyperref[\nogloxy{SWEDesigner::Client::Model::CellTypes::ClassDiagramElement}]{\nogloxy{\texttt{ClassDiagramElement}}}
\end{itemize}
\end{itemize}

\paragraph{\nogloxy{SWEDesigner::Client::Model::CellTypes::HxReturn}}
\label{\nogloxy{SWEDesigner::Client::Model::CellTypes::HxReturn}}
\begin{itemize}
\item \textbf{Descrizione}\\
questa classe rappresenta il blocco \emph{return} del diagramma delle attività.
\item \textbf{Utilizzo}\\
la classe \texttt{NewCellFactory} ritorna un'istanza di questa classe ogni volta che l'utente richiede un nuovo blocco \emph{return}.
\item \textbf{Classi ereditate}:
\begin{itemize}
\item \hyperref[\nogloxy{SWEDesigner::Client::Model::CellTypes::ActivityDiagramElement}]{\nogloxy{\texttt{ActivityDiagramElement}}}
\end{itemize}
\end{itemize}

\paragraph{\nogloxy{SWEDesigner::Client::Model::CellTypes::HxStereotype}}
\label{\nogloxy{SWEDesigner::Client::Model::CellTypes::HxStereotype}}
\begin{itemize}
\item \textbf{Descrizione}\\
questa classe rappresenta le caratteristiche dello stereotipo che viene assegnato ad un \texttt{ClassDiagramElement} come i meta-attributi e meta-metodi, i quali sono essere ridefiniti dalla classe o interfaccia implementata tramite \proj{}. 
\item \textbf{Utilizzo}\\
ogni elemento possiederà un unico stereotipo per semplificare l'implementazione di \proj{}. Gli stereotipi disponibili sono presenti all'interno della classe \emph{ProjectStereotypes} (all'interno di \emph{Model::Utility}).
\item \textbf{Relazioni con altre classi}:
\begin{itemize}
\item \textit{IN} \hyperref[\nogloxy{SWEDesigner::Client::Model::CellTypes::ClassDiagramElement}]{\nogloxy{\texttt{ClassDiagramElement}}}\\
questa classe è la base di tutte le classi che rappresentano gli elementi del diagramma delle classi.
\item \textit{IN} \hyperref[\nogloxy{SWEDesigner::Client::Model::Utility::ProjectStereotypes}]{\nogloxy{\texttt{ProjectStereotypes}}}\\
questa classe contiene al suo interno i possibili stereotipi utilizzabili recuperati dal server in modo asincrono. Conterrà una lista di elementi \texttt{HxStereotype}.
\end{itemize}
\end{itemize}

\paragraph{\nogloxy{SWEDesigner::Client::Model::CellTypes::HxWhile}}
\label{\nogloxy{SWEDesigner::Client::Model::CellTypes::HxWhile}}
\begin{itemize}
\item \textbf{Descrizione}\\
questa classe rappresenta il blocco \emph{while} del diagramma delle attività.
\item \textbf{Utilizzo}\\
la classe \texttt{NewCellFactory} ritorna un'istanza di questa classe ogni volta che l'utente richiede un nuovo blocco \emph{while}.
\item \textbf{Classi ereditate}:
\begin{itemize}
\item \hyperref[\nogloxy{SWEDesigner::Client::Model::CellTypes::ActivityDiagramElement}]{\nogloxy{\texttt{ActivityDiagramElement}}}
\end{itemize}
\end{itemize}

\paragraph{\nogloxy{SWEDesigner::Client::Model::CellTypes::ImplementationCell}}
\label{\nogloxy{SWEDesigner::Client::Model::CellTypes::ImplementationCell}}
\begin{itemize}
\item \textbf{Descrizione}\\
questa classe rappresenta la relazione di generalizzazione tra due celle del diagramma delle classi.
\item \textbf{Utilizzo}\\
eredita dalla classe \texttt{ClassDiagramLink}.
\item \textbf{Classi ereditate}:
\begin{itemize}
\item \hyperref[\nogloxy{SWEDesigner::Client::Model::CellTypes::ClassDiagramLink}]{\nogloxy{\texttt{ClassDiagramLink}}}
\end{itemize}
\end{itemize}
\subsection{\nogloxy{SWEDesigner::Client::Model::Utility}}
\label{\nogloxy{SWEDesigner::Client::Model::Utility}}
\subsubsection{Informazioni generali}
\begin{itemize}
\item \textbf{Descrizione}\\
questo package racchiude le classi che definiscono i principali comandi dell'applicazione. Esse fanno parte di un unico pattern Command, usato dalla AppView principale.
\item \textbf{Padre}: \hyperref[\nogloxy{SWEDesigner::Client::Model}]{\nogloxy{\texttt{Model}}}
\end{itemize}
\subsubsection{Classi}
\paragraph{\nogloxy{SWEDesigner::Client::Model::Utility::ProjectGenerator}}
\label{\nogloxy{SWEDesigner::Client::Model::Utility::ProjectGenerator}}
\begin{itemize}
\item \textbf{Descrizione}\\
questa classe si occupa di fornire il comando necessario ad inviare al server il progetto e di ricevere il codice del progetto quando l'utente lo richiede.
\item \textbf{Utilizzo}\\
viene utilizzata da \texttt{AppView} essendo una realizzazione dell'interfaccia quando l'utente chiede di generare il codice del progetto e usa \texttt{ProjectModel} come modello su cui operare.
\item \textbf{Relazioni con altre classi}:
\begin{itemize}
\item \textit{OUT} \hyperref[\nogloxy{SWEDesigner::Client::Model::ProjectCommand}]{\nogloxy{\texttt{ProjectCommand}}}\\
questa interfaccia descrive la struttura di un comando che viene chiamato da \texttt{AppView} quando l'utente decide di creare un nuovo progetto, di caricarne uno esistente, di salvarlo o di generare il codice dal diagramma. Il pattern realizzato è il pattern Command.
\end{itemize}
\end{itemize}

\paragraph{\nogloxy{SWEDesigner::Client::Model::Utility::ProjectInitializer}}
\label{\nogloxy{SWEDesigner::Client::Model::Utility::ProjectInitializer}}
\begin{itemize}
\item \textbf{Descrizione}\\
questa classe si occupa di fornire il comando per la creazione di nuovi progetti vuoti quando l'utente lo richiede.
\item \textbf{Utilizzo}\\
viene utilizzata da \texttt{AppView} essendo una realizzazione dell'interfaccia quando l'utente chiede di creare un nuovo progetto e usa \texttt{ProjectModel} come modello su cui operare.
\item \textbf{Relazioni con altre classi}:
\begin{itemize}
\item \textit{OUT} \hyperref[\nogloxy{SWEDesigner::Client::Model::ProjectCommand}]{\nogloxy{\texttt{ProjectCommand}}}\\
questa interfaccia descrive la struttura di un comando che viene chiamato da \texttt{AppView} quando l'utente decide di creare un nuovo progetto, di caricarne uno esistente, di salvarlo o di generare il codice dal diagramma. Il pattern realizzato è il pattern Command.
\end{itemize}
\end{itemize}

\paragraph{\nogloxy{SWEDesigner::Client::Model::Utility::ProjectLoader}}
\label{\nogloxy{SWEDesigner::Client::Model::Utility::ProjectLoader}}
\begin{itemize}
\item \textbf{Descrizione}\\
questa classe si occupa di fornire il comando per il caricamento di un progetto già esistente su disco dell'utente quando l'utente lo richiede. 
\item \textbf{Utilizzo}\\
viene utilizzata da \texttt{AppView} essendo una realizzazione dell'interfaccia quando l'utente chiede di caricare un progetto esistente e usa \texttt{ProjectModel} come modello su cui operare.
È possibile prevedere in futuro la modifica di questa classe per effettuare il recupero di queste informazioni anche in remoto (ad esempio, dei progetti demo forniti dal server) tramite interfaccia REST.
\item \textbf{Relazioni con altre classi}:
\begin{itemize}
\item \textit{OUT} \hyperref[\nogloxy{SWEDesigner::Client::Model::ProjectCommand}]{\nogloxy{\texttt{ProjectCommand}}}\\
questa interfaccia descrive la struttura di un comando che viene chiamato da \texttt{AppView} quando l'utente decide di creare un nuovo progetto, di caricarne uno esistente, di salvarlo o di generare il codice dal diagramma. Il pattern realizzato è il pattern Command.
\end{itemize}
\end{itemize}

\paragraph{\nogloxy{SWEDesigner::Client::Model::Utility::ProjectSaver}}
\label{\nogloxy{SWEDesigner::Client::Model::Utility::ProjectSaver}}
\begin{itemize}
\item \textbf{Descrizione}\\
questa classe si occupa di fornire il comando per il salvataggio di un progetto quando l'utente lo richiede.
\item \textbf{Utilizzo}\\
viene utilizzata da \texttt{AppView} essendo una realizzazione dell'interfaccia quando l'utente chiede di salvare un progetto e usa \texttt{ProjectModel} come modello su cui operare.
\item \textbf{Relazioni con altre classi}:
\begin{itemize}
\item \textit{OUT} \hyperref[\nogloxy{SWEDesigner::Client::Model::ProjectCommand}]{\nogloxy{\texttt{ProjectCommand}}}\\
questa interfaccia descrive la struttura di un comando che viene chiamato da \texttt{AppView} quando l'utente decide di creare un nuovo progetto, di caricarne uno esistente, di salvarlo o di generare il codice dal diagramma. Il pattern realizzato è il pattern Command.
\end{itemize}
\end{itemize}

\paragraph{\nogloxy{SWEDesigner::Client::Model::Utility::ProjectStereotypes}}
\label{\nogloxy{SWEDesigner::Client::Model::Utility::ProjectStereotypes}}
\begin{itemize}
\item \textbf{Descrizione}\\
questa classe contiene al suo interno i possibili stereotipi utilizzabili recuperati dal server in modo asincrono. Conterrà una lista di elementi \texttt{HxStereotype}.
\item \textbf{Utilizzo}\\
\texttt{DetailsView} necessita degli stereotipi esistenti per permettere l'inserimento e la modifica dello stereotipo di una classe. È possibile che la \texttt{NewCellFactory} faccia una richiesta simile a questa classe, al fine di poter fornire all'utente la possibilità di inserire una classe già stereotipata. 
\item \textbf{Relazioni con altre classi}:
\begin{itemize}
\item \textit{IN} \hyperref[\nogloxy{SWEDesigner::Client::View::DetailsView}]{\nogloxy{\texttt{DetailsView}}}\\
questa classe si occupa di visualizzare tutti i campi di un blocco o di una relazione (come il nome di una classe, i suoi attributi, la condizione di un blocco \texttt{if} o il blocco di partenza di una relazione) permettendone anche la modifica. È possibile anche specificare uno stereotipo per la classe.

\item \textit{OUT} \hyperref[\nogloxy{SWEDesigner::Client::Model::CellTypes::HxStereotype}]{\nogloxy{\texttt{HxStereotype}}}\\
questa classe rappresenta le caratteristiche dello stereotipo che viene assegnato ad un \texttt{ClassDiagramElement} come i meta-attributi e meta-metodi, i quali sono essere ridefiniti dalla classe o interfaccia implementata tramite \proj{}. 
\end{itemize}
\end{itemize}
\subsection{\nogloxy{SWEDesigner::Client::View}}
\label{\nogloxy{SWEDesigner::Client::View}}
\subsubsection{Informazioni generali}
\begin{itemize}
\item \textbf{Descrizione}\\
il package raccoglie le classi che descrivono le classi che interagiscono con il browser, che popolano template e si sottoscrivono alla view tramite un pattern Observer. I template non sono contenuti in questo package.
\item \textbf{Padre}: \hyperref[\nogloxy{SWEDesigner::Client}]{\nogloxy{\texttt{Client}}}
\end{itemize}
\subsubsection{Classi}
\paragraph{\nogloxy{SWEDesigner::Client::View::AppView}}
\label{\nogloxy{SWEDesigner::Client::View::AppView}}
\begin{itemize}
\item \textbf{Descrizione}\\
questa classe si occupa di gestire l'intera view dell'applicazione, componendo l'interfaccia principale e richiamando le altre view tramite il sistema a template di \backbonejs{}.
\item \textbf{Utilizzo}\\
essa è la prima classe costruita dall'entry point del programma.
\item \textbf{Relazioni con altre classi}:
\begin{itemize}
\item \textit{OUT} \hyperref[\nogloxy{SWEDesigner::Client::Model::ProjectCommand}]{\nogloxy{\texttt{ProjectCommand}}}\\
questa interfaccia descrive la struttura di un comando che viene chiamato da \texttt{AppView} quando l'utente decide di creare un nuovo progetto, di caricarne uno esistente, di salvarlo o di generare il codice dal diagramma. Il pattern realizzato è il pattern Command.
\item \textit{OUT} \hyperref[\nogloxy{SWEDesigner::Client::Model::ProjectModel}]{\nogloxy{\texttt{ProjectModel}}}\\
questa classe ci permette di aggiungere della logica alla collezione di tutti i diagrammi che possediamo.
\item \textit{OUT} \hyperref[\nogloxy{SWEDesigner::Client::View::DetailsView}]{\nogloxy{\texttt{DetailsView}}}\\
questa classe si occupa di visualizzare tutti i campi di un blocco o di una relazione (come il nome di una classe, i suoi attributi, la condizione di un blocco \texttt{if} o il blocco di partenza di una relazione) permettendone anche la modifica. È possibile anche specificare uno stereotipo per la classe.

\item \textit{OUT} \hyperref[\nogloxy{SWEDesigner::Client::View::NewCellView}]{\nogloxy{\texttt{NewCellView}}}\\
questa classe si occupa di visualizzare tutti i possibili blocchi e relazioni che si possono inserire nel diagramma delle classi o delle attività.
\item \textit{OUT} \hyperref[\nogloxy{SWEDesigner::Client::View::ProjectView}]{\nogloxy{\texttt{ProjectView}}}\\
questa classe rappresenta l'area di disegno principale dell'applicazione, che necessita di essere cambiata tra diagramma delle classi e diagramma delle attività. 
\end{itemize}
\end{itemize}

\paragraph{\nogloxy{SWEDesigner::Client::View::DetailsView}}
\label{\nogloxy{SWEDesigner::Client::View::DetailsView}}
\begin{itemize}
\item \textbf{Descrizione}\\
questa classe si occupa di visualizzare tutti i campi di un blocco o di una relazione (come il nome di una classe, i suoi attributi, la condizione di un blocco \texttt{if} o il blocco di partenza di una relazione) permettendone anche la modifica. È possibile anche specificare uno stereotipo per la classe.

\item \textbf{Utilizzo}\\
utilizza come model uno tra le classi contenute nel package \texttt{CellTypes} in base alla selezione fatta dall'utente e ne modifica i campi in base all'input. Essa eredita dalla classe della libreria \jointjs{} chiamata \texttt{CellView} in modo da usare il codice della libreria già esistente, fornendo già il collegamento tra \texttt{Cell} e \texttt{CellView}.
In futuro, qualora l'implementazione di questa classe risulti troppo pesante o complicata da testare, è possibile subclassare questa classe in \emph{view} multiple e prevedere una nuova \emph{factory} di view.
\item \textbf{Relazioni con altre classi}:
\begin{itemize}
\item \textit{IN} \hyperref[\nogloxy{SWEDesigner::Client::View::AppView}]{\nogloxy{\texttt{AppView}}}\\
questa classe si occupa di gestire l'intera view dell'applicazione, componendo l'interfaccia principale e richiamando le altre view tramite il sistema a template di \backbonejs{}.
\item \textit{OUT} \hyperref[\nogloxy{SWEDesigner::Client::Model::Utility::ProjectStereotypes}]{\nogloxy{\texttt{ProjectStereotypes}}}\\
questa classe contiene al suo interno i possibili stereotipi utilizzabili recuperati dal server in modo asincrono. Conterrà una lista di elementi \texttt{HxStereotype}.
\end{itemize}
\end{itemize}

\paragraph{\nogloxy{SWEDesigner::Client::View::NewCellView}}
\label{\nogloxy{SWEDesigner::Client::View::NewCellView}}
\begin{itemize}
\item \textbf{Descrizione}\\
questa classe si occupa di visualizzare tutti i possibili blocchi e relazioni che si possono inserire nel diagramma delle classi o delle attività.
\item \textbf{Utilizzo}\\
utilizza \texttt{NewCellModel} per recuperare i blocchi e relazioni inseribili nel diagramma corrente (che può essere o delle classi o delle attività).
\item \textbf{Relazioni con altre classi}:
\begin{itemize}
\item \textit{IN} \hyperref[\nogloxy{SWEDesigner::Client::View::AppView}]{\nogloxy{\texttt{AppView}}}\\
questa classe si occupa di gestire l'intera view dell'applicazione, componendo l'interfaccia principale e richiamando le altre view tramite il sistema a template di \backbonejs{}.
\item \textit{OUT} \hyperref[\nogloxy{SWEDesigner::Client::Model::NewCellModel}]{\nogloxy{\texttt{NewCellModel}}}\\
questa classe si occupa di fornire tutti i tipi di cell (tutti i blocchi e relazioni) da poter inserire nel diagramma corrente (o classi o attività).
\end{itemize}
\end{itemize}

\paragraph{\nogloxy{SWEDesigner::Client::View::ProjectView}}
\label{\nogloxy{SWEDesigner::Client::View::ProjectView}}
\begin{itemize}
\item \textbf{Descrizione}\\
questa classe rappresenta l'area di disegno principale dell'applicazione, che necessita di essere cambiata tra diagramma delle classi e diagramma delle attività. 
\item \textbf{Utilizzo}\\
seleziona l'elemento \emph{Graph} da visualizzare. esso è presente nel relativo \emph{model}, che sarà di volta in volta cambiato selezionandolo dalla sua \emph{collection}.
\item \textbf{Relazioni con altre classi}:
\begin{itemize}
\item \textit{IN} \hyperref[\nogloxy{SWEDesigner::Client::View::AppView}]{\nogloxy{\texttt{AppView}}}\\
questa classe si occupa di gestire l'intera view dell'applicazione, componendo l'interfaccia principale e richiamando le altre view tramite il sistema a template di \backbonejs{}.
\item \textit{OUT} \hyperref[\nogloxy{SWEDesigner::Client::Model::ProjectModel}]{\nogloxy{\texttt{ProjectModel}}}\\
questa classe ci permette di aggiungere della logica alla collezione di tutti i diagrammi che possediamo.
\end{itemize}
\end{itemize}
