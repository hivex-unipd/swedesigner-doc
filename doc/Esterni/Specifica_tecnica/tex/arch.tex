%%%%%%%%%%%%%%%%%%%%%%%%%%%%%%%%%
%%  Descrizione dell'architettura
%%%%%%%%%%%%%%%%%%%%%%%%%%%%%%%%%



\subsection{Metodo e formalismo di specifica}
Suddividiamo la descrizione dell'architettura di \proj{} in tre sezioni:
\begin{itemize}
	\item §\ref{sec:arch_gen}, che illustra gli aspetti generali dell'architettura del software;
	\item §\ref{sec:arch_client}, che descrive l'architettura del \emph{front end} dell'applicazione;
	\item §\ref{sec:arch_server}, che descrive l'architettura del \emph{back end} dell'applicazione;
	\item §\ref{sec:arch_proto}, che descrive il protocollo che lega le due interfacce precedenti.
\end{itemize}
Per ognuna di queste sezioni, specifichiamo l'architettura con metodo \emph{top-down} --- dal generale al particolare --- utilizzando i formalismi di UML 2.0.



\subsection{Architettura generale} \label{sec:arch_gen}
L'architettura dell'applicazione è suddivisa in due moduli:
\begin{enumerate}
	\item il client, che vive nel browser dell'utente;
	\item il server, che fornisce la pagina dell'applicazione al client e riceve da esso delle richieste di generazione di codice.
\end{enumerate}



\subsection{Architetura del client} \label{sec:arch_client}
menos e berga



\subsection{Architettura del server} \label{sec:arch_server}
Il server offre due servizi:
\begin{enumerate}
	\item fornisce al client la pagina HTML dove disegnare i diagrammi;
	\item elabora un file JSON in arrivo dal client e gli fornisce l'applicazione generata a partire da tale file.
\end{enumerate}
Questi due servizi rispettano lo stile architetturale REST, come spiegato nella sezione \ref{sec:arch_proto}.

Il \emph{back end} del server è suddiviso nei seguenti due moduli (penso):
\begin{itemize}
	\item controller: rappresenta il core del \emph{back end}. Si occupa di ricevere ed elaborare le richieste provenienti dal client.
	\item endpoints: per la REST (ce l'abbiamo???)
\end{itemize}



\subsection{Protocollo di comunicazione client-server} \label{sec:arch_proto}
% client                   server
%           ----GET---->   home.html
%           <---html----
% [disegna]
%           ----POST--->   json
%                          [genera codice]
%           <---zip-----
I due servizi offerti dal server (ottenere la pagina HTML e ottenere l'applicazione generata) seguono lo stile architetturale REST:
\begin{itemize}
	\item La richiesta per ottenere la pagina HTML usa il metodo HTTP GET; la pagina è quindi \emph{cachable} (i router tra client e server possono decidere di ottimizzarne la fornitura).
	\item Il file JSON viene spedito con il metodo HTTP POST, che permette la persistenza di tale file nel server; la persistenza del file JSONè indifferente alla nostra applicazione ma ne aumenta l'estensibilità, in quanto un giorno i manutentori potrebbe voler offrire un servizio di condivisione dei diagrammi disegnati oppure un database di utenti che abbiano i propri diagrammi sul server.
	\item Ognuno dei due servizi (la pagina HTML e la generazione di codice) è una risorsa distinta: questo disaccoppia i due servizi, aumentando ancora la manutenibilità del sistema.
	\item La richiesta per ottenere la pagina HTML è per forza idempotente: lo stato del server non può influire su una pagina statica.
	\item La richiesta di generazione di codice è idempotente: l'unica dipendenza esterna al programma è il file JSON, che viene passato dal client.
\end{itemize}
