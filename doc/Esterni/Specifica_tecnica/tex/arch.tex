%%%%%%%%%%%%%%%%%%%%%%%%%%%%%%%%%
%%  Descrizione dell'architettura
%%%%%%%%%%%%%%%%%%%%%%%%%%%%%%%%%

\subsection{Metodo e formalismo di specifica}
Suddividiamo la descrizione dell'architettura di \proj{} in tre sezioni:
\begin{itemize}
	\item §\ref{sec:arch_client}, che descrive il \emph{front end} dell'applicazione;
	\item §\ref{sec:arch_server}, che descrive il \emph{back end} dell'applicazione;
	\item §\ref{sec:arch_proto}, che descrive il protocollo che lega le due interfacce precedenti.
\end{itemize}
Per ognuna di queste sezioni, descriviamo l'architettura del sistema con metodo \emph{top-down} --- dal generale al particolare --- utilizzando i formalismi di UML 2.0.

\subsection{Architettura generale}
L'architettura dell'applicazione è suddivisa in due moduli (REST ecce ecce).....
\begin{enumerate}
	\item Client
	\item Server
\end{enumerate}

\subsection{Architetura Client} \label{sec:arch_client}
menos e berga

\subsection{Architettura Server} \label{sec:arch_server}
Il server si occupa di fornire servizi REST ed elaborare le richieste (i dati) in arrivo dal client. Il back-end del server è stato suddiviso nei seguenti due moduli (penso):
\begin{itemize}
	\item controller: rappresenta il core del back-end. Si occupa di ricevere ed elaborare le richieste provenienti dal client.
	\item endpoints: per la REST (ce l'abbiamo???)
\end{itemize}

\subsection{Protocollo di comunicazione client-server} \label{sec:arch_proto}
%%% [...]
