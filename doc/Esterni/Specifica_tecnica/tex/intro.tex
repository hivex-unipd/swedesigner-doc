%%%%%%%%%%%%%%%%
%%  Introduzione
%%%%%%%%%%%%%%%%

\subsection{Scopo del documento}
Questo documento ha lo scopo di definire, ad alto livello, l'architettura software di \proj; ci focalizziamo quindi sulla \emph{struttura} del software, cioè sulla descrizione delle sue componenti. L'implementazione del nostro software è a oggetti; per questo, le componenti descritte sono classi e loro istanze.
% Ecco una prova di pattern: \pattern{Observer}

Riteniamo che l'architettura esposta di seguito, accompagnata da un documento più preciso che verrà prodotto nelle prossime settimane (Definizione di Prodotto), sia ragionevole e fattibile. La difficoltà di generare codice eseguibile è stata superata grazie ad un uso accorto di librerie open-source, stili architetturali e \emph{design patter}. Il tutto è stato svolto con grande attenzione ai princìpi della programmazione ad oggetti, per aumentare il grado di coesione di un'architettura che rischia ad ogni momento di gonfiarsi inutilmente.

Ci è parso utile indicizzare i \emph{design patter} utilizzati; per questo, li abbiamo riportati in un elenco (dopo l'indice generale).

\subsection{Scopo del prodotto}
\scopo

\subsection{Glossario}
\presgloss

\subsection{Riferimenti} \label{sec:ref}

\subsubsection{Riferimenti normativi}
% ...

\subsubsection{Riferimenti informativi}
\begin{itemize}
	\item E. Gamma, R. Helm, R. Johnson, J. Vlissides, \emph{Design Patterns: Elements of Reusable Object-Oriented Software}.
	\item Server Apache Tomcat: \url{tomcat.apache.org}, visitato in data 01/03/2017.
	\item Libreria Pivotal Spring: \url{spring.io}, visitato in data 01/03/2017.
	\item Libreria StringTemplate: \url{stringtemplate.org}, visitato in data 01/03/2017.
	\item Libreria Google Gson: \url{github.com/google/gson}, visitato in data 01/03/2017.
\end{itemize}
