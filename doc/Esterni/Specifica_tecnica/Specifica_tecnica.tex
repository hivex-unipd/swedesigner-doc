% Norme di progetto
% da compilare con il comando pdflatex Specifica_tecnica_x.x.x.tex

% Dichiarazioni di ambiente e inclusione di pacchetti
% da usare tramite il comando % Dichiarazioni di ambiente e inclusione di pacchetti
% da usare tramite il comando % Dichiarazioni di ambiente e inclusione di pacchetti
% da usare tramite il comando \input{../../util/hx-ambiente}

\documentclass[a4paper,titlepage]{article}
\usepackage[T1]{fontenc}
\usepackage[utf8]{inputenc}
\usepackage[english,italian]{babel}
\usepackage{microtype}
\usepackage{lmodern}
\usepackage{underscore}
\usepackage{graphicx}
\usepackage{eurosym}
\usepackage{float}
\usepackage{fancyhdr}
\usepackage[table,dvipsnames]{xcolor}
\usepackage{multirow}
\usepackage{longtable}
\usepackage{chngpage}
\usepackage{grffile}
\usepackage[titles]{tocloft}
\usepackage{hyperref}
\hypersetup{hidelinks}

\usepackage{../../util/hx-vers}
\usepackage{../../util/hx-macro}
\usepackage{../../util/hx-front}

% solo se si vuole una nuova pagina ad ogni \section:
\usepackage{titlesec}
\newcommand{\sectionbreak}{\clearpage}

% stile di pagina:
\pagestyle{fancy}

% solo se si vuole eliminare l'indentazione ad ogni paragrafo:
\setlength{\parindent}{0pt}

% intestazione:
\lhead{\Large{\proj}}
\rhead{\includegraphics[keepaspectratio=true,width=50px]{../../util/hivex_logo2.png}}
\renewcommand{\headrulewidth}{0.4pt}

% pie' di pagina:
\lfoot{\email}
\rfoot{\thepage}
\cfoot{}
\renewcommand{\footrulewidth}{0.4pt}

% spazio verticale tra le celle di una tabella:
\renewcommand{\arraystretch}{1.5}

% profondità di indicizzazione:
\setcounter{tocdepth}{4}
\setcounter{secnumdepth}{4}

% numerazione innestata per elenchi numerati:
\renewcommand{\labelenumii}{\theenumii}
\renewcommand{\theenumii}{\theenumi.\arabic{enumii}.}


\documentclass[a4paper,titlepage]{article}
\usepackage[T1]{fontenc}
\usepackage[utf8]{inputenc}
\usepackage[english,italian]{babel}
\usepackage{microtype}
\usepackage{lmodern}
\usepackage{underscore}
\usepackage{graphicx}
\usepackage{eurosym}
\usepackage{float}
\usepackage{fancyhdr}
\usepackage[table,dvipsnames]{xcolor}
\usepackage{multirow}
\usepackage{longtable}
\usepackage{chngpage}
\usepackage{grffile}
\usepackage[titles]{tocloft}
\usepackage{hyperref}
\hypersetup{hidelinks}

\usepackage{../../util/hx-vers}
\usepackage{../../util/hx-macro}
\usepackage{../../util/hx-front}

% solo se si vuole una nuova pagina ad ogni \section:
\usepackage{titlesec}
\newcommand{\sectionbreak}{\clearpage}

% stile di pagina:
\pagestyle{fancy}

% solo se si vuole eliminare l'indentazione ad ogni paragrafo:
\setlength{\parindent}{0pt}

% intestazione:
\lhead{\Large{\proj}}
\rhead{\includegraphics[keepaspectratio=true,width=50px]{../../util/hivex_logo2.png}}
\renewcommand{\headrulewidth}{0.4pt}

% pie' di pagina:
\lfoot{\email}
\rfoot{\thepage}
\cfoot{}
\renewcommand{\footrulewidth}{0.4pt}

% spazio verticale tra le celle di una tabella:
\renewcommand{\arraystretch}{1.5}

% profondità di indicizzazione:
\setcounter{tocdepth}{4}
\setcounter{secnumdepth}{4}

% numerazione innestata per elenchi numerati:
\renewcommand{\labelenumii}{\theenumii}
\renewcommand{\theenumii}{\theenumi.\arabic{enumii}.}


\documentclass[a4paper,titlepage]{article}
\usepackage[T1]{fontenc}
\usepackage[utf8]{inputenc}
\usepackage[english,italian]{babel}
\usepackage{microtype}
\usepackage{lmodern}
\usepackage{underscore}
\usepackage{graphicx}
\usepackage{eurosym}
\usepackage{float}
\usepackage{fancyhdr}
\usepackage[table,dvipsnames]{xcolor}
\usepackage{multirow}
\usepackage{longtable}
\usepackage{chngpage}
\usepackage{grffile}
\usepackage[titles]{tocloft}
\usepackage{hyperref}
\hypersetup{hidelinks}

\usepackage{../../util/hx-vers}
\usepackage{../../util/hx-macro}
\usepackage{../../util/hx-front}

% solo se si vuole una nuova pagina ad ogni \section:
\usepackage{titlesec}
\newcommand{\sectionbreak}{\clearpage}

% stile di pagina:
\pagestyle{fancy}

% solo se si vuole eliminare l'indentazione ad ogni paragrafo:
\setlength{\parindent}{0pt}

% intestazione:
\lhead{\Large{\proj}}
\rhead{\includegraphics[keepaspectratio=true,width=50px]{../../util/hivex_logo2.png}}
\renewcommand{\headrulewidth}{0.4pt}

% pie' di pagina:
\lfoot{\email}
\rfoot{\thepage}
\cfoot{}
\renewcommand{\footrulewidth}{0.4pt}

% spazio verticale tra le celle di una tabella:
\renewcommand{\arraystretch}{1.5}

% profondità di indicizzazione:
\setcounter{tocdepth}{4}
\setcounter{secnumdepth}{4}

% numerazione innestata per elenchi numerati:
\renewcommand{\labelenumii}{\theenumii}
\renewcommand{\theenumii}{\theenumi.\arabic{enumii}.}


\version{0.0.1}
\creaz{12 febbraio 2017}
\author{\GG, \MM}
\supervisor{\LB, \AZ}
\uso{esterno}
\dest{\ALL}
\title{Specifica Tecnica}
% \date{9 gennaio 2017}

\begin{document}
\maketitle
% diario delle modifiche per l'analisi dei requisiti
% da includere con % diario delle modifiche per l'analisi dei requisiti
% da includere con % diario delle modifiche per l'analisi dei requisiti
% da includere con \include{diario}

\begin{diario}
	4.0.0 & {\LB} (Responsabile) & 02/05/2017 & Approvazione del documento \\ \hline
	3.1.0 & {\PB} (Verificatore) & 02/05/2017 & Verifica del documento \\ \hline
	3.0.1 & {\MM} (Analista) & 01/05/2017 & 
	\begin{itemize}
	\item Inserimento UC5.35 e relativo requisito;
	\item Inserimento UC8 e relativo requisito;
	\item Inserimento tabella Requisiti Implementati come appendice.
\end{itemize}\\ \hline
	3.0.0 & {\AZ} (Responsabile) & 19/03/2017 & Approvazione del documento \\ \hline
	2.1.0 & {\MM} (Verificatore) & 19/03/2017 & Verifica del documento \\ \hline
	2.0.3 & {\PB} (Progettista) & 18/03/2017 &  
\begin{itemize}
	\item Modifica tabella Tracciamento Fonti-Requisiti;
	\item Modifica tabella Requisiti-Fonti;
	\item Modifica Estensione UC7.
\end{itemize}\\ \hline
	2.0.2 & {\PB} (Progettista) & 17/03/2017 &  Ristrutturato UC5 e relativi requisiti\\ \hline
	2.0.1 & {\PB} (Progettista) & 16/03/2017 &  Ristrutturato UC4 e relativi requisiti\\ \hline
	2.0.0 & {\LS} (Responsabile) & 01/02/2017 & Approvazione del documento \\ \hline
	1.1.0 & {\GG} (Verificatore) & 01/02/2017 & Verifica del documento \\ \hline
	1.0.4 & {\AZ} (Analista) & 31/01/2017 & Inserito UC5.26 con relativo requisito e tracciamento nelle tabelle e inseriti i requisiti RFO7, RFO8, RFO8.1, RFO8.2, RFO9, RFO10 e RFO11\\ \hline
	1.0.3 & {\AZ} (Analista) & 29/01/2017 & Corretta la descrizione dello UC5 e approfondita la descrizione dello UC7 \\ \hline
	1.0.2 & {\AZ} (Analista) & 28/01/2017 & Corretti UC4.1.6.3.2, UC4.2.1 e inserito perimetro sistema del UC5\\ \hline
	1.0.1 & {\AZ} (Analista) & 26/01/2017 & Inserimento scenario alternativo allo UC2, creazione UC3.1 con relativo requisito e tracciamento nelle tabelle e corrette alcune postcondizioni \\ \hline
	1.0.0 & {\LB} (Responsabile) & 09/01/2017 & Approvazione documento \\ \hline
	0.4.0 & {\LS} (Verificatore) & 06/01/2017 & Verifica introduzione, descrizione generale e requisiti \\ \hline
	0.3.0 & {\MM} (Verificatore) & 06/01/2017 & Verifica UC5.3-UC7 \\ \hline
	0.2.0 & {\LB} (Verificatore) & 06/01/2017 & Verifica UC4.2-UC5.2 \\ \hline
	0.1.0 & {\AZ} (Verificatore) & 06/01/2017 & Verifica UC1-4.1.8 \\ \hline
	0.0.11 & {\LS} (Analista) & 04/01/2017 & Stesura UC6-UC7 \\ \hline
	0.0.10 & {\GG} (Analista) & 03/01/2017 & Stesura UC5.6-UC5.18 \\ \hline
	0.0.9 & {\LS} (Analista) & 03/01/2017 & Stesura UC5.3-UC5.5.6.1 \\ \hline
	0.0.8 & {\PB} (Analista) & 02/01/2017 & Stesura UC5-UC5.2 \\ \hline
	0.0.7 & {\AZ} (Analista) & 02/01/2017 & Stesura UC4.3.3.1-UC4.11 \\ \hline
	0.0.6 & {\MM} (Analista) & 30/12/2016 & Stesura UC4.2-UC4.3.3.1 \\ \hline
	0.0.5 & {\GG} (Analista) & 29/12/2016 & Stesura UC4.1.6-UC4.1.8 \\ \hline
	0.0.4 & {\PB} (Analista) & 29/12/2016 & Stesura UC4-UC4.1.5 \\ \hline
	0.0.3 & {\LB} (Analista) & 28/12/2016 & Stesura UC1-UC2-UC3 \\ \hline
	0.0.2 & {\LS} (Analista) & 27/12/2016 & Stesura introduzione e descrizione generale \\ \hline
	0.0.1 & {\AZ} (Analista) & 27/12/2016 & Stesura scheletro \\ \hline
\end{diario}


\begin{diario}
	4.0.0 & {\LB} (Responsabile) & 02/05/2017 & Approvazione del documento \\ \hline
	3.1.0 & {\PB} (Verificatore) & 02/05/2017 & Verifica del documento \\ \hline
	3.0.1 & {\MM} (Analista) & 01/05/2017 & 
	\begin{itemize}
	\item Inserimento UC5.35 e relativo requisito;
	\item Inserimento UC8 e relativo requisito;
	\item Inserimento tabella Requisiti Implementati come appendice.
\end{itemize}\\ \hline
	3.0.0 & {\AZ} (Responsabile) & 19/03/2017 & Approvazione del documento \\ \hline
	2.1.0 & {\MM} (Verificatore) & 19/03/2017 & Verifica del documento \\ \hline
	2.0.3 & {\PB} (Progettista) & 18/03/2017 &  
\begin{itemize}
	\item Modifica tabella Tracciamento Fonti-Requisiti;
	\item Modifica tabella Requisiti-Fonti;
	\item Modifica Estensione UC7.
\end{itemize}\\ \hline
	2.0.2 & {\PB} (Progettista) & 17/03/2017 &  Ristrutturato UC5 e relativi requisiti\\ \hline
	2.0.1 & {\PB} (Progettista) & 16/03/2017 &  Ristrutturato UC4 e relativi requisiti\\ \hline
	2.0.0 & {\LS} (Responsabile) & 01/02/2017 & Approvazione del documento \\ \hline
	1.1.0 & {\GG} (Verificatore) & 01/02/2017 & Verifica del documento \\ \hline
	1.0.4 & {\AZ} (Analista) & 31/01/2017 & Inserito UC5.26 con relativo requisito e tracciamento nelle tabelle e inseriti i requisiti RFO7, RFO8, RFO8.1, RFO8.2, RFO9, RFO10 e RFO11\\ \hline
	1.0.3 & {\AZ} (Analista) & 29/01/2017 & Corretta la descrizione dello UC5 e approfondita la descrizione dello UC7 \\ \hline
	1.0.2 & {\AZ} (Analista) & 28/01/2017 & Corretti UC4.1.6.3.2, UC4.2.1 e inserito perimetro sistema del UC5\\ \hline
	1.0.1 & {\AZ} (Analista) & 26/01/2017 & Inserimento scenario alternativo allo UC2, creazione UC3.1 con relativo requisito e tracciamento nelle tabelle e corrette alcune postcondizioni \\ \hline
	1.0.0 & {\LB} (Responsabile) & 09/01/2017 & Approvazione documento \\ \hline
	0.4.0 & {\LS} (Verificatore) & 06/01/2017 & Verifica introduzione, descrizione generale e requisiti \\ \hline
	0.3.0 & {\MM} (Verificatore) & 06/01/2017 & Verifica UC5.3-UC7 \\ \hline
	0.2.0 & {\LB} (Verificatore) & 06/01/2017 & Verifica UC4.2-UC5.2 \\ \hline
	0.1.0 & {\AZ} (Verificatore) & 06/01/2017 & Verifica UC1-4.1.8 \\ \hline
	0.0.11 & {\LS} (Analista) & 04/01/2017 & Stesura UC6-UC7 \\ \hline
	0.0.10 & {\GG} (Analista) & 03/01/2017 & Stesura UC5.6-UC5.18 \\ \hline
	0.0.9 & {\LS} (Analista) & 03/01/2017 & Stesura UC5.3-UC5.5.6.1 \\ \hline
	0.0.8 & {\PB} (Analista) & 02/01/2017 & Stesura UC5-UC5.2 \\ \hline
	0.0.7 & {\AZ} (Analista) & 02/01/2017 & Stesura UC4.3.3.1-UC4.11 \\ \hline
	0.0.6 & {\MM} (Analista) & 30/12/2016 & Stesura UC4.2-UC4.3.3.1 \\ \hline
	0.0.5 & {\GG} (Analista) & 29/12/2016 & Stesura UC4.1.6-UC4.1.8 \\ \hline
	0.0.4 & {\PB} (Analista) & 29/12/2016 & Stesura UC4-UC4.1.5 \\ \hline
	0.0.3 & {\LB} (Analista) & 28/12/2016 & Stesura UC1-UC2-UC3 \\ \hline
	0.0.2 & {\LS} (Analista) & 27/12/2016 & Stesura introduzione e descrizione generale \\ \hline
	0.0.1 & {\AZ} (Analista) & 27/12/2016 & Stesura scheletro \\ \hline
\end{diario}


\begin{diario}
	4.0.0 & {\LB} (Responsabile) & 02/05/2017 & Approvazione del documento \\ \hline
	3.1.0 & {\PB} (Verificatore) & 02/05/2017 & Verifica del documento \\ \hline
	3.0.1 & {\MM} (Analista) & 01/05/2017 & 
	\begin{itemize}
	\item Inserimento UC5.35 e relativo requisito;
	\item Inserimento UC8 e relativo requisito;
	\item Inserimento tabella Requisiti Implementati come appendice.
\end{itemize}\\ \hline
	3.0.0 & {\AZ} (Responsabile) & 19/03/2017 & Approvazione del documento \\ \hline
	2.1.0 & {\MM} (Verificatore) & 19/03/2017 & Verifica del documento \\ \hline
	2.0.3 & {\PB} (Progettista) & 18/03/2017 &  
\begin{itemize}
	\item Modifica tabella Tracciamento Fonti-Requisiti;
	\item Modifica tabella Requisiti-Fonti;
	\item Modifica Estensione UC7.
\end{itemize}\\ \hline
	2.0.2 & {\PB} (Progettista) & 17/03/2017 &  Ristrutturato UC5 e relativi requisiti\\ \hline
	2.0.1 & {\PB} (Progettista) & 16/03/2017 &  Ristrutturato UC4 e relativi requisiti\\ \hline
	2.0.0 & {\LS} (Responsabile) & 01/02/2017 & Approvazione del documento \\ \hline
	1.1.0 & {\GG} (Verificatore) & 01/02/2017 & Verifica del documento \\ \hline
	1.0.4 & {\AZ} (Analista) & 31/01/2017 & Inserito UC5.26 con relativo requisito e tracciamento nelle tabelle e inseriti i requisiti RFO7, RFO8, RFO8.1, RFO8.2, RFO9, RFO10 e RFO11\\ \hline
	1.0.3 & {\AZ} (Analista) & 29/01/2017 & Corretta la descrizione dello UC5 e approfondita la descrizione dello UC7 \\ \hline
	1.0.2 & {\AZ} (Analista) & 28/01/2017 & Corretti UC4.1.6.3.2, UC4.2.1 e inserito perimetro sistema del UC5\\ \hline
	1.0.1 & {\AZ} (Analista) & 26/01/2017 & Inserimento scenario alternativo allo UC2, creazione UC3.1 con relativo requisito e tracciamento nelle tabelle e corrette alcune postcondizioni \\ \hline
	1.0.0 & {\LB} (Responsabile) & 09/01/2017 & Approvazione documento \\ \hline
	0.4.0 & {\LS} (Verificatore) & 06/01/2017 & Verifica introduzione, descrizione generale e requisiti \\ \hline
	0.3.0 & {\MM} (Verificatore) & 06/01/2017 & Verifica UC5.3-UC7 \\ \hline
	0.2.0 & {\LB} (Verificatore) & 06/01/2017 & Verifica UC4.2-UC5.2 \\ \hline
	0.1.0 & {\AZ} (Verificatore) & 06/01/2017 & Verifica UC1-4.1.8 \\ \hline
	0.0.11 & {\LS} (Analista) & 04/01/2017 & Stesura UC6-UC7 \\ \hline
	0.0.10 & {\GG} (Analista) & 03/01/2017 & Stesura UC5.6-UC5.18 \\ \hline
	0.0.9 & {\LS} (Analista) & 03/01/2017 & Stesura UC5.3-UC5.5.6.1 \\ \hline
	0.0.8 & {\PB} (Analista) & 02/01/2017 & Stesura UC5-UC5.2 \\ \hline
	0.0.7 & {\AZ} (Analista) & 02/01/2017 & Stesura UC4.3.3.1-UC4.11 \\ \hline
	0.0.6 & {\MM} (Analista) & 30/12/2016 & Stesura UC4.2-UC4.3.3.1 \\ \hline
	0.0.5 & {\GG} (Analista) & 29/12/2016 & Stesura UC4.1.6-UC4.1.8 \\ \hline
	0.0.4 & {\PB} (Analista) & 29/12/2016 & Stesura UC4-UC4.1.5 \\ \hline
	0.0.3 & {\LB} (Analista) & 28/12/2016 & Stesura UC1-UC2-UC3 \\ \hline
	0.0.2 & {\LS} (Analista) & 27/12/2016 & Stesura introduzione e descrizione generale \\ \hline
	0.0.1 & {\AZ} (Analista) & 27/12/2016 & Stesura scheletro \\ \hline
\end{diario}

\tableofcontents
\newpage



% introduzione standard
\section{Introduzione}
%%%%%%%%%%%%%%%%
%%  Introduzione
%%%%%%%%%%%%%%%%

\subsection{Scopo del documento}
Questo documento ha lo scopo di definire, ad alto livello, l'architettura software di \proj; ci focalizziamo quindi sulla \emph{struttura} del software, cioè sulla descrizione delle sue componenti. L'implementazione del nostro software è a oggetti; per questo, le componenti descritte sono classi e loro istanze.

% \subsection{Struttura del documento}
% utile?

\subsection{Scopo del prodotto}
\scopo

\subsection{Glossario}
\presgloss

\subsection{Riferimenti} \label{sec:ref}

\subsubsection{Riferimenti normativi}
% ...

\subsubsection{Riferimenti informativi}
% ...

\subsection{Stime di fattibilità}
% [fattibilità di ciò che progettiamo]


% - scelta di framework, librerie etc...
% - interfacce con tali tecnologie
\section{Tecnologie utilizzate}
%%%%%%%%%%%%%%%%%%%%%%%%%
%%  Tecnologie utilizzate
%%%%%%%%%%%%%%%%%%%%%%%%%



\subsection{Client}
Per implementare il client di \proj{} abbiamo scelto di utilizzare HTML5 e CSS3 assieme a JavaScript. Di seguito elenchiamo e descriviamo le librerie JavaScript di cui il nostro client necessita. Tutte le librerie sono open-source, come richiesto dal capitolato.

\subsubsection{JointJS}
JointJS (\url{https://www.jointjs.com/opensource}) è una libreria open-source per visualizzare e manipolare grafi (e quindi diagrammi). Dipende da HTML5 e da alcune altre librerie JavaScript ma offre numerosi plug-in per realizzare particolari tipi di grafi.

JointJS dipende da:
\begin{itemize}
	\item HTML5: JointJS richiede che una pagina HTML5 sia popolata con un tag \texttt{<canvas>}, che conterrà un grafo.
	\item Le seguenti librerie JavaScript:
	\begin{itemize}
		\item JQuery;
		\item Lodash;
		\item Backbone.js.
	\end{itemize}
\end{itemize}

il client di \proj{} utilizza un plug-in di JointJS che facilita la creazione di una particolare categoria di grafi: i diagrammi UML. [\dots]

\subsubsection{Backbone.js}
% ...

\subsubsection{JQuery}
% ...

%%% [altro...]



\subsection{Server}

\subsubsection{Apache Tomcat}
% ...

\subsubsection{Pivotal Spring}
% ...

%%% [altro...]



\subsection{Linguaggio target}

%%% [...]


% - diagramma dei package
% - scelte architetturali
% - altre cose ad alto livello
\section{Descrizione dell'architettura}
\input{tex/architettura}

% ---> distinguere STRUTTURA da COMPORTAMENTO ???
% - diagrammi delle classi
% - se serve, qualche diagramma di attività/sequenza
\section{Componenti}
%%%%%%%%%%%%%%
%%  Componenti
%%%%%%%%%%%%%%



% [Breve introduzione...]



\subsection{Client}

\subsubsection{MainApp}
% ...

\subsubsection{AppView}
% ...

% eccetera...



\subsection{Server}
% server==package principale
%Controller== classe che fa da Front Controller
%generator== package dentro server che fa il lavoro di creare il zip
%Project==classe (forse interface o abstract) dentro generator che fa il lavoro di creare il zip
\subsubsection{server}
--immagine package server con la classe controller che comunica con la classe Project contenuta nel package generator--
\begin{description}
\item[Descrizione] è il package che racchiude tutti i componenti del back-end che si occupano di soddisfare le richieste provenienti dal front-end ed elaborarle;
\item[Package contenuti] 
	\begin{itemize}
	\item server::endpoints (in teoria nn c'è);
	\item server::generator;
	\end{itemize}
\item[Interazione con altri componenti] interagisce con il client definito nella sez xxx tramite i servizi REST offerti. (in teoria non c'è)
\end{description}

\paragraph{Classi}
\subparagraph{server::Controller}
\begin{description}
\item[Descrizione] è la classe che implementa il pattern Front Controller e si occupa di raccogliere le richieste dal client (anche Singleton??)
\item[Utilizzo] viene invocata ""dal client"" e trasferisce la richiesta alla classe Project contenuta nel package server::generator che si occuperà di invocare il relativo comando per soddisfare la richiesta; 
\item[Relazioni con altre classi] 
	\begin{itemize}
	\item server::generator::Project.
	\end{itemize}
\end{description}

\subsubsection{server::generator}
--immagine--
\begin{description}
\item[Descrizione] è il package che contiene tutti i componenti che definiscono la logica necessaria per soddisfare le richieste il arrivo dal client;
\item[Padre] server; 
\item[Package contenuti]
	\begin{itemize}
	...
	\end{itemize}
\item[Interazione con altri componenti] 
	\begin{itemize}
	\item server::Controller.
	\end{itemize}
\end{description}

% JSON, client-server... ???
\section{Modelli dei dati e interfacce}
%%%%%%%%%%%%%%%%%%%%%%%%%%%%%%%%%
%%  Modelli dei dati e interfacce
%%%%%%%%%%%%%%%%%%%%%%%%%%%%%%%%%



%%% [...]


% ---> ha senso?
% \section{Design pattern utilizzati}
% %%%%%%%%%%%%%%%%%%%%%%%%%%%%%%
%%  Design pattern creazionali
%%%%%%%%%%%%%%%%%%%%%%%%%%%%%%

\subsection{Dependency injection}

\begin{figure}[H] \label{fig:injector}
	\includegraphics[scale=0.8]{img/injector.png}
	\caption{Dependency Injection.}
\end{figure}

\subsubsection{Scopo} Permette di disaccoppiare il comportamento di una componente dalla risoluzione delle sue dipendenze, semplificando perciò lo sviluppo di un software di grandi dimensioni e allo stesso tempo migliorandone la testabilità.

\subsubsection{Motivazione} Grazie a questo pattern è possibile separare la responsabilità di uso e creazione di un oggetto. Il componente dipendente dovrà solo sapere come usare un servizio richiesto, mentre il compito di creare ed iniettare quest'ultimo spetta ad un injector. Questo permette al componente dipendente di essere altamente configurabile in quanto è fisso solo il suo comportamento.

\subsubsection{Struttura} Sono coinvolti 3 componenti nella dependency injection:
\begin{enumerate}
	\item gli oggetti servizi(ossia un qualsiasi oggetto che potrebbe essere usato);
	\item l'oggetto dipendente da questi servizi;
	\item un injector responsabile di creare ed iniettare i servizi.
\end{enumerate}

\subsubsection{Applicabilità} è possibile applicare il pattern in tre modi differenti:
\begin{itemize}
	\item \textbf{setter injection:} la dipendenza viene iniettata tramite dei metodi setter del component dipendente;
	\item \textbf{costruction injection:} la dipendenza viene iniettata tramite un paramento del costruttore;
	\item \textbf{interface injection:} l'iniezione viene eseguita attraverso l'interfaccia che fornirà un setter a chiunque la implementa.
\end{itemize}

\subsection{Command}

\begin{figure}[H] \label{fig:command}
	\includegraphics[scale=0.6]{img/command.png}
	\caption{Command.}
\end{figure}

\subsubsection{Scopo} Permette di isolare la porzione di codice che effettua un'azione dal codice che ne richiede l'esecuzione. Tale azione è incapsulata nell'oggetto Command.

\subsubsection{Struttura} Sono coinvolti i seguenti componenti:

\begin{enumerate}
	\item Client: colui che richiede il comando  ed imposta il Receiver;
	\item Invoker: colui che effettua l'invocazione del comando;
	\item Command: interfaccia generica per l'esecuzione del comando;
	\item ConcreteCommand: implementazione del comando che consente di collegare l'invoker con il Receiver;
	\item Receiver: colui che riceve il comando e sa come eseguirlo.
\end{enumerate}

\subsubsection{Applicabilità} è possibile applicare il pattern per:

\begin{itemize}
	\item parametrizzare gli oggetti sull'azione da eseguire;
	\item specificare,accodare ed eseguire richieste molteplici volte;
	\item supportare operazioni di Undo e Redo;
	\item supportare la transazione: un comando equivale ad un'operazione atomica.
\end{itemize}

\subsection{Factory}

\begin{figure}[H] \label{fig:factory}
	\includegraphics[scale=0.34]{img/factory.png}
	\caption{Factory.}
\end{figure}

\subsubsection{Scopo} Indirizza il problema della creazione di oggetti senza specificarne la classe esatta fornendo un'interfaccia per creare l'oggetto, ma lascia che le sottoclassidecidano quale oggetto istanziare.

\subsubsection{Struttura} Sono coinvolti i seguenti componenti:

\begin{enumerate}
	\item Creator: dichiara la Factory che avrà il compito di ritornare l'oggetto appropriato;
	\item ConcreteCreator: effettua l'overwrite del metodo della Factoryal fine di ritornare l'implentazione dell'oggetto;
	\item Product: definisce l'interfaccia dell'oggetto che deve essere creato dalla Factory;
	\item ConcreteProduct: implementa l'oggetto in base ai metodi  definiti dall'interfaccia Product.
\end{enumerate}

\subsubsection{Applicabilità} è possibile applicare il pattern quando:

\begin{itemize}
	\item la creazione di un oggetto preclude il suo riuso senza una duplicazione di codice;
	\item la creazione di un oggetto richide l'accesso ad informazioni o risorse che non dovrebbero essere contenute nella classe di composizione;
	\item la gestione del ciclo di vita degli oggetti gestiti deve essere centralizzata in modo da assicurare un comportamento coerente all'interno dell'applicazione.
\end{itemize}

\subsection{Observer}

\begin{figure}[H] \label{fig:observer}
	\includegraphics[scale=0.6]{img/observer.png}
	\caption{Observer.}
\end{figure}

\subsubsection{Scopo} Questo pattern viene utilizzato quando si vuole realizzare una dipendenza uno-a-molti in cui il cambiamento di stato di un soggetto venga notiifcato a tutui i soggetti che si sono mostrati interessati.

\subsubsection{Struttura} Sono coinvolti i seguenti componenti:

\begin{enumerate}
	\item Subject: espone l’interfaccia che consente agli osservatori di iscriversi e cancellarsi mantenendi una reference a tutti gli osservatori iscritti;
	\item Observer: espone l’interfaccia che consente di aggiornare gli osservatori in caso di cambio di stato del soggetto osservato;
	\item ConcreteSubject: mantiene lo stato del soggetto osservato e notifica gli osservatori in caso di un cambio di stato;
	\item ConcreteObserver: implementa l’interfaccia dell’Observer definendo il comportamento in caso di cambio di stato del soggetto osservato.
\end{enumerate}

\subsubsection{Applicabilità} è possibile applicare il pattern quando:

\begin{itemize}
	\item in un problema ci sono due aspetti tra loro dipendenti, che possono essere rappresentati come classi che possono essere usati indipendentemente tra loro;
	\item quando il cambiamento di un oggetto provoca un cambiamento in un altro oggetto;
	\item quando un oggetto ha la necessità di comunicare con altri oggetti, senza fare assunzioni sugli altri oggetti.
\end{itemize}


% ---> meglio come subsection di "Componenti"?
% - tracciamento componenti-reqq
% - tracciamento reqq-componenti
\section{Tracciamento}
%%%%%%%%%%%%%%%%
%%  Tracciamento
%%%%%%%%%%%%%%%%



\subsection{Tracciamento Classi-Requisiti}
\normalsize
\begin{longtable}{|>{\centering}m{10cm}|m{3cm}<{\centering}|}
\hline 
\textbf{Classe} & \textbf{Requisiti}\\
\hline
\endhead
\hyperref[\nogloxy{swedesigner::client::model::celltypes::activity::ActivityDiagramElement}]{\nogloxy{\texttt{swedesigner::client::model::celltypes::-\linebreak activity::ActivityDiagramElement}}} & RFO4\\ \hline

\hyperref[\nogloxy{swedesigner::client::model::celltypes::class::ClassDiagramElement}]{\nogloxy{\texttt{swedesigner::client::model::celltypes::-\linebreak class::ClassDiagramElement}}} & RFO3\\
& RFD3.7\\
& RFD3.8\\
& RFD3.9\\
& RFD3.10\\ \hline

\hyperref[\nogloxy{swedesigner::client::model::celltypes::class::ClassDiagramLink}]{\nogloxy{\texttt{swedesigner::client::model::celltypes::-\linebreak class::ClassDiagramLink}}} & RFO3.2\\
& RFO3.2.1\\
& RFO3.2.2\\
& RFO3.2.3\\
& RFO3.2.4\\
& RFO3.2.5\\
& RFO3.2.6\\
& RFO3.2.7\\ \hline

\hyperref[\nogloxy{swedesigner::client::model::celltypes::class::HxGeneralization}]{\nogloxy{\texttt{swedesigner::client::model::celltypes::-\linebreak class::HxGeneralization}}} & RFO3.2\\ \hline

\hyperref[\nogloxy{swedesigner::client::model::celltypes::HxClass}]{\nogloxy{\texttt{swedesigner::client::model::celltypes::-\linebreak HxClass}}} & RFO3.1\\ \hline

\hyperref[\nogloxy{swedesigner::client::model::celltypes::HxComment}]{\nogloxy{\texttt{swedesigner::client::model::celltypes::-\linebreak HxComment}}} & RFO3.3\\ \hline

\hyperref[\nogloxy{swedesigner::client::model::celltypes::HxCustom}]{\nogloxy{\texttt{swedesigner::client::model::celltypes::-\linebreak HxCustom}}} & RFO4.6\\ \hline

\hyperref[\nogloxy{swedesigner::client::model::celltypes::HxFor}]{\nogloxy{\texttt{swedesigner::client::model::celltypes::-\linebreak HxFor}}} & RFO4.5\\ \hline

\hyperref[\nogloxy{swedesigner::client::model::celltypes::HxIf}]{\nogloxy{\texttt{swedesigner::client::model::celltypes::-\linebreak HxIf}}} & RFO4.3\\ \hline

\hyperref[\nogloxy{swedesigner::client::model::celltypes::HxInterface}]{\nogloxy{\texttt{swedesigner::client::model::celltypes::-\linebreak HxInterface}}} & RFO3.1\\ \hline

\hyperref[\nogloxy{swedesigner::client::model::celltypes::HxReturn}]{\nogloxy{\texttt{swedesigner::client::model::celltypes::-\linebreak HxReturn}}} & RFO4.26\\ \hline

\hyperref[\nogloxy{swedesigner::client::model::celltypes::HxStereotype}]{\nogloxy{\texttt{swedesigner::client::model::celltypes::-\linebreak HxStereotype}}} & RFO3.1.3\\ \hline

\hyperref[\nogloxy{swedesigner::client::model::celltypes::HxVariable}]{\nogloxy{\texttt{swedesigner::client::model::celltypes::-\linebreak HxVariable}}} & RFO4.1\\ \hline

\hyperref[\nogloxy{swedesigner::client::model::celltypes::HxWhile}]{\nogloxy{\texttt{swedesigner::client::model::celltypes::-\linebreak HxWhile}}} & RFO4.4\\ \hline

\hyperref[\nogloxy{swedesigner::client::model::celltypes::ImplementationCell}]{\nogloxy{\texttt{swedesigner::client::model::celltypes::-\linebreak ImplementationCell}}} & RFO3.2\\ \hline

\hyperref[\nogloxy{swedesigner::client::model::NewCellFactory}]{\nogloxy{\texttt{swedesigner::client::model::-\linebreak NewCellFactory}}} & RFO3.1\\
& RFO3.2\\
& RFO3.3\\
& RFO4.1\\
& RFO4.2\\
& RFO4.3\\
& RFO4.4\\
& RFO4.5\\
& RFO4.6\\ \hline

\hyperref[\nogloxy{swedesigner::client::model::NewCellModel}]{\nogloxy{\texttt{swedesigner::client::model::-\linebreak NewCellModel}}} & RFO3.1\\
& RFO3.2\\
& RFO3.3\\
& RFO4.1\\
& RFO4.2\\
& RFO4.3\\
& RFO4.4\\
& RFO4.5\\
& RFO4.6\\ \hline

\hyperref[\nogloxy{swedesigner::client::model::ProjectCommand}]{\nogloxy{\texttt{swedesigner::client::model::-\linebreak ProjectCommand}}} & RFO2\\
& RFO5\\
& RFO6\\ \hline

\hyperref[\nogloxy{swedesigner::client::model::ProjectModel}]{\nogloxy{\texttt{swedesigner::client::model::-\linebreak ProjectModel}}} & RFO1.1\\
& RFO3\\
& RFO4\\ \hline

\hyperref[\nogloxy{swedesigner::client::model::utility::ProjectGenerator}]{\nogloxy{\texttt{swedesigner::client::model::utility::-\linebreak ProjectGenerator}}} & RFO6\\ \hline

\hyperref[\nogloxy{swedesigner::client::model::utility::ProjectInitializer}]{\nogloxy{\texttt{swedesigner::client::model::utility::-\linebreak ProjectInitializer}}} & RFO2\\ \hline

\hyperref[\nogloxy{swedesigner::client::model::utility::ProjectLoader}]{\nogloxy{\texttt{swedesigner::client::model::utility::-\linebreak ProjectLoader}}} & RFO1\\ \hline

\hyperref[\nogloxy{swedesigner::client::model::utility::ProjectSaver}]{\nogloxy{\texttt{swedesigner::client::model::utility::-\linebreak ProjectSaver}}} & RFO5\\ \hline

\hyperref[\nogloxy{swedesigner::client::model::utility::ProjectStereotypes}]{\nogloxy{\texttt{swedesigner::client::model::utility::-\linebreak ProjectStereotypes}}} & RFO3.1.3\\ \hline

\hyperref[\nogloxy{swedesigner::client::view::AppView}]{\nogloxy{\texttt{swedesigner::client::view::AppView}}} & RFO1\\
& RFO2\\
& RFO3\\
& RFO4\\
& RFO5\\
& RFO6\\ \hline

\hyperref[\nogloxy{swedesigner::client::view::DetailsView}]{\nogloxy{\texttt{swedesigner::client::view::DetailsView}}} & RFO3.1\\
& RFO3.1.4.6.1\\
& RFO3.1.6.3.3.1\\
& RFO3.1.6.5.1\\
& RFO3.1.8.1\\
& RFO3.2\\
& RFO3.2.6.1\\
& RFO3.3\\
& RFO3.3.3.1\\
& RFO3.11\\
& RFO3.12\\
& RFO3.13\\
& RFO4.1\\
& RFO4.1.4.1\\
& RFO4.2\\
& RFO4.2.3.1\\
& RFO4.3\\
& RFO4.3.5.1\\
& RFO4.4\\
& RFO4.4.4.1\\
& RFO4.5\\
& RFO4.5.6.1\\
& RFO4.6\\
& RFO4.20\\
& RFO4.21\\
& RFO4.22\\
& RFO4.23\\
& RFO4.24\\
& RFO4.25\\ \hline

\hyperref[\nogloxy{swedesigner::client::view::NewCellView}]{\nogloxy{\texttt{swedesigner::client::view::NewCellView}}} & RFO3.1\\
& RFO3.2\\
& RFO3.3\\
& RFO4.1\\
& RFO4.2\\
& RFO4.3\\
& RFO4.4\\
& RFO4.5\\
& RFO4.6\\
& RFO4.26\\ \hline

\hyperref[\nogloxy{swedesigner::client::view::ProjectView}]{\nogloxy{\texttt{swedesigner::client::view::ProjectView}}} & RFO3\\
& RFO4\\ \hline

\hyperref[\nogloxy{swedesigner::server::compiler::Compiler}]{\nogloxy{\texttt{swedesigner::server::compiler::-\linebreak Compiler}}} & RFO9\\ \hline

\hyperref[\nogloxy{swedesigner::server::compiler::CompilerAssembler}]{\nogloxy{\texttt{swedesigner::server::compiler::-\linebreak CompilerAssembler}}} & RFO9\\ \hline

\hyperref[\nogloxy{swedesigner::server::compiler::java::JavaCompiler}]{\nogloxy{\texttt{swedesigner::server::compiler::java::-\linebreak JavaCompiler}}} & RFO9\\ \hline

\hyperref[\nogloxy{swedesigner::server::controller::RequestHandlerController}]{\nogloxy{\texttt{swedesigner::server::controller::-\linebreak RequestHandlerController}}} & RFO7\\
& RFO8\\
& RFO9\\
& RFO10\\
& RFO11\\ \hline

\hyperref[\nogloxy{swedesigner::server::generator::Generator}]{\nogloxy{\texttt{swedesigner::server::generator::-\linebreak Generator}}} & RFO8.2\\ \hline

\hyperref[\nogloxy{swedesigner::server::generator::java::JavaGenerator}]{\nogloxy{\texttt{swedesigner::server::generator::java::-\linebreak JavaGenerator}}} & RFO8.2\\ \hline

\hyperref[\nogloxy{swedesigner::server::parser::Parser}]{\nogloxy{\texttt{swedesigner::server::parser::Parser}}} & RFO8.1\\ \hline

\hyperref[\nogloxy{swedesigner::server::project::ParsedAttribute}]{\nogloxy{\texttt{swedesigner::server::project::-\linebreak ParsedAttribute}}} & RFO8.1\\ \hline

\hyperref[\nogloxy{swedesigner::server::project::ParsedClass}]{\nogloxy{\texttt{swedesigner::server::project::-\linebreak ParsedClass}}} & RFO8.1\\ \hline

\hyperref[\nogloxy{swedesigner::server::project::ParsedCustom}]{\nogloxy{\texttt{swedesigner::server::project::-\linebreak ParsedCustom}}} & RFO8.1\\ \hline

\hyperref[\nogloxy{swedesigner::server::project::ParsedElement}]{\nogloxy{\texttt{swedesigner::server::project::-\linebreak ParsedElement}}} & RFO8.1\\ \hline

\hyperref[\nogloxy{swedesigner::server::project::ParsedElse}]{\nogloxy{\texttt{swedesigner::server::project::-\linebreak ParsedElse}}} & RFO8.1\\ \hline

\hyperref[\nogloxy{swedesigner::server::project::ParsedException}]{\nogloxy{\texttt{swedesigner::server::project::-\linebreak ParsedException}}} & RFO6.1\\ \hline

\hyperref[\nogloxy{swedesigner::server::project::ParsedFor}]{\nogloxy{\texttt{swedesigner::server::project::-\linebreak ParsedFor}}} & RFO8.1\\ \hline

\hyperref[\nogloxy{swedesigner::server::project::ParsedIf}]{\nogloxy{\texttt{swedesigner::server::project::ParsedIf}}} & RFO8.1\\ \hline

\hyperref[\nogloxy{swedesigner::server::project::ParsedInstruction}]{\nogloxy{\texttt{swedesigner::server::project::-\linebreak ParsedInstruction}}} & RFO8.1\\ \hline

\hyperref[\nogloxy{swedesigner::server::project::ParsedInterface}]{\nogloxy{\texttt{swedesigner::server::project::-\linebreak ParsedInterface}}} & RFO8.1\\ \hline

\hyperref[\nogloxy{swedesigner::server::project::ParsedMethod}]{\nogloxy{\texttt{swedesigner::server::project::-\linebreak ParsedMethod}}} & RFO8.1\\ \hline

\hyperref[\nogloxy{swedesigner::server::project::ParsedProgram}]{\nogloxy{\texttt{swedesigner::server::project::-\linebreak ParsedProgram}}} & RFO8.1\\ \hline

\hyperref[\nogloxy{swedesigner::server::project::ParsedReturn}]{\nogloxy{\texttt{swedesigner::server::project::-\linebreak ParsedReturn}}} & RFO8.1\\ \hline

\hyperref[\nogloxy{swedesigner::server::project::ParsedStatement}]{\nogloxy{\texttt{swedesigner::server::project::-\linebreak ParsedStatement}}} & RFO8.1\\ \hline

\hyperref[\nogloxy{swedesigner::server::project::ParsedType}]{\nogloxy{\texttt{swedesigner::server::project::-\linebreak ParsedType}}} & RFO8.1\\ \hline

\hyperref[\nogloxy{swedesigner::server::project::ParsedWhile}]{\nogloxy{\texttt{swedesigner::server::project::-\linebreak ParsedWhile}}} & RFO8.1\\ \hline

\hyperref[\nogloxy{swedesigner::server::template::java::JavaTemplate}]{\nogloxy{\texttt{swedesigner::server::template::java::-\linebreak JavaTemplate}}} & RFO8.2\\ \hline

\hyperref[\nogloxy{swedesigner::server::template::Template}]{\nogloxy{\texttt{swedesigner::server::template::-\linebreak Template}}} & RFO8.2\\ \hline

\hyperref[\nogloxy{swedesigner::server::utility::Compressor}]{\nogloxy{\texttt{swedesigner::server::utility::-\linebreak Compressor}}} & RFO10\\ \hline

\caption[Tracciamento Classi-Requisiti]{Tracciamento Classi-Requisiti}
\label{tabella:class-requi}
\end{longtable}
\clearpage



\subsection{Tracciamento Requisiti-Classi}
\normalsize
\begin{longtable}{|>{\centering}m{3cm}|m{10cm}<{\centering}|}
\hline 
\textbf{Requisito} & \textbf{Classi}\\
\hline
\endhead
RFO1 & \hyperref[\nogloxy{swedesigner::client::model::utility::ProjectLoader}]{\nogloxy{\texttt{swedesigner::client::model::utility::-\linebreak ProjectLoader}}}\\
& \hyperref[\nogloxy{swedesigner::client::view::AppView}]{\nogloxy{\texttt{swedesigner::client::view::AppView}}}\\ \hline

RFO1.1 & \hyperref[\nogloxy{swedesigner::client::model::ProjectModel}]{\nogloxy{\texttt{swedesigner::client::model::-\linebreak ProjectModel}}}\\ \hline

RFO2 & \hyperref[\nogloxy{swedesigner::client::model::ProjectCommand}]{\nogloxy{\texttt{swedesigner::client::model::-\linebreak ProjectCommand}}}\\
& \hyperref[\nogloxy{swedesigner::client::model::utility::ProjectInitializer}]{\nogloxy{\texttt{swedesigner::client::model::utility::-\linebreak ProjectInitializer}}}\\
& \hyperref[\nogloxy{swedesigner::client::view::AppView}]{\nogloxy{\texttt{swedesigner::client::view::AppView}}}\\ \hline

RFO3 & \hyperref[\nogloxy{swedesigner::client::model::celltypes::class::ClassDiagramElement}]{\nogloxy{\texttt{swedesigner::client::model::celltypes::-\linebreak class::ClassDiagramElement}}}\\
& \hyperref[\nogloxy{swedesigner::client::model::ProjectModel}]{\nogloxy{\texttt{swedesigner::client::model::-\linebreak ProjectModel}}}\\
& \hyperref[\nogloxy{swedesigner::client::view::AppView}]{\nogloxy{\texttt{swedesigner::client::view::AppView}}}\\
& \hyperref[\nogloxy{swedesigner::client::view::ProjectView}]{\nogloxy{\texttt{swedesigner::client::view::ProjectView}}}\\ \hline

RFO3.1 & \hyperref[\nogloxy{swedesigner::client::model::celltypes::HxClass}]{\nogloxy{\texttt{swedesigner::client::model::celltypes::-\linebreak HxClass}}}\\
& \hyperref[\nogloxy{swedesigner::client::model::celltypes::HxInterface}]{\nogloxy{\texttt{swedesigner::client::model::celltypes::-\linebreak HxInterface}}}\\
& \hyperref[\nogloxy{swedesigner::client::model::NewCellFactory}]{\nogloxy{\texttt{swedesigner::client::model::-\linebreak NewCellFactory}}}\\
& \hyperref[\nogloxy{swedesigner::client::model::NewCellModel}]{\nogloxy{\texttt{swedesigner::client::model::-\linebreak NewCellModel}}}\\
& \hyperref[\nogloxy{swedesigner::client::view::DetailsView}]{\nogloxy{\texttt{swedesigner::client::view::DetailsView}}}\\
& \hyperref[\nogloxy{swedesigner::client::view::NewCellView}]{\nogloxy{\texttt{swedesigner::client::view::NewCellView}}}\\ \hline

RFO3.1.3 & \hyperref[\nogloxy{swedesigner::client::model::celltypes::HxStereotype}]{\nogloxy{\texttt{swedesigner::client::model::celltypes::-\linebreak HxStereotype}}}\\
& \hyperref[\nogloxy{swedesigner::client::model::utility::ProjectStereotypes}]{\nogloxy{\texttt{swedesigner::client::model::utility::-\linebreak ProjectStereotypes}}}\\ \hline

RFO3.1.4.6.1 & \hyperref[\nogloxy{swedesigner::client::view::DetailsView}]{\nogloxy{\texttt{swedesigner::client::view::DetailsView}}}\\ \hline

RFO3.1.6.3.3.1 & \hyperref[\nogloxy{swedesigner::client::view::DetailsView}]{\nogloxy{\texttt{swedesigner::client::view::DetailsView}}}\\ \hline

RFO3.1.6.5.1 & \hyperref[\nogloxy{swedesigner::client::view::DetailsView}]{\nogloxy{\texttt{swedesigner::client::view::DetailsView}}}\\ \hline

RFO3.1.8.1 & \hyperref[\nogloxy{swedesigner::client::view::DetailsView}]{\nogloxy{\texttt{swedesigner::client::view::DetailsView}}}\\ \hline

RFO3.2 & \hyperref[\nogloxy{swedesigner::client::model::celltypes::class::ClassDiagramLink}]{\nogloxy{\texttt{swedesigner::client::model::celltypes::-\linebreak class::ClassDiagramLink}}}\\
& \hyperref[\nogloxy{swedesigner::client::model::celltypes::class::HxGeneralization}]{\nogloxy{\texttt{swedesigner::client::model::celltypes::-\linebreak class::HxGeneralization}}}\\
& \hyperref[\nogloxy{swedesigner::client::model::celltypes::ImplementationCell}]{\nogloxy{\texttt{swedesigner::client::model::celltypes::-\linebreak ImplementationCell}}}\\
& \hyperref[\nogloxy{swedesigner::client::model::NewCellFactory}]{\nogloxy{\texttt{swedesigner::client::model::-\linebreak NewCellFactory}}}\\
& \hyperref[\nogloxy{swedesigner::client::model::NewCellModel}]{\nogloxy{\texttt{swedesigner::client::model::-\linebreak NewCellModel}}}\\
& \hyperref[\nogloxy{swedesigner::client::view::DetailsView}]{\nogloxy{\texttt{swedesigner::client::view::DetailsView}}}\\
& \hyperref[\nogloxy{swedesigner::client::view::NewCellView}]{\nogloxy{\texttt{swedesigner::client::view::NewCellView}}}\\ \hline

RFO3.2.1 & \hyperref[\nogloxy{swedesigner::client::model::celltypes::class::ClassDiagramLink}]{\nogloxy{\texttt{swedesigner::client::model::celltypes::-\linebreak class::ClassDiagramLink}}}\\ \hline

RFO3.2.2 & \hyperref[\nogloxy{swedesigner::client::model::celltypes::class::ClassDiagramLink}]{\nogloxy{\texttt{swedesigner::client::model::celltypes::-\linebreak class::ClassDiagramLink}}}\\ \hline

RFO3.2.3 & \hyperref[\nogloxy{swedesigner::client::model::celltypes::class::ClassDiagramLink}]{\nogloxy{\texttt{swedesigner::client::model::celltypes::-\linebreak class::ClassDiagramLink}}}\\ \hline

RFO3.2.4 & \hyperref[\nogloxy{swedesigner::client::model::celltypes::class::ClassDiagramLink}]{\nogloxy{\texttt{swedesigner::client::model::celltypes::-\linebreak class::ClassDiagramLink}}}\\ \hline

RFO3.2.5 & \hyperref[\nogloxy{swedesigner::client::model::celltypes::class::ClassDiagramLink}]{\nogloxy{\texttt{swedesigner::client::model::celltypes::-\linebreak class::ClassDiagramLink}}}\\ \hline

RFO3.2.6 & \hyperref[\nogloxy{swedesigner::client::model::celltypes::class::ClassDiagramLink}]{\nogloxy{\texttt{swedesigner::client::model::celltypes::-\linebreak class::ClassDiagramLink}}}\\ \hline

RFO3.2.6.1 & \hyperref[\nogloxy{swedesigner::client::view::DetailsView}]{\nogloxy{\texttt{swedesigner::client::view::DetailsView}}}\\ \hline

RFO3.2.7 & \hyperref[\nogloxy{swedesigner::client::model::celltypes::class::ClassDiagramLink}]{\nogloxy{\texttt{swedesigner::client::model::celltypes::-\linebreak class::ClassDiagramLink}}}\\ \hline

RFO3.3 & \hyperref[\nogloxy{swedesigner::client::model::celltypes::HxComment}]{\nogloxy{\texttt{swedesigner::client::model::celltypes::-\linebreak HxComment}}}\\
& \hyperref[\nogloxy{swedesigner::client::model::NewCellFactory}]{\nogloxy{\texttt{swedesigner::client::model::-\linebreak NewCellFactory}}}\\
& \hyperref[\nogloxy{swedesigner::client::model::NewCellModel}]{\nogloxy{\texttt{swedesigner::client::model::-\linebreak NewCellModel}}}\\
& \hyperref[\nogloxy{swedesigner::client::view::DetailsView}]{\nogloxy{\texttt{swedesigner::client::view::DetailsView}}}\\
& \hyperref[\nogloxy{swedesigner::client::view::NewCellView}]{\nogloxy{\texttt{swedesigner::client::view::NewCellView}}}\\ \hline

RFO3.3.3.1 & \hyperref[\nogloxy{swedesigner::client::view::DetailsView}]{\nogloxy{\texttt{swedesigner::client::view::DetailsView}}}\\ \hline

RFD3.7 & \hyperref[\nogloxy{swedesigner::client::model::celltypes::class::ClassDiagramElement}]{\nogloxy{\texttt{swedesigner::client::model::celltypes::-\linebreak class::ClassDiagramElement}}}\\ \hline

RFD3.8 & \hyperref[\nogloxy{swedesigner::client::model::celltypes::class::ClassDiagramElement}]{\nogloxy{\texttt{swedesigner::client::model::celltypes::-\linebreak class::ClassDiagramElement}}}\\ \hline

RFD3.9 & \hyperref[\nogloxy{swedesigner::client::model::celltypes::class::ClassDiagramElement}]{\nogloxy{\texttt{swedesigner::client::model::celltypes::-\linebreak class::ClassDiagramElement}}}\\ \hline

RFD3.10 & \hyperref[\nogloxy{swedesigner::client::model::celltypes::class::ClassDiagramElement}]{\nogloxy{\texttt{swedesigner::client::model::celltypes::-\linebreak class::ClassDiagramElement}}}\\ \hline

RFO3.11 & \hyperref[\nogloxy{swedesigner::client::view::DetailsView}]{\nogloxy{\texttt{swedesigner::client::view::DetailsView}}}\\ \hline

RFO3.12 & \hyperref[\nogloxy{swedesigner::client::view::DetailsView}]{\nogloxy{\texttt{swedesigner::client::view::DetailsView}}}\\ \hline

RFO3.13 & \hyperref[\nogloxy{swedesigner::client::view::DetailsView}]{\nogloxy{\texttt{swedesigner::client::view::DetailsView}}}\\ \hline

RFO4 & \hyperref[\nogloxy{swedesigner::client::model::celltypes::activity::ActivityDiagramElement}]{\nogloxy{\texttt{swedesigner::client::model::celltypes::-\linebreak activity::ActivityDiagramElement}}}\\
& \hyperref[\nogloxy{swedesigner::client::model::ProjectModel}]{\nogloxy{\texttt{swedesigner::client::model::-\linebreak ProjectModel}}}\\
& \hyperref[\nogloxy{swedesigner::client::view::AppView}]{\nogloxy{\texttt{swedesigner::client::view::AppView}}}\\
& \hyperref[\nogloxy{swedesigner::client::view::ProjectView}]{\nogloxy{\texttt{swedesigner::client::view::ProjectView}}}\\ \hline

RFO4.1 & \hyperref[\nogloxy{swedesigner::client::model::celltypes::HxVariable}]{\nogloxy{\texttt{swedesigner::client::model::celltypes::-\linebreak HxVariable}}}\\
& \hyperref[\nogloxy{swedesigner::client::model::NewCellFactory}]{\nogloxy{\texttt{swedesigner::client::model::-\linebreak NewCellFactory}}}\\
& \hyperref[\nogloxy{swedesigner::client::model::NewCellModel}]{\nogloxy{\texttt{swedesigner::client::model::-\linebreak NewCellModel}}}\\
& \hyperref[\nogloxy{swedesigner::client::view::DetailsView}]{\nogloxy{\texttt{swedesigner::client::view::DetailsView}}}\\
& \hyperref[\nogloxy{swedesigner::client::view::NewCellView}]{\nogloxy{\texttt{swedesigner::client::view::NewCellView}}}\\ \hline

RFO4.1.4.1 & \hyperref[\nogloxy{swedesigner::client::view::DetailsView}]{\nogloxy{\texttt{swedesigner::client::view::DetailsView}}}\\ \hline

RFO4.2 & \hyperref[\nogloxy{swedesigner::client::model::NewCellFactory}]{\nogloxy{\texttt{swedesigner::client::model::-\linebreak NewCellFactory}}}\\
& \hyperref[\nogloxy{swedesigner::client::model::NewCellModel}]{\nogloxy{\texttt{swedesigner::client::model::-\linebreak NewCellModel}}}\\
& \hyperref[\nogloxy{swedesigner::client::view::DetailsView}]{\nogloxy{\texttt{swedesigner::client::view::DetailsView}}}\\
& \hyperref[\nogloxy{swedesigner::client::view::NewCellView}]{\nogloxy{\texttt{swedesigner::client::view::NewCellView}}}\\ \hline

RFO4.2.3.1 & \hyperref[\nogloxy{swedesigner::client::view::DetailsView}]{\nogloxy{\texttt{swedesigner::client::view::DetailsView}}}\\ \hline

RFO4.3 & \hyperref[\nogloxy{swedesigner::client::model::celltypes::HxIf}]{\nogloxy{\texttt{swedesigner::client::model::celltypes::-\linebreak HxIf}}}\\
& \hyperref[\nogloxy{swedesigner::client::model::NewCellFactory}]{\nogloxy{\texttt{swedesigner::client::model::-\linebreak NewCellFactory}}}\\
& \hyperref[\nogloxy{swedesigner::client::model::NewCellModel}]{\nogloxy{\texttt{swedesigner::client::model::-\linebreak NewCellModel}}}\\
& \hyperref[\nogloxy{swedesigner::client::view::DetailsView}]{\nogloxy{\texttt{swedesigner::client::view::DetailsView}}}\\
& \hyperref[\nogloxy{swedesigner::client::view::NewCellView}]{\nogloxy{\texttt{swedesigner::client::view::NewCellView}}}\\ \hline

RFO4.3.5.1 & \hyperref[\nogloxy{swedesigner::client::view::DetailsView}]{\nogloxy{\texttt{swedesigner::client::view::DetailsView}}}\\ \hline

RFO4.4 & \hyperref[\nogloxy{swedesigner::client::model::celltypes::HxWhile}]{\nogloxy{\texttt{swedesigner::client::model::celltypes::-\linebreak HxWhile}}}\\
& \hyperref[\nogloxy{swedesigner::client::model::NewCellFactory}]{\nogloxy{\texttt{swedesigner::client::model::-\linebreak NewCellFactory}}}\\
& \hyperref[\nogloxy{swedesigner::client::model::NewCellModel}]{\nogloxy{\texttt{swedesigner::client::model::-\linebreak NewCellModel}}}\\
& \hyperref[\nogloxy{swedesigner::client::view::DetailsView}]{\nogloxy{\texttt{swedesigner::client::view::DetailsView}}}\\
& \hyperref[\nogloxy{swedesigner::client::view::NewCellView}]{\nogloxy{\texttt{swedesigner::client::view::NewCellView}}}\\ \hline

RFO4.4.4.1 & \hyperref[\nogloxy{swedesigner::client::view::DetailsView}]{\nogloxy{\texttt{swedesigner::client::view::DetailsView}}}\\ \hline

RFO4.5 & \hyperref[\nogloxy{swedesigner::client::model::celltypes::HxFor}]{\nogloxy{\texttt{swedesigner::client::model::celltypes::-\linebreak HxFor}}}\\
& \hyperref[\nogloxy{swedesigner::client::model::NewCellFactory}]{\nogloxy{\texttt{swedesigner::client::model::-\linebreak NewCellFactory}}}\\
& \hyperref[\nogloxy{swedesigner::client::model::NewCellModel}]{\nogloxy{\texttt{swedesigner::client::model::-\linebreak NewCellModel}}}\\
& \hyperref[\nogloxy{swedesigner::client::view::DetailsView}]{\nogloxy{\texttt{swedesigner::client::view::DetailsView}}}\\
& \hyperref[\nogloxy{swedesigner::client::view::NewCellView}]{\nogloxy{\texttt{swedesigner::client::view::NewCellView}}}\\ \hline

RFO4.5.6.1 & \hyperref[\nogloxy{swedesigner::client::view::DetailsView}]{\nogloxy{\texttt{swedesigner::client::view::DetailsView}}}\\ \hline

RFO4.6 & \hyperref[\nogloxy{swedesigner::client::model::celltypes::HxCustom}]{\nogloxy{\texttt{swedesigner::client::model::celltypes::-\linebreak HxCustom}}}\\
& \hyperref[\nogloxy{swedesigner::client::model::NewCellFactory}]{\nogloxy{\texttt{swedesigner::client::model::-\linebreak NewCellFactory}}}\\
& \hyperref[\nogloxy{swedesigner::client::model::NewCellModel}]{\nogloxy{\texttt{swedesigner::client::model::-\linebreak NewCellModel}}}\\
& \hyperref[\nogloxy{swedesigner::client::view::DetailsView}]{\nogloxy{\texttt{swedesigner::client::view::DetailsView}}}\\
& \hyperref[\nogloxy{swedesigner::client::view::NewCellView}]{\nogloxy{\texttt{swedesigner::client::view::NewCellView}}}\\ \hline

RFO4.20 & \hyperref[\nogloxy{swedesigner::client::view::DetailsView}]{\nogloxy{\texttt{swedesigner::client::view::DetailsView}}}\\ \hline

RFO4.21 & \hyperref[\nogloxy{swedesigner::client::view::DetailsView}]{\nogloxy{\texttt{swedesigner::client::view::DetailsView}}}\\ \hline

RFO4.22 & \hyperref[\nogloxy{swedesigner::client::view::DetailsView}]{\nogloxy{\texttt{swedesigner::client::view::DetailsView}}}\\ \hline

RFO4.23 & \hyperref[\nogloxy{swedesigner::client::view::DetailsView}]{\nogloxy{\texttt{swedesigner::client::view::DetailsView}}}\\ \hline

RFO4.24 & \hyperref[\nogloxy{swedesigner::client::view::DetailsView}]{\nogloxy{\texttt{swedesigner::client::view::DetailsView}}}\\ \hline

RFO4.25 & \hyperref[\nogloxy{swedesigner::client::view::DetailsView}]{\nogloxy{\texttt{swedesigner::client::view::DetailsView}}}\\ \hline

RFO4.26 & \hyperref[\nogloxy{swedesigner::client::model::celltypes::HxReturn}]{\nogloxy{\texttt{swedesigner::client::model::celltypes::-\linebreak HxReturn}}}\\
& \hyperref[\nogloxy{swedesigner::client::view::NewCellView}]{\nogloxy{\texttt{swedesigner::client::view::NewCellView}}}\\ \hline

RFO5 & \hyperref[\nogloxy{swedesigner::client::model::ProjectCommand}]{\nogloxy{\texttt{swedesigner::client::model::-\linebreak ProjectCommand}}}\\
& \hyperref[\nogloxy{swedesigner::client::model::utility::ProjectSaver}]{\nogloxy{\texttt{swedesigner::client::model::utility::-\linebreak ProjectSaver}}}\\
& \hyperref[\nogloxy{swedesigner::client::view::AppView}]{\nogloxy{\texttt{swedesigner::client::view::AppView}}}\\ \hline

RFO6 & \hyperref[\nogloxy{swedesigner::client::model::ProjectCommand}]{\nogloxy{\texttt{swedesigner::client::model::-\linebreak ProjectCommand}}}\\
& \hyperref[\nogloxy{swedesigner::client::model::utility::ProjectGenerator}]{\nogloxy{\texttt{swedesigner::client::model::utility::-\linebreak ProjectGenerator}}}\\
& \hyperref[\nogloxy{swedesigner::client::view::AppView}]{\nogloxy{\texttt{swedesigner::client::view::AppView}}}\\ \hline

RFO6.1 & \hyperref[\nogloxy{swedesigner::server::project::ParsedException}]{\nogloxy{\texttt{swedesigner::server::project::-\linebreak ParsedException}}}\\ \hline

RFO7 & \hyperref[\nogloxy{swedesigner::server::controller::RequestHandlerController}]{\nogloxy{\texttt{swedesigner::server::controller::-\linebreak RequestHandlerController}}}\\ \hline

RFO8 & \hyperref[\nogloxy{swedesigner::server::controller::RequestHandlerController}]{\nogloxy{\texttt{swedesigner::server::controller::-\linebreak RequestHandlerController}}}\\ \hline

RFO8.1 & \hyperref[\nogloxy{swedesigner::server::parser::Parser}]{\nogloxy{\texttt{swedesigner::server::parser::Parser}}}\\
& \hyperref[\nogloxy{swedesigner::server::project::ParsedAttribute}]{\nogloxy{\texttt{swedesigner::server::project::-\linebreak ParsedAttribute}}}\\
& \hyperref[\nogloxy{swedesigner::server::project::ParsedClass}]{\nogloxy{\texttt{swedesigner::server::project::-\linebreak ParsedClass}}}\\
& \hyperref[\nogloxy{swedesigner::server::project::ParsedCustom}]{\nogloxy{\texttt{swedesigner::server::project::-\linebreak ParsedCustom}}}\\
& \hyperref[\nogloxy{swedesigner::server::project::ParsedElement}]{\nogloxy{\texttt{swedesigner::server::project::-\linebreak ParsedElement}}}\\
& \hyperref[\nogloxy{swedesigner::server::project::ParsedElse}]{\nogloxy{\texttt{swedesigner::server::project::-\linebreak ParsedElse}}}\\
& \hyperref[\nogloxy{swedesigner::server::project::ParsedFor}]{\nogloxy{\texttt{swedesigner::server::project::-\linebreak ParsedFor}}}\\
& \hyperref[\nogloxy{swedesigner::server::project::ParsedIf}]{\nogloxy{\texttt{swedesigner::server::project::ParsedIf}}}\\
& \hyperref[\nogloxy{swedesigner::server::project::ParsedInstruction}]{\nogloxy{\texttt{swedesigner::server::project::-\linebreak ParsedInstruction}}}\\
& \hyperref[\nogloxy{swedesigner::server::project::ParsedInterface}]{\nogloxy{\texttt{swedesigner::server::project::-\linebreak ParsedInterface}}}\\
& \hyperref[\nogloxy{swedesigner::server::project::ParsedMethod}]{\nogloxy{\texttt{swedesigner::server::project::-\linebreak ParsedMethod}}}\\
& \hyperref[\nogloxy{swedesigner::server::project::ParsedProgram}]{\nogloxy{\texttt{swedesigner::server::project::-\linebreak ParsedProgram}}}\\
& \hyperref[\nogloxy{swedesigner::server::project::ParsedReturn}]{\nogloxy{\texttt{swedesigner::server::project::-\linebreak ParsedReturn}}}\\
& \hyperref[\nogloxy{swedesigner::server::project::ParsedStatement}]{\nogloxy{\texttt{swedesigner::server::project::-\linebreak ParsedStatement}}}\\
& \hyperref[\nogloxy{swedesigner::server::project::ParsedType}]{\nogloxy{\texttt{swedesigner::server::project::-\linebreak ParsedType}}}\\
& \hyperref[\nogloxy{swedesigner::server::project::ParsedWhile}]{\nogloxy{\texttt{swedesigner::server::project::-\linebreak ParsedWhile}}}\\ \hline

RFO8.2 & \hyperref[\nogloxy{swedesigner::server::generator::Generator}]{\nogloxy{\texttt{swedesigner::server::generator::-\linebreak Generator}}}\\
& \hyperref[\nogloxy{swedesigner::server::generator::java::JavaGenerator}]{\nogloxy{\texttt{swedesigner::server::generator::java::-\linebreak JavaGenerator}}}\\
& \hyperref[\nogloxy{swedesigner::server::template::java::JavaTemplate}]{\nogloxy{\texttt{swedesigner::server::template::java::-\linebreak JavaTemplate}}}\\
& \hyperref[\nogloxy{swedesigner::server::template::Template}]{\nogloxy{\texttt{swedesigner::server::template::-\linebreak Template}}}\\ \hline

RFO9 & \hyperref[\nogloxy{swedesigner::server::compiler::Compiler}]{\nogloxy{\texttt{swedesigner::server::compiler::-\linebreak Compiler}}}\\
& \hyperref[\nogloxy{swedesigner::server::compiler::CompilerAssembler}]{\nogloxy{\texttt{swedesigner::server::compiler::-\linebreak CompilerAssembler}}}\\
& \hyperref[\nogloxy{swedesigner::server::compiler::java::JavaCompiler}]{\nogloxy{\texttt{swedesigner::server::compiler::java::-\linebreak JavaCompiler}}}\\
& \hyperref[\nogloxy{swedesigner::server::controller::RequestHandlerController}]{\nogloxy{\texttt{swedesigner::server::controller::-\linebreak RequestHandlerController}}}\\ \hline

RFO10 & \hyperref[\nogloxy{swedesigner::server::controller::RequestHandlerController}]{\nogloxy{\texttt{swedesigner::server::controller::-\linebreak RequestHandlerController}}}\\
& \hyperref[\nogloxy{swedesigner::server::utility::Compressor}]{\nogloxy{\texttt{swedesigner::server::utility::-\linebreak Compressor}}}\\ \hline

RFO11 & \hyperref[\nogloxy{swedesigner::server::controller::RequestHandlerController}]{\nogloxy{\texttt{swedesigner::server::controller::-\linebreak RequestHandlerController}}}\\ \hline

\caption[Tracciamento Requisiti-Classi]{Tracciamento Requisiti-Classi}
\label{tabella:requi-class}
\end{longtable}




\end{document}
