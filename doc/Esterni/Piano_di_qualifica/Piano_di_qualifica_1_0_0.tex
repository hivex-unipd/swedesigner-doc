%Piano di Qualifica
% da compilare con il comando pdflatex Piano_di_Qualifica_x.x.x.tex

% Dichiarazioni di ambiente e inclusione di pacchetti
% da usare tramite il comando % Dichiarazioni di ambiente e inclusione di pacchetti
% da usare tramite il comando % Dichiarazioni di ambiente e inclusione di pacchetti
% da usare tramite il comando \input{../../util/hx-ambiente}

\documentclass[a4paper,titlepage]{article}
\usepackage[T1]{fontenc}
\usepackage[utf8]{inputenc}
\usepackage[english,italian]{babel}
\usepackage{microtype}
\usepackage{lmodern}
\usepackage{underscore}
\usepackage{graphicx}
\usepackage{eurosym}
\usepackage{float}
\usepackage{fancyhdr}
\usepackage[table,dvipsnames]{xcolor}
\usepackage{multirow}
\usepackage{longtable}
\usepackage{chngpage}
\usepackage{grffile}
\usepackage[titles]{tocloft}
\usepackage{hyperref}
\hypersetup{hidelinks}

\usepackage{../../util/hx-vers}
\usepackage{../../util/hx-macro}
\usepackage{../../util/hx-front}

% solo se si vuole una nuova pagina ad ogni \section:
\usepackage{titlesec}
\newcommand{\sectionbreak}{\clearpage}

% stile di pagina:
\pagestyle{fancy}

% solo se si vuole eliminare l'indentazione ad ogni paragrafo:
\setlength{\parindent}{0pt}

% intestazione:
\lhead{\Large{\proj}}
\rhead{\includegraphics[keepaspectratio=true,width=50px]{../../util/hivex_logo2.png}}
\renewcommand{\headrulewidth}{0.4pt}

% pie' di pagina:
\lfoot{\email}
\rfoot{\thepage}
\cfoot{}
\renewcommand{\footrulewidth}{0.4pt}

% spazio verticale tra le celle di una tabella:
\renewcommand{\arraystretch}{1.5}

% profondità di indicizzazione:
\setcounter{tocdepth}{4}
\setcounter{secnumdepth}{4}

% numerazione innestata per elenchi numerati:
\renewcommand{\labelenumii}{\theenumii}
\renewcommand{\theenumii}{\theenumi.\arabic{enumii}.}


\documentclass[a4paper,titlepage]{article}
\usepackage[T1]{fontenc}
\usepackage[utf8]{inputenc}
\usepackage[english,italian]{babel}
\usepackage{microtype}
\usepackage{lmodern}
\usepackage{underscore}
\usepackage{graphicx}
\usepackage{eurosym}
\usepackage{float}
\usepackage{fancyhdr}
\usepackage[table,dvipsnames]{xcolor}
\usepackage{multirow}
\usepackage{longtable}
\usepackage{chngpage}
\usepackage{grffile}
\usepackage[titles]{tocloft}
\usepackage{hyperref}
\hypersetup{hidelinks}

\usepackage{../../util/hx-vers}
\usepackage{../../util/hx-macro}
\usepackage{../../util/hx-front}

% solo se si vuole una nuova pagina ad ogni \section:
\usepackage{titlesec}
\newcommand{\sectionbreak}{\clearpage}

% stile di pagina:
\pagestyle{fancy}

% solo se si vuole eliminare l'indentazione ad ogni paragrafo:
\setlength{\parindent}{0pt}

% intestazione:
\lhead{\Large{\proj}}
\rhead{\includegraphics[keepaspectratio=true,width=50px]{../../util/hivex_logo2.png}}
\renewcommand{\headrulewidth}{0.4pt}

% pie' di pagina:
\lfoot{\email}
\rfoot{\thepage}
\cfoot{}
\renewcommand{\footrulewidth}{0.4pt}

% spazio verticale tra le celle di una tabella:
\renewcommand{\arraystretch}{1.5}

% profondità di indicizzazione:
\setcounter{tocdepth}{4}
\setcounter{secnumdepth}{4}

% numerazione innestata per elenchi numerati:
\renewcommand{\labelenumii}{\theenumii}
\renewcommand{\theenumii}{\theenumi.\arabic{enumii}.}


\documentclass[a4paper,titlepage]{article}
\usepackage[T1]{fontenc}
\usepackage[utf8]{inputenc}
\usepackage[english,italian]{babel}
\usepackage{microtype}
\usepackage{lmodern}
\usepackage{underscore}
\usepackage{graphicx}
\usepackage{eurosym}
\usepackage{float}
\usepackage{fancyhdr}
\usepackage[table,dvipsnames]{xcolor}
\usepackage{multirow}
\usepackage{longtable}
\usepackage{chngpage}
\usepackage{grffile}
\usepackage[titles]{tocloft}
\usepackage{hyperref}
\hypersetup{hidelinks}

\usepackage{../../util/hx-vers}
\usepackage{../../util/hx-macro}
\usepackage{../../util/hx-front}

% solo se si vuole una nuova pagina ad ogni \section:
\usepackage{titlesec}
\newcommand{\sectionbreak}{\clearpage}

% stile di pagina:
\pagestyle{fancy}

% solo se si vuole eliminare l'indentazione ad ogni paragrafo:
\setlength{\parindent}{0pt}

% intestazione:
\lhead{\Large{\proj}}
\rhead{\includegraphics[keepaspectratio=true,width=50px]{../../util/hivex_logo2.png}}
\renewcommand{\headrulewidth}{0.4pt}

% pie' di pagina:
\lfoot{\email}
\rfoot{\thepage}
\cfoot{}
\renewcommand{\footrulewidth}{0.4pt}

% spazio verticale tra le celle di una tabella:
\renewcommand{\arraystretch}{1.5}

% profondità di indicizzazione:
\setcounter{tocdepth}{4}
\setcounter{secnumdepth}{4}

% numerazione innestata per elenchi numerati:
\renewcommand{\labelenumii}{\theenumii}
\renewcommand{\theenumii}{\theenumi.\arabic{enumii}.}

\usepackage{float}

\version{1.0.0}
\creaz{28 dicembre 2016}
\author{\LS, \AZ}
\supervisor{\GG}
\uso{esterno}
\dest{\TV, \RC, \ZU}
\title{Piano di Qualifica}
% \date{9 gennaio 2017}

\begin{document}
\maketitle
% diario delle modifiche per l'analisi dei requisiti
% da includere con % diario delle modifiche per l'analisi dei requisiti
% da includere con % diario delle modifiche per l'analisi dei requisiti
% da includere con \include{diario}

\begin{diario}
	4.0.0 & {\LB} (Responsabile) & 02/05/2017 & Approvazione del documento \\ \hline
	3.1.0 & {\PB} (Verificatore) & 02/05/2017 & Verifica del documento \\ \hline
	3.0.1 & {\MM} (Analista) & 01/05/2017 & 
	\begin{itemize}
	\item Inserimento UC5.35 e relativo requisito;
	\item Inserimento UC8 e relativo requisito;
	\item Inserimento tabella Requisiti Implementati come appendice.
\end{itemize}\\ \hline
	3.0.0 & {\AZ} (Responsabile) & 19/03/2017 & Approvazione del documento \\ \hline
	2.1.0 & {\MM} (Verificatore) & 19/03/2017 & Verifica del documento \\ \hline
	2.0.3 & {\PB} (Progettista) & 18/03/2017 &  
\begin{itemize}
	\item Modifica tabella Tracciamento Fonti-Requisiti;
	\item Modifica tabella Requisiti-Fonti;
	\item Modifica Estensione UC7.
\end{itemize}\\ \hline
	2.0.2 & {\PB} (Progettista) & 17/03/2017 &  Ristrutturato UC5 e relativi requisiti\\ \hline
	2.0.1 & {\PB} (Progettista) & 16/03/2017 &  Ristrutturato UC4 e relativi requisiti\\ \hline
	2.0.0 & {\LS} (Responsabile) & 01/02/2017 & Approvazione del documento \\ \hline
	1.1.0 & {\GG} (Verificatore) & 01/02/2017 & Verifica del documento \\ \hline
	1.0.4 & {\AZ} (Analista) & 31/01/2017 & Inserito UC5.26 con relativo requisito e tracciamento nelle tabelle e inseriti i requisiti RFO7, RFO8, RFO8.1, RFO8.2, RFO9, RFO10 e RFO11\\ \hline
	1.0.3 & {\AZ} (Analista) & 29/01/2017 & Corretta la descrizione dello UC5 e approfondita la descrizione dello UC7 \\ \hline
	1.0.2 & {\AZ} (Analista) & 28/01/2017 & Corretti UC4.1.6.3.2, UC4.2.1 e inserito perimetro sistema del UC5\\ \hline
	1.0.1 & {\AZ} (Analista) & 26/01/2017 & Inserimento scenario alternativo allo UC2, creazione UC3.1 con relativo requisito e tracciamento nelle tabelle e corrette alcune postcondizioni \\ \hline
	1.0.0 & {\LB} (Responsabile) & 09/01/2017 & Approvazione documento \\ \hline
	0.4.0 & {\LS} (Verificatore) & 06/01/2017 & Verifica introduzione, descrizione generale e requisiti \\ \hline
	0.3.0 & {\MM} (Verificatore) & 06/01/2017 & Verifica UC5.3-UC7 \\ \hline
	0.2.0 & {\LB} (Verificatore) & 06/01/2017 & Verifica UC4.2-UC5.2 \\ \hline
	0.1.0 & {\AZ} (Verificatore) & 06/01/2017 & Verifica UC1-4.1.8 \\ \hline
	0.0.11 & {\LS} (Analista) & 04/01/2017 & Stesura UC6-UC7 \\ \hline
	0.0.10 & {\GG} (Analista) & 03/01/2017 & Stesura UC5.6-UC5.18 \\ \hline
	0.0.9 & {\LS} (Analista) & 03/01/2017 & Stesura UC5.3-UC5.5.6.1 \\ \hline
	0.0.8 & {\PB} (Analista) & 02/01/2017 & Stesura UC5-UC5.2 \\ \hline
	0.0.7 & {\AZ} (Analista) & 02/01/2017 & Stesura UC4.3.3.1-UC4.11 \\ \hline
	0.0.6 & {\MM} (Analista) & 30/12/2016 & Stesura UC4.2-UC4.3.3.1 \\ \hline
	0.0.5 & {\GG} (Analista) & 29/12/2016 & Stesura UC4.1.6-UC4.1.8 \\ \hline
	0.0.4 & {\PB} (Analista) & 29/12/2016 & Stesura UC4-UC4.1.5 \\ \hline
	0.0.3 & {\LB} (Analista) & 28/12/2016 & Stesura UC1-UC2-UC3 \\ \hline
	0.0.2 & {\LS} (Analista) & 27/12/2016 & Stesura introduzione e descrizione generale \\ \hline
	0.0.1 & {\AZ} (Analista) & 27/12/2016 & Stesura scheletro \\ \hline
\end{diario}


\begin{diario}
	4.0.0 & {\LB} (Responsabile) & 02/05/2017 & Approvazione del documento \\ \hline
	3.1.0 & {\PB} (Verificatore) & 02/05/2017 & Verifica del documento \\ \hline
	3.0.1 & {\MM} (Analista) & 01/05/2017 & 
	\begin{itemize}
	\item Inserimento UC5.35 e relativo requisito;
	\item Inserimento UC8 e relativo requisito;
	\item Inserimento tabella Requisiti Implementati come appendice.
\end{itemize}\\ \hline
	3.0.0 & {\AZ} (Responsabile) & 19/03/2017 & Approvazione del documento \\ \hline
	2.1.0 & {\MM} (Verificatore) & 19/03/2017 & Verifica del documento \\ \hline
	2.0.3 & {\PB} (Progettista) & 18/03/2017 &  
\begin{itemize}
	\item Modifica tabella Tracciamento Fonti-Requisiti;
	\item Modifica tabella Requisiti-Fonti;
	\item Modifica Estensione UC7.
\end{itemize}\\ \hline
	2.0.2 & {\PB} (Progettista) & 17/03/2017 &  Ristrutturato UC5 e relativi requisiti\\ \hline
	2.0.1 & {\PB} (Progettista) & 16/03/2017 &  Ristrutturato UC4 e relativi requisiti\\ \hline
	2.0.0 & {\LS} (Responsabile) & 01/02/2017 & Approvazione del documento \\ \hline
	1.1.0 & {\GG} (Verificatore) & 01/02/2017 & Verifica del documento \\ \hline
	1.0.4 & {\AZ} (Analista) & 31/01/2017 & Inserito UC5.26 con relativo requisito e tracciamento nelle tabelle e inseriti i requisiti RFO7, RFO8, RFO8.1, RFO8.2, RFO9, RFO10 e RFO11\\ \hline
	1.0.3 & {\AZ} (Analista) & 29/01/2017 & Corretta la descrizione dello UC5 e approfondita la descrizione dello UC7 \\ \hline
	1.0.2 & {\AZ} (Analista) & 28/01/2017 & Corretti UC4.1.6.3.2, UC4.2.1 e inserito perimetro sistema del UC5\\ \hline
	1.0.1 & {\AZ} (Analista) & 26/01/2017 & Inserimento scenario alternativo allo UC2, creazione UC3.1 con relativo requisito e tracciamento nelle tabelle e corrette alcune postcondizioni \\ \hline
	1.0.0 & {\LB} (Responsabile) & 09/01/2017 & Approvazione documento \\ \hline
	0.4.0 & {\LS} (Verificatore) & 06/01/2017 & Verifica introduzione, descrizione generale e requisiti \\ \hline
	0.3.0 & {\MM} (Verificatore) & 06/01/2017 & Verifica UC5.3-UC7 \\ \hline
	0.2.0 & {\LB} (Verificatore) & 06/01/2017 & Verifica UC4.2-UC5.2 \\ \hline
	0.1.0 & {\AZ} (Verificatore) & 06/01/2017 & Verifica UC1-4.1.8 \\ \hline
	0.0.11 & {\LS} (Analista) & 04/01/2017 & Stesura UC6-UC7 \\ \hline
	0.0.10 & {\GG} (Analista) & 03/01/2017 & Stesura UC5.6-UC5.18 \\ \hline
	0.0.9 & {\LS} (Analista) & 03/01/2017 & Stesura UC5.3-UC5.5.6.1 \\ \hline
	0.0.8 & {\PB} (Analista) & 02/01/2017 & Stesura UC5-UC5.2 \\ \hline
	0.0.7 & {\AZ} (Analista) & 02/01/2017 & Stesura UC4.3.3.1-UC4.11 \\ \hline
	0.0.6 & {\MM} (Analista) & 30/12/2016 & Stesura UC4.2-UC4.3.3.1 \\ \hline
	0.0.5 & {\GG} (Analista) & 29/12/2016 & Stesura UC4.1.6-UC4.1.8 \\ \hline
	0.0.4 & {\PB} (Analista) & 29/12/2016 & Stesura UC4-UC4.1.5 \\ \hline
	0.0.3 & {\LB} (Analista) & 28/12/2016 & Stesura UC1-UC2-UC3 \\ \hline
	0.0.2 & {\LS} (Analista) & 27/12/2016 & Stesura introduzione e descrizione generale \\ \hline
	0.0.1 & {\AZ} (Analista) & 27/12/2016 & Stesura scheletro \\ \hline
\end{diario}


\begin{diario}
	4.0.0 & {\LB} (Responsabile) & 02/05/2017 & Approvazione del documento \\ \hline
	3.1.0 & {\PB} (Verificatore) & 02/05/2017 & Verifica del documento \\ \hline
	3.0.1 & {\MM} (Analista) & 01/05/2017 & 
	\begin{itemize}
	\item Inserimento UC5.35 e relativo requisito;
	\item Inserimento UC8 e relativo requisito;
	\item Inserimento tabella Requisiti Implementati come appendice.
\end{itemize}\\ \hline
	3.0.0 & {\AZ} (Responsabile) & 19/03/2017 & Approvazione del documento \\ \hline
	2.1.0 & {\MM} (Verificatore) & 19/03/2017 & Verifica del documento \\ \hline
	2.0.3 & {\PB} (Progettista) & 18/03/2017 &  
\begin{itemize}
	\item Modifica tabella Tracciamento Fonti-Requisiti;
	\item Modifica tabella Requisiti-Fonti;
	\item Modifica Estensione UC7.
\end{itemize}\\ \hline
	2.0.2 & {\PB} (Progettista) & 17/03/2017 &  Ristrutturato UC5 e relativi requisiti\\ \hline
	2.0.1 & {\PB} (Progettista) & 16/03/2017 &  Ristrutturato UC4 e relativi requisiti\\ \hline
	2.0.0 & {\LS} (Responsabile) & 01/02/2017 & Approvazione del documento \\ \hline
	1.1.0 & {\GG} (Verificatore) & 01/02/2017 & Verifica del documento \\ \hline
	1.0.4 & {\AZ} (Analista) & 31/01/2017 & Inserito UC5.26 con relativo requisito e tracciamento nelle tabelle e inseriti i requisiti RFO7, RFO8, RFO8.1, RFO8.2, RFO9, RFO10 e RFO11\\ \hline
	1.0.3 & {\AZ} (Analista) & 29/01/2017 & Corretta la descrizione dello UC5 e approfondita la descrizione dello UC7 \\ \hline
	1.0.2 & {\AZ} (Analista) & 28/01/2017 & Corretti UC4.1.6.3.2, UC4.2.1 e inserito perimetro sistema del UC5\\ \hline
	1.0.1 & {\AZ} (Analista) & 26/01/2017 & Inserimento scenario alternativo allo UC2, creazione UC3.1 con relativo requisito e tracciamento nelle tabelle e corrette alcune postcondizioni \\ \hline
	1.0.0 & {\LB} (Responsabile) & 09/01/2017 & Approvazione documento \\ \hline
	0.4.0 & {\LS} (Verificatore) & 06/01/2017 & Verifica introduzione, descrizione generale e requisiti \\ \hline
	0.3.0 & {\MM} (Verificatore) & 06/01/2017 & Verifica UC5.3-UC7 \\ \hline
	0.2.0 & {\LB} (Verificatore) & 06/01/2017 & Verifica UC4.2-UC5.2 \\ \hline
	0.1.0 & {\AZ} (Verificatore) & 06/01/2017 & Verifica UC1-4.1.8 \\ \hline
	0.0.11 & {\LS} (Analista) & 04/01/2017 & Stesura UC6-UC7 \\ \hline
	0.0.10 & {\GG} (Analista) & 03/01/2017 & Stesura UC5.6-UC5.18 \\ \hline
	0.0.9 & {\LS} (Analista) & 03/01/2017 & Stesura UC5.3-UC5.5.6.1 \\ \hline
	0.0.8 & {\PB} (Analista) & 02/01/2017 & Stesura UC5-UC5.2 \\ \hline
	0.0.7 & {\AZ} (Analista) & 02/01/2017 & Stesura UC4.3.3.1-UC4.11 \\ \hline
	0.0.6 & {\MM} (Analista) & 30/12/2016 & Stesura UC4.2-UC4.3.3.1 \\ \hline
	0.0.5 & {\GG} (Analista) & 29/12/2016 & Stesura UC4.1.6-UC4.1.8 \\ \hline
	0.0.4 & {\PB} (Analista) & 29/12/2016 & Stesura UC4-UC4.1.5 \\ \hline
	0.0.3 & {\LB} (Analista) & 28/12/2016 & Stesura UC1-UC2-UC3 \\ \hline
	0.0.2 & {\LS} (Analista) & 27/12/2016 & Stesura introduzione e descrizione generale \\ \hline
	0.0.1 & {\AZ} (Analista) & 27/12/2016 & Stesura scheletro \\ \hline
\end{diario}

\tableofcontents



\section{Introduzione}
	\subsection{Scopo del documento}
	Questo documento ha lo scopo di fissare le strategie di verifica e validazione che il gruppo \hx{} ha deciso di adottare per perseguire gli obiettivi di qualità di processo e di prodotto relativi al progetto \proj{}. Il presente documento si propone di descrivere l'approccio del gruppo alle diverse fasi di verifica per poter ottenere il miglior risultato auspicabile in termini di qualità. Per perseguire tali obiettivi e risultati occorre verificare continuamente le attività svolte in modo da individuare e correggere velocemente le anomalie, minimizzando così l'utilizzo di risorse e allo stesso tempo mantenendo la correttezza del prodotto.

	\subsection{Scopo del prodotto}
	\scopo
	
	\subsection{Glossario}
	\presgloss
	
	\subsection{Riferimenti}
		\subsubsection{Riferimenti normativi}
		\begin{itemize}
			\item \NdP;
			\item \textbf{Capitolato d'Appalto C6, \proj}: \\
			\url{www.math.unipd.it/~tullio/IS-1/2016/Progetto/C6.pdf}, visitato il 08/02/2017;
			\item \textbf{Standard ISO/IEC 12207:2008}: \\
			\url{ieeexplore.ieee.org/document/4475826/}, visitato il 08/02/2017;
			\item \textbf{Standard ISO/IEC 15504}: \\
			\url{en.wikipedia.org/wiki/ISO/IEC_15504}, visitato il 08/02/2017;
			\item \textbf{Standard ISO/IEC 9126}: \\
			\url{it.wikipedia.org/wiki/ISO/IEC_9126}, visitato il 08/02/2017;
			\item \textbf{PDCA (\emph{Plan, Do, Check, Act})}: \\
			\url{it.wikipedia.org/wiki/Ciclo_di_Deming}, visitato il 08/02/2017.
		\end{itemize}
		
		\subsubsection{Riferimenti informativi}
		\begin{itemize}
			\item \textbf{Analisi dei Requisiti}: \AdR;
			\item \textbf{Piano di Progetto}: \PdP;
			\item \textbf{Glossario}: \Glossario;
			\item \textbf{Indice di Gulpease}: \\
			\url{it.wikipedia.org/wiki/Indice_Gulpease}, visitato il 08/02/2017;
			\item \textbf{Regole del Progetto Didattico}: \\
			\url{www.math.unipd.it/~tullio/IS-1/2016/Dispense/L09.pdf}, visitato il 08/02/2017;
			\item \textbf{Slide Qualità del Software}: \\
			\url{www.math.unipd.it/~tullio/IS-1/2016/Dispense/L10.pdf}, visitato il 08/02/2017;
			\item \textbf{Slide Qualità del Prodotto}: \\
			\url{www.math.unipd.it/~tullio/IS-1/2016/Dispense/L11.pdf}, visitato il 08/02/2017.
		\end{itemize}



\section{Visione generale della strategia di gestione della qualità}
	\subsection{Obiettivi di qualità}
	In questa sezione vengono illustrati in modo quanto più completo ed esaustivo gli obiettivi di qualità che il gruppo intende perseguire nel corso 	dello svolgimento del progetto, opportunamente declinati nelle due sottosezioni Qualità di processo e Qualità di prodotto.

	\subsection{Qualità di processo}
	È indubbio che la qualità di un prodotto software non possa in alcun modo prescindere dalla qualità dei diversi processi che concorrono nel definirlo e 	che anzi ne sia intrinsecamente dipendente. Con tale consapevolezza il gruppo ha scelto come riferimento per la valutazione della qualità dei propri processi lo standard ISO/IEC 15504 altresì conosciuto sotto l'acronimo di \gloss{SPICE}.
		\subsubsection{Procedure per il controllo della qualità di processo}
		L'adozione dello standard SPICE, in combinazione con l'approccio automigliorativo del ciclo PDCA, è alla base della strategia prevista del gruppo per garantire una sempre maggiore qualità dei processi in atto e dei prodotti risultanti.
		\\Il documento \PdP{} stabilisce gli obiettivi del gruppo in relazione alla pianificazione in dettaglio dei vari processi e alla ripartizione delle risorse ad essi assegnate per un efficace svolgimento. Altri obiettivi di qualità sono illustrati successivamente in questo documento.
		\\Le metriche utilizzate dal gruppo per valutare la qualità dei processi, commisurarla agli obiettivi stabiliti e stabilire eventuali azioni correttive di miglioramento sono definiti in dettaglio nella sezione \ref{S5} del seguente documento. Da sottolineare che anche le metriche relative alla qualità del prodotto software costituiscono un'indicazione preziosa per valutare la qualità dei processi in atto: infatti un prodotto di bassa qualità è indicativo di un processo da migliorarsi.

		\subsubsection{Obiettivi di qualità di processo}
		In questa sezione vengono elencati i principali processi software identificati dal team sulla base dello standard ISO/IEC 12207:2008 e per ognuno di essi gli obiettivi di qualità perseguiti.
			
			\paragraph{Pianificazione di progetto e processo di verifica e controllo}
			Questo macro-processo, derivato dall'unione dei processi 6.3.1 e 6.3.2 previsti dallo standard ISO/IEC 12207:2008, ha lo scopo di pianificare le attività richieste dal progetto.
			\\Gli obiettivi di qualità stabiliti per questo processo sono i seguenti:
				\subparagraph{Rispetto della pianificazione} 
				Ogni attività di un processo verrà svolta da parte di colui a cui è stata assegnata, rispettando la pianificazione temporale stabilita nel documento \PdP{} e svolgendo tutti i compiti in cui essa si articola.
				Gli obiettivi di pianificazione e le metriche ad essi associati sono i seguenti:
		 		\begin{itemize}
					\item \textbf{schedule variance}: l'obiettivo stabilito per la metrica è il valore 0.
				\end{itemize}
				\subparagraph{Rispetto del budget}   
				I costi di ogni processo dovranno rientrare nel budget previsto dal documento \PdP.
				Gli obiettivi di efficienza e le metriche ad essi associati sono i seguenti:
		 		\begin{itemize}
					\item \textbf{budget variance}: l'obiettivo stabilito per la metrica è il valore 0.
				\end{itemize}
				
			\paragraph{Processo di gestione della documentazione del software}
			Il processo, corrispondente al processo 7.2.1 definito dallo standard ISO/IEC 12207:2008, ha lo scopo di produrre e gestire la documentazione relativa alle funzionalità e alle caratteristiche del sistema software prodotto.
			\\Gli obiettivi di qualità stabiliti per questo processo sono i seguenti:
				\subparagraph{Leggibilità della documentazione}
				La documentazione prodotta sarà quanto più chiara e comprensibile a tutti gli \gloss{stakeholder} coinvolti nel progetto.
				Gli obiettivi relativi alla documentazione e le metriche ad essi associati sono i seguenti:
		 		\begin{itemize}
					\item \textbf{indice di Gulpease}: l'obiettivo stabilito per la metrica è il range 60 - 100.
				\end{itemize}
				
			\paragraph{Processo di architettura del software}
			Il processo, corrispondente al processo 7.1.3 dello standard ISO/IEC 12207:2008, ha lo scopo di definire un'architettura software che implementi i requisiti corrispondenti e identifichi le diverse componenti del sistema.
			Gli obiettivi di qualità stabiliti per questo processo sono i seguenti:
				\subparagraph{Completezza}
				L'architettura prodotta rispetta le regole di progettazione che ne garantiscono la completezza. Ad esempio si evitano package senza elementi, classi non usate, parametri senza un tipo, ecc.
				Gli obiettivi relativi alla completezza dell'architettura e le metriche ad essi associati sono i seguenti:
		 		\begin{itemize}
					\item \textbf{Numero di violazioni della completezza architetturale di alta importanza}: l'obiettivo stabilito per la metrica è il valore 0;
					\item \textbf{Numero di violazioni della completezza architetturale di media importanza}: l'obiettivo stabilito per la metrica è il valore 0;
					\item \textbf{Numero di violazioni della completezza architetturale di bassa importanza}: l'obiettivo stabilito per la metrica è il valore 0.
				\end{itemize}
				\subparagraph{Correttezza}
				L'architettura prodotta rispetta le regole di progettazione che ne garantiscono la correttezza rispetto allo standard UML.
				Gli obiettivi relativi alla correttezza dell'architettura e le metriche ad essi associati sono i seguenti:
		 		\begin{itemize}
					\item \textbf{Numero di violazioni della correttezza architetturale di alta importanza}: l'obiettivo stabilito per la metrica è il valore 0;
					\item \textbf{Numero di violazioni della correttezza architetturale di media importanza}: l'obiettivo stabilito per la metrica è il valore 0;
					\item \textbf{Numero di violazioni della correttezza architetturale di bassa importanza}: l'obiettivo stabilito per la metrica è il valore 0.
				\end{itemize}
				\subparagraph{Stile}
				L'architettura prodotta non presenta problemi di design, che pur essendo legali per lo standard UML, posso compromettere la qualità generale del sistema. Ad esempio si evitano dipendenze circolari tra i pacchetti, lunghe liste di parametri, ecc.
				Gli obiettivi relativi allo stile dell'architettura e le metriche ad essi associati sono i seguenti:
		 		\begin{itemize}
					\item \textbf{Numero di violazioni dello stile architetturale di alta importanza}: l'obiettivo stabilito per la metrica è il valore 0;
					\item \textbf{Numero di violazioni dello stile architetturale di media importanza}: l'obiettivo stabilito per la metrica è il valore 0;
					\item \textbf{Numero di violazioni dello stile architetturale di bassa importanza}: l'obiettivo stabilito per la metrica è il valore 0.
				\end{itemize}
				
			\paragraph{Processo di costruzione del software}
			Il processo, corrispondente al processo 7.1.5 previsto dallo standard ISO/IEC 12207:2008, definisce le attività principali volte alla produzione di unità software eseguibili che riflettano quanto identificato a livello di progettazione.
			\\Gli obiettivi di qualità stabiliti per questo processo sono i seguenti:
				\subparagraph{Rispetto delle norme di codifica}
				Durante il processo di codifica vengono rispettate le norme stabilite nel \NdP{}.
				Gli obiettivi di rispetto delle norme di codifica e le metriche ad essi associati sono i seguenti:
		 		\begin{itemize}
					\item \textbf{numero di violazioni di alta importanza delle norme di codifica}: l'obiettivo stabilito per la metrica è il valore 0;
					\item \textbf{numero di violazioni di media importanza delle norme di codifica}: l'obiettivo stabilito per la metrica è il valore 0;
					\item \textbf{numero di violazioni di bassa importanza delle norme di codifica}: l'obiettivo stabilito per la metrica è il valore 0.
				\end{itemize}
				
			\paragraph{Processo di test per la qualifica del software/sistema}
			Il macro-processo, derivante dall'unione dei processi 6.4.6 e 7.1.7 dello standard ISO/IEC 12207:2008, ha lo scopo che ogni requisito individuato sia stato implementato nel prodotto.
			\\Gli obiettivi di qualità stabiliti per questo processo sono i seguenti:
				\subparagraph{Corretto funzionamento del sistema e integrazione delle componenti}
				Le funzionalità richieste sono state integrate all'interno di un prodotto software le cui diverse componenti interagiscono fra esse in modo coerente e senza criticità, a formare un sistema coeso e funzionante.
				Gli obiettivi di e le metriche ad essi associati sono i seguenti:
		 		\begin{itemize}
					\item \textbf{test di unità eseguiti}: l'obiettivo stabilito per la metrica è il valore 100;
					\item \textbf{test di integrazione eseguiti}: l'obiettivo stabilito per la metrica è il valore 100;
					\item \textbf{test di sistema eseguiti}: l'obiettivo stabilito per la metrica è il valore 100;
					\item \textbf{test di accettazione eseguiti}: l'obiettivo stabilito per la metrica è il valore 100;
				\end{itemize}
				
				\subparagraph{Copertura codice}
				I test effettuati sulle diverse componenti del prodotto software andranno a considerare una gran parte delle possibili casistiche d'utilizzo e ad esplorare una notevole fetta del codice sorgente ad esso relativo.
				\begin{itemize}
					\item \textbf{statement coverage}: l'obiettivo stabilito per la metrica è il valore 100;
					\item \textbf{branch coverage}: l'obiettivo stabilito per la metrica è il valore 100;
				\end{itemize}
	
	\subsection{Qualità di prodotto}
	Per garantire una buona qualità di prodotto, il gruppo \hx{} ha individuato dallo standard \gloss{ISO/IEC 9126:2001} le qualità che ritiene di maggior importanza durante il ciclo di vita del prodotto e ha individuato gli obiettivi e le metriche coerenti con i livelli di qualità stabiliti.
		\subsubsection{Procedure per il controllo delle qualità di prodotto}
		Il controllo di qualità del prodotto verrà garantito da:
		\begin{itemize}
			\item \textbf{quality assurance}: l'insieme di attività realizzate per garantire il raggiungimento degli obiettivi di qualità. Tali attività prevedono la realizzazione di tecniche di analisi statica e dinamica.
			\item \textbf{verifica}: il processo che stabilisce se il prodotto in uscita da una fase è consistente, corretto e completo. Per tutta la durata del progetto verranno svolte attività di verifica.
			\item \textbf{validazione}: la conferma oggettiva che il sistema soddisfa i requisiti.
		\end{itemize}
		\subsubsection{Obiettivi di qualità del prodotto}
		Gli obiettivi di qualità del software che il gruppo \hx{} desidera raggiungere nell'arco del progetto sono un sottoinsieme di quelli enunciati nello standard ISO/IEC 9126:2001.
		
		 	\paragraph{funzionalità}
		 	Il prodotto possiede tutte le funzionalità descritte all'interno dei requisiti obbligatori e gran parte delle funzionalità descritte all'interno dei requisiti desiderabili.
		 	Gli obiettivi di funzionalità e le metriche ad essi associati sono i seguenti:
		 	\begin{itemize}
				\item \textbf{copertura dei requisiti obbligatori}: l'obiettivo stabilito per la metrica è il valore 100;
				\item \textbf{copertura dei requisiti desiderabili}: l'obiettivo stabilito per la metrica è il range 80 - 100.
			\end{itemize}
		 	\paragraph{affidabilità}
		 	Il prodotto è testato negli aspetti più importanti e in determinate situazioni nelle quali esso si può trovare.
		 	Gli obiettivi di affidabilità e le metriche ad essi associati sono i seguenti:
		 	\begin{itemize}
				\item \textbf{test superati}: l'obiettivo stabilito per la metrica è il valore 100.
			\end{itemize}
		 	\paragraph{efficienza}
		 	Il prodotto presenta codice senza elevati gradi di complessità relativamente ad alcuni vincoli definiti.
		 	Gli obiettivi di efficienza e le metriche ad essi associati sono i seguenti:
		 	\begin{itemize}
				\item \textbf{Profondità di annidamento dei blocchi}: l'obiettivo stabilito per la metrica è il range 0 - 4;
			\end{itemize}
		 	\paragraph{manutenibilità}
		 	Il codice risulta manutenibile e facilmente comprensibili.
		 	Gli obiettivi di manutenibilità e le metriche ad essi associati sono i seguenti:
		 	\begin{itemize}
				\item \textbf{Numero di linee di codice per metodo}: l'obiettivo stabilito per la metrica è il range 0 - 20; 
				\item \textbf{Numero di parametri per metodo}: l'obiettivo stabilito per la metrica è il range 0 - 4;
				\item \textbf{Numero di campi dati per classe}: l'obiettivo stabilito per la metrica è il range 0 - 10;
				\item \textbf{Numero di metodi per classe}: l'obiettivo stabilito per la metrica è il range 0 - 10;
				\item \textbf{Grado di accoppiamento afferente per package}: l'obiettivo stabilito per la metrica è il 0 - 7;
				\item \textbf{Grado di accoppiamento efferente per package}: l'obiettivo stabilito per la metrica è il 0 - 6;
				\item \textbf{Complessità ciclomatica per metodo}: l'obiettivo stabilito per la metrica è il range 0 - 8;
				\item \textbf{Numero di tipi per package}: l'obiettivo stabilito per la metrica è il range 0 - 20;
				\item \textbf{Distanza dalla sequenza principale normalizzata}: l'obiettivo stabilito per la metrica è il range 0.0 - 0.5; 
				\item \textbf{Instabilità}: l'obiettivo stabilito per la metrica sono i range 0.0 - 0.3 e 0.7 - 1.0; 
				\item \textbf{Percentuale linee di commento su linee di codice}: l'obiettivo stabilito per la metrica è il range 20 - 40;	
				\item \textbf{Numero di figli diretti}: l'obiettivo stabilito per la metrica è il range 0 - 2;
				\item \textbf{Profondità nella gerarchia}: l'obiettivo stabilito per la metrica è il range 1 - 2;
				\item \textbf{Profondità nella gerarchia}: l'obiettivo stabilito per la metrica è il range 1 - 2;
			\end{itemize}
			\paragraph{portabilità}
		 	Il prodotto deve poter garantire le sue funzionalità in browser differenti.
		 	Gli obiettivi di portabilità e le metriche ad essi associati sono i seguenti:
		 	\begin{itemize}
				\item \textbf{Validazione W3C}: l'obiettivo stabilito per la metrica è il valore 0.
			\end{itemize}
	\subsection{Scadenze temporali}
	Le scadenze temporali stabilite dal gruppo sono definite in dettaglio nel documento \PdP.




\section{La strategia di gestione della qualità nel dettaglio}
	\subsection{Risorse}
		\subsubsection{Risorse necessarie}
		Per svolgere al meglio il processo di verifica saranno necessarie le seguenti risorse umane, hardware e software:
		\begin{itemize}
			\item \textbf{Risorse umane}
			I ruoli di progetto coinvolti a pieno titolo nel processo di verifica sono il Responsabile di Progetto e i Verificatori. Per quanto riguarda una descrizione più dettagliata delle relative responsabilità, fare riferimento al documento \NdP.
			\item \textbf{Risorse hardware}
			  È necessario avere a disposizione calcolatori dotati di sufficiente potenza di calcolo, in grado di connettersi alla rete internet e con installati gli strumenti software indispensabili per lo svolgimento di verifica (come individuati nelle \NdP).
			\item \textbf{Risorse software}
			Le risorse software necessarie al processo di verifica sono gli strumenti software in grado di eseguire controlli sui documenti, aiutare lo sviluppo nei linguaggi di programmazione scelti, gestire l'analisi statica del codice e l'esecuzione dei test previsti.
		\end{itemize}
		\subsubsection{Risorse disponibili}
		Seguono le risorse umane, hardware e software a disposizione del team per lo svolgimento del processo di verifica.
		\begin{itemize}
			\item \textbf{Risorse umane}
			Ogni membro del gruppo assumerà, a rotazione, il ruolo di Responsabile del Progetto o Verificatore, come indicato in dettaglio nel documento \PdP. Pertanto tutti i componenti verranno coinvolti a pieno titolo nelle diverse fasi del processo di Verifica.
			\item \textbf{Risorse hardware}
			Ognuno dei membri del gruppo con l'incarico di ricoprire il ruolo di Responsabile del Progetto o Verificatore ha a sua disposizione almeno un computer personale (portatile o fisso) dotato degli strumenti software necessari allo svolgimento del compito. Risultano inoltre a disposizione anche i calcolatori resi disponibili dai laboratori informatici dell'Ateneo.
			\item \textbf{Risorse software}
			Le risorse software disponibili sono costituite dagli \gloss{editor} di testo utilizzati per la stesura dei documenti in \gloss{Latex}, gli \gloss{IDE} impiegati a supporto dell'attività di codifica, le applicazioni online e le estensioni per i controlli sulla leggibilità e la complessità dei documenti in riferimento all'indice Gulpease. Più in dettaglio gli strumenti software impiegati per l'attività di Verifica sono illustrati nel documento \NdP.
			\end{itemize}
		


\section{Test}
	\subsection{Tipi di test}
	Sono stati individuati quattro tipologie di test:
	\begin{itemize}
		\item \textbf{Test di unità [TU]}: test con i quali si cerca di verificare la più piccola parte di lavoro prodotta da un programmatore, quali metodi e funzioni scritte;
		\item \textbf{Test di integrazione [TI]}: test per verificare le componenti di sistema con l'obiettivo di verificare il funzionamento dei vari package prodotti, singolarmente o nel loro insieme;
		\item \textbf{Test di sistema [TS]}: test con i quali si tenta di verificare che il funzionamento e il comportamento dell'architettura siano corretti;
		\item \textbf{Test di accettazione [TA]}: test per verificare che il prodotto soddisfi le richieste del proponente.
	\end{itemize}
	Si sottolinea che la specifica dei test delle diverse tipologie verrà inserita nel presente documento a seguito della fase di progettazione del prodotto in esame.



\section{Resoconto attività di verifica}
All'interno di questa sezione vengono riportati gli esiti delle attivit di verifica svolte sui processi attivati e sui relativi prodotti secondo quanto stabilito nel \PdP.
	\subsection{Revisione dei requisiti}
		\subsubsection{Verifica processo di documentazione}
		Le attività di verifica svolte sui documenti prodotti in questa prima fase sono state di due tipi:
		\begin{itemize}		
			\item attività di verifica manuali;
			\item attività di verifica automatizzate.
		\end{itemize}
		
		Le prime sono state svolte dai verificatori assegnati ad ogni documento utilizzando la tecnica di 					analisi statica walkthrough. Grazie ad essa è stato possibile correggere una discreta quantità di 					errori e imprecisioni presenti all'interno dei documenti prodotti fra cui: 
		\begin{itemize}	
			\item nuovi termini da inserire nel glossario;
			\item termini presenti nel glossario ma non correttamente segnati;
			\item violazioni delle norme tipografiche e grammaticali come definite nel documento \NdP ;
			\item refusi ed errori grammaticali;
			\item periodi troppo lunghi e complessi da spezzare, ridurre e semplificare.
		\end{itemize}
		A partire dalla natura e frequenza degli errori identificati il gruppo ha iniziato a stilare una lista
		di controllo da applicare nei successivi momenti di verifica nel contesto della strategia di verifica
		inspection.

		Le attività di verifica automatizzate, invece, sono state svolte calcolando l'indice Gulpease dei 	
		diversi documenti prodotti attraverso l'uso di appositi strumenti web automatici. 
		\\I risultati ottenuti sono elencati nella seguente tabella:
		\begin{table}[H]
		\begin{tabular}{|l|l|l|}
		\hline
		\textbf{Documento} 		&\textbf{Valutazione} &\textbf{~~~~~~Esito~~~~~~} \\
		\hline
		\PdQ 					&51		&~~~~~~Superato~~~~~~\\
		\NdP 					&41		&~~~~~~Superato~~~~~~\\
		\SdF 					&49		&~~~~~~Superato~~~~~~\\	
		\AdR 					&72		&~~~~~~Superato~~~~~~\\
		\PdP 					&55		&~~~~~~Superato~~~~~~\\
		\Glossario 				&48		&~~~~~~Superato~~~~~~\\
		\textit{Verbali.pdf} 		&49		&~~~~~~Superato~~~~~~\\
		\hline
		\end{tabular}
		\caption{Esiti del calcolo dell'indice di Gulpease dei documenti consegnati}
		\end{table}
		
		\subsubsection{Verifica processo di pianificazione di progetto e processo di verifica e controllo}
		Per verificare il soddisfacimento degli obiettivi di qualita definiti per tale processo, a partire dal 				consultivo redatto nel \PdP sono stati calcolati gli indici di budget variance e schedule variance per 				ogniuna delle attività previste.
		\\I risultati ottenuti sono elencati nella seguente tabella:	
		\begin{table}[H]
		\begin{tabular}{|l|l|l|}
		\hline
		\textbf{Documento} 		&\textbf{Schedule variance} &\textbf{Budget variance} \\
		\hline
		\PdQ 					&0\%		&0\%\\
		\NdP 					&0\%		&0\%\\
		\SdF 					&0\%		&0\%\\
		\AdR 					&0\%		&0\%\\
		\PdP 					&-25\%		&0\%\\
		\Glossario 				&0\%		&0\%\\
		\textit{Verbali.pdf} 	&0\%		&0\%\\
		\hline
		\end{tabular}
		\caption{Esiti del calcolo degli indici di schedule e budget variance}
		\end{table}



\appendix
\section{Standard adottati}
	\subsection{Qualità di processo - SPICE}
	Lo standard prevede sei diversi livelli di maturità (o capacità), e 9 attributi di processo che, se posseduti, lo portano ad un certo livello di 			capacità.
	\begin{itemize}
	\item \textbf{Livello 0: Incompleto} il processo non è implementato o non raggiunge il suo obiettivo;   non esiste evidenza di esecuzione 					sistematica delle attività che lo compongono.
	 \item \textbf{Livello 1: Attuato} il processo è implementato e raggiunge il suo obiettivo; esiste evidenza di semplice attuazione delle attività 			che lo compongono.
	Il relativo attributo di processo:
		\begin{itemize}
			\item \emph{Esecuzione del Processo}: la misura in cui il processo raggiunge i propri obiettivi trasformando   prodotti in ingresso 						identificabili in prodotti in uscita identificabili.
		\end{itemize}
	\item \textbf{Livello 2: Gestito} il processo è gestito e i suoi prodotti sono stabiliti, controllati e manutenuti; le attività sono pianificate e 			controllate e il loro svolgimento risulta documentato.
	I relativi attributi di processo:
		\begin{itemize}
			\item \emph{Gestione delle Prestazioni}: la misura in cui il processo produce un risultato coerente con gli obiettivi attesi.
			\item \emph{Gestione dei Prodotti}: la misura in cui il processo viene gestito per elaborare prodotti documentati, controllati e verificati 				in modo appropriato.
		\end{itemize}
	
	\item \textbf{Livello 3: Definito} il processo viene eseguito in base ai principi dell'ingegneria del software. Le procedure sono definite e 				adattate ai progetti e ruoli, competenze e responsabilità sono definiti e controllati.
	I relativi attributi di processo:
		\begin{itemize}
			\item \emph{Definizione del Processo}: la misura in cui il processo raggiunge i risultati attesi aderendo ad un particolare standard di 					processo.
			\item \emph{Utilizzo del Processo}: la misura in cui il processo attinge alle risorse allocate per la sua esecuzione.	
		\end{itemize}
		
	\item \textbf{Livello 4: Predicibile} il processo è messo in atto costantemente entro limiti definiti. Le attività che lo compongono, la loro 				gestione e i relativi risultati sono controllati quantitativamente.
	I relativi attributi di processo:
		\begin{itemize}
			\item \emph{Misurazione del Processo}: la misura in cui il processo utilizza i risultati raggiunti e le misure ricavate durante 							l'esecuzione per garantire il raggiungimento dei traguardi definiti.
			\item \emph{Controllo del Processo}: la misura in cui il processo viene controllato tramite la raccolta, l'analisi e la messa in uso di 					misurazioni di prodotto e processo allo scopo di correggere, ove necessario, la sua esecuzione per raggiungere i risultati attesi.	
		\end{itemize}

	\item \textbf{Livello 5: Ottimizzante} il processo è continuativamente migliorato per soddisfare i rilevanti traguardi di business attuali e 				previsti. I cambiamenti del processo sono valutati e lo studio per il miglioramento è un'attività costante.
	I relativi attributi di processo:
		\begin{itemize}
			\item \emph{Innovazione del Processo}: la misura in cui cambiamenti relativi alla definizione, alla gestione e all'esecuzione del processo 					sono controllati per raggiungere gli obiettivi di business dell'organizzazione.
			\item \emph{Ottimizzazione del Processo}: la misura in cui vengono identificati e implementati cambiamenti relativamente all'esecuzione del 				processo in modo tale da assicurare un miglioramento continuo nel raggiungimento degli obiettivi rilevanti dell'organizzazione.
		\end{itemize}
	\end{itemize}

Ogni attributo di processo è misurabile e sono definiti dallo standard 4 diversi gradi di possesso:
\begin{itemize}
	\item \textbf{N}: non posseduto (0\% - 15\%);
	\item \textbf{P}: parzialmente posseduto (16\% - 50\%);
	\item \textbf{L}: largamente posseduto (51\% - 85\%);
	\item \textbf{F}: completamente posseduto (86\% - 100\%).
\end{itemize}

Le misurazioni e le valutazioni risultanti dal monitoraggio dei diversi processi sono usate nel contesto di una strategia di miglioramento continuo della qualità, realizzata attraverso il ciclo PDCA.
\\Il ciclo PDCA, altresì noto come ciclo di Deming, definisce un'organizzazione interna dei processi incentrata sul principio del miglioramento continuo allo scopo di renderli automigliorativi.
\\Le fasi in cui esso si articola sono quattro:
	\begin{itemize}
		\item \textbf{Pianificare (Plan)}: in tale fase vengono definite attività, scadenze, responsabilità, risorse utili a raggiungere specifici obiettivi di miglioramento opportunamente pianificati.
		\item \textbf{Eseguire (Do)}: in questa fase vengono attuate le azioni migliorative pianificate al passo precedente. Si procede inoltre ad eseguire misurazioni e raccogliere dati utili per le successive fasi di analisi e controllo.
		\item \textbf{Valutare (Check)}: si tratta di una fase di verifica in cui l'esito delle azioni di miglioramento viene confrontato rispetto alle attese e agli obiettivi pianificati.
		\item \textbf{Agire (Act)}: se l'esito delle valutazioni effettuate al passo precedente risulta positivo, i cambiamenti introdotti nell'esecuzione del processo vengono incorporati stabilmente in esso e standardizzati.
	\end{itemize}
	\subsection{Qualità di prodotto - ISO/IEC:9126}
	Lo Standard ISO/IEC 9126:2001 si articola in quattro parti:
	\begin{enumerate}
	\item Modello della qualità del software (9126-1);
		\item Metriche per la qualità esterna (9126-2);
		\item Metriche per la qualità interna (9126-3);
		\item Metriche per la qualità in uso (9126-4).
	\end{enumerate}
	Lo standard analizza la qualità del software sotto tre diversi punti di vista:
	\begin{itemize}
		\item \textbf{Qualità interna}: è la qualità del prodotto software che fa riferimento alle caratteristiche implementative del software come l'architettura e il codice derivante da quest'ultima.
		\item \textbf{Qualità esterna}: è la qualità del prodotto software relativa a quando esso viene eseguito e testato in un ambiente di prova. 
		\item \textbf{Qualità in uso}: è la qualità del prodotto software dal lato di chi utilizza tale prodotto all'interno di uno specifico sistema.
	\end{itemize}
		\subsubsection{Modello della qualità del software}
		Nella prima parte dello standard vengono presentati i modelli per la qualità esterna, interna ed in uso.
			\paragraph{Modello della qualità esterna ed interna}
			Il modello di qualità esterna ed interna sancito nella prima parte dello standard è suddiviso nelle seguenti sei caratteristiche generali misurabili attraverso delle metriche:
			\begin{itemize}
				\item \textbf{funzionalità}: è la capacità del software di fornire le funzioni che soddisfano determinate esigenze, necessarie per operare in determinate condizioni. 
				\item \textbf{affidabilità}: rappresenta la capacità del prodotto software di mantenere uno specifico livello di prestazioni quando viene usato in certe condizioni e per un periodo di tempo determinato.
				\item \textbf{usabilità}: è la capacità di un prodotto software di essere facilmente comprensibile e attraente in ogni sua parte per un utente qualsiasi. Un software è considerato usabile proporzionalmente alla facilità con cui un utente opera per sfruttare al massimo le funzionalità che il software mette a disposizione.
				\item \textbf{efficienza}: è la capacità di un prodotto di eseguire le funzioni richieste nel minor tempo possibile ed utilizzando le risorse necessarie nel modo migliore.
				\item \textbf{manutenibilità}: rappresenta la capacità di un prodotto software di essere modificato in tempi rapidi e a costi accessibili. Le modifiche possono riguardare correzioni o adattamenti del prodotto a variazioni negli ambienti, nei requisiti e nelle specifiche funzionali.
				\item \textbf{portabilità}: è la capacità di un prodotto software di poter essere spostato da un ambiente all'altro velocemente. L'ambiente include sia aspetti hardware che software.
			\end{itemize}
			
			\paragraph{Modello della qualità in uso}
			Gli attributi presenti nel modello relativo alla qualità del software in uso sono rappresentati da quattro grandi categorie:
			\begin{itemize}
				\item \textbf{efficacia}: è la capacità di consentire all'utente di raggiungere obiettivi specifici con precisione e completezza.
				\item \textbf{produttività}: rappresenta la capacità di permettere all'utente di utilizzare un numero stabilito di risorse, in relazione all'efficienza raggiunta in uno specifico contesto di utilizzo.
				\item \textbf{sicurezza fisica}: è la capacità di raggiungere un livello accettabile di rischio di danni a dati, persone, proprietà o ambienti.
				\item \textbf{soddisfazione}: rappresenta la capacità di soddisfare gli utenti.
			\end{itemize}
		\subsubsection{Metriche per la qualità del software}
		Nelle restanti tre parti vengono trattate le metriche per la qualità esterna, interna e in uso.
			\paragraph{Metriche per la qualità esterna}
			Le metriche esterne misurano i comportamenti del software che si possono rilevare dai test, dall'operatività e dall'osservazione durante la sua esecuzione sulla base degli obiettivi stabiliti. Le metriche esterne sono scelte in base alle caratteristiche che il prodotto finale dovrà dimostrare una volta utilizzato.
			\paragraph{Metriche per la qualità interna}
			Le metriche interne si applicano al software non eseguibile (un esempio è il codice sorgente) e alla documentazione. Le misure effettuate permettono di prevedere il livello di qualità esterna ed in uso del prodotto finale poiché gli attributi interni influenzano le caratteristiche esterne e quelle in uso.
			\paragraph{Metriche per la qualità in uso}
			Le metriche della qualità in uso valutano il livello con cui il software consente agli utenti di svolgere le proprie attività con efficacia, produttività, sicurezza e soddisfazione nel contesto operativo previsto.
	
\end{document}
