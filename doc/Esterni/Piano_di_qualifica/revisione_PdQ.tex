\documentclass[a4paper]{article}
\usepackage[T1]{fontenc}
\usepackage[utf8]{inputenc}
\usepackage[english,italian]{babel}
\usepackage{microtype}
\usepackage{hyperref}
\usepackage{../../util/hx-macro}
\title{Revisione PdQ}
\author{\GG}

\begin{document}
\maketitle

\paragraph{}
Mi mancano da leggere le ultime otto pagine. Intanto, in tabella trovate le correzioni che consiglierei di apportare. Invece ho corretto gli errori di battitura “banali” direttamente nel file sorgente.

\paragraph{}
\begin{center}
\begin{tabular}{| p{4cm} | p{7cm} |}
	\hline
	\textbf{dove} & \textbf{correzione} \\ \hline
	\hline
	§ “Obiettivi di qualità” & “team” non è più in glossario. \\ \hline
	§ “Obiettivi di qualità di processo” & Piuttosto di \texttt{subparagraph}, usare l'ambiente \texttt{description} per ogni obiettivo di qualità; in questo caso mi sembra (anche graficamente) la scelta migliore. \\ \hline
	§ “Standard adottato”, sotto “Qualità del prodotto” & Cambiare “dal lato dell’utilizzatore che utilizza tale prodotto [\dots]” in “dal lato di chi utilizza tale prodotto [\dots]”. \\ \hline
	§ “Modello della qualità esterna ed interna” & Esplicitare “[\dots] è suddiviso nelle seguenti sei caratteristiche generali [\dots]”, anziché solo “è suddiviso in sei caratteristiche generali [\dots]” \\ \hline
	§ “Risorse” (in § “La strategia di gestione della qualità nel dettaglio”) & Piuttosto dei paragrafi, usare un elenco: questi paragrafi sono abbastanza brevi (quindi va bene un elenco) e non credo debbano essere indicizzati nell'indice generale. \\ \hline
	§ “Risorse disponibili” & Cambiare “Le risorse software disponibili sono costituite dagli editor di testo latex e utilizzati per la stesura dei documenti, gli \gloss{IDE} [\dots]” in “Le risorse software disponibili sono costituite dagli editor di testo utilizzati per la stesura dei documenti in Latex, gli \gloss{IDE} [\dots]” \\ \hline
	§ “Misure” (in § “Misure e Metriche”) & Propongo di scegliere uno solo tra \emph{range} e \emph{intervallo}, che hanno lo stesso significato. \\ \hline
	§ “Misure” (in § “Misure e Metriche”) & Nel range ottimale non bisognerebbe scrivere “risulta ottimale”, anziché “risulta accettabile”? Ma può darsi che mi sbagli io. \\ \hline
	§ “Misure” (in § “Misure e Metriche”) & Nel range accettabile, spezzare la frase lunga sei righe in: “Risulta implicito che valori non rientranti nell'intervallo di accettabilità sono eccessivamente bassi e non possono quindi essere accettati; essi richiedono immediate strategie correttive [\dots da qui in poi tutto ok]” \\ \hline
	§ “Metriche per i processi” & Mettere \emph{schedule variance} in corsivo; ci sono solo quattro occorrenze nel testo, quindi il corsivo non dovrebbe essere pesante da leggere. \\ \hline
	§ “Schedule Variance” & Sostiruire “[\dots] la differenza fra la data pianificata di fine di un’attività e la data di fine reale di quell'attività” con “[\dots] la differenza fra la data pianificata di fine di un’attività e la data reale di fine di quell'attività” \\ \hline
	§ “Schedule Variance” & Sostiruire “periodi di \gloss{slack}” con “\gloss{periodi di slack}”, per non confonderli con l'applicazione Slack. \\ \hline
\end{tabular}
\end{center}

\paragraph{}
Osservazioni generali:
\begin{itemize}
	\item Nelle liste, eviterei di mettere i due punti (:) in grassetto; meglio evidenziare solo il termine che si vuole evidenziare, senza la punteggiatura:
	\begin{itemize}
		\item \textbf{Termine}: così;
		\item \textbf{Termine:} non così.
	\end{itemize}
	\item Da quel che ho capito, sarebbe meglio usare un nuovo paragrafo anonimo (andando a capo due volte) anziché usare il doppio backslash.
	\item Aggiungerei qualche virgola, per dare più respiro alle frasi lunghe.
	\item All'interno degli elementi di una lista, vi consiglio di preferire il punto e virgola al punto fermo.
	\item Per fare l'inciso, in Latex, si usano tre trattini attaccati --- mi sono permesso di correggere direttamente nel sorgente anche questo aspetto, che è standard nell'utilizzo di Latex.
	\item In italiano non si alzano le iniziali con troppa generosità: terrei bassi i nomi dei ruoli e soprattutto quelli dei processi (ad es. “verifica”).
	\item 
\end{itemize}

\end{document}
