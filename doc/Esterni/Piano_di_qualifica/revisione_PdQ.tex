\documentclass[a4paper]{article}
\usepackage[T1]{fontenc}
\usepackage[utf8]{inputenc}
\usepackage[english,italian]{babel}
\usepackage{microtype}
\usepackage{longtable}
\usepackage{hyperref}
\usepackage{../../util/hx-macro}
\title{Revisione PdQ}
\author{\GG}

\begin{document}
\maketitle

\paragraph{}
In tabella trovate le correzioni che consiglierei di apportare. Invece ho corretto gli errori di battitura “banali” direttamente nel file sorgente.

\paragraph{}
\begin{longtable}{| p{3cm} | p{10cm} |}
	\hline
	\textbf{sezione del testo} & \textbf{correzione} \\ \hline
	\hline
	“Obiettivi di qualità” & \emph{Team} non è più in glossario. \\ \hline
	“Obiettivi di qualità di processo” & Piuttosto di \texttt{subparagraph}, usare l'ambiente \texttt{description} per ogni obiettivo di qualità; in questo caso mi sembra (anche graficamente) la scelta migliore. \\ \hline
	“Standard adottato”, sotto “Qualità del prodotto” & Cambiare “dal lato dell’utilizzatore che utilizza tale prodotto [\dots]” in “dal lato di chi utilizza tale prodotto [\dots]”. \\ \hline
	“Modello della qualità esterna ed interna” & Esplicitare “[\dots] è suddiviso nelle seguenti sei caratteristiche generali [\dots]”, anziché solo “è suddiviso in sei caratteristiche generali [\dots]” \\ \hline
	“Risorse” (in “La strategia di gestione della qualità nel dettaglio”) & Piuttosto dei paragrafi, usare un elenco: questi paragrafi sono abbastanza brevi (quindi va bene un elenco) e non credo debbano essere indicizzati nell'indice generale. \\ \hline
	“Risorse disponibili” & Cambiare “Le risorse software disponibili sono costituite dagli \gloss{editor} di testo \gloss{latex} e utilizzati per la stesura dei documenti, gli \gloss{IDE} [\dots]” in “Le risorse software disponibili sono costituite dagli editor di testo utilizzati per la stesura dei documenti in Latex, gli IDE [\dots]” \\ \hline
	“Misure” (in “Misure e Metriche”) & Propongo di scegliere uno solo tra \emph{range} e \emph{intervallo}. \\ \hline
	“Misure” (in “Misure e Metriche”) & Nel range ottimale non bisognerebbe scrivere “risulta ottimale”, anziché “risulta accettabile”? Ma può darsi che mi sbagli io. \\ \hline
	“Misure” (in “Misure e Metriche”) & Nel range accettabile, spezzare la frase lunga sei righe in: “Risulta implicito che valori non rientranti nell'intervallo di accettabilità sono eccessivamente bassi e non possono quindi essere accettati; essi richiedono immediate strategie correttive [\dots da qui in poi tutto ok]” \\ \hline
	“Metriche per i processi” e altrove & Mettere \emph{schedule variance} e \emph{budget variance} in corsivo. Queste espressioni si trovano poche volte quindi il corsivo non dovrebbe essere pesante da leggere. \\ \hline
	“Schedule Variance” & Sostiruire “[\dots] la differenza fra la data pianificata di fine di un’attività e la data di fine reale di quell'attività” con “[\dots] la differenza fra la data pianificata di fine di un’attività e la data reale di fine di quell'attività” (oppure anche “fine attività” al posto di “fine di un'attività”). \\ \hline
	“Schedule Variance” e altrove & Sostiruire “periodi di \gloss{slack}” con “periodi di slack”, per non confonderli con l'applicazione Slack. \\ \hline
	“Schedule Variance” & Errore di punteggiatura; meglio usare i due punti e il punto e virgola, così: “Viene calcolata come la differenza fra la data pianificata di fine di un'attività e la data di fine reale di quell'attività: un valore di schedule variance minore di zero indica che l'attività di progetto considerata sta richiedendo più tempo di quanto pianificato; se pari a zero indica invece [\dots]” \\ \hline
	“Produttività di codifica” & Propongo di aggiungere una breve definizione di “ore/persona”, dato che dobbiamo essere il più espliciti possibile. \\ \hline
	“Copertura del codice” & Sostituire “Si può ridurre tale indice [\dots]” con “Si può aumentare la copertura [\dots]”, sennò ci si confonde tra aumento della copertura e aumento degli errori\dots{} ma forse non ho capito bene: i metodi \emph{\gloss{getter}} e \emph{\gloss{setter}} aumentano o diminuiscono la copertura? \\ \hline
	“Funzionalità” & Credo sia bene esplicitare perché si è scelto di non trattare i requisiti opzionali. \\ \hline
	“Profondità di annidamento” & Sostituire “istruzioni \emph{if}/cicli \emph{for}” con “istruzioni condizionali (\emph{if}) o iterative (\emph{for}, \emph{while})”. \\ \hline
	“Chiamate innestate di metodi” & Esplicitare cosa s'intende per \emph{chiamate innestate di metodi}, anche se a noi sembra chiaro. \\ \hline
	“Numero di \gloss{statement} per metodo” & Si potrebbe cambiare \emph{statement} in \emph{istruzioni} o esplicitarne il significato. P.S. 50 istruzioni sono più di quanto si possa vedere a schermo intero. \\ \hline
	“Numero di campi dati per classe” & Secondo me, 25 campi dati sono troppi anche per essere accettabili in un diagramma \gloss{UML}: scenderei a 20. \\ \hline
	“Lunghezza media di un'unità” & Sostituire “La metrica è una misura della dimensione [\dots]” con “Questa metrica è una misura della dimensione [\dots]” \\ \hline
	“Rapporto linee di commento su linee di codice” & Nell'equazione, sostituire “numero linee di commento scritte” con “numero linee di commento”; per lo stesso motivo, sostituire “numero linee di codice sorgente prodotte” con “numero linee di codice sorgente”. \\ \hline
	“Validazione \gloss{W3C}” & Dobbiamo ricordarci di mettere “W3C” nel glossario e magari aggiungere il sito nei riferimenti a inizio documento. \\ \hline
	“Tipi di test” & La “più piccola parte di lavoro prodotta da un programmatore” potrebbe essere il singolo carattere; forse è meglio “la più piccola parte di lavoro che conviene verificare” (più o meno come nelle slide del prof.). Inoltre, unità non potrebbe essere anche una singola classe, sufficientemente piccola? \\ \hline
	“Tipi di test” & \emph{Prodotto} non è in glossario. \\ \hline
\end{longtable}

\paragraph{}
Osservazioni generali:
\begin{itemize}
	\item Una \texttt{newpage} dopo ogni \texttt{section} non è nelle NdP.
	\item Aggiungerei qualche virgola, per dare più respiro alle frasi lunghe.
	\item Nelle liste, eviterei di mettere i due punti (:) in grassetto; meglio evidenziare solo il termine che si vuole evidenziare, senza la punteggiatura:
	\begin{itemize}
		\item \textbf{Termine}: così;
		\item \textbf{Termine:} non così.
	\end{itemize}
	\item Da quel che ho capito, sarebbe meglio usare un nuovo paragrafo anonimo (andando a capo due volte) anziché usare il doppio backslash.
	\item All'interno degli elementi di una lista, vi consiglio di preferire il punto e virgola al punto fermo.
	\item Per fare l'inciso, in Latex, si usano tre trattini attaccati --- mi sono permesso di correggere direttamente nel sorgente anche questo aspetto, che è standard nell'utilizzo di Latex.
	\item In italiano non si alzano le iniziali con troppa generosità: terrei bassi i nomi dei ruoli e soprattutto quelli dei processi (ad es. “verifica”).
	\item 
\end{itemize}

\end{document}
