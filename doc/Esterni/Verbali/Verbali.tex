% Tutti i verbali di riunione
% da compilare con il comando pdflatex Esterni/Verbali/Verbali.tex

% Dichiarazioni di ambiente e inclusione di pacchetti
% da usare tramite il comando % Dichiarazioni di ambiente e inclusione di pacchetti
% da usare tramite il comando % Dichiarazioni di ambiente e inclusione di pacchetti
% da usare tramite il comando \input{../../util/hx-ambiente}

\documentclass[a4paper,titlepage]{article}
\usepackage[T1]{fontenc}
\usepackage[utf8]{inputenc}
\usepackage[english,italian]{babel}
\usepackage{microtype}
\usepackage{lmodern}
\usepackage{underscore}
\usepackage{graphicx}
\usepackage{eurosym}
\usepackage{float}
\usepackage{fancyhdr}
\usepackage[table,dvipsnames]{xcolor}
\usepackage{multirow}
\usepackage{longtable}
\usepackage{chngpage}
\usepackage{grffile}
\usepackage[titles]{tocloft}
\usepackage{hyperref}
\hypersetup{hidelinks}

\usepackage{../../util/hx-vers}
\usepackage{../../util/hx-macro}
\usepackage{../../util/hx-front}

% solo se si vuole una nuova pagina ad ogni \section:
\usepackage{titlesec}
\newcommand{\sectionbreak}{\clearpage}

% stile di pagina:
\pagestyle{fancy}

% solo se si vuole eliminare l'indentazione ad ogni paragrafo:
\setlength{\parindent}{0pt}

% intestazione:
\lhead{\Large{\proj}}
\rhead{\includegraphics[keepaspectratio=true,width=50px]{../../util/hivex_logo2.png}}
\renewcommand{\headrulewidth}{0.4pt}

% pie' di pagina:
\lfoot{\email}
\rfoot{\thepage}
\cfoot{}
\renewcommand{\footrulewidth}{0.4pt}

% spazio verticale tra le celle di una tabella:
\renewcommand{\arraystretch}{1.5}

% profondità di indicizzazione:
\setcounter{tocdepth}{4}
\setcounter{secnumdepth}{4}

% numerazione innestata per elenchi numerati:
\renewcommand{\labelenumii}{\theenumii}
\renewcommand{\theenumii}{\theenumi.\arabic{enumii}.}


\documentclass[a4paper,titlepage]{article}
\usepackage[T1]{fontenc}
\usepackage[utf8]{inputenc}
\usepackage[english,italian]{babel}
\usepackage{microtype}
\usepackage{lmodern}
\usepackage{underscore}
\usepackage{graphicx}
\usepackage{eurosym}
\usepackage{float}
\usepackage{fancyhdr}
\usepackage[table,dvipsnames]{xcolor}
\usepackage{multirow}
\usepackage{longtable}
\usepackage{chngpage}
\usepackage{grffile}
\usepackage[titles]{tocloft}
\usepackage{hyperref}
\hypersetup{hidelinks}

\usepackage{../../util/hx-vers}
\usepackage{../../util/hx-macro}
\usepackage{../../util/hx-front}

% solo se si vuole una nuova pagina ad ogni \section:
\usepackage{titlesec}
\newcommand{\sectionbreak}{\clearpage}

% stile di pagina:
\pagestyle{fancy}

% solo se si vuole eliminare l'indentazione ad ogni paragrafo:
\setlength{\parindent}{0pt}

% intestazione:
\lhead{\Large{\proj}}
\rhead{\includegraphics[keepaspectratio=true,width=50px]{../../util/hivex_logo2.png}}
\renewcommand{\headrulewidth}{0.4pt}

% pie' di pagina:
\lfoot{\email}
\rfoot{\thepage}
\cfoot{}
\renewcommand{\footrulewidth}{0.4pt}

% spazio verticale tra le celle di una tabella:
\renewcommand{\arraystretch}{1.5}

% profondità di indicizzazione:
\setcounter{tocdepth}{4}
\setcounter{secnumdepth}{4}

% numerazione innestata per elenchi numerati:
\renewcommand{\labelenumii}{\theenumii}
\renewcommand{\theenumii}{\theenumi.\arabic{enumii}.}


\documentclass[a4paper,titlepage]{article}
\usepackage[T1]{fontenc}
\usepackage[utf8]{inputenc}
\usepackage[english,italian]{babel}
\usepackage{microtype}
\usepackage{lmodern}
\usepackage{underscore}
\usepackage{graphicx}
\usepackage{eurosym}
\usepackage{float}
\usepackage{fancyhdr}
\usepackage[table,dvipsnames]{xcolor}
\usepackage{multirow}
\usepackage{longtable}
\usepackage{chngpage}
\usepackage{grffile}
\usepackage[titles]{tocloft}
\usepackage{hyperref}
\hypersetup{hidelinks}

\usepackage{../../util/hx-vers}
\usepackage{../../util/hx-macro}
\usepackage{../../util/hx-front}

% solo se si vuole una nuova pagina ad ogni \section:
\usepackage{titlesec}
\newcommand{\sectionbreak}{\clearpage}

% stile di pagina:
\pagestyle{fancy}

% solo se si vuole eliminare l'indentazione ad ogni paragrafo:
\setlength{\parindent}{0pt}

% intestazione:
\lhead{\Large{\proj}}
\rhead{\includegraphics[keepaspectratio=true,width=50px]{../../util/hivex_logo2.png}}
\renewcommand{\headrulewidth}{0.4pt}

% pie' di pagina:
\lfoot{\email}
\rfoot{\thepage}
\cfoot{}
\renewcommand{\footrulewidth}{0.4pt}

% spazio verticale tra le celle di una tabella:
\renewcommand{\arraystretch}{1.5}

% profondità di indicizzazione:
\setcounter{tocdepth}{4}
\setcounter{secnumdepth}{4}

% numerazione innestata per elenchi numerati:
\renewcommand{\labelenumii}{\theenumii}
\renewcommand{\theenumii}{\theenumi.\arabic{enumii}.}


\version{1.0.0}
\creaz{1 dicembre 2016}
\author{\PB}
\supervisor{\MM}
\uso{interno}
\dest{Tutti i membri del gruppo}
\title{Verbali delle riunioni}
\date{9 gennaio 2017}

\begin{document}

\maketitle
\tableofcontents
\newpage

\section{Riunione del 29 novembre 2016}

\begin{description}
	\item[Orario] 13:30 - 15:30;
	\item[Durata] 2 ore;
	\item[Luogo incontro] Aula 1C150 di Torre Archimede; 
	\item[Oggetto] scelta del nome e del logo per il gruppo, scelta del capitolato d'appalto, degli strumenti per l'organizzazione e il versionamento che si andranno a utilizzare e definizione dei ruoli;
	\item[Segretario] \LB; 
	\item[Partecipanti] \ALL.
\end{description}

Durante la riunione sono state prese le seguenti decisioni:
il nome e il logo del gruppo e dopo varie proposte è stato scelto all'unanimità il nome \emph{Hivex}, dall'inglese \emph{hive} (la casetta delle api), dato che il nostro gruppo si compone di sei persone --- tante quanti i lati dell'esagono; inoltre Francis Bacon scriveva, nel \emph{Novum Organum},
\begin{quote}
	The men of experiment are like the ant, they only collect and use; the reasoners resemble spiders, who make cobwebs out of their own substance. But the bee takes the middle course, it gathers its material from the flowers of the garden and field, but transforms and digests it by a power of its own.
\end{quote}

Si è scelto il capitolato C6 \proj{} proposto dall'azienda \ZU{} e si è discusso su come distribuire i ruoli ai vari componenti del gruppo.
Infine abbiamo scelto gli strumenti che verranno utilizzati per l'organizzazione e il versionamento: più in particolare come strumento di organizzazione abbiamo scelto di utilizzare \gloss{Asana} in quanto offre, per esempio, l'integrazione con \gloss{Slack}; si è reso necessario integrare questo strumento con \gloss{Instagantt}, allo scopo di visualizzare in maniera immediata tramite il \gloss{diagramma di Gantt} tutti i task inseriti in \gloss{Asana}. Prima di scegliere \gloss{Asana} avevamo rivolto la nostra attenzione a \gloss{Trello} ma durante l'analisi delle funzionalità offerte sono sorte varie criticità. Per la comunicazione abbiamo scelto di utilizzare \gloss{Slack}: esso permette la suddivisione in diversi canali e l'integrazione di diversi servizi, come l'interfacciamento ai servizi di \gloss{Asana}. Per quanto riguarda il versionamento, abbiamo scelto di utilizzare  \gloss{GitHub} perché offre un ottimo servizio gratuitamente.



\section{Riunione del 12 dicembre 2016}

\begin{description}
	\item[Orario] 13:30 - 15:00;
	\item[Durata] 1 ora e 30 minuti;
	\item[Luogo incontro] Aula 1C150 di Torre Archimede;
	\item[Oggetto] scelta degli strumenti per lo sviluppo del software e preparazione di domande da porre al proponente il 15 dicembre;
	\item[Segretario] \PB; 
	\item[Partecipanti] \ALL.
\end{description}

Durante la riunione si è discusso degli strumenti da utilizzare nel corso del progetto. Alcuni criteri di scelta sono stati: che gli strumenti fossero “standard”; che non fossero eccessivamente complicati da utilizzare né da imparare; che si integrassero il più possibile tra di loro. Per generare la documentazione abbiamo adottato \gloss{Latex}, al fine di ottenere una maggiore automazione e modularità. Per la gestione automatizzata dei requisiti, ci è venuta in aiuto un'interfaccia open-source \gloss{PHP}/\gloss{MySQL} chiamata “\gloss{PragmaDB}”, ideata da alcuni nostri predecessori. Sono stati inoltre analizzati alcuni software per il controllo del codice e dei documenti quali \gloss{Sonarqube} e \gloss{Travis} che si integra bene con \gloss{Github}, e altre tecnologie circa \gloss{java}, \gloss{javascript}, \gloss{javaDoc} ed infine \gloss{IntelliJIDEA}.

Sono state scelte le domande da proporre circa la definizione dei requisiti del progetto. In particolare, è stato pensato di porre alcune domande basilari come la scelta del linguaggio da generare (\gloss{Java} o \gloss{Javascript}) e la scelta della lingua per la documentazione del software(italiano o inglese); inoltre, sono stati formalizzati i dubbi sorti sulla natura stessa di questo progetto così particolare.



\section{Riunione del 15 dicembre 2016}

\begin{description}
	\item[Orario] 14:30 - 17:00;
	\item[Durata] 2 ore e 30 minuti;
	\item[Luogo incontro] \ZU, sede di via Cittadella 7, Padova;
	\item[Oggetto] discussione del capitolato d'appalto;
	\item[Segretario] \LS; 
	\item[Partecipanti] \GP, \ALL.
\end{description}

Durante la riunione sono state discusse alcune scelte prese dal gruppo riguardanti le potenzialità del software e suggerite altre di nuove. Abbiamo compreso che il dominio di programmi software che il nostro editor dovrà modellare dev'essere mirato ad un particolare ambiente applicativo. Sono stati chiariti alcuni dubbi e il proponente ha lasciato ampia libertà sul linguaggio da usare e sulla natura del software (applicazione web o desktop).



\section{Riunione del 20 dicembre 2016}

\begin{description}
	\item[Orario] 14:30 - 17:00;
	\item[Durata] 2 ore e 30 minuti;
	\item[Luogo incontro] Aula P4 Paolotti;
	\item[Oggetto] scelta del dominio applicativo e analisi dei casi d'uso generali;
	\item[Segretario] \PB; 
	\item[Partecipanti] \ALL.
\end{description}

Durante la riunione è stato scelto di realizzare un'\gloss{applicazione web} con \gloss{HTML5}, \gloss{CSS3} e \gloss{JavaScript} dal lato \gloss{client}, che genera codice \gloss{Java}; dopo varie proposte il dominio applicativo del software da realizzare è risultato “giochi da tavolo”. 
Sono stati analizzati i casi d'uso generali che hanno fatto sorgere domande esposte il giorno seguente al \GP.



\section{Riunione del 21 dicembre 2016}

\begin{description}
	\item[Orario] 14:30 - 16:00;
	\item[Durata] 1 ore e 30 minuti;
	\item[Luogo incontro] \ZU, sede di via Cittadella 7, Padova
	\item[Oggetto] discussione del capitolato d'appalto;
	\item[Segretario] \AZ; 
	\item[Partecipanti] \GP, \ALL.
\end{description}

Durante la riunione è stato chiarito il concetto di stereotipo e dove poter utilizzarlo all'interno del progetto; inoltre è sorta la necessità di dover generare anche l'interfaccia del prodotto (gioco da tavolo) generato dal software. Il diagramma delle attività sarà ispirato al \gloss{diagramma Nassi–Shneiderman} utilizzando una rappresentazione “a blocchi”.



\section{Riunione del 22 dicembre 2016}

\begin{description}
	\item[Orario] 11:30 - 13:30;
	\item[Durata] 2 ore;
	\item[Luogo incontro] 2AB45 di Torre Archimede;
	\item[Oggetto] discussione dei casi d'uso;
	\item[Segretario] \PB; 
	\item[Partecipanti] \ALL.
\end{description}

Durante la riunione sono stati analizzati una parte dei casi d'uso e dei requisiti sorti dalla riunione precedente col proponente.



\section{Riunione del 23 dicembre 2016}

\begin{description}
	\item[Orario] 15:00 - 18:00;
	\item[Durata] 3 ore;
	\item[Luogo incontro] 2AB45 di Torre Archimede; 
	\item[Oggetto] discussione dei casi d'uso;
	\item[Segretario] \PB; 
	\item[Partecipanti] \ALL.
\end{description}

Durante la riunione sono stati analizzati i casi d'uso, in particolar modo riguardo la parte del \gloss{diagramma delle attività}.



\section{Riunione del 2 gennaio 2017}

\begin{description}
	\item[Orario] 09:30 - 13:30;
	\item[Durata] 4 ore;
	\item[Luogo incontro] abitazione di \PB;
	\item[Oggetto] discussione dei requisiti, linguaggio da utilizzare lato server, progresso documentazione, modifiche \gloss{PragmaDB};
	\item[Segretario] \PB; 
	\item[Partecipanti] \ALL.
\end{description}

Durante la riunione sono stati esposti i progressi sulla documentazione da tutti i membri del gruppo. Dopo ciò, è stato deciso di utilizzare \gloss{Java} come linguaggio di programmazione per il \gloss{lato server} dell'applicativo da realizzare e, infine, sono stati analizzati tutti i requisiti da inserire nell'\AdR{} e le modifiche da apportare alla parte \gloss{PHP} di \gloss{PragmaDB} sull'esportazione del glossario in \gloss{Latex}.



% \section{Riunione del 10 gennaio 2017}

% \begin{description}
% 	\item[Orario] 14:30 - 16:00;
% 	\item[Durata] 1 ora e 30 minuti;
% 	\item[Luogo incontro] \ZU, sede di via Cittadella 7, Padova;
% 	\item[Oggetto] presentazione al \GP di un breve caso d'uso che esemplifica le funzionalità più importanti del prodotto;
% 	\item[Segretario] \MM; 
% 	\item[Partecipanti] \GP, \LB, \GG, \MM.
% \end{description}

% Abbiamo presentato al \GP un breve esempio (nella forma di un caso d'uso del sistema), in cui un utente crea un nuovo progetto con la nostra applicazione e si interfaccia con il \gloss{diagramma delle classi} e il \gloss{diagramma delle attività}.

\end{document}
