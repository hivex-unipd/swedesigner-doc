% Tutti i verbali di riunione
% da compilare con il comando pdflatex Verbali.tex

% Dichiarazioni di ambiente e inclusione di pacchetti
% da usare tramite il comando % Dichiarazioni di ambiente e inclusione di pacchetti
% da usare tramite il comando % Dichiarazioni di ambiente e inclusione di pacchetti
% da usare tramite il comando \input{../../util/hx-ambiente}

\documentclass[a4paper,titlepage]{article}
\usepackage[T1]{fontenc}
\usepackage[utf8]{inputenc}
\usepackage[english,italian]{babel}
\usepackage{microtype}
\usepackage{lmodern}
\usepackage{underscore}
\usepackage{graphicx}
\usepackage{eurosym}
\usepackage{float}
\usepackage{fancyhdr}
\usepackage[table,dvipsnames]{xcolor}
\usepackage{multirow}
\usepackage{longtable}
\usepackage{chngpage}
\usepackage{grffile}
\usepackage[titles]{tocloft}
\usepackage{hyperref}
\hypersetup{hidelinks}

\usepackage{../../util/hx-vers}
\usepackage{../../util/hx-macro}
\usepackage{../../util/hx-front}

% solo se si vuole una nuova pagina ad ogni \section:
\usepackage{titlesec}
\newcommand{\sectionbreak}{\clearpage}

% stile di pagina:
\pagestyle{fancy}

% solo se si vuole eliminare l'indentazione ad ogni paragrafo:
\setlength{\parindent}{0pt}

% intestazione:
\lhead{\Large{\proj}}
\rhead{\includegraphics[keepaspectratio=true,width=50px]{../../util/hivex_logo2.png}}
\renewcommand{\headrulewidth}{0.4pt}

% pie' di pagina:
\lfoot{\email}
\rfoot{\thepage}
\cfoot{}
\renewcommand{\footrulewidth}{0.4pt}

% spazio verticale tra le celle di una tabella:
\renewcommand{\arraystretch}{1.5}

% profondità di indicizzazione:
\setcounter{tocdepth}{4}
\setcounter{secnumdepth}{4}

% numerazione innestata per elenchi numerati:
\renewcommand{\labelenumii}{\theenumii}
\renewcommand{\theenumii}{\theenumi.\arabic{enumii}.}


\documentclass[a4paper,titlepage]{article}
\usepackage[T1]{fontenc}
\usepackage[utf8]{inputenc}
\usepackage[english,italian]{babel}
\usepackage{microtype}
\usepackage{lmodern}
\usepackage{underscore}
\usepackage{graphicx}
\usepackage{eurosym}
\usepackage{float}
\usepackage{fancyhdr}
\usepackage[table,dvipsnames]{xcolor}
\usepackage{multirow}
\usepackage{longtable}
\usepackage{chngpage}
\usepackage{grffile}
\usepackage[titles]{tocloft}
\usepackage{hyperref}
\hypersetup{hidelinks}

\usepackage{../../util/hx-vers}
\usepackage{../../util/hx-macro}
\usepackage{../../util/hx-front}

% solo se si vuole una nuova pagina ad ogni \section:
\usepackage{titlesec}
\newcommand{\sectionbreak}{\clearpage}

% stile di pagina:
\pagestyle{fancy}

% solo se si vuole eliminare l'indentazione ad ogni paragrafo:
\setlength{\parindent}{0pt}

% intestazione:
\lhead{\Large{\proj}}
\rhead{\includegraphics[keepaspectratio=true,width=50px]{../../util/hivex_logo2.png}}
\renewcommand{\headrulewidth}{0.4pt}

% pie' di pagina:
\lfoot{\email}
\rfoot{\thepage}
\cfoot{}
\renewcommand{\footrulewidth}{0.4pt}

% spazio verticale tra le celle di una tabella:
\renewcommand{\arraystretch}{1.5}

% profondità di indicizzazione:
\setcounter{tocdepth}{4}
\setcounter{secnumdepth}{4}

% numerazione innestata per elenchi numerati:
\renewcommand{\labelenumii}{\theenumii}
\renewcommand{\theenumii}{\theenumi.\arabic{enumii}.}


\documentclass[a4paper,titlepage]{article}
\usepackage[T1]{fontenc}
\usepackage[utf8]{inputenc}
\usepackage[english,italian]{babel}
\usepackage{microtype}
\usepackage{lmodern}
\usepackage{underscore}
\usepackage{graphicx}
\usepackage{eurosym}
\usepackage{float}
\usepackage{fancyhdr}
\usepackage[table,dvipsnames]{xcolor}
\usepackage{multirow}
\usepackage{longtable}
\usepackage{chngpage}
\usepackage{grffile}
\usepackage[titles]{tocloft}
\usepackage{hyperref}
\hypersetup{hidelinks}

\usepackage{../../util/hx-vers}
\usepackage{../../util/hx-macro}
\usepackage{../../util/hx-front}

% solo se si vuole una nuova pagina ad ogni \section:
\usepackage{titlesec}
\newcommand{\sectionbreak}{\clearpage}

% stile di pagina:
\pagestyle{fancy}

% solo se si vuole eliminare l'indentazione ad ogni paragrafo:
\setlength{\parindent}{0pt}

% intestazione:
\lhead{\Large{\proj}}
\rhead{\includegraphics[keepaspectratio=true,width=50px]{../../util/hivex_logo2.png}}
\renewcommand{\headrulewidth}{0.4pt}

% pie' di pagina:
\lfoot{\email}
\rfoot{\thepage}
\cfoot{}
\renewcommand{\footrulewidth}{0.4pt}

% spazio verticale tra le celle di una tabella:
\renewcommand{\arraystretch}{1.5}

% profondità di indicizzazione:
\setcounter{tocdepth}{4}
\setcounter{secnumdepth}{4}

% numerazione innestata per elenchi numerati:
\renewcommand{\labelenumii}{\theenumii}
\renewcommand{\theenumii}{\theenumi.\arabic{enumii}.}


\author{\PB}
\supervisor{\GG}
\dest{Uso interno}
\title{Verbali delle riunioni}

\begin{document}

\maketitle



\section{Riunione del 29 novembre 2016}

\begin{itemize}
	\item Orario: 13:30 - 15:30;
	\item Durata: 2 ore;
	\item Luogo incontro: Aula 1C150 di Torre Archimede; 
	\item Oggetto: scelta del nome e del logo per il gruppo, scelta del capitolato d'appalto, degli strumenti per l'organizzazione e il versionamento che si andranno a utilizzare e definizione dei ruoli;
	\item Segretario: \LB; 
	\item Partecipanti: \AZ, \GG, \LB, \LS, \MM, \PB.
\end{itemize}

Durante la riunione sono stati discussi i seguenti argomenti: il nome e il logo del gruppo, il capitolato da scegliere, “”””le modalità con le quali verranno distribuiti i ruoli ai vari componenti del gruppo””” e gli strumenti che verranno utilizzati per l'organizzazione e il versionamento.

Dopo varie proposte è stato scelto all'unanimità il nome \textit{Hivex}, dall'inglese \textit{hive} (la casetta delle api), dato che il nostro gruppo si compone di sei persone, tante quanti i lati dell'esagono; inoltre Francis Bacon scriveva, nel \textit{Novum Organum},
\begin{quote}
	The men of experiment are like the ant, they only collect and use; the reasoners resemble spiders, who make cobwebs out of their own substance. But the bee takes the middle course, it gathers its material from the flowers of the garden and field, but transforms and digests it by a power of its own.
\end{quote}

Si è scelto il capitolato C6 “SWEDesigner” proposto dall'azienda \ZU; 
“””Si è discusso su come distribuire i ruoli ai vari componenti del gruppo. Questa decisione è stata riportata ed è consultabile nel file Ruoli.txt in Google Drive; “””
Sono stati definiti gli strumenti e i servizi che si andranno ad utilizzare nel corso del progetto. Come strumento di organizzazione abbiamo scelto di utilizzare Asana [PARLARE DELLE MOTIVAZIONI DELLA SCELTA], per la comunicazione abbiamo scelto di utilizzare Slack [PARLARE DELLE MOTIVAZIONI DELLA SCELTA ES:INTEGRAZIONE CON LE ALTRE APP,DIAGRAMMA GANTT]. Per quanto riguarda il versionamento, abbiamo scelto di utilizzare GitHub perché offre un ottimo servizio gratuitamente per l'open-source [...]



\section{Riunione del 12 dicembre 2016}

\begin{itemize}
	\item Orario: 13:30 - 15:00;
	\item Durata: 1 ora 30 minuti;
	\item Luogo incontro: Aula 1C150 di Torre Archimede; 
	\item Oggetto: scelta degli strumenti per lo sviluppo del software e preparazione di domande da porre al proponente il 15 dicembre;
	\item Segretario: \PB; 
	\item Partecipanti: \AZ, \GG, \LB, \LS, \MM, \PB.
\end{itemize}

Si è discusso degli strumenti da utilizzare nel corso del progetto. Alcuni criteri di scelta sono stati: che gli strumenti fossero “standard”; che non fossero eccessivamente complicati da utilizzare né da imparare; che si integrassero il più possibile tra di loro. Come sistema di repository abbiamo adottato Github [...] Per generare la documentazione abbiamo adottato Latex, al fine di ottenere una maggiore automazione e modularità. Per la gestione automatizzata dei requisiti, ci è venuta in aiuto un'interfaccia open-source PHP/MySQL chiamata “PragmaDB”, ideata da alcuni nostri predecessori. Con Github si integrava bene Travis [...] (ant, maven, gradle [...]), [Sonarqube, java o javascript??, javaDoc, IntelliJ IDEA...].

Sono state scelte le domande da proporre circa la definizione dei requisiti del progetto. In particolare, abbiamo pensato di porre alcune domande basilari come la scelta del linguaggio (Java o Javascript) e la scelta della lingua (italiano o inglese); inoltre, abbiamo formalizzato i dubbi che avevamo sulla natura stessa di questo progetto così particolare.



\section{Riunione del 15 dicembre 2016}

\begin{itemize}
	\item Orario: 14:30 - 17:00;
	\item Durata: 2 ore 30 minuti;
	\item Luogo incontro: \ZU, sede di via Cittadella 7, Padova; 
	\item Oggetto: discussione del capitolato d'appalto;
	\item Segretario: \LS; 
	\item Partecipanti: \GP, \AZ, \GG, \LB, \LS, \MM, \PB.
\end{itemize}

Durante la discussione sono state discusse alcune scelte prese dal gruppo riguardanti le potenzialità del software e suggerite altre di nuove. Abbiamo compreso che il dominio di programmi software che il nostro editor dovrà modellare dev'essere mirato ad un particolare ambiente applicativo \textbf{[...]}. Abbiamo chiarito alcuni dubbi che avevamo [...]. Il proponente ci ha lasciato ampia libertà sul linguaggio da usare e sulla natura del software (applicazione web o desktop)[...]

\end{document}
