% Riunione interna del 17 marzo 2017
% da compilare con il comando pdflatex Esterni/Verbali/VI_17-03-17.tex

% Dichiarazioni di ambiente e inclusione di pacchetti
% da usare tramite il comando % Dichiarazioni di ambiente e inclusione di pacchetti
% da usare tramite il comando % Dichiarazioni di ambiente e inclusione di pacchetti
% da usare tramite il comando \input{../../util/hx-ambiente}

\documentclass[a4paper,titlepage]{article}
\usepackage[T1]{fontenc}
\usepackage[utf8]{inputenc}
\usepackage[english,italian]{babel}
\usepackage{microtype}
\usepackage{lmodern}
\usepackage{underscore}
\usepackage{graphicx}
\usepackage{eurosym}
\usepackage{float}
\usepackage{fancyhdr}
\usepackage[table,dvipsnames]{xcolor}
\usepackage{multirow}
\usepackage{longtable}
\usepackage{chngpage}
\usepackage{grffile}
\usepackage[titles]{tocloft}
\usepackage{hyperref}
\hypersetup{hidelinks}

\usepackage{../../util/hx-vers}
\usepackage{../../util/hx-macro}
\usepackage{../../util/hx-front}

% solo se si vuole una nuova pagina ad ogni \section:
\usepackage{titlesec}
\newcommand{\sectionbreak}{\clearpage}

% stile di pagina:
\pagestyle{fancy}

% solo se si vuole eliminare l'indentazione ad ogni paragrafo:
\setlength{\parindent}{0pt}

% intestazione:
\lhead{\Large{\proj}}
\rhead{\includegraphics[keepaspectratio=true,width=50px]{../../util/hivex_logo2.png}}
\renewcommand{\headrulewidth}{0.4pt}

% pie' di pagina:
\lfoot{\email}
\rfoot{\thepage}
\cfoot{}
\renewcommand{\footrulewidth}{0.4pt}

% spazio verticale tra le celle di una tabella:
\renewcommand{\arraystretch}{1.5}

% profondità di indicizzazione:
\setcounter{tocdepth}{4}
\setcounter{secnumdepth}{4}

% numerazione innestata per elenchi numerati:
\renewcommand{\labelenumii}{\theenumii}
\renewcommand{\theenumii}{\theenumi.\arabic{enumii}.}


\documentclass[a4paper,titlepage]{article}
\usepackage[T1]{fontenc}
\usepackage[utf8]{inputenc}
\usepackage[english,italian]{babel}
\usepackage{microtype}
\usepackage{lmodern}
\usepackage{underscore}
\usepackage{graphicx}
\usepackage{eurosym}
\usepackage{float}
\usepackage{fancyhdr}
\usepackage[table,dvipsnames]{xcolor}
\usepackage{multirow}
\usepackage{longtable}
\usepackage{chngpage}
\usepackage{grffile}
\usepackage[titles]{tocloft}
\usepackage{hyperref}
\hypersetup{hidelinks}

\usepackage{../../util/hx-vers}
\usepackage{../../util/hx-macro}
\usepackage{../../util/hx-front}

% solo se si vuole una nuova pagina ad ogni \section:
\usepackage{titlesec}
\newcommand{\sectionbreak}{\clearpage}

% stile di pagina:
\pagestyle{fancy}

% solo se si vuole eliminare l'indentazione ad ogni paragrafo:
\setlength{\parindent}{0pt}

% intestazione:
\lhead{\Large{\proj}}
\rhead{\includegraphics[keepaspectratio=true,width=50px]{../../util/hivex_logo2.png}}
\renewcommand{\headrulewidth}{0.4pt}

% pie' di pagina:
\lfoot{\email}
\rfoot{\thepage}
\cfoot{}
\renewcommand{\footrulewidth}{0.4pt}

% spazio verticale tra le celle di una tabella:
\renewcommand{\arraystretch}{1.5}

% profondità di indicizzazione:
\setcounter{tocdepth}{4}
\setcounter{secnumdepth}{4}

% numerazione innestata per elenchi numerati:
\renewcommand{\labelenumii}{\theenumii}
\renewcommand{\theenumii}{\theenumi.\arabic{enumii}.}


\documentclass[a4paper,titlepage]{article}
\usepackage[T1]{fontenc}
\usepackage[utf8]{inputenc}
\usepackage[english,italian]{babel}
\usepackage{microtype}
\usepackage{lmodern}
\usepackage{underscore}
\usepackage{graphicx}
\usepackage{eurosym}
\usepackage{float}
\usepackage{fancyhdr}
\usepackage[table,dvipsnames]{xcolor}
\usepackage{multirow}
\usepackage{longtable}
\usepackage{chngpage}
\usepackage{grffile}
\usepackage[titles]{tocloft}
\usepackage{hyperref}
\hypersetup{hidelinks}

\usepackage{../../util/hx-vers}
\usepackage{../../util/hx-macro}
\usepackage{../../util/hx-front}

% solo se si vuole una nuova pagina ad ogni \section:
\usepackage{titlesec}
\newcommand{\sectionbreak}{\clearpage}

% stile di pagina:
\pagestyle{fancy}

% solo se si vuole eliminare l'indentazione ad ogni paragrafo:
\setlength{\parindent}{0pt}

% intestazione:
\lhead{\Large{\proj}}
\rhead{\includegraphics[keepaspectratio=true,width=50px]{../../util/hivex_logo2.png}}
\renewcommand{\headrulewidth}{0.4pt}

% pie' di pagina:
\lfoot{\email}
\rfoot{\thepage}
\cfoot{}
\renewcommand{\footrulewidth}{0.4pt}

% spazio verticale tra le celle di una tabella:
\renewcommand{\arraystretch}{1.5}

% profondità di indicizzazione:
\setcounter{tocdepth}{4}
\setcounter{secnumdepth}{4}

% numerazione innestata per elenchi numerati:
\renewcommand{\labelenumii}{\theenumii}
\renewcommand{\theenumii}{\theenumi.\arabic{enumii}.}


\version{1.0.0}
\creaz{22 marzo 2017}
\author{\GG}
\supervisor{\LB}
\uso{interno}
\dest{\ALL}
\title{Riunione interna del 17 marzo 2017}

\begin{document}

\maketitle

\begin{description}
	\item[Orario] 09:30 - 12:30;
	\item[Durata] 3 ore;
	\item[Luogo incontro] Laboratorio di Torre Archimede;
	\item[Oggetto] discussione sull'esito della RP;
	\item[Segretario] \AZ;
	\item[Partecipanti] \ALL.
\end{description}

Decisioni prese:
\begin{enumerate}
	\item \textbf{colloquio con il \GP} --- Abbiamo deciso di fissare un incontro con il proponente, \GP, con i seguenti obiettivi:
	\begin{enumerate}
		\item presentazione delle scelte architetturali adottate;
		\item modifica dei requisiti (rimozione della gestione automatica della visibilità delle variabili) e spiegazione del perché;
		\item esposizione del compromesso adottato tra:
		\begin{itemize}
			\item estensibilità dell'applicazione, sul lungo termine;
			\item sua immediata ricchezza di funzionalità.
		\end{itemize}
		\item tecnologie usate.
	\end{enumerate}
	\item \textbf{domande per il \RC} --- Abbiamo deciso di chiedere al \RC{} delle precisazioni sulla revisione dei documenti; in particolare sul termine \emph{service} e sui diagrammi più adatti (tra quelli di attività e di sequenza) per una Specifica Tecnica.
	\item \textbf{colloquio con il \TV} --- Avendo letto la revisione, abbiamo deciso di chiedere al \TV{} delle precisazioni sui seguenti punti:
	\begin{enumerate}
		\item possibile relazione tra identificatore numerico dei verbali e il registro delle modifiche;
		\item nell'Analisi dei Requisiti, rapporto tra requisiti relativi alla modifica di elementi e requisiti relativi all'inserimento di elementi;
		\item obiettivo e significato delle presentazioni ad ogni revisione di avanzamento (per poter migliorare le prossime presentazioni);
		\item nell'Analisi dei Requisiti, necessità o meno di rimuovere la ridondanza tra estensione e scenario alternativo;
		\item quando inserire i test di unità nel prodotto.
	\end{enumerate}
	\item \textbf{miglioramento delle presentazioni} --- Per migliorare le nostre presentazioni, abbiamo deciso di concentrarci sul rispetto degli intervalli di tempo assegnati ad ognuno durante la presentazione; inoltre, abbiamo pattuito di fare delle prove in più (cronometrate).
	\item \textbf{correzioni per la Specifica Tecnica} --- Abbiamo discusso dell'utilizzo di \emph{Spring Boot} (invece di \emph{Spring}) e abbiamo convenuto di adottarlo; inoltre, abbiamo chiarito alcune correzioni che dovremo apportare al documento di Specifica Tecnica.
\end{enumerate}

\end{document}
